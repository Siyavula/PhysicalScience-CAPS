         \chapter{The periodic table}\fancyfoot[LO,RE]{Chemistry: Matter and materials}
    \setcounter{figure}{1}
    \setcounter{subfigure}{1}
 \label{m38760*cid9}
            \section{The arrangement of atoms in the periodic table}
            \nopagebreak
            \label{m38760} $ \hspace{-5pt}\begin{array}{cccccccccccc}   \includegraphics[width=0.75cm]{col11305.imgs/summary_video.png} &   \end{array} $ \hspace{2 pt}\raisebox{-5 pt}{} {(section shortcode: P10025 )} \par 
      \label{m38760*id261491}The \textbf{periodic table of the elements} is a method of showing the chemical elements in a table with the elements arranged in order of increasing atomic number. Most of the work that was done to arrive at the periodic table that we know, can be attributed to a man called \textbf{Dmitri Mendeleev} in 1869. Mendeleev was a Russian chemist who designed the table in such a way that recurring ("periodic") trends or patterns in the properties of the elements could be shown. Using the trends he observed, he left gaps for those elements that he thought were 'missing'. He also predicted the properties that he thought the missing elements would have when they were discovered. Many of these elements were indeed discovered and Mendeleev's predictions were proved to be correct.\par 
      \label{m38760*id261511}To show the recurring properties that he had observed, Mendeleev began new rows in his table so that elements with similar properties were in the same vertical columns, called \textbf{groups}. Each row was referred to as a \textbf{period}. Figure~\ref{fig:atom:periodic} shows a simplified version of the periodic table. The full periodic table is reproduced at the front of this book. You can view an online periodic table at \textsl{http://periodictable.com/}. \par 
    \setcounter{subfigure}{0}
	\begin{figure}[H] % horizontal\label{m38760*uid133}
 \begin{center}
\begin{pspicture}(-7.8,-1)(6.8,4)
%\psgrid[gridcolor=gray]
\psset{unit=0.75}
\psline(-9,4)(-9,0)
\psline(-8,4)(-8,0)
\psline(-7,3)(-7,0)
\psline(-6,1)(-6,0)
\psline(-5,1)(-5,0)
\psline(-4,1)(-4,0)
\psline(-3,1)(-3,0)
\psline(-2,1)(-2,0)
\psline(-1,1)(-1,0)
\psline(0,1)(0,0)
\psline(1,1)(1,0)
\psline(2,1)(2,0)
\psline(3,3)(3,0)
\psline(4,3)(4,0)
\psline(5,3)(5,0)
\psline(6,3)(6,0)
\psline(7,3)(7,0)
\psline(8,4)(8,0)
\psline(9,4)(9,0)
\psline(-9,4)(-8,4)
\psline(-9,3)(-7,3)
\psline(-9,2)(-7,2)
\psline(-9,1)(9,1)
\psline(-9,0)(9,0)
\psline(3,1)(9,1)
\psline(3,2)(9,2)
\psline(3,3)(9,3)
\psline(8,4)(9,4)
\rput(-8.5,3.5){\textbf{H}}
\rput(-8.5,2.5){\textbf{Li}}
\rput(-8.5,1.5){\textbf{Na}}
\rput(-8.5,0.5){\textbf{K}}
\rput(-7.5,2.5){\textbf{Be}}
\rput(-7.5,1.5){\textbf{Mg}}
\rput(-7.5,0.5){\textbf{Ca}}
\rput(-6.5,0.5){\textbf{Sc}}
\rput(-5.5,0.5){\textbf{Ti}}
\rput(-4.5,0.5){\textbf{V}}
\rput(-3.5,0.5){\textbf{Cr}}
\rput(-2.5,0.5){\textbf{Mn}}
\rput(-1.5,0.5){\textbf{Fe}}
\rput(-0.5,0.5){\textbf{Co}}
\rput(0.5,0.5){\textbf{Ni}}
\rput(1.5,0.5){\textbf{Cu}}
\rput(2.5,0.5){\textbf{Zn}}
\rput(3.5,2.5){\textbf{B}}
\rput(3.5,1.5){\textbf{Al}}
\rput(3.5,0.5){\textbf{Ga}}
\rput(4.5,2.5){\textbf{C}}
\rput(4.5,1.5){\textbf{Si}}
\rput(4.5,0.5){\textbf{Ge}}
\rput(5.5,2.5){\textbf{N}}
\rput(5.5,1.5){\textbf{P}}
\rput(5.5,0.5){\textbf{As}}
\rput(6.5,2.5){\textbf{O}}
\rput(6.5,1.5){\textbf{S}}
\rput(6.5,0.5){\textbf{Se}}
\rput(7.5,2.5){\textbf{F}}
\rput(7.5,1.5){\textbf{Cl}}
\rput(7.5,0.5){\textbf{Br}}
\rput(8.5,3.5){\textbf{He}}
\rput(8.5,2.5){\textbf{Ne}}
\rput(8.5,1.5){\textbf{Ar}}
\rput(8.5,0.5){\textbf{Kr}}
\psline[linewidth=0.1,arrows=<->](-9.5,4)(-9.5,0)
\pcline[linestyle=none](-9.5,0)(-9.5,4)
\aput{:U}{Group}
\rput(-8.5,5){group number}
\rput(-8.5,4.6){1}
\rput(-7.5,3.5){2}
\rput(3.5,3.5){13}
\rput(4.5,3.5){14}
\rput(5.5,3.5){15}
\rput(6.5,3.5){16}
\rput(7.5,3.5){17}
\rput(8.5,4.5){18}
\psline[linewidth=0.1,arrows=->](-7,-0.5)(5,-0.5)
\pcline[linestyle=none](-7,-0.5)(5,-0.5)
\bput{:U}{Period}
\end{pspicture}
\end{center}
\caption{A simplified diagram showing part of the Periodic Table}
\label{fig:atom:periodic}
 \end{figure}       
            
\subsection*{Definitions and important concepts}
Before we can talk about the trends in the periodic table, we first need to define some terms that are used. 
\begin{itemize}
\item \textbf{Atomic radius} \\
The atomic radius is a measure of the size of an atom. 
\item \textbf{Ionisation energy}\\
The first ionisation energy is the energy needed to remove one electron from an atom in the gas phase. The ionisation energy is different for each element. We can also define second, third, fourth, etc. ionisation energies. These are the energies needed to remove the second, third, or fourth electron respectively. 
\item \textbf{Electron affinity}\\
Electron affinity can be thought of as how much an element wants electrons.
\item \textbf{Electronegativity}\\
Electronegativity is the tendency of atoms to attract electrons.
\item A \textbf{group} is a vertical column in the periodic table and is considered to be the most important way of classifying the elements. If you look at a periodic table, you will see the groups numbered at the top of each column. The groups are numbered from left to right starting with 1 and ending with 18. This is the convention that we will use in this book. On some periodic tables you may see that the groups are numbered from left to right  as follows: 1, 2, then an open space which contains the \textbf{transition elements}, followed by groups 3 to 8. Another way to label the groups is using Roman numerals.
\item A \textbf{period} is a horizontal row in the periodic table of the elements. The periods are labelled from top to bottom, starting with 1 and ending with 7.
\end{itemize}


      \label{m38760*uid146}
            \subsection*{Periods in the periodic table}
            \nopagebreak
            \label{m38760*id261855} The following diagram illustrates some of the key trends in the periods: \\
\begin{figure}[H]

\begin{center}
\scalebox{0.7}{
\begin{pspicture}(0,0)(10,10)
%\psgrid[gridcolor=gray]
\rput(5,5){
\psset{unit=0.75}
\pspolygon(0,0)(0,7)(1,7)(1,6)(2,6)(2,4)(12,4)(12,6)(17,6)(17,7)(18,7)(18,0)(0,0)
\rput(1,8){\Large{period number}}
\rput(-1,6.5){\Large{1}}
\rput(-1,5.5){\Large{2}}
\rput(-1,4.5){\Large{3}}
\rput(-1,3.5){\Large{4}}
\rput(-1,2.5){\Large{5}}
\rput(-1,1.5){\Large{6}}
\rput(-1,0.5){\Large{7}}
\psline[linewidth=0.08,arrowinset=0]{<-}(0.2,0.5)(17.8,0.5)
\rput(8.8,1.1){\Large{Atomic radius}}
\psline[linewidth=0.08,arrowinset=0]{->}(0.2,1.8)(17.8,1.8)
\rput(8.8,2.3){\Large{Ionization energy}}
\psline[linewidth=0.08,arrowinset=0]{->}(0.2,3.1)(17.8,3.1)
\rput(8.8,3.5){\Large{Electronegativity}}
}
\end{pspicture}
}
\end{center}

\caption{Trends on the periodic table.}
\label{fig:atom:periodic2}
 \end{figure} 
From this we see that the ionisation energy increases across a period. The atomic radius decreases across a period. The electronegativity increases across a period. Table~\ref{tab:period3trends} summarises the patterns or trends in the properties of the elements in period 3. Similar trends are observed in the other periods of the periodic table. \\
\begin{table}[H]
 \begin{center}
  \begin{tabular}{|p{2cm}|p{1.5cm}|p{1.5cm}|p{1.5cm}|p{1.5cm}|p{1.5cm}|p{1.5cm}|p{1.5cm}|p{1.5cm}|} \hline
 & $^{23}_{11}\mathsf{Na}$ & $^{24}_{12}\mathsf{Mg}$ &  $^{27}_{13}\mathsf{Al}$ & $^{28}_{14}\mathsf{Si}$ &  $^{31}_{15}\mathsf{P}$ & $^{32}_{16}\mathsf{S}$ & $^{35}_{17}\mathsf{Cl}$ & $^{40}_{18}\mathsf{Ar}$\\ \hline
   Compounds with chlorine & $\mathsf{NaCl}$ & $\mathsf{MgCl}_2$ & $\mathsf{AlCl}_{3}$ & $\mathsf{SiCl}_{4}$ & $\mathsf{PCl}_{5}$ or $\mathsf{PCl}_{3}$ & $\mathsf{S}_{2}\mathsf{Cl}_{2}$ & no chlorides & no compounds \\ \hline
Compounds with oxygen & $\mathsf{Na}_{2}\mathsf{O}$ & $\mathsf{MgO}$ & $\mathsf{Al}_{2}\mathsf{O}_{3}$ & $\mathsf{SiO}_{2}$ & $\mathsf{P}_{4}\mathsf{O}_{6}$ or $\mathsf{P}_{4}\mathsf{O}_{10}$ & $\mathsf{SO}_{3}$ or $\mathsf{SO}_{4}$ & $\mathsf{Cl}_{2}\mathsf{O}_{7}$ or $\mathsf{Cl}_{2}\mathsf{O}$ & no compounds \\ \hline
% Electron configuration & $1s^{2}2s^{2}2p^{6}3s^{1}$ & $1s^{2}2s^{2}2p^{6}3s^{2}$ & $1s^{2}2s^{2}2p^{6}3s^{2}3p^{1}$ & $1s^{2}2s^{2}2p^{6}3s^{2}3p^{2}$ & $1s^{2}2s^{2}2p^{6}3s^{2}3p^{3}$ & $1s^{2}2s^{2}2p^{6}3s^{2}3p^{4}$ & $1s^{2}2s^{2}2p^{6}3s^{2}3p^{5}$ & $1s^{2}2s^{2}2p^{6}3s^{2}3p^{6}$ \\ \hline
Atomic radius & \multicolumn{8}{p{8cm}|}{Decreases across a period.} \\ \hline
First ionisation energy & \multicolumn{8}{p{8cm}|}{The general trend is an increase across the period.} \\ \hline
Electro-negativity & \multicolumn{8}{p{8cm}|}{Increases across the period.} \\ \hline
Melting and boiling point & \multicolumn{8}{p{8cm}|}{Increases to silicon and then decreases to argon.} \\ \hline
Electrical conductivity & \multicolumn{8}{p{10cm}|}{Increases from sodium to aluminium. Silicon is a semi-conductor. The rest are insulators.} \\ \hline
  \end{tabular}
\caption{Summary of the trends in group 1}
\label{tab:period3trends}
 \end{center}

\end{table}
\begin{exercises}{Periods on the periodic table}
Use Table~\ref{tab:period3trends} and Figure~\ref{fig:atom:periodic2} to help you produce a similar table for the elements in period 2.	
\end{exercises}
\label{m38760*eip-151}Using the properties of the groups and the trends that we observe in certain properties (ionization energy, formation of halides and oxides, melting and boiling points, atomic diameter) we can predict the the properties of unknown elements. For example, the properties of the unfamiliar elements Francium (Fr), Barium (Ba), Astatine (At), and Xenon (Xe) can be predicted by knowing their position on the periodic table. Using the periodic table we can say: Francium (Group 1) is an alkali metal, very reactive and needs to lose 1 electron to obtain a full outer energy shell; Barium (Group 2) is an alkali earth metal and needs to lose 2 electrons to achieve stability; Astatine (Group 7) is a halogen, very reactive and needs to gain 1 electron to obtain a full outer energy shell; and Xenon (Group 8) is a noble gas and thus stable due to its full outer energy shell.   This is how scientists are able to say what sort of properties the atoms in the last period have. Almost all of the elements in this period do not occur naturally on earth and are made in laboratories. These atoms do not exist for very long (they are very unstable and break apart easily) and so measuring their properties is difficult.\par \label{m38760*secfhsst!!!underscore!!!id1062}
\subsection*{Groups in the periodic table}
            \nopagebreak
            \label{m38760*id261554} In some groups, the elements display very similar chemical properties and the groups are even given special names to identify them.\par 
        \label{m38760*id261833}The following diagram illustrates some of the key trends in the groups of the periodic table: \\
\begin{figure}[H]

\begin{center}
\scalebox{0.7}{
\begin{pspicture}(0,0)(20,20)
%\psgrid[gridcolor=gray]
\rput(5,5){
\psset{unit=0.75}
\pspolygon(0,0)(0,7)(1,7)(1,6)(2,6)(2,4)(12,4)(12,6)(17,6)(17,7)(18,7)(18,0)(0,0)
\rput(0,8){\Large{group number}}
\rput(0.5,7.5){\Large{1}}
\rput(1.5,6.5){\Large{2}}
\rput(12.5,6.5){\Large{13}}
\rput(13.5,6.5){\Large{14}}
\rput(14.5,6.5){\Large{15}}
\rput(15.5,6.5){\Large{16}}
\rput(16.5,6.5){\Large{17}}
\rput(17.5,7.5){\Large{18}}
\psline[linewidth=0.1,arrowinset=0]{->}(0.8,6.8)(0.8,0.2)
\uput[0](0.8,3){\Large{Atomic}}
\uput[0](0.8,2.3){\Large{diameter}}
\psline[linewidth=0.1,arrowinset=0]{<-}(4.2,6.8)(4.2,0.2)
\uput[0](4.2,3){\Large{Ionization}}
\uput[0](4.2,2.3){\Large{energy}}
\psline[linewidth=0.1,arrowinset=0]{<-}(7.5,6.8)(7.5,0.2)
\uput[0](7.5,3){\Large{Electro-}}
\uput[0](7.5,2.3){\Large{negativity}}
\psline[linewidth=0.1,arrowinset=0]{<-}(10.9,6.8)(10.9,0.2)
\uput[0](10.9,3){\Large{Melting and}}
\uput[0](10.9,2.3){\Large{boiling point}}
\psline[linewidth=0.1,arrowinset=0]{->}(14.8,6.8)(14.8,0.2)
\uput[0](14.8,3){\Large{Density}}
}
\end{pspicture}
}
\end{center}
\caption{Trends in the groups on the periodic table.}
\label{fig:atom:periodic1}
 \end{figure} 
From this we see that that the diameter of the atoms increases down a group. The ionisation energy and the electronegativity both decrease down a group. \\
Table~\ref{tab:group1trends} summarises the patterns or trends in the properties of the elements in group 1. Similar trends are observed for the elements in the other groups of the periodic table. \\
\begin{table}[H]
 \begin{center}
  \begin{tabular}{|l|p{1cm}|p{1cm}|p{1cm}|p{1cm}|p{1cm}|} \hline
 & $^{7}_{3}\mathsf{Li}$ & $^{7}_{3}\mathsf{Na}$ &  $^{7}_{3}\mathsf{K}$ & $^{7}_{3}\mathsf{Rb}$ &  $^{7}_{3}\mathsf{Cs}$ \\ \hline
   \multirow{2}{*}{Group 1 chlorides} & $\mathsf{LiCl}$ & $\mathsf{NaCl}$ & $\mathsf{KCl}$ & $\mathsf{RbCl}$ & $\mathsf{CsCl}$ \\ \cline{2-6}
   & \multicolumn{5}{|p{6cm}|}{Group 1 elements all form halogen compounds in a 1:1 ratio} \\ \hline
\multirow{2}{*}{Group 1 oxides} & $\mathsf{Li}_{2}\mathsf{O}}$ & $\mathsf{Na}_{2}\mathsf{O}}$ & $\mathsf{K}_{2}\mathsf{O}}$ & $\mathsf{Rb}_{2}\mathsf{O}}$ & $\mathsf{Cs}_{2}\mathsf{O}}$\\ \cline{2-6}
   & \multicolumn{5}{|l|}{Group 1 elements all form oxides in a 1:1 ratio} \\ \hline
Atomic radius & \multicolumn{5}{p{6cm}|}{Increases as you move down the group.} \\ \hline
First ionisation energy & \multicolumn{5}{p{6cm}|}{Decreases as you move down the group.} \\ \hline
Electronegativity & \multicolumn{5}{p{6cm}|}{Decreases as you move down the group.} \\ \hline
Melting and boiling point & \multicolumn{5}{p{6cm}|}{Decreases as you move down the group.} \\ \hline
Density & \multicolumn{5}{p{6cm}|}{Increases as you move down the group.} \\ \hline
  \end{tabular}
\caption{Summary of the trends in group 1}
\label{tab:group1trends}
 \end{center}

\end{table}
\begin{exercises}{Groups on the periodic table}
Use Table~\ref{tab:group1trends} and Figure~\ref{fig:atom:periodic1} to help you produce similar tables for group 2 and group 17.
\end{exercises}

\par \label{m38760*eip-148}
    \setcounter{subfigure}{0}
	\begin{figure}[H] % horizontal\label{m38760*periodictable-1}
    \textnormal{Khan academy video on the periodic table - 1}\vspace{.1in} \nopagebreak
  \label{m38760*yt-media1}\label{m38760*yt-video1}
            \raisebox{-5 pt}{ \includegraphics[width=0.5cm]{col11305.imgs/summary_www.png}} { (Video:  P10026 )}
      \vspace{2pt}
    \vspace{.1in}
 \end{figure}       \par 
\begin{exercises}{Elements in the periodic table}
            \nopagebreak
\begin{enumerate}[noitemsep, label=\textbf{\arabic*}. ]
 \item The following two atoms are given. Compare these elements in terms of the following properties. Explain the differences in each case.
$^{24}_{12}\mathsf{Mg}$ and $^{40}_{20}\mathsf{Ca}$. 
\begin{enumerate}[noitemsep, label=\textbf{\arabic*}. ]
 \item Size of the atom (atomic radius)
\item Electronegativity
\item First ionisation energy
\item Boiling point
\end{enumerate}
 \item Study the following graph and explain the trend in electronegativity of the group 2 elements.
\begin{pspicture}(-2.25,-2)(29,12)
  \psaxes[axesstyle=axes,Dx=1,Dy=0.5,labels=y,ticks=y]{-}(5,1.5)
  \listplot[plotstyle=bar,barwidth=0.5cm]{0.5 1.5
1.5 1.3
2.5 1
3.5 1
4.5 0.9}
\rput{0}(0.5,-0.2){$\mathsf{Be}$}
\rput{0}(1.5,-0.2){$\mathsf{Mg}$}
\rput{0}(2.5,-0.2){$\mathsf{Ca}$}
\rput{0}(3.5,-0.2){$\mathsf{Sr}$}
\rput{0}(4.5,-0.2){$\mathsf{Ba}$}
\end{pspicture}
 
\item            \label{m38760*id262476}Refer to the elements listed below: \label{m38760*id7632}\begin{itemize}[noitemsep]
            \item Lithium ($\mathrm{Li}$)\item Chlorine ($\mathrm{Cl}$)\item Magnesium ($\mathrm{Mg}$)\item Neon ($\mathrm{Ne}$)\item Oxygen ($\mathrm{O}$)\item Calcium ($\mathrm{Ca}$)\item Carbon ($\mathrm{C}$)\end{itemize}
         Which of the elements listed above:
        \label{m38760*id262499}\begin{enumerate}[noitemsep, label=\textbf{\arabic*}. ] 
            \label{m38760*uid158}\item belongs to Group 1
\label{m38760*uid159}\item is a halogen
\label{m38760*uid160}\item is a noble gas
\label{m38760*uid161}\item is an alkali metal
\label{m38760*uid162}\item has an atomic number of 12
\label{m38760*uid163}\item has 4 neutrons in the nucleus of its atoms
\label{m38760*uid164}\item contains electrons in the 4th energy level
\label{m38760*uid165}\item has only one valence electron
\label{m38760*uid166}\item has all its energy orbitals full
\label{m38760*uid167}\item will have chemical properties that are most similar
\label{m38760*uid168}\item will form positive ions
\end{enumerate}
\end{enumerate}
         \par 
\label{m38760**end}
\par \raisebox{-5 pt}{\includegraphics[width=0.5cm]{col11305.imgs/summary_www.png}} Find the answers with the shortcodes:
 \par \begin{tabular}[h]{cccccc}
 (1.) liw  & \end{tabular}

\end{exercises}

         \subsection*{Trends}
    \nopagebreak
            \label{m38757} $ \hspace{-5pt}\begin{array}{cccccccccccc}   \includegraphics[width=0.75cm]{col11305.imgs/summary_video.png} &   \includegraphics[width=0.75cm]{col11305.imgs/summary_presentation.png} &   \end{array} $ \hspace{2 pt}\raisebox{-5 pt}{} {(section shortcode: P10027 )} \par 
      \label{m38757*uid120}
            \subsubsection*{Ions}
            \nopagebreak
            \label{m38757*id260625}In the previous section, we focused our attention on the electron configuration of \textsl{neutral} atoms. In a neutral atom, the number of protons is the same as the number of electrons. But what happens if an atom \textsl{gains} or \textsl{loses} electrons? Does it mean that the atom will still be part of the same element?\par 
        \label{m38757*id260647}A change in the number of electrons of an atom does not change the type of atom that it is. However, the \textsl{charge} of the atom will change. If electrons are added, then the atom will become \textsl{more negative}. If electrons are taken away, then the atom will become \textsl{more positive}. The atom that is formed in either of these cases is called an \textbf{ion}. An ion is a charged atom.\par 


\Definition{   \label{id2424842} Ion } { \label{m38757*meaningfhsst!!!underscore!!!id815}
        \label{m38757*id260685}An ion is a charged atom. A positively charged ion is called a \textbf{cation} e.g. ${\mathrm{Na}}^{+}$, and a negatively charged ion is called an \textbf{anion} e.g. ${\mathrm{F}}^{-}$. The charge on an ion depends on the number of electrons that have been lost or gained.} 
        

\label{m38757*eip-879}But how do we know how many electrons an atom will gain or lose? Remember what we said about stability? We said that all atoms are trying to get a full outer shell. For the elements on the left hand side of the periodic table the easiest way to do this is to lose electrons and for the elements on the right of the periodic table the easiest way to do this is to gain electrons. So the elements on the left of the periodic table will form cations and the elements on the right hand side of the periodic table will form anions. By doing this the elements can be in the most stable electronic configuration and so be as stable as the noble gases.\par \label{m38757*id260742}Look at the following examples. Notice the number of valence electrons in the neutral atom, the number of electrons that are lost or gained and the final charge of the ion that is formed.\par 
        \label{m38757*id260747}\noindent{}\textbf{Lithium}
A lithium atom loses one electron to form a positive ion:

\begin{figure}[H] % horizontal\label{m38757*uid121}
\begin{center}
\begin{pspicture}(-5,1)(7,4)
\SpecialCoor
%\psgrid[gridcolor=lightgray]
\rput(0,3){
\pscircle(0,0){1.25}
\pscircle(0,0){0.75}
\pscircle[fillcolor=lightgray,fillstyle=solid](0,0){0.25}
\multido{\n=90+180}{2}{\pscircle[fillcolor=black,fillstyle=solid]({0.75;\n}){0.1}}
\pscircle[fillcolor=black,fillstyle=solid]({1.25;0}){0.1}
\psline(2,0.8)(0.9,0.8)
\uput[r](2,0.8){second energy level}
\psline(2,-0.4)(0.6,-0.4)
\uput[r](2,-0.4){first energy level}
\psline(2,0)({0.75;90})
\psline(2,0)({1.25;0})
\uput[r](2,0){electrons}
\psline({0.25;-45})(0.8,-1.2)(2,-1.2)
\uput[r](2,-1.2){\parbox[l]{4cm}{nucleus, containing 3 protons and 4 neutrons}}
}
\end{pspicture}
\caption{The arrangement of electrons in a lithium atom.}
\label{fig:atom:lithium}
\end{center}
\end{figure}       

In this example, the lithium atom loses an electron to form the cation ${\mathrm{Li}}^{+}$.\par 
        \label{m38757*id260803}\noindent{}\textbf{Fluorine}
A fluorine atom gains one electron to form a negative ion:
    \setcounter{subfigure}{0}
	\begin{figure}[H] % horizontal\label{m38757*uid122}
   \begin{center}
\begin{pspicture}(-5.5,-2)(7,1.5)
\pscircle(0,0){1.25}
\pscircle(0,0){0.75}
\pscircle[fillcolor=lightgray,fillstyle=solid](0,0){0.25}
\multido{\n=90+180}{2}{\pscircle[fillcolor=black,fillstyle=solid]({0.75;\n}){0.1}}
\multido{\n=0+45}{7}{\pscircle[fillcolor=black,fillstyle=solid]({1.25;\n}){0.1}}
\end{pspicture}
\caption{The arrangement of electrons in a fluorine atom.}
\label{fig:atom:fluorine}
\end{center}

 \end{figure}       \par 
\label{m38757*eip-273}You should have noticed in both these examples that each element lost or gained electrons to make a full outer shell.\par \label{m38757*secfhsst!!!underscore!!!id842}
\begin{activity}{The formation of ions}
            \nopagebreak
        \label{m38757*id260856}\begin{enumerate}[noitemsep, label=\textbf{\arabic*}. ] 
            \label{m38757*uid123}\item Use the diagram for lithium as a guide and draw similar diagrams to show how each of the following ions is formed:
\label{m38757*id260872}\begin{enumerate}[noitemsep, label=\textbf{\alph*}. ] 
            \label{m38757*uid124}\item ${\mathrm{Mg}}^{2+}$ \label{m38757*uid125}\item ${\mathrm{Na}}^{+}$ \label{m38757*uid126}\item ${\mathrm{Cl}}^{-}$ \label{m38757*uid127}\item ${\mathrm{O}}^{2+}$ \end{enumerate}
        \label{m38757*uid128}\item Do you notice anything interesting about the charge on each of these ions? Hint: Look at the number of valence electrons in the neutral atom and the charge on the final ion.
\end{enumerate}
        \label{m38757*id261008}\noindent{}\textbf{Observations: }\newline
    Once you have completed the activity, you should notice that:
        \label{m38757*id261012}\begin{itemize}[noitemsep]
            \label{m38757*uid129}\item In each case the number of electrons that is either gained or lost, is the same as the number of electrons that are needed for the atoms to achieve a full outer energy level.
\label{m38757*uid130}\item If you look at an energy level diagram for sodium ($Na$), you will see that in a neutral atom, there is only one valence electron. In order to achieve a full outer energy level, and therefore a more stable state for the atom, this electron will be \textsl{lost}.
\label{m38757*uid131}\item In the case of oxygen ($\mathrm{O}$), there are six valence electrons. To achieve a full energy level, it makes more sense for this atom to \textsl{gain} two electrons. A negative ion is formed.
\end{itemize}
        \par \label{m38757*eip-531}
\end{activity}            


\begin{exercises}{The formation of ions}
            \nopagebreak
            \label{m38757*id261135}Match the information in column A with the information in column B by writing only the letter (A to I) next to the question number (1 to 7)
\begin{center}
\begin{tabular}{|p{6cm}|p{2cm}|}\hline
1. A positive ion that has 3 less electrons than its neutral atom & A. Mg$^{2+}$ \\\hline
2. An ion that has 1 more electron than its neutral atom & B. Cl$^{-}$ \\\hline
3. The anion that is formed when bromine gains an electron & C. CO$_{3}^{2-}$ \\\hline
4. The cation that is formed from a magnesium atom & D. Al$^{3+}$ \\\hline
5. An example of a compound ion & E. Br$^{2-}$ \\\hline
6. A positive ion with the electron configuration of argon & F. K$^{+}$ \\\hline
7. A negative ion with the electron configuration of neon & G. Mg$^{+}$ \\\hline
 & H. O$^{2-}$ \\\hline
 & I. Br$^{-}$ \\\hline
\end{tabular}
\end{center}
      
\label{m38757*uid132}
\par \raisebox{-5 pt}{\includegraphics[width=0.5cm]{col11305.imgs/summary_www.png}} Find the answers with the shortcodes:
 \par \begin{tabular}[h]{cccccc}
 (1.) lid  & \end{tabular}

\end{exercises}

            \subsubsection*{Ionisation Energy}
            \nopagebreak
            \label{m38757*id261078}Ionisation energy is the energy that is needed to remove one electron from an atom in the gas phase. The ionisation energy will be different for different atoms.\par 
        \label{m38757*eip-622}

      \Tip{When we talk of ionisation energies and calculate these energies the atoms involved are in the gas phase.}


      \label{m38757*id261083}The second ionisation energy is the energy that is needed to remove a second electron from an atom, and so on across a period. As an energy level becomes more full, it becomes more and more difficult to remove an electron and the ionisation energy \textsl{increases}. The ionisation energy \textsl{increases} across a period. Down a group, the outermost electrons are further away from the nucleus. It becomes easier to remove an electron, less energy is required, so the ionisation energy decreases as you move down a group.\par 
\label{m38757*eip-865}

\begin{exercises}{Trends in ionisation energy}
Refer to the data table below which gives the ionisation energy (in kJ.mol$^{-1}$) and atomic number (Z) for a number of elements in the periodic table:\\

\begin{center}
\begin{tabular}{|l|l|c|l|l|c|}\hline
\textbf{Z} & Name of element & Ionisation energy & \textbf{Z} & Name of element & Ionisation energy \\\hline
1 & Hydrogen  & 1310 & 10 & Neon       & 2072 \\\hline
2 & Helium    & 2360 & 11 & Sodium     & 494  \\\hline
3 & Lithium   & 517  & 12 & Magnesium  & 734  \\\hline
4 & Beryllium & 895  & 13 & Aluminium  & 575  \\\hline
5 & Boron     & 797  & 14 & Silicon    & 783  \\\hline
6 & Carbon    & 1087 & 15 & Phosphorus & 1051 \\\hline
7 & Nitrogen  & 1397 & 16 & Sulphur    & 994  \\\hline
8 & Oxygen    & 1307 & 17 & Chlorine   & 1250 \\\hline
9 & Fluorine  & 1673 & 18 & Argon      & 1540 \\\hline
\end{tabular}
\end{center}

\begin{enumerate}[noitemsep, label=\textbf{\arabic*}. ]
\item{Draw a line graph to show the relationship between atomic number (on the x-axis) and ionisation energy (y-axis).}
\item{Describe any trends that you observe.}
\item{Explain why...
	\begin{enumerate}[noitemsep, label=\textbf{\arabic*}. ]
	\item{the ionisation energy for Z=2 is higher than for Z=1}
	\item{the ionisation energy for Z=3 is lower than for Z=2}
	\item{the ionisation energy increases between Z=5 and Z=7}
	\end{enumerate}
}
\end{enumerate}
      \label{m38757*secfhsst!!!underscore!!!id936}
\par \raisebox{-5 pt}{\includegraphics[width=0.5cm]{col11305.imgs/summary_www.png}} Find the answers with the shortcodes:
 \par \begin{tabular}[h]{cccccc}
 (1.) liv  & \end{tabular}
\end{exercises}
        \par 
        \label{m38757*eip-6}
    \setcounter{subfigure}{0}
	\begin{figure}[H] % horizontal\label{m38757*periodictable-3}
    \textnormal{Khan academy video on periodic table - 2}\vspace{.1in} \nopagebreak
  \label{m38757*yt-media3}\label{m38757*yt-video3}
            \raisebox{-5 pt}{ \includegraphics[width=0.5cm]{col11305.imgs/summary_www.png}} { (Video:  P10028 )}
      \vspace{2pt}
    \vspace{.1in}
 \end{figure}       \par 
        \label{m38757*id261577}The characteristics of each group are mostly determined by the electron configuration of the atoms of the elements in the group.\par 
        \label{m38757*id261581}\begin{itemize}[noitemsep]
            \label{m38757*uid135}\item \textsl{Group 1:} These elements are known as the \textbf{alkali metals} and they are very reactive. Note that although hydrogen appears in group 1, it is not an alkali metal.
    \setcounter{subfigure}{0}
	\begin{figure}[H] % horizontal\label{m38757*uid139}
    \begin{center}
\begin{pspicture}(-6,-2)(4,2)
\SpecialCoor
%\psgrid[gridcolor=lightgray]

\rput(-3,-1.85){Lithium}
\rput(-3,0){
\pscircle[fillcolor=lightgray,fillstyle=solid](0,0){0.25}
\pscircle(0,0){0.5}
\pscircle(0,0){1}
\pscircle[fillcolor=black,fillstyle=solid]({0.5;90}){0.1}
\pscircle[fillcolor=black,fillstyle=solid]({0.5;270}){0.1}
\pscircle[fillcolor=black,fillstyle=solid]({1;45}){0.1}
}

\rput(-0.5,-1.85){Sodium}
\rput(-0.5,0){
\pscircle[fillcolor=lightgray,fillstyle=solid](0,0){0.25}
\pscircle(0,0){0.5}
\pscircle(0,0){1}
}

\rput(2.5,-1.85){Potassium}
\rput(2.5,0){
\pscircle[fillcolor=lightgray,fillstyle=solid](0,0){0.25}
\pscircle(0,0){0.5}
\pscircle(0,0){1}
\pscircle(0,0){1.5}
}
\end{pspicture}
\caption{Electron diagrams for some of the Group 1 elements, with sodium and potasium incomplete; to be completed as an exercise.} 
\label{fig:Group1Elements}
\end{center}
 \end{figure}       \label{m38757*uid140}\item \textsl{Group 2:} These elements are known as the \textbf{alkali earth metals}. Each element only has two valence electrons and so in chemical reactions, the group 2 elements tend to \textsl{lose} these electrons so that the energy shells are complete. These elements are less reactive than those in group 1 because it is more difficult to lose two electrons than it is to lose one.
\label{m38757*uid141}\item \textsl{Group 13} elements have three valence electrons.
\item \textsl{Group 14:} These elements have 4 valence electrons. They can lose all 4 electrons to become cations of charge +4 or gain 4 electrons to become anions of charge -4.
\item \textsl{Group 15:} These elements have 5 valence electrons. These elements are sometimes called the pnictogens.
\label{m38757*id6232}\item \textsl{Group 16:} These elements are sometimes known as the chalcogens. These elements are fairly reactive and tend to gain two electrons to fill their outer shell.
\label{m38757*uid142}\item \textsl{Group 17:} These elements are known as the \textbf{halogens}. Each element is missing just one electron from its outer energy shell. These elements tend to \textsl{gain} electrons to fill this shell, rather than losing them. These elements are also very reactive.
\label{m38757*uid143}\item \textsl{Group 18:} These elements are the \textbf{noble gases}. All of the energy shells of the halogens are full and so these elements are very unreactive.
    \setcounter{subfigure}{0}
	\begin{figure}[H] % horizontal\label{m38757*uid144}
    \begin{center}
\begin{pspicture}(-6,-2)(-2,2)
\SpecialCoor
%\psgrid[gridcolor=lightgray]
\rput(-5,-1.85){Helium}
\rput(-5,0){\pscircle(0,0){0.5}
\pscircle[fillcolor=lightgray,fillstyle=solid](0,0){0.25}
\pscircle[fillcolor=black,fillstyle=solid]({0.5;90}){0.1}
\pscircle[fillcolor=black,fillstyle=solid]({0.5;270}){0.1}
}
\rput(-3,-1.85){Neon}
\rput(-3,0){
\pscircle[fillcolor=lightgray,fillstyle=solid](0,0){0.25}
\pscircle(0,0){0.5}
\pscircle(0,0){1}
\multido{\n=90+180}{2}{\pscircle[fillcolor=black,fillstyle=solid]({0.5;\n}){0.1}}
\multido{\n=0+45}{8}{\pscircle[fillcolor=black,fillstyle=solid]({1;\n}){0.1}}
}
\end{pspicture}
\caption{Electron diagrams for two of the noble gases, helium (He) and neon (Ne).}
\label{fig:NobleGases}
\end{center}

 \end{figure}       \label{m38757*uid145}\item \textsl{Transition metals:} Groups 3 - 12. For example, copper is a transition metal. Copper can lose one, two or three elecrons to become a cation. 
\end{itemize}            

\begin{activity}{The properties of elements }
            \nopagebreak
            \label{m38757*uid798724}Refer to Figure~\ref{fig:Group1Elements}.
\label{m38757*id261630}\begin{enumerate}[noitemsep, label=\textbf{\arabic*}. ] 
            \label{m38757*uid136}\item MAgnesium and calcium are two elements in group 2. Draw electron diagrams for magnesium and calcium.
\label{m38757*uid137}\item What do you notice about the number of electrons in the valence energy level in each case?
\label{m38757*uid138}\item Explain why elements from group 1 are more reactive than elements from group 2 on the periodic table (Hint: Think about the 'ionisation energy').
\end{enumerate}
        \par \end{activity}
\section{Chemical properties}
% \label{m38757*eip-921}You should also be able to indicate where metals, non-metals and metalloids are found on the periodic table. If you do not recall where these lie, then refer to classification of matter. \label{m38757*eip-215}By now you should have an appreciation of what the periodic table can tell us. The periodic table does not just list the elements, but tells chemists what the properties of elements are, how the elements will combine and many other useful facts. The periodic table is truly an amazing resource. Into one simple table, chemists have packed so many facts and data that can easily be seen with a glance. The periodic table is a crucial part of chemistry and you should never go to science class without it. \label{m38757*eip-325}


The following presentation provides a summary of the periodic table
    \setcounter{subfigure}{0}
	\begin{figure}[H] % horizontal\label{m38757*slidesharefigure}
    \label{m38757*slidesharemedia}\label{m38757*slideshareflash}\raisebox{-5 pt}{ \includegraphics[width=0.5cm]{col11305.imgs/summary_www.png}} { (Presentation:  P10029 )}
      \vspace{2pt}
    \vspace{.1in}
 \end{figure}       \par 
    \label{m38757*eip-572}

\begin{activity}{Inventing your own periodic table}
            \nopagebreak
            \label{m38760*eip-603}
\begin{minipage}{.5\textwidth}
You are the official chemist for the planet Zog. You have discovered all the same elements that we have here on Earth, but you don't have a periodic table. The citizens of Zog want to know how all these elements relate to each other. How would you invent the periodic table? Think about how you would organize the data that you have and what properties you would include. Do not simply copy Mendeleev's ideas, be creative and come up with some of your own. Research other forms of the periodic table and make one that makes sense to you. Present your ideas to your class. 
\end{minipage}
\begin{minipage}{.5\textwidth}
\begin{center}
\includegraphics[width=.8\textwidth]{photos/Circular_periodic_table.png}\\
\textsl{Picture from Wikimedia commons}
\end{center}
\end{minipage}

\end{activity}
            \section{Summary}
            \nopagebreak
            \label{m38757*uid0123}\begin{itemize}[noitemsep]
            \label{m38757*id79342}\item Elements are arranged in periods and groups on the periodic table. The elements are arranged according to increasing atomic number. 
\label{m38757*id97342}\item A \textbf{group} is a column on the periodic table containing elements with similar properties. A \textbf{period} is a row on the periodic table.
\item The groups on the periodic table are labeled from 1 to 18. Group 1 is known as the alkali metals, group 2 is known as the alkali earth metals, the group 17 is known as the halogens and the group 18 is known as the noble gases. The elements in a group have similar properties.\item Several trends such as ionisation energy and atomic diameter can be seen across the periods of the periodic table. \label{m38757*uid184}\item An \textbf{ion} is a charged atom. A \textbf{cation} is a positively charged ion and an \textbf{anion} is a negatively charged ion.
\label{m38757*uid185}\item When forming an ion, an atom will lose or gain the number of electrons that will make its valence (or outermost) energy level full.
\label{m38757*uid186}\item An element's \textbf{ionisation energy} is the energy that is needed to remove one electron from an atom.
\label{m38757*uid187}\item Ionisation energy increases across a \textbf{period} in the periodic table.
\label{m38757*uid188}\item Ionisation energy decreases down a \textbf{group} in the periodic table.
\end{itemize}
        \label{m38757*eip-219}
            



\begin{eocexercises}{ Periodic Table}
            \nopagebreak
            \label{m38757*uid091221}\begin{enumerate}[noitemsep, label=\textbf{\arabic*}. ] 
            \item For the following questions state whether they are true or false. If they are false, correct the statement. \label{m38757*id073324}\begin{enumerate}[noitemsep, label=\textbf{\alph*}. ] 
            \item The group 1 elements are sometimes known as the alkali earth metals.
\item The group 2 elements tend to lose 2 electrons to form cations.
\item The group 8 elements are known as the noble gases.\item Group 7 elements are very unreactive.\item The transition elements are found between groups 3 and 4.\end{enumerate}
\item Give one word or term for each of the following:
\label{m38757*id0734}\begin{enumerate}[noitemsep, label=\textbf{\alph*}. ] 
            \item A positive ion
\item The energy that is needed to remove one electron from an atom\item A horizontal row on the periodic table\item A very reactive group of elements that is missing just one electron from their outer shells.\end{enumerate}
\item For each of the following elements give the ion that will be formed:
\label{m38757*id07324}\begin{enumerate}[noitemsep, label=\textbf{\alph*}. ] 
            \item sodium
\item bromine
\item magnesium
\item oxygen
\end{enumerate}
\label{m38757*uid222}\item The following table shows the first ionisation energies for the elements of period 1 and 2.
    % \textbf{m38757*id263866}\par
 \begin{center} 
\begin{tabular}{|c|c|c|} \hline 
Period&Element&First ionisation energy ($kJ.mol^{-1}$)\\[6pt] \hline 
1&H&1312\\ &He&2372\\ \hline 
&Li&520\\ &Be&899\\ &B&801\\ &C&1086\\ 2&N&1402\\ &O&1314\\ &F&1681\\ &Ne&2081\\ \hline 
\end{tabular}
\end{center} 
    \par
  \label{m38757*id264201}\begin{enumerate}[noitemsep, label=\textbf{\alph*}. ] 
            \label{m38757*uid223}\item What is the meaning of the term \textsl{first ionisation energy}?
\label{m38757*uid224}\item Identify the pattern of first ionisation energies in a period.
\label{m38757*uid225}\item Which TWO elements exert the strongest attractive forces on their electrons? Use the data in the table to give a reason for your answer.
\label{m38757*uid226}\item Draw Aufbau diagrams for the TWO elements you listed in the previous question and explain why these elements are so stable.
\label{m38757*uid227}\item It is safer to use helium gas than hydrogen gas in balloons. Which property of helium makes it a safer option?
\label{m38757*uid228}\item 'Group 1 elements readily form positive ions'.
Is this statement correct? Explain your answer by referring to the table.
\end{enumerate}
        \end{enumerate}
\label{m38757**end}
  \label{4e3d8e3d8992782b4e5d6fd958df32f9**end}
\par \raisebox{-5 pt}{\includegraphics[width=0.5cm]{col11305.imgs/summary_www.png}} Find the answers with the shortcodes:
 \par \begin{tabular}[h]{cccccc}
 (1.) l4Z  &  (2.) l4K  &  (3.) l4B  &  (4.) li6  & \end{tabular}
\end{eocexercises}
