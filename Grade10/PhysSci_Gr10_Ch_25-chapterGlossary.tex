      \chapter{Glossary}
      \label{col11305*Glossary}
      \end{indexheading}
      \vspace{.3cm}
      
    \begin{description}\setlength{\topsep}{0cm}\setlength{\itemsep}{0cm}
    \setlength{\parskip}{0cm}\setlength{\parsep}{0cm}
    \setlength{\partopsep}{0cm}
    \setlength{\labelwidth}{.6cm}\setlength{\labelsep}{0cm}
    \setlength{\leftmargin}{1cm}

    
	    \vspace{.3cm}
	    \item[{\large \bfseries A}]\noindent\raggedright
	    {\bf  Acceleration }\\\begin{description}\item{\hspace{.3cm}}\hspace{.3cm}
      \label{m38794*id67550}Acceleration is the rate of change of velocity. \par 
      \\\end{description}

	    \item[] \noindent\raggedright {\bf  Acid rain }\\\begin{description}\item{\hspace{.3cm}}\hspace{.3cm}
       Acid rain refers to the deposition of acidic components in rain, snow and dew. Acid rain occurs when sulphur dioxide and nitrogen oxides are emitted into the atmosphere, undergo chemical transformations and are absorbed by water droplets in clouds. The droplets then fall to earth as rain, snow, mist, dry dust, hail, or sleet. This increases the acidity of the soil and affects the chemical balance of lakes and streams. 
        \\\end{description}

	    \item[] \noindent\raggedright {\bf  Amplitude }\\\begin{description}\item{\hspace{.3cm}}\hspace{.3cm}
        The \textbf{amplitude} is the maximum displacement from equilibrium. For a longitudinal wave which is a pressure wave this would be the maximum increase (or decrease) in pressure from the equilibrium pressure that is cause when a peak (or trough) passes a point.
        \\\end{description}

	    \item[] \noindent\raggedright {\bf  Amplitude }\\\begin{description}\item{\hspace{.3cm}}\hspace{.3cm}
        \label{m38806*id318011}The \textsl{amplitude} is the maximum displacement of a particle from its equilibrium position. \par 
        \\\end{description}

	    \item[] \noindent\raggedright {\bf  Amplitude }\\\begin{description}\item{\hspace{.3cm}}\hspace{.3cm}
        \label{m38801*id312963}The amplitude of a pulse is a measurement of how far the medium is displaced from rest. \par 
        \\\end{description}

	    \item[] \noindent\raggedright {\bf  Atomic mass number (A) }\\\begin{description}\item{\hspace{.3cm}}\hspace{.3cm}
      \label{m38745*id255874}The number of protons and neutrons in the nucleus of an atom \par 
      \\\end{description}

	    \item[] \noindent\raggedright {\bf  Atomic number (Z) }\\\begin{description}\item{\hspace{.3cm}}\hspace{.3cm}
      \label{m38745*id255833}The number of protons in an atom \par 
      \\\end{description}

	    \item[] \noindent\raggedright {\bf  Atomic orbital }\\\begin{description}\item{\hspace{.3cm}}\hspace{.3cm}
        \label{m38741*id259495}An atomic orbital is the region in which an electron may be found around a single atom.
 \par 
        \\\end{description}

	    \item[] \noindent\raggedright {\bf  Attraction and Repulsion }\\\begin{description}\item{\hspace{.3cm}}\hspace{.3cm}
        \label{m37830*id128774}\textsl{Like} poles of magnets \textsl{repel}
each other whilst \textsl{unlike} poles \textsl{attract} each other. \par 
        \\\end{description}

	    \item[] \noindent\raggedright {\bf  Average velocity }\\\begin{description}\item{\hspace{.3cm}}\hspace{.3cm}
      \label{m38791*id64246}Average velocity is the total displacement of a body over a time interval. \par 
      \\\end{description}

	    \item[] \noindent\raggedright {\bf  Avogadro's number }\\\begin{description}\item{\hspace{.3cm}}\hspace{.3cm}
      \label{m38717*id276010}The number of particles in a mole, equal to \begin{math}6,022\ensuremath{\times}{10}^{23}\end{math}. \par 
      \\\end{description}
	    \vspace{.3cm}
	    \item[{\large \bfseries B}]\noindent\raggedright
	    {\bf  Boiling point }\\\begin{description}\item{\hspace{.3cm}}\hspace{.3cm}
The temperature at which a \textsl{liquid} changes 
its phase to become a \textsl{gas}. The process is 
called evaporation and the reverse process is called condensation 
\\\end{description}
	    \vspace{.3cm}
	    \item[{\large \bfseries C}]\noindent\raggedright
	    {\bf  Chemical bond }\\\begin{description}\item{\hspace{.3cm}}\hspace{.3cm}
      A chemical bond is the physical process that causes atoms and molecules to be attracted to each other, and held together in more stable chemical compounds. 
      \\\end{description}

	    \item[] \noindent\raggedright {\bf  Chemical change }\\\begin{description}\item{\hspace{.3cm}}\hspace{.3cm}
      The formation of new substances in a chemical reaction. One type of matter is changed into something different. 
      \\\end{description}

	    \item[] \noindent\raggedright {\bf  Compound }\\\begin{description}\item{\hspace{.3cm}}\hspace{.3cm}
      A compound is a group of two or more different atoms that are 
attracted to each other by relatively strong forces or bonds. 
      \\\end{description}

	    \item[] \noindent\raggedright {\bf  Compound }\\\begin{description}\item{\hspace{.3cm}}\hspace{.3cm}
        A substance made up of two or more elements that are joined together in a fixed ratio.
        \\\end{description}

	    \item[] \noindent\raggedright {\bf  Compression }\\\begin{description}\item{\hspace{.3cm}}\hspace{.3cm}
      A \textbf{compression} is a region in a longitudinal wave where the particles are closest together. 
      \\\end{description}

	    \item[] \noindent\raggedright {\bf  Concentration }\\\begin{description}\item{\hspace{.3cm}}\hspace{.3cm}
      \label{m38712*id282955}Concentration is a measure of the amount of solute that is dissolved in a given volume of liquid. It is measured in \begin{math}\mathrm{mol}\ensuremath{\cdot}{\mathrm{dm}}^{-3}\end{math}. Another term that is used for concentration is \textbf{molarity (M)} \par 
      \\\end{description}

	    \item[] \noindent\raggedright {\bf  Conductivity }\\\begin{description}\item{\hspace{.3cm}}\hspace{.3cm}
      Conductivity is a measure of a solution's ability to conduct an electric current.
      \\\end{description}

	    \item[] \noindent\raggedright {\bf  Conductors and insulators }\\\begin{description}\item{\hspace{.3cm}}\hspace{.3cm}
      \label{m38706*id66124}A conductor allows the easy movement or flow of something such as heat or electrical charge through it. Insulators are the opposite to conductors because they \textsl{inhibit} or reduce the flow of heat, electrical charge, sound etc through them. \par 

      \\\end{description}

	    \item[] \noindent\raggedright {\bf  Conservation of energy principle }\\\begin{description}\item{\hspace{.3cm}}\hspace{.3cm}
      Energy cannot be created or destroyed. It can only be changed from one form to another. 
      \\\end{description}

	    \item[] \noindent\raggedright {\bf  Conservation of Energy }\\\begin{description}\item{\hspace{.3cm}}\hspace{.3cm}
        \label{m38786*id68470}The Law of Conservation of Energy: Energy cannot be created or destroyed, but is merely changed from one form into another. \par 
        \\\end{description}

	    \item[] \noindent\raggedright {\bf  Conservation of Mechanical Energy }\\\begin{description}\item{\hspace{.3cm}}\hspace{.3cm}
        \label{m38786*id68506}Law of Conservation of Mechanical Energy: The total amount of mechanical energy in a closed system remains constant. \par 
        \\\end{description}

	    \item[] \noindent\raggedright {\bf  Constructive interference}\\\begin{description}\item{\hspace{.3cm}}\hspace{.3cm}
      Constructive interference is when two pulses meet, resulting in a bigger pulse. 
      \\\end{description}

	    \item[] \noindent\raggedright {\bf  Core electrons }\\\begin{description}\item{\hspace{.3cm}}\hspace{.3cm}
        \label{m38741*id259989}All the electrons in an atom, excluding the valence electrons \par 
        \\\end{description}

	    \item[] \noindent\raggedright {\bf  Covalent bond }\\\begin{description}\item{\hspace{.3cm}}\hspace{.3cm}
        Covalent bonding is a form of chemical bonding where pairs of electrons are shared between atoms. 
        \\\end{description}

	    \item[] \noindent\raggedright {\bf  Current }\\\begin{description}\item{\hspace{.3cm}}\hspace{.3cm}
        \label{m38773*id66743}Current is the rate of flow of charge, i.e. the rate at which charges move past a fixed point in a circuit. We use the symbol I to show current and it is measured in amperes (A). One ampere is one coulomb of charge moving in one second. The relationship between current, charge and time is given by:\par 
        \label{m38773*id66751}\nopagebreak\noindent{}
          \settowidth{\mymathboxwidth}{\begin{equation}
    I=\frac{Q}{\Delta t}\tag{16.26}
      \end{equation}
    }
    \typeout{Columnwidth = \the\columnwidth}\typeout{math as usual width = \the\mymathboxwidth}
    \ifthenelse{\lengthtest{\mymathboxwidth < \columnwidth}}{% if the math fits, do it again, for real
    \begin{equation}
    I=\frac{Q}{\Delta t}\tag{16.26}
      \end{equation}
    }{% else, if it doesn't fit
    \setlength{\mymathboxwidth}{\columnwidth}
      \addtolength{\mymathboxwidth}{-48pt}
    \par\vspace{12pt}\noindent\begin{minipage}{\columnwidth}
    \parbox[t]{\mymathboxwidth}{\large\begin{math}
    I=\frac{Q}{\Delta t}\end{math}}\hfill
    \parbox[t]{48pt}{\raggedleft 
    (16.26)}
    \end{minipage}\vspace{12pt}\par
    }% end of conditional for this bit of math
    \typeout{math as usual width = \the\mymathboxwidth}
    
        
        
        \\\end{description}
	    \vspace{.3cm}
	    \item[{\large \bfseries D}]\noindent\raggedright
	    {\bf Density}\\\begin{description}\item{\hspace{.3cm}}\hspace{.3cm}Density is a measure of the mass of a substance per 
unit volume.\\\end{description}

	    \item[] \noindent\raggedright {\bf  Destructive interference}\\\begin{description}\item{\hspace{.3cm}}\hspace{.3cm}
      Destructive interference is when two pulses meet, resulting in a smaller pulse. 
      \\\end{description}

	    \item[] \noindent\raggedright {\bf  Displacement }\\\begin{description}\item{\hspace{.3cm}}\hspace{.3cm}
      \label{m38788*id62992}Displacement is the change in an object's position. \par 
      \\\end{description}

	    \item[] \noindent\raggedright {\bf  Dissociation }\\\begin{description}\item{\hspace{.3cm}}\hspace{.3cm}
        Dissociation in chemistry and biochemistry is a general process in which ionic compounds separate or split into smaller molecules or ions, usually in a reversible manner.  
        \\\end{description}
	    \vspace{.3cm}
	    \item[{\large \bfseries E}]\noindent\raggedright
	    {\bf  Electric circuit }\\\begin{description}\item{\hspace{.3cm}}\hspace{.3cm}
        \label{m38771*id62792}An electric circuit is a closed path (with no breaks or gaps) along which electrical charges (electrons) flow powered by an energy source. \par 
        \\\end{description}

	    \item[] \noindent\raggedright {\bf  Electrolyte }\\\begin{description}\item{\hspace{.3cm}}\hspace{.3cm}
        An electrolyte is a substance that contains free ions and behaves as an electrically conductive medium. Because they generally consist of ions in solution, electrolytes are also known as ionic solutions. 
        \\\end{description}

	    \item[] \noindent\raggedright {\bf  Electron configuration }\\\begin{description}\item{\hspace{.3cm}}\hspace{.3cm}
        \label{m38741*id259615}Electron configuration is the arrangement of electrons in an atom, molecule or other physical structure. \par 
        \\\end{description}

	    \item[] \noindent\raggedright {\bf  Element }\\\begin{description}\item{\hspace{.3cm}}\hspace{.3cm}
        An element is a substance that cannot be broken down into other substances through chemical means.
        \\\end{description}

	    \item[] \noindent\raggedright {\bf emf}\\\begin{description}\item{\hspace{.3cm}}\hspace{.3cm}
The emf (electromotive force) is the voltage measured across the terminals of a battery when \textsl{no current is flowing} through the battery.
\\\end{description}

	    \item[] \noindent\raggedright {\bf  Empirical formula }\\\begin{description}\item{\hspace{.3cm}}\hspace{.3cm}
      \label{m38712*id280341}The empirical formula of a chemical compound gives the relative number of each type of atom in that compound. \par 
      \\\end{description}

	    \item[] \noindent\raggedright {\bf  Empirical formula }\\\begin{description}\item{\hspace{.3cm}}\hspace{.3cm}
This is a way of expressing the \textsl{relative} 
number of each type of atom in a chemical compound. In most cases, the empirical 
formula does not show the exact number of atoms, but rather the simplest 
\textsl{ratio} of the atoms in the compound. 
\\\end{description}

	    \item[] \noindent\raggedright {\bf  Equivalent resistance in a parallel circuit, Rp }\\\begin{description}\item{\hspace{.3cm}}\hspace{.3cm}
          \label{m38776*id6613206}For \begin{math}n\end{math} resistors in parallel, the equivalent resistance is:\par 
          \label{m38776*uid2944}\nopagebreak\noindent{}
            \settowidth{\mymathboxwidth}{\begin{equation}
    \frac{1}{{R}_{p}}=\left(\frac{1}{{R}_{1}}+\frac{1}{{R}_{2}}+\frac{1}{{R}_{3}}+\cdots +\frac{1}{{R}_{n}}\right)\tag{16.38}
      \end{equation}
    }
    \typeout{Columnwidth = \the\columnwidth}\typeout{math as usual width = \the\mymathboxwidth}
    \ifthenelse{\lengthtest{\mymathboxwidth < \columnwidth}}{% if the math fits, do it again, for real
    \begin{equation}
    \frac{1}{{R}_{p}}=\left(\frac{1}{{R}_{1}}+\frac{1}{{R}_{2}}+\frac{1}{{R}_{3}}+\cdots +\frac{1}{{R}_{n}}\right)\tag{16.38}
      \end{equation}
    }{% else, if it doesn't fit
    \setlength{\mymathboxwidth}{\columnwidth}
      \addtolength{\mymathboxwidth}{-48pt}
    \par\vspace{12pt}\noindent\begin{minipage}{\columnwidth}
    \parbox[t]{\mymathboxwidth}{\large\begin{math}
    \frac{1}{{R}_{p}}=\left(\frac{1}{{R}_{1}}+\frac{1}{{R}_{2}}+\frac{1}{{R}_{3}}+\cdots +\frac{1}{{R}_{n}}\right)\end{math}}\hfill
    \parbox[t]{48pt}{\raggedleft 
    (16.38)}
    \end{minipage}\vspace{12pt}\par
    }% end of conditional for this bit of math
    \typeout{math as usual width = \the\mymathboxwidth}
    
          
          
          \\\end{description}

	    \item[] \noindent\raggedright {\bf  Equivalent resistance in a series circuit, Rs }\\\begin{description}\item{\hspace{.3cm}}\hspace{.3cm}
          \label{m38776*id64628}For \begin{math}n\end{math} resistors in series the equivalent resistance is:\par 
          \label{m38776*uid2532}\nopagebreak\noindent{}
            \settowidth{\mymathboxwidth}{\begin{equation}
    {R}_{s}={R}_{1}+{R}_{2}+{R}_{3}+\cdots +{R}_{n}\tag{16.31}
      \end{equation}
    }
    \typeout{Columnwidth = \the\columnwidth}\typeout{math as usual width = \the\mymathboxwidth}
    \ifthenelse{\lengthtest{\mymathboxwidth < \columnwidth}}{% if the math fits, do it again, for real
    \begin{equation}
    {R}_{s}={R}_{1}+{R}_{2}+{R}_{3}+\cdots +{R}_{n}\tag{16.31}
      \end{equation}
    }{% else, if it doesn't fit
    \setlength{\mymathboxwidth}{\columnwidth}
      \addtolength{\mymathboxwidth}{-48pt}
    \par\vspace{12pt}\noindent\begin{minipage}{\columnwidth}
    \parbox[t]{\mymathboxwidth}{\large\begin{math}
    {R}_{s}={R}_{1}+{R}_{2}+{R}_{3}+\cdots +{R}_{n}\end{math}}\hfill
    \parbox[t]{48pt}{\raggedleft 
    (16.31)}
    \end{minipage}\vspace{12pt}\par
    }% end of conditional for this bit of math
    \typeout{math as usual width = \the\mymathboxwidth}
    
          
          
          \\\end{description}

	    \item[] \noindent\raggedright {\bf  Equivalent resistance of two parallel resistor, Rp }\\\begin{description}\item{\hspace{.3cm}}\hspace{.3cm}
          \label{m38776*eip-id1170816625786}For \begin{math}2\end{math} resistors in parallel with resistances \begin{math}{R}_{1}\end{math} and \begin{math}{R}_{2}\end{math}, the equivalent resistance is:\par 
          \label{m38776*eip-id1170814512227}\nopagebreak\noindent{}\settowidth{\mymathboxwidth}{\begin{equation}
    {R}_{p}=\frac{{R}_{1}{R}_{2}}{{R}_{1}+{R}_{2}}\tag{16.37}
      \end{equation}
    }
    \typeout{Columnwidth = \the\columnwidth}\typeout{math as usual width = \the\mymathboxwidth}
    \ifthenelse{\lengthtest{\mymathboxwidth < \columnwidth}}{% if the math fits, do it again, for real
    \begin{equation}
    {R}_{p}=\frac{{R}_{1}{R}_{2}}{{R}_{1}+{R}_{2}}\tag{16.37}
      \end{equation}
    }{% else, if it doesn't fit
    \setlength{\mymathboxwidth}{\columnwidth}
      \addtolength{\mymathboxwidth}{-48pt}
    \par\vspace{12pt}\noindent\begin{minipage}{\columnwidth}
    \parbox[t]{\mymathboxwidth}{\large\begin{math}
    {R}_{p}=\frac{{R}_{1}{R}_{2}}{{R}_{1}+{R}_{2}}\end{math}}\hfill
    \parbox[t]{48pt}{\raggedleft 
    (16.37)}
    \end{minipage}\vspace{12pt}\par
    }% end of conditional for this bit of math
    \typeout{math as usual width = \the\mymathboxwidth}
    
          
          \\\end{description}
	    \vspace{.3cm}
	    \item[{\large \bfseries F}]\noindent\raggedright
	    {\bf  Frame of Reference }\\\begin{description}\item{\hspace{.3cm}}\hspace{.3cm}
        \label{m38787*id62637}A frame of reference is a reference point combined with a set of directions. \par 
        \\\end{description}

	    \item[] \noindent\raggedright {\bf  Frequency }\\\begin{description}\item{\hspace{.3cm}}\hspace{.3cm}
        The frequency is the number of successive peaks (or troughs) passing a given point in 1 second.
        \\\end{description}

	    \item[] \noindent\raggedright {\bf  Frequency }\\\begin{description}\item{\hspace{.3cm}}\hspace{.3cm}
        The \textbf{frequency} of a wave is the number of wavelengths per second.  
        \\\end{description}
	    \vspace{.3cm}
	    \item[{\large \bfseries G}]\noindent\raggedright
	    {\bf  Gradient }\\\begin{description}\item{\hspace{.3cm}}\hspace{.3cm}
        \label{m38795*id69226}The gradient of a line can be calculated by dividing the change in the \begin{math}y\end{math}-value by the change in the \begin{math}x\end{math}-value.\par 
        \label{m38795*id69250}m = \begin{math}\frac{\Delta y}{\Delta x}\end{math} \par 
        \\\end{description}
	    \vspace{.3cm}
	    \item[{\large \bfseries H}]\noindent\raggedright
	    {\bf  Heterogeneous mixture }\\\begin{description}\item{\hspace{.3cm}}\hspace{.3cm}
        A heterogeneous mixture is one that is non-uniform and the different components of the mixture can be seen.
        \\\end{description}

	    \item[] \noindent\raggedright {\bf  Homogeneous mixture }\\\begin{description}\item{\hspace{.3cm}}\hspace{.3cm}
        A homogeneous mixture is one that is uniform, and where the different components of the mixture cannot be seen. 
        \\\end{description}
	    \vspace{.3cm}
	    \item[{\large \bfseries I}]\noindent\raggedright
	    {\bf  Instantaneous velocity }\\\begin{description}\item{\hspace{.3cm}}\hspace{.3cm}
      \label{m38791*id64227}Instantaneous velocity is the velocity of a body at a specific instant in time. \par 
      \\\end{description}

	    \item[] \noindent\raggedright {\bf  Ion }\\\begin{description}\item{\hspace{.3cm}}\hspace{.3cm}
        \label{m38757*id260685}An ion is a charged atom. A positively charged ion is called a \textbf{cation} e.g. \begin{math}{\mathrm{Na}}^{+}\end{math}, and a negatively charged ion is called an \textbf{anion} e.g. \begin{math}{\mathrm{F}}^{-}\end{math}. The charge on an ion depends on the number of electrons that have been lost or gained. \par 
        \\\end{description}

	    \item[] \noindent\raggedright {\bf Ion exchange reaction}\\\begin{description}\item{\hspace{.3cm}}\hspace{.3cm}A type of reaction where the positive ions exchange their respective negative ions due to a driving force.\\\end{description}

	    \item[] \noindent\raggedright {\bf  Ionic bond }\\\begin{description}\item{\hspace{.3cm}}\hspace{.3cm}
        An ionic bond is a type of chemical bond based on the electrostatic forces between two oppositely-charged ions. When ionic bonds form, a metal donates one or more electrons, due to having a low electronegativity, to form a positive ion or cation. The non-metal atom has a high electronegativity, and therefore readily gains electrons to form a negative ion or anion. The two ions are then attracted to each other by electrostatic forces.
 
        \\\end{description}

	    \item[] \noindent\raggedright {\bf  Isotope }\\\begin{description}\item{\hspace{.3cm}}\hspace{.3cm}
        \label{m38753*id257386}The \textbf{isotope} of a particular element is made up of atoms which have the same number of protons as the atoms in the original element, but a different number of neutrons.  \par 
        \\\end{description}
	    \vspace{.3cm}
	    \item[{\large \bfseries K}]\noindent\raggedright
	    {\bf  Kinetic Energy }\\\begin{description}\item{\hspace{.3cm}}\hspace{.3cm}
      \label{m38785*id66784}Kinetic energy is the energy an object has due to its motion. \par 
      \\\end{description}
	    \vspace{.3cm}
	    \item[{\large \bfseries L}]\noindent\raggedright
	    {\bf  Longitudinal waves }\\\begin{description}\item{\hspace{.3cm}}\hspace{.3cm}
      A longitudinal wave is a wave where the particles in the medium move parallel to the direction of propagation of the wave. 
      \\\end{description}
	    \vspace{.3cm}
	    \item[{\large \bfseries M}]\noindent\raggedright
	    {\bf  Magnetism }\\\begin{description}\item{\hspace{.3cm}}\hspace{.3cm}
      \label{m38706*id67174}Magnetism is one of the phenomena by which materials exert attractive or repulsive forces on other materials. \par 
      \\\end{description}

	    \item[] \noindent\raggedright {\bf  Medium }\\\begin{description}\item{\hspace{.3cm}}\hspace{.3cm}
      \label{m38801*id312830}A medium is the substance or material in which a wave will move. \par 
      \\\end{description}

	    \item[] \noindent\raggedright {\bf  Melting point }\\\begin{description}\item{\hspace{.3cm}}\hspace{.3cm}
The temperature at which a \textsl{solid} changes 
its phase or state to become a \textsl{liquid}. The 
process is called melting and the reverse process (change in phase from liquid 
to solid) is called \textbf{freezing}. 
\\\end{description}

	    \item[] \noindent\raggedright {\bf  Metallic bond }\\\begin{description}\item{\hspace{.3cm}}\hspace{.3cm}
        Metallic bonding is the electrostatic attraction between the positively charged atomic nuclei of metal atoms and the delocalised electrons in the metal.
 
        \\\end{description}

	    \item[] \noindent\raggedright {\bf  Mixture }\\\begin{description}\item{\hspace{.3cm}}\hspace{.3cm}
      A \textbf{mixture} is a combination of two or more substances, where these substances are not bonded (or joined) to each other. 
      \\\end{description}

	    \item[] \noindent\raggedright {\bf  Model }\\\begin{description}\item{\hspace{.3cm}}\hspace{.3cm}
      \label{m38756*id254584}A model is a representation of a system in the real world. Models help us to understand systems and their properties. For example, an \textsl{atomic model} represents what the structure of an atom \textsl{could} look like, based on what we know about how atoms behave. It is not necessarily a true picture of the exact structure of an atom. \par 
      \\\end{description}

	    \item[] \noindent\raggedright {\bf  Molar mass }\\\begin{description}\item{\hspace{.3cm}}\hspace{.3cm}
      \label{m38717*id276397}Molar mass (M) is the mass of 1 mole of a chemical substance. The unit for molar mass is \textbf{grams per mole} or \begin{math}\mathrm{g}\ensuremath{\cdot}\mathrm{mol}{}^{-1}\end{math}. \par 
      \\\end{description}

	    \item[] \noindent\raggedright {\bf Molar volume of gases}\\\begin{description}\item{\hspace{.3cm}}\hspace{.3cm}1 mole of gas occupies \begin{math}22,4{\mathrm{dm}}^{3}\end{math} at S.T.P.\\\end{description}

	    \item[] \noindent\raggedright {\bf  Mole }\\\begin{description}\item{\hspace{.3cm}}\hspace{.3cm}
      \label{m38717*id275969}The mole (abbreviation 'n') is the SI (Standard International) unit for 'amount of substance'. \par 
      \\\end{description}

	    \item[] \noindent\raggedright {\bf  Molecular formula }\\\begin{description}\item{\hspace{.3cm}}\hspace{.3cm}
      \label{m38712*id280360}The molecular formula of a chemical compound gives the exact number of atoms of each element in one molecule of that compound. \par 
      \\\end{description}

	    \item[] \noindent\raggedright {\bf  Molecular formula }\\\begin{description}\item{\hspace{.3cm}}\hspace{.3cm}
This is a concise way of expressing information about the atoms that make up a 
particular chemical compound. The molecular formula gives the exact number of 
each type of atom in the molecule. 
\\\end{description}
	    \vspace{.3cm}
	    \item[{\large \bfseries P}]\noindent\raggedright
	    {\bf  Parallel circuit }\\\begin{description}\item{\hspace{.3cm}}\hspace{.3cm}
          \label{m38771*id63335}In a parallel circuit, the charge flowing from the battery can flow along \textbf{multiple} paths to return to the battery. \par 
          \\\end{description}

	    \item[] \noindent\raggedright {\bf  Peaks and troughs }\\\begin{description}\item{\hspace{.3cm}}\hspace{.3cm}
        \label{m38806*id317968}A \textsl{peak} is a point on the wave where the displacement of the medium is at a maximum. A point on the wave is a \textsl{trough} if the displacement of the medium at that point is at a minimum.  \par 
        \\\end{description}

	    \item[] \noindent\raggedright {\bf  Period (T) }\\\begin{description}\item{\hspace{.3cm}}\hspace{.3cm}The period (\begin{math}\mathrm{T}\end{math}) is the time taken for two successive peaks (or troughs) to pass a fixed point.
        
        \\\end{description}

	    \item[] \noindent\raggedright {\bf  Period }\\\begin{description}\item{\hspace{.3cm}}\hspace{.3cm}
       The \textbf{period} of a wave is the time taken by the wave to move one wavelength.
        \\\end{description}

	    \item[] \noindent\raggedright {\bf  Photon }\\\begin{description}\item{\hspace{.3cm}}\hspace{.3cm}
      \label{m38777*id187051}A photon is a quantum (energy packet) of light. \par 
      \\\end{description}

	    \item[] \noindent\raggedright {\bf  Physical change }\\\begin{description}\item{\hspace{.3cm}}\hspace{.3cm}
      A change that can be seen or felt, but that doesn't involve the break up of the particles in the reaction. During a physical change, the \textsl{form} of matter may change, but not its \textsl{identity}. A change in temperature is an example of a physical change. 
      \\\end{description}

	    \item[] \noindent\raggedright {\bf  Physical Quantity }\\\begin{description}\item{\hspace{.3cm}}\hspace{.3cm}
      \label{m30853*id62558}A physical quantity is anything
that you can measure. For example, length, temperature, distance
and time are physical quantities. \par 
      \\\end{description}

	    \item[] \noindent\raggedright {\bf  Planck's constant }\\\begin{description}\item{\hspace{.3cm}}\hspace{.3cm}
      \label{m38778*id188843}Planck's constant is a physical constant named after Max Planck.\par 
      \label{m38778*id188849}\begin{math}h=6,626\ensuremath{\times}{10}^{-34}\end{math} J \begin{math}\ensuremath{\cdot}\end{math} s
 \par 
      \\\end{description}

	    \item[] \noindent\raggedright {\bf  Position }\\\begin{description}\item{\hspace{.3cm}}\hspace{.3cm}
        \label{m38787*id62726}Position is a measurement of a location, with reference to an origin. \par 
        \\\end{description}

	    \item[] \noindent\raggedright {\bf  Potential Difference }\\\begin{description}\item{\hspace{.3cm}}\hspace{.3cm}
        \label{m38772*id63970}Electrical potential difference is the difference in electrical potential energy per unit charge between two points. The unit of potential difference is the volt\label{m38772*uid41}\footnote{named after the Italian physicist Alessandro Volta (1745--1827)} (V). 
The potential difference of a battery is the voltage measured across it when current is flowing through it.
\par 
        \\\end{description}

	    \item[] \noindent\raggedright {\bf  Potential energy }\\\begin{description}\item{\hspace{.3cm}}\hspace{.3cm}
      \label{m38784*id66155}Potential energy is the energy an object has due to its position or state. \par 
      \\\end{description}

	    \item[] \noindent\raggedright {\bf  Precipitate }\\\begin{description}\item{\hspace{.3cm}}\hspace{.3cm}
      A precipitate is the solid that forms in a solution during a chemical reaction.
      \\\end{description}

	    \item[] \noindent\raggedright {\bf  Prefix }\\\begin{description}\item{\hspace{.3cm}}\hspace{.3cm}
      \label{m30853*id65218}A prefix is a group of letters that are placed in front of a word. The effect of the prefix is to change meaning of the word. For example, the prefix \textsl{un} is often added to a word to mean \textsl{not}, as in \textsl{un}necessary which means \textsl{not necessary}. \par 
      \\\end{description}

	    \item[] \noindent\raggedright {\bf  Pulse }\\\begin{description}\item{\hspace{.3cm}}\hspace{.3cm}
      \label{m38801*id312926}A pulse is a single disturbance that moves through a medium. \par 
      \\\end{description}

	    \item[] \noindent\raggedright {\bf  Pulse Speed }\\\begin{description}\item{\hspace{.3cm}}\hspace{.3cm}
        \label{m38801*id313292}Pulse speed is the distance a pulse travels per unit time. \par 
        \\\end{description}
	    \vspace{.3cm}
	    \item[{\large \bfseries R}]\noindent\raggedright
	    {\bf  Rarefaction }\\\begin{description}\item{\hspace{.3cm}}\hspace{.3cm}
      A \textbf{rarefaction} is a region in a longitudinal wave where the particles are furthest apart. 
      \\\end{description}

	    \item[] \noindent\raggedright {\bf  Relative atomic mass }\\\begin{description}\item{\hspace{.3cm}}\hspace{.3cm}
        \label{m38756*id258580}Relative atomic mass is the average mass of one atom of all the naturally occurring isotopes of a particular chemical element, expressed in atomic mass units.
 \par 
        \\\end{description}

	    \item[] \noindent\raggedright {\bf  Representing circuits }\\\begin{description}\item{\hspace{.3cm}}\hspace{.3cm}
          \label{m38771*id63196}A \textbf{physical circuit} is the electric circuit you create with real components.\par 
          \label{m38771*id63206}A \textbf{circuit diagram} is a drawing which uses symbols to represent the different components in the physical circuit.
 \par 
          \\\end{description}

	    \item[] \noindent\raggedright {\bf  Resistance }\\\begin{description}\item{\hspace{.3cm}}\hspace{.3cm}
        \label{m38776*id67260} The resistance of a conductor is defined as the potential difference across it divided by the current flowing though it. We use the symbol \textbf{R} to show resistance and it is measured in units called \textbf{Ohms} with the symbol \begin{math}\mathrm{\Omega }\end{math}.\par 
        \label{m38776*id67288}\nopagebreak\noindent{}
          \settowidth{\mymathboxwidth}{\begin{equation}
    1\phantom{\rule{4pt}{0ex}}\mathrm{Ohm}=1\frac{\mathrm{Volt}}{\mathrm{Ampere}}.\tag{16.30}
      \end{equation}
    }
    \typeout{Columnwidth = \the\columnwidth}\typeout{math as usual width = \the\mymathboxwidth}
    \ifthenelse{\lengthtest{\mymathboxwidth < \columnwidth}}{% if the math fits, do it again, for real
    \begin{equation}
    1\phantom{\rule{4pt}{0ex}}\mathrm{Ohm}=1\frac{\mathrm{Volt}}{\mathrm{Ampere}}.\tag{16.30}
      \end{equation}
    }{% else, if it doesn't fit
    \setlength{\mymathboxwidth}{\columnwidth}
      \addtolength{\mymathboxwidth}{-48pt}
    \par\vspace{12pt}\noindent\begin{minipage}{\columnwidth}
    \parbox[t]{\mymathboxwidth}{\large\begin{math}
    1\phantom{\rule{4pt}{0ex}}\mathrm{Ohm}=1\frac{\mathrm{Volt}}{\mathrm{Ampere}}.\end{math}}\hfill
    \parbox[t]{48pt}{\raggedleft 
    (16.30)}
    \end{minipage}\vspace{12pt}\par
    }% end of conditional for this bit of math
    \typeout{math as usual width = \the\mymathboxwidth}
    
        
        
        \\\end{description}

	    \item[] \noindent\raggedright {\bf  Resultant of Vectors }\\\begin{description}\item{\hspace{.3cm}}\hspace{.3cm}
        \label{m38813*id188362}The resultant of a number of vectors is the single vector whose effect is the same as the individual vectors acting together. \par 
        \\\end{description}
	    \vspace{.3cm}
	    \item[{\large \bfseries S}]\noindent\raggedright
	    {\bf  Scalar }\\\begin{description}\item{\hspace{.3cm}}\hspace{.3cm}
      \label{m38812*id186740}A scalar is a quantity that has only magnitude (size). \par 
      \\\end{description}

	    \item[] \noindent\raggedright {\bf  Series circuit }\\\begin{description}\item{\hspace{.3cm}}\hspace{.3cm}
          \label{m38771*id63313}In a series circuit, the charge flowing from the battery can only flow along a \textbf{single} path to return to the battery. \par 
          \\\end{description}

	    \item[] \noindent\raggedright {\bf  SI Units }\\\begin{description}\item{\hspace{.3cm}}\hspace{.3cm}
        \label{m30853*id62598}The name \textsl{SI units} comes from the
French \textsl{Syst\`{e}me International d'Unit\'{e}s}, which means
\textsl{international system of units}. \par 
        \\\end{description}

	    \item[] \noindent\raggedright {\bf  Speed of sound }\\\begin{description}\item{\hspace{.3cm}}\hspace{.3cm}
     The speed of sound in air, at sea level, at a temperature of \begin{math}21{}^{\circ }\mathrm{C}\end{math} and under normal atmospheric conditions, is \begin{math}344\phantom{\rule{2pt}{0ex}}\mathrm{m}\ensuremath{\cdot}\mathrm{s}{}^{-1}\end{math}. 
      \\\end{description}
	    \vspace{.3cm}
	    \item[{\large \bfseries T}]\noindent\raggedright
	    {\bf  The law of conservation of mass }\\\begin{description}\item{\hspace{.3cm}}\hspace{.3cm}
        \label{m38726*id63208}The mass of a closed system of substances will remain constant, regardless of the processes acting inside the system. Matter can change form, but cannot be created or destroyed. For any chemical process in a closed system, the mass of the reactants must equal the mass of the products. \par 
        \\\end{description}

	    \item[] \noindent\raggedright {\bf  Transverse Pulse }\\\begin{description}\item{\hspace{.3cm}}\hspace{.3cm}
      \label{m38801*id3129262}A pulse where all of the particles disturbed by the pulse move perpendicular (at a right angle) to the direction in which the pulse is moving. \par 
      \\\end{description}

	    \item[] \noindent\raggedright {\bf  Transverse wave }\\\begin{description}\item{\hspace{.3cm}}\hspace{.3cm}
      \label{m38806*id317741}A \textsl{transverse wave} is a wave where the movement of the particles of the medium is perpendicular (at a right angle) to the direction of propagation of the wave. \par 
      \\\end{description}
	    \vspace{.3cm}
	    \item[{\large \bfseries V}]\noindent\raggedright
	    {\bf  Valence electrons }\\\begin{description}\item{\hspace{.3cm}}\hspace{.3cm}
        \label{m38741*id259971}The electrons in the outer energy level of an atom \par 
        \\\end{description}

	    \item[] \noindent\raggedright {\bf  Valency }\\\begin{description}\item{\hspace{.3cm}}\hspace{.3cm}
        The number of electrons in the outer shell of an atom which are able to be used to form bonds with other atoms. 
        \\\end{description}

	    \item[] \noindent\raggedright {\bf  Vectors }\\\begin{description}\item{\hspace{.3cm}}\hspace{.3cm}
      \label{m38812*id186786}A vector is a quantity that has both magnitude and direction. \par 
      \\\end{description}

	    \item[] \noindent\raggedright {\bf  Velocity }\\\begin{description}\item{\hspace{.3cm}}\hspace{.3cm}
      \label{m38791*id64209}Velocity is the rate of change of displacement. \par 
      \\\end{description}

	    \item[] \noindent\raggedright {\bf Viscosity}\\\begin{description}\item{\hspace{.3cm}}\hspace{.3cm}Viscosity is a measure of how resistant a liquid is to 
flowing (in other words, how easy it is to pour the liquid from one container to 
another).\\\end{description}
	    \vspace{.3cm}
	    \item[{\large \bfseries W}]\noindent\raggedright
	    {\bf  Water hardness }\\\begin{description}\item{\hspace{.3cm}}\hspace{.3cm}
        Water hardness is a measure of the mineral content of water. Minerals are substances such as calcite, quartz and mica that occur naturally as a result of geological processes. 
        \\\end{description}

	    \item[] \noindent\raggedright {\bf  Wave }\\\begin{description}\item{\hspace{.3cm}}\hspace{.3cm}
      \label{m38806*id317713}A \textsl{wave} is a periodic, continuous disturbance that consists of a \textsl{train} of pulses. \par 
      \\\end{description}

	    \item[] \noindent\raggedright {\bf  Wavelength of wave }\\\begin{description}\item{\hspace{.3cm}}\hspace{.3cm}
        \label{m38806*id319098}The wavelength of a wave is the distance between any two adjacent points that are in phase. \par 
        \\\end{description}

	    \item[] \noindent\raggedright {\bf  Wavelength }\\\begin{description}\item{\hspace{.3cm}}\hspace{.3cm}
        The \textbf{wavelength} in a longitudinal wave is the distance between two consecutive points that are in phase. 
        \\\end{description}
    \end{description}
  
      \newpage 
      \def\leftmark{INDEX}
      \def\rightmark{INDEX}
      \begin{indexheading}
      
      
