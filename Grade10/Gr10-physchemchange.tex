         \chapter{Physical and chemical change}\fancyfoot[LO,RE]{Chemistry: Chemical change}
    \setcounter{figure}{1}
    \setcounter{subfigure}{1}
    \label{m38709*cid1}
            \section{Introduction}
            \nopagebreak
            \label{m38709} $ \hspace{-5pt}\begin{array}{cccccccccccc}   \end{array} $ \hspace{2 pt}\raisebox{-5 pt}{\includegraphics[width=0.5cm]{col11305.imgs/summary_www.png}} {(section shortcode: P10056 )} \par 
      \label{m38709*id62175}Matter is all around us. The desks we sit at, the air we breathe and the water we drink are all examples of matter. But matter doesn't always stay the same. It can change in many different ways. In this chapter, we are going to take a closer look at \textbf{physical} and \textbf{chemical} changes that occur in matter.\par 
    \label{m38709*cid2}
            \subsection*{Physical changes in matter}
            \nopagebreak
      \label{m38709*id62200}A \textbf{physical change} is one where the particles of the substances that are involved in the change are not broken up in any way. When water is heated for example, the temperature and energy of the water molecules increases and the liquid water evaporates to form water vapour. When this happens, some kind of change has taken place, but the molecular structure of the water has not changed. This is an example of a \textsl{physical change}.\par 
      \label{m38709*id62556}$\mathsf{H}{}_{2}\mathsf{O}\left(\mathsf{l}\right)\to \mathsf{H}{}_{2}\mathsf{O}\left(\mathsf{g}\right)$
      \par 
      \label{m38709*id62600}Conduction (the transfer of energy through a material) is another example of a physical change. As energy is transferred from one material to another, the \textsl{energy} of each material is changed, but not its chemical makeup. Dissolving one substance in another is also a physical change.\par 
\label{m38709*fhsst!!!underscore!!!id76}
 \Definition{   \label{id2458225}Physical change } { \label{m38709*meaningfhsst!!!underscore!!!id76}
      A change that can be seen or felt, but that doesn't involve the break up of the particles in the reaction. During a physical change, the \textsl{form} of matter may change, but not its \textsl{identity}. A change in temperature is an example of a physical change. 
       } 

      \label{m38709*id62640}There are some important things to remember about physical changes in matter:\par 
      \label{m38709*id62644}\begin{enumerate}[noitemsep, label=\textbf{\arabic*}. ] 
            \label{m38709*uid1}\item \textsl{Arrangement of particles}\newline
When a physical change occurs, the compounds may re-arrange themselves, but the bonds in between the atoms will not break. For example when liquid water boils, the molecules will move apart but the molecule will stay intact. In other words water will not break up into hydrogen and oxygen atoms.

Figure~\ref{fig:physical change:water phases} shows this phase change. Note that the water molecules themselves stay the same, but their arrangement changed.
    \setcounter{subfigure}{0}
	\begin{figure}[H] % horizontal\label{m38709*uid2}
\begin{figure}[h]
\begin{center}
\begin{pspicture}(0,0)(10,2.6)
\SpecialCoor
%\psgrid[gridcolor=lightgray]
\def\water{\psset{unit=0.25}\rput{150}{\pscircle(0,0){2}
\psarc[fillcolor=white,fillstyle=solid](-1.5,1){1.5}{30}{260}
\psarc[fillcolor=white,fillstyle=solid](1.5,1){1.5}{280}{150}
\rput(-1.5,1){\pscurve(1.5;30)(-1;142.5)(1.5;260)}
\rput(1.5,1){\pscurve(1.5;150)(-1;37.5)(1.5;280)}}}

\rput(2,0){\psframe(0,0.5)(3,2.5)
\rput(1.5,1){\psset{unit=0.5}\rput(-0.7,-0.2){\water}
\rput{185}(-1,0.9){\water}
\rput{120}(2,0.6){\water}
\rput{310}(-2.1,0){\water}
\rput{60}(0.6,1.1){\water}
\rput(1.2,-0.2){\water}}}

\rput(5,0){\psframe(0,0.5)(3,2.5)
\rput(1.5,1){\psset{unit=0.5}\rput{120}(2,1){\water}
\rput{250}(-1,2){\water}
\rput{70}(0,.5){\water}
\rput{150}(-1.5,-.2){\water}}}

\uput[d](3,0.5){liquid}
\uput[d](6,0.5){gas}

\end{pspicture}
\end{center}
\caption{The arrangement of water molecules in the liquid and gas phase}
\label{fig:physical change:water phases}
\end{figure}
 \end{figure}   
\IFact{The bonding of hydrogen and oxygen to form water is explosive and if the water molecule broke apart everytime water boiled, life on Earth would not exist for very long!}    
\label{m38709*uid221}\item \textsl{Conservation of mass}\newline
    In a physical change, the total mass, the number of atoms and the number of molecules will always stay the same. In other words you will always have the same number of molecules or atoms at the end of the change as you had at the beginning. 
\label{m38709*uid3}\item \textsl{Energy changes}\newline
Energy changes may take place when there is a physical change in matter, but these energy changes are normally smaller than the energy changes that take place during a chemical change.
\label{m38709*uid4}\item \textsl{Reversibility}\newline
Physical changes in matter are usually easier to reverse than chemical changes. Water vapour for example, can be changed back to liquid water if the temperature is lowered. Liquid water can be changed into ice by simply decreasing the temperature.
\end{enumerate}
        \label{m38709*eip-904}\begin{activity}{Physical change}Use plastic pellets or marbles to represent water in the solid state. What do you need to do to the pellets to represent the change from solid to liquid? \par 
\end{activity}
    \label{m38709*cid3}
            \subsection*{Chemical Changes in Matter}
            \nopagebreak
      \label{m38709*id62778}When a \textbf{chemical change} takes place, new substances are formed in a chemical reaction. These new products may have very different properties from the substances that were there at the start of the reaction.\par 
            \label{m38709*fhsst!!!underscore!!!id107}
 \Definition{   \label{id2458579}\textbf{ Chemical change }} { \label{m38709*meaningfhsst!!!underscore!!!id107}
      The formation of new substances in a chemical reaction. One type of matter is changed into something different. 
       } 
      \label{m38709*id62788}The breakdown of hydrogen peroxide ($\mathsf{H}_{2}\mathsf{O}_{2}$) to form water ($\mathsf{H}_{2}\mathsf{O}$) and oxygen gas ($\mathsf{O}_{2}$) is an example of chemical change. A simplified diagram of this reaction is shown in Figure~\ref{fig:chemical change:decomposition}. The chemical bonds between $\mathsf{O}$ and $\mathsf{H}$ in $\mathsf{H}_{2}\mathsf{O}_{2}$ are broken and new bonds between $\mathsf{H}$ and $\mathsf{O}$ (to form $\mathsf{H}_{2}\mathsf{O}$) and between $\mathsf{O}$ and $\mathsf{O}$ (to form $\mathsf{O}_{2}$) are formed. A chemical change has taken place..\par 
    \setcounter{subfigure}{0}
\begin{figure}[h]
\begin{center}
\begin{pspicture}(-6,-1.5)(12,1.5)
%\psgrid[gridcolor=lightgray]
%reactants
\rput(-2,0){
\psellipse(-3,0)(0.5,0.5)
\rput(-3,0){$\mathsf{O}$}
\psellipse(-2,0)(0.5,0.5)
\rput(-2,0){$\mathsf{O}$}
\psellipse(-3.3,-0.75)(0.3,0.3)
\rput(-3.3,-0.75){$\mathsf{H}$}
\psellipse(-1.7,0.75)(0.3,0.3)
\rput(-1.7,0.75){$\mathsf{H}$}
\rput(3,0){
\psellipse(-3,0)(0.5,0.5)
\rput(-3,0){$\mathsf{O}$}
\psellipse(-2,0)(0.5,0.5)
\rput(-2,0){$\mathsf{O}$}
\psellipse(-3.3,-0.75)(0.3,0.3)
\rput(-3.3,-0.75){$\mathsf{H}$}
\psellipse(-1.7,0.75)(0.3,0.3)
\rput(-1.7,0.75){$\mathsf{H}$}
}
%products
\psline[arrows=->](2,0)(4,0)
\psellipse(5,0)(0.5,0.5)
\rput(5,0){$\mathsf{O}$}
\psellipse(4.5,0.65)(0.3,0.3)
\rput(4.5,0.65){$\mathsf{H}$}
\psellipse(5.5,0.65)(0.3,0.3)
\rput(5.5,0.65){$\mathsf{H}$}
\rput(2,-0.5){
\psellipse(5,0)(0.5,0.5)
\rput(5,0){$\mathsf{O}$}
\psellipse(4.5,0.65)(0.3,0.3)
\rput(4.5,0.65){$\mathsf{H}$}
\psellipse(5.5,0.65)(0.3,0.3)
\rput(5.5,0.65){$\mathsf{H}$}
}
\rput(8.5,0){\textbf{+}}
\psellipse(9.5,0)(0.5,0.5)
\rput(9.5,0){$\mathsf{O}$}
\psellipse(10.5,0)(0.5,0.5)
\rput(10.5,0){$\mathsf{O}$}
}
\end{pspicture}
\end{center}
\caption{The decomposition of H$_{2}$O$_{2}$ to form H$_{2}$O and O$_{2}$}
\label{fig:chemical change:decomposition}
\end{figure}     
\par
\begin{table}[H]
 \begin{center}
  \begin{tabular}{|l|l|l|} \hline
& \multicolumn{2}{|c|}{$2\mathsf{H}_{2}\mathsf{O}_{2} \rightarrow 2\mathsf{H}_{2}\mathsf{O} + \mathsf{O}_{2}$} \\ 
& \includegraphics[width=.1\textwidth]{photos/H2O2_models.png} & \includegraphics[width=.1\textwidth]{photos/H2O_O2.png} \\ \hline
   \textbf{Molecules} & two molecules & three molecules \\ \hline
\textbf{Energy changes} & energy taken in when bonds are broken & energy given off when bonds are formed \\ \hline
\textbf{Mass is conserved} & $4(1,01) + 4(16,0) = 68,04$ & $2(18,02) + 2(16,0) = 68,04$ \\ \hline
\textbf{Atoms are conserved} & 4 oxygen atoms, 4 hydrogen atoms & 4 oxygen atoms, 4 hydrogen atoms \\ \hline
  \end{tabular}
 \end{center}
\end{table}

      \label{m38709*id62865}There are some important things to remember about chemical changes:\par 
      \label{m38709*id62869}\begin{enumerate}[noitemsep, label=\textbf{\arabic*}. ] 
            \label{m38709*uid6}\item \textsl{Arrangement of particles}\newline
During a chemical change, the particles themselves are changed in some way. In the example of hydrogen peroxide that was used earlier, the $\mathsf{H}_{2}\mathsf{O}_{2}$ molecules were split up into their component atoms. The number of particles will change because each $\mathsf{H}_{2}\mathsf{O}_{2}$ molecule breaks down into two water molecules ($\mathsf{H}_{2}\mathsf{O}$) and one oxygen molecule ($\mathsf{O}_{2}$).
\label{m38709*uid7}\item \textsl{Energy changes}\newline
The energy changes that take place during a chemical reaction are much greater than those that take place during a physical change in matter. During a chemical reaction, energy is used up in order to break bonds, and then energy is released when the new product is formed. This will be discussed in more detail in "Energy changes in chemical reactions".
\label{m38709*uid8}\item \textsl{Reversibility}\newline
Chemical changes are far more difficult to reverse than physical changes. When hydrogen peroxide decomposes into water and oxygen, it is almost impossible to get back to hydrogen peroxide.
\item \textsl{Mass conservation}\newline
Mass is conserved during a chemical change, but the number of molecules may change. In the example of the decomposition of hydrogen peroxide, for every two molecules of hydrogen peroxide that decomposes, thre molecules are formed (two water and one oxygen).
\end{enumerate}
      \label{m38709*id62997}We will consider two types of chemical reactions: \textbf{decomposition reactions} and \textbf{synthesis reactions}.\par 
      \label{m38709*uid9}
            \subsubsection*{Decomposition reactions}
            \nopagebreak
            \label{m38709*id63019}A \textbf{decomposition reaction} occurs when a chemical compound is broken down into elements or smaller compounds. The generalised equation for a decomposition reaction is:\par 
        \label{m38709*id63029}$\mathsf{AB}\to \mathsf{A}+\mathsf{B}$\par 
        \label{m38709*id63040}One example of such a reaction is the decomposition of mercury (II) oxide to form mercury and oxygen according to the following equation:
\label{m38709*id734}\nopagebreak\noindent{}

    \begin{equation*}
    2\mathsf{HgO (s)}\to 2\mathsf{Hg (\ell)}+{\mathsf{O}}_{2}\mathsf{(g)}
      \end{equation*}
%     \setcounter{subfigure}{0}
% 	\begin{figure}[H] % horizontal\label{m38709*uid10}
%     \begin{center}
%     \label{m38709*uid10!!!underscore!!!media}\label{m38709*uid10!!!underscore!!!printimage}\includegraphics[width=300px]{col11305.imgs/m38709_CG10C4_003.png} % m38709;CG10C4\_003.png;;;6.0;8.5;
%       \vspace{2pt}
%     \vspace{\rubberspace}\par \begin{cnxcaption}
% 	  \small \textbf{Figure 12.3: }The decomposition of $\mathsf{HgO}$ to form $\mathsf{Hg}$ and ${\mathsf{O}}_{2}$
% 	\end{cnxcaption}
%     \vspace{.1in}
%     \end{center}
%  \end{figure}       \par 
\label{m38709*secfhsst!!!underscore!!!id163}
            \begin{g_experiment}{The decomposition of hydrogen peroxide}
            \nopagebreak
            \label{m38709*id63175}\noindent{}\textbf{Aim:}\newline
    To observe the decomposition of hydrogen peroxide when it is heated.\par 
        \label{m38709*id63194}\noindent{}\textbf{Apparatus:}\newline
    Dilute hydrogen peroxide (about 3\%); manganese dioxide; test tubes; a water bowl; stopper and delivery tube, bunsen burner\par 
        \label{m38709*eip-470}
\Warning{Hydrogen peroxide can cause chemical burns. Work carefully with it.}
	\par
      \label{m38709*id63199}
    \setcounter{subfigure}{0}
	\begin{figure}[H] % horizontal\label{m38709*id63200}
    \begin{center}
\scalebox{0.8}{
\begin{pspicture}(0,0)(5,5)
\psset{unit=0.5cm,glassType=erlen,recuperationGaz,niveauLiquide1=20}
\pstChauffageBallon[tubeRecourbe]
\end{pspicture}
}
    \end{center}
 \end{figure}       
        \par 
        \label{m38709*id63206}\noindent{}\textbf{Method:}\label{m38709*id63212}\begin{enumerate}[noitemsep, label=\textbf{\arabic*}. ] 
            \label{m38709*uid11}\item Put a small amount (about $5~\mathsf{ml}$) of hydrogen peroxide in a test tube.
\label{m38709*uid12}\item Set up the apparatus as shown above.
\label{m38709*uid13}\item Very carefully add a small amount (about $0,5~\mathsf{g}$) of manganese dioxide to the test tube containing hydrogen peroxide. 
\end{enumerate}
        \par 
        \label{m38709*id63254}\noindent{}\textbf{Results:}\newline
    You should observe a gas bubbling up into the second test tube. This reaction happens quite rapidly. \par 
        \label{m38709*id63302}\noindent{}\textbf{Conclusions:}\newline
    When hydrogen peroxide is added to manganese dioxide it decomposes to form oxygen and water. The chemical decomposition reaction that takes place can be written as follows:
        \label{m38709*id63313}\nopagebreak\noindent{}
    \begin{equation*}
    2{\mathsf{H}}_{2}{\mathsf{O}}_{2} \mathsf{(aq)} \to 2\mathsf{H}_{2}\mathsf{O}\mathsf{(\ell)}+{\mathsf{O}}_{2}\mathsf{(g)}
      \end{equation*}
Note that the manganese dioxide is a catalyst and is not shown in the reaction. (A catalyst helps speed up a chemical reaction.)    \par 
\end{g_experiment}
\label{m38709*eip-619}
\Note{This experiment used the downward displacement of water to collect a gas. This is a very common way to collect a gas in chemistry. The oxygen that is evolved in this reaction moves along the delivery tube and then collects in the top of the test tube. It does this because it is less dense than water and does not dissolve in water, so the water is displaced downwards. If you use a test tube with an outlet attached, you could collect the oxygen into jars and store it for use in other experiments.} 
	\par
      \label{m38709*eip-633}The above experiment can be very vigourous and produce a lot of oxygen very rapidly. For this reason you use dilute hydrogen peroxide and only a small amount of manganese dioxide.
\IFact{This reaction is often performed without collecting the oxygen gas and is commonly known as the elephant's toothpaste reaction.} \par
      \label{m38709*uid17}
            \subsubsection*{Synthesis reactions}
            \nopagebreak
            \label{m38709*id63365}During a \textbf{synthesis reaction}, a new product is formed from elements or smaller compounds. The generalised equation for a synthesis reaction is as follows:
        \label{m38709*id63374}\nopagebreak\noindent
    \begin{equation*}
    \mathsf{A}+\mathsf{B}\to \mathsf{AB}
      \end{equation*}
    \par 
        \label{m38709*id63386}One example of a synthesis reaction is the burning of magnesium in oxygen to form magnesium oxide(Figure~12.5). The equation for the reaction is:
        \label{m38709*id63390}\nopagebreak\noindent
    \begin{equation*}
    2\mathsf{Mg (s)}+{\mathsf{O}}_{2} \mathsf{(g)} \to 2\mathsf{MgO (s)}
      \end{equation*}
%     \setcounter{subfigure}{0}
% \begin{figure}[H]
% \begin{center}
% \begin{pspicture}(-6,-1)(6,1.5)
% %\psgrid[gridcolor=lightgray]
% \rput(-1,0){
% \psellipse(-5,0)(0.75,0.75)
% \rput(-5,0){Mg}
% \psellipse(-3.2,0)(0.75,0.75)
% \rput(-3.2,0){Mg}
% \rput(-2,0){\textbf{+}}
% \psellipse(-1,0)(0.5,0.5)
% \rput(-1,0){O}
% \psellipse(0,0)(0.5,0.5)
% \rput(0,0){O}
% \psline[arrows=->](1,0)(2.5,0)
% \psellipse(3.5,0)(0.75,0.75)
% \rput(3.5,0){Mg}
% \psellipse(4.75,0)(0.5,0.5)
% \rput(4.75,0){O}
% \rput(3,0){
% \psellipse(3.5,0)(0.75,0.75)
% \rput(3.5,0){Mg}
% \psellipse(4.75,0)(0.5,0.5)
% \rput(4.75,0){O}
% }
% }
% \end{pspicture}
% \end{center}
% \caption{The synthesis of magnesium oxide (MgO) from magnesium and oxygen}
% \label{fig:chemical change:synthesis}
% \end{figure}      \par 
\begin{figure}[H]
 \begin{center}
\includegraphics[width=.3\textwidth]{photos/Magnesium_ribbon_burning.jpg}
 \end{center}
\caption{Magnesium ribbon burning in oxygen, photo by Capt. John Yossarian on wikimedia}
\end{figure}

\label{m38709*secfhsst!!!underscore!!!id243}
            \begin{g_experiment}{Chemical reactions involving iron and sulphur }
            \nopagebreak
            \label{m38709*id63437}\noindent{}\textbf{Aim:}
          \newline
     To demonstrate the synthesis of iron sulphide from iron and sulphur.\par 
        \label{m38709*id63447}\noindent{}\textbf{Apparatus:}
          \newline
$5,6~\mathsf{g}$ iron filings and $3,2~\mathsf{g}$ powdered sulphur; porcelain dish; test tube; Bunsen burner\par 
        \label{m38709*id63457}
    \setcounter{subfigure}{0}
	\begin{figure}[H] % horizontal\label{m38709*id63460}
    \begin{center}
\begin{pspicture}(0,0)(5,5)
\psset{unit=0.5cm,tubeSeul=true,pince=true}
\newpsstyle{gray} {linestyle=none,fillstyle=solid,fillcolor=darkgray}
\newpsstyle{orange} {linestyle=none,fillstyle=solid,fillcolor=orange}
%playing with colours, try tweak this to look like the real thing, add photo of the real thing?
\pstChauffageTube[aspectLiquide1=gray,aspectLiquide2=orange,niveauLiquide1=40,niveauLiquide2=20]
\end{pspicture}
    \end{center}
 \end{figure}       
        \par 
        \label{m38709*id63467}\noindent{}\textbf{Method:}
          \newline
        \label{m38709*id63473}\begin{enumerate}[noitemsep, label=\textbf{\arabic*}. ] 
            \label{m38709*uid20}\item Measure the quantity of iron and sulphur that you need and mix them in a porcelain dish.
\label{m38709*uid21}\item Take some of this mixture and place it in the test tube. The test tube should be about one third full.
\label{m38709*uid22}\item This reaction should take place in a fume cupboard or a well ventialted room. Heat the test tube containing the mixture over the Bunsen burner. Increase the heat if no reaction takes place. Once the reaction begins, you will need to remove the test tube from the flame. Record your observations.
\label{m38709*uid23}\item Wait for the product to cool before breaking the test tube with a hammer. Make sure that the test tube is rolled in paper before you do this, otherwise the glass will shatter everywhere and you may be hurt.
\label{m38709*uid24}\item What does the product look like? Does it look anything like the original reactants? Does it have any of the properties of the reactants (e.g. the magnetism of iron)?
\end{enumerate}
        \par 
        \label{m38709*eip-963}
      \Warning{When working with a bunsen burner work in a well ventilated space and ensure that there are no flammable substances close by. Always tuck loose clothing in and ensure that long hair is tied back.}
	\par
      \label{m38709*id63554}\noindent{}\textbf{Results:}
          \newline
        \label{m38709*id63560}\begin{enumerate}[noitemsep, label=\textbf{\arabic*}. ] 
            \label{m38709*uid25}\item After you removed the test tube from the flame, the mixture glowed a bright red colour. The reaction is exothermic and \textsl{produces heats}.
\label{m38709*uid26}\item The product, iron sulphide, is a dark colour and does not share any of the properties of the original reactants. It is an entirely new product.
\end{enumerate}
        \par 
        \label{m38709*id63594}\noindent{}\textbf{Conclusions:}
          \newline
A synthesis reaction has taken place. The equation for the reaction is:
        \label{m38709*id63604}\nopagebreak\noindent{}
    \begin{equation*}
    \mathsf{Fe (s)}+\mathsf{S (s)}\to \mathsf{FeS (s)}
      \end{equation*}
    \par 
\end{g_experiment}
    \label{m38711*cid4}
            \subsection*{Energy changes in chemical reactions}
            \nopagebreak
            \label{m38711} $ \hspace{-5pt}\begin{array}{cccccccccccc}   \includegraphics[width=0.75cm]{col11305.imgs/summary_video.png} &   \end{array} $ \hspace{2 pt}\raisebox{-5 pt}{} {(section shortcode: P10057 )} \par 
      \label{m38711*id64254}All reactions involve some change in energy. During a \textsl{physical} change in matter, such as the evaporation of liquid water to water vapour, the energy of the water molecules increases. However, the change in energy is much smaller than in chemical reactions.\par 
      \label{m38711*id64265}When a chemical reaction occurs, some bonds will \textsl{break}, while new bonds may \textsl{form}. Energy changes in chemical reactions result from the breaking and forming of bonds. Energy is needed to \textsl{break} bonds. When new bonds \textsl{form}, energy will be \textsl{released} because the new product has a lower energy. \par 
      \label{m38711*id64305}In some reactions, the energy that must be \textsl{absorbed} to break the bonds in the reactants is less than
the total energy that is \textsl{released} when new bonds are formed. This means that in the overall reaction, energy is \textsl{released}. This type of reaction is known as an \textbf{exothermic} reaction. In other reactions, the energy that must be \textsl{absorbed} to break the bonds in the reactants is more than the total energy that is \textsl{released} when new bonds are formed. This means that in the overall reaction, energy must be \textsl{absorbed} from the surroundings. This type of reaction is known as an \textbf{endothermic} reaction. Most decomposition reactions are endothermic and heating is needed for the reaction to occur. Most synthesis reactions are exothermic, meaning that energy is given off in the form of heat or light.\par 
      \label{m38711*id64360}More simply, we can describe the energy changes that take place during a chemical reaction as:\par 
      \label{m38711*id64364}
        \textsl{Total energy absorbed to break bonds - Total energy released when new bonds form}
      \par 
      \label{m38711*id64371}So, for example, in the reaction...\par 
      \label{m38711*id64375}$2\mathsf{Mg}+{\mathsf{O}}_{2}\to  2\mathsf{MgO}$
      \par 
      \label{m38711*id64411}Energy is needed for the magnesium atoms to break away from the metal lattice. Energy is also needed to break the $\mathsf{O}-\mathsf{O}$ bonds in the oxygen molecule so that new $\mathsf{Mg}-\mathsf{O}$ bonds can be formed, and energy is released when the product ($\mathsf{MgO}$) forms.\par 
      \label{m38711*id64416}Despite all the energy changes that seem to take place during reactions, it is important to remember that energy cannot be created or destroyed. Energy that enters a system will have come from the surrounding environment and energy that leaves a system will again become part of that environment. This is known as the \textbf{conservation of energy} principle.\par 
\label{m38711*fhsst!!!underscore!!!id409}
\Definition{   \label{id2460317}\textbf{ Conservation of energy principle }} { \label{m38711*meaningfhsst!!!underscore!!!id409}
      Energy cannot be created or destroyed. It can only be changed from one form to another. 
       } 
\begin{minipage}{.5\textwidth}
      \label{m38711*id64445}Chemical reactions may produce some very visible and often violent changes. An explosion, for example, is a sudden increase in volume and release of energy when high temperatures are generated and gases are released. For example, $\mathsf{NH}_{4}{\mathsf{NO}}_{3}$ can be heated to generate nitrous oxide. Under these conditions, it is highly sensitive and can detonate easily in an explosive exothermic reaction.\par
\end{minipage}
\begin{minipage}{.5\textwidth}
\begin{center}
 \includegraphics[width=.3\textwidth]{photos/explosionby_CTBTOphotostream_flickr.jpg} \\
\textsl{Photo by CTBTOphotostream on flickr}
\end{center}
\end{minipage}
 
    \label{m38711*cid5}
            \section{Conservation of atoms and mass in reactions}
            \nopagebreak
      \label{m38711*id64489}The total mass of all the substances taking part in a chemical reaction is conserved during a chemical reaction. This is known as the \textbf{law of conservation of mass}. The total number of \textbf{atoms} of each element also remains the same during a reaction, although these may be arranged differently in the products.\par 
      \label{m38711*id64505}We will use two of our earlier examples of chemical reactions to demonstrate this:\par 
      \label{m38711*id64509}1. The decomposition of hydrogen peroxide into water and oxygen\par 
      \label{m38711*id64513}$2{\mathsf{H}}_{2}{\mathsf{O}}_{2}\to 2\mathsf{H}{}_{2}\mathsf{O}+{\mathsf{O}}_{2}$
      \par 
      \label{m38711*id64563}
    \setcounter{subfigure}{0}
\begin{figure}[h]
\begin{center}
\begin{pspicture}(-6,-1.5)(12,1.5)
%\psgrid[gridcolor=lightgray]
%reactants
\rput(-2,0){
\psellipse(-3,0)(0.5,0.5)
\rput(-3,0){$\mathsf{O}$}
\psellipse(-2,0)(0.5,0.5)
\rput(-2,0){$\mathsf{O}$}
\psellipse(-3.3,-0.75)(0.3,0.3)
\rput(-3.3,-0.75){$\mathsf{H}$}
\psellipse(-1.7,0.75)(0.3,0.3)
\rput(-1.7,0.75){$\mathsf{H}$}
\rput(3,0){
\psellipse(-3,0)(0.5,0.5)
\rput(-3,0){$\mathsf{O}$}
\psellipse(-2,0)(0.5,0.5)
\rput(-2,0){$\mathsf{O}$}
\psellipse(-3.3,-0.75)(0.3,0.3)
\rput(-3.3,-0.75){$\mathsf{H}$}
\psellipse(-1.7,0.75)(0.3,0.3)
\rput(-1.7,0.75){$\mathsf{H}$}
}
%products
\psline[arrows=->](2,0)(4,0)
\psellipse(5,0)(0.5,0.5)
\rput(5,0){$\mathsf{O}$}
\psellipse(4.5,0.65)(0.3,0.3)
\rput(4.5,0.65){$\mathsf{H}$}
\psellipse(5.5,0.65)(0.3,0.3)
\rput(5.5,0.65){$\mathsf{H}$}
\rput(2,-0.5){
\psellipse(5,0)(0.5,0.5)
\rput(5,0){$\mathsf{O}$}
\psellipse(4.5,0.65)(0.3,0.3)
\rput(4.5,0.65){$\mathsf{H}$}
\psellipse(5.5,0.65)(0.3,0.3)
\rput(5.5,0.65){$\mathsf{H}$}
}
\rput(8.5,0){\textbf{+}}
\psellipse(9.5,0)(0.5,0.5)
\rput(9.5,0){$\mathsf{O}$}
\psellipse(10.5,0)(0.5,0.5)
\rput(10.5,0){$\mathsf{O}$}
}
\end{pspicture}
\end{center}
\end{figure}       
      \par 
      \label{m38711*id64573}
        \textsl{Left hand side of the equation}
      \par 
      \label{m38711*id64579}$\mathsf{Total\; atomic\; mass}=\left(4\ensuremath{\times}1\right)+\left(4\ensuremath{\times}16\right)=68\phantom{\rule{2pt}{0ex}}\mathsf{u}$\par 
      \label{m38711*id64601}$\mathsf{Number\; of\; atoms\; of\; each\; element}=\left(4\ensuremath{\times}\mathsf{H}\right)+\left(4\ensuremath{\times}\mathsf{O}\right)$\par 
      \label{m38711*id64623}
        \textsl{Right hand side of the equation}
      \par 
      \label{m38711*id64630}$\mathsf{Total\; atomic\; mass}=\left(4\ensuremath{\times}1\right)+\left(4\ensuremath{\times}16\right)=68\phantom{\rule{2pt}{0ex}}\mathsf{u}$\par 
      \label{m38711*id64660}$\mathsf{Number\; of\; atoms\; of\; each\; element}=\left(4\ensuremath{\times}\mathsf{H}\right)+\left(4\ensuremath{\times}\mathsf{O}\right)$\par 
      \label{m38711*id64682}Both the atomic mass and the number of atoms of each element are conserved in the reaction.\par 
      \label{m38711*id64686}2. The synthesis of magnesium and oxygen to form magnesium oxide\par 
      \label{m38711*eip-233}\nopagebreak\noindent{}
    \begin{equation*}
    2\mathsf{Mg}+{\mathsf{O}}_{2}\to 
            2\mathsf{MgO}
      \end{equation*}
    \label{m38711*id64723}
    \setcounter{subfigure}{0}
\begin{figure}[h]
\begin{center}
\begin{pspicture}(-6,-1)(6,1.5)
%\psgrid[gridcolor=lightgray]
\rput(-1,0){
\psellipse(-5,0)(0.75,0.75)
\rput(-5,0){Mg}
\psellipse(-3.2,0)(0.75,0.75)
\rput(-3.2,0){Mg}
\rput(-2,0){\textbf{+}}
\psellipse(-1,0)(0.5,0.5)
\rput(-1,0){O}
\psellipse(0,0)(0.5,0.5)
\rput(0,0){O}
\psline[arrows=->](1,0)(2.5,0)
\psellipse(3.5,0)(0.75,0.75)
\rput(3.5,0){Mg}
\psellipse(4.75,0)(0.5,0.5)
\rput(4.75,0){O}
\rput(3,0){
\psellipse(3.5,0)(0.75,0.75)
\rput(3.5,0){Mg}
\psellipse(4.75,0)(0.5,0.5)
\rput(4.75,0){O}
}
}
\end{pspicture}
\end{center}
\end{figure}     
      \par 
      \label{m38711*id64732}
        \textsl{Left hand side of the equation}
      \par 
      \label{m38711*id64739}$\mathsf{Total\; atomic\; mass}=\left(2\ensuremath{\times}24,3\right)+\left(2\ensuremath{\times}16\right)=80,6\phantom{\rule{2pt}{0ex}}\mathsf{u}$\par 
      \label{m38711*id64761}$\mathsf{Number\; of\; atoms\; of\; each\; element}=\left(2\ensuremath{\times}\mathsf{Mg}\right)+\left(2\ensuremath{\times}\mathsf{O}\right)$\par 
      \label{m38711*id64783}
        \textsl{Right hand side of the equation}
      \par 
      \label{m38711*id64790}$\mathsf{Total\; atomic\; mass}=\left(2\ensuremath{\times}24,3\right)+\left(2\ensuremath{\times}16\right)=80,6\phantom{\rule{2pt}{0ex}}\mathsf{u}$\par 
      \label{m38711*id64811}$\mathsf{Number\; of\; atoms\; of\; each\; element}=\left(2\ensuremath{\times}\mathsf{Mg}\right)+\left(2\ensuremath{\times}\mathsf{O}\right)$\par 
      \label{m38711*id64833}Both the atomic mass and the number of atoms of each element are conserved in the reaction.\par 
\label{m38711*secfhsst!!!underscore!!!id486}
            \begin{activity}{The conservation of atoms in chemical reactions }
            \nopagebreak
            \label{m38711*id64844}\noindent
\textbf{Materials:} \\ Coloured modelling clay rolled into balls or marbles and prestick to represent atoms. Each colour will represent a different element.
        \par 
      \label{m38711*id64882}\noindent
\textbf{Method:}\\
\begin{minipage}{.6\textwidth}
      \label{m38711*id64889}\begin{enumerate}[noitemsep, label=\textbf{\arabic*}. ] 
\label{m38711*uid36}\item Build your reactants. Use marbles and prestick or modelling clay to represent the reactants and put these on one side of your table. Make at least five (${\mathsf{CaCO}}_{3}$) units.
\label{m38711*uid37}\item Place the (${\mathsf{CaCO}}_{3}$) units on a table. The table represents the 'test tube' where the reaction is going to take place. 
\label{m38711*uid38}\item Now count the number of atoms ($\mathsf{Ca}, \mathsf{O}, \mathsf{C}$) you have in your 'test tube'. Fill in the reactants column in the table below. 
\label{m38711*uid39}\item Let the reaction take place. Each person can now take a ${\mathsf{CaCO}}_{3}$ unit and 'decompose' it. Break it apart and build ${\mathsf{CO}}_{2}$ and ${\mathsf{CaO}}$ units with it. These are the products. Place the products on the table.
\item When the 'reaction' has finished (i.e. when all the ${\mathsf{CaCO}}_{3}$ units have decomposed) count the number of atoms ($\mathsf{Ca}, \mathsf{O}, \mathsf{C}$) and complete the table.
\item What do you notice about the number of atoms for the reactants, compared to the products?
\item Write a balanced equation for this reaction and use your models to build this equation.
\end{enumerate}
\end{minipage}
\begin{minipage}{.4\textwidth}
 \begin{center}
 \includegraphics[width=.6\textwidth]{photos/balancing3.png}\par
\end{center}
\end{minipage}
        \par 
\begin{table}[H]
 \begin{center}
  \begin{tabular}{|l|l|l||l|l|l|l|l|} \hline
   \multicolumn{3}{|c||}{\textbf{Reactants}} & \multicolumn{4}{|c|}{\textbf{Products}} \\ \hline
$\mathsf{Ca}$ & $\mathsf{O}$ & $\mathsf{C}$ & $\mathsf{C}$ & $\mathsf{O}$ & $\mathsf{Ca}$ & $\mathsf{O}$ \\ \hline
&&&&&& \\ \hline
&&&&&& \\ \hline
&&&&&& \\ \hline
&&&&&& \\ \hline
&&&&&& \\ \hline
  \end{tabular}

 \end{center}

\end{table}

      \label{m38711*id65031}\noindent{}\textbf{Discussion}
     You should have noticed that the number of atoms in the reactants is the same as the number of atoms in the product. The number of atoms is conserved during the reaction. However, you will also see that the molecules in the reactants and products is not the same. The number of molecules is not conserved during the reaction.
 \par 
\end{activity}
\label{m38711*eip-14}
            \begin{i_experiment}{Conservation of matter}
            \nopagebreak
            \label{m38711*eip-453}\noindent{}\textbf{Aim:}
To prove the law of conservation of matter experimentally.
\par 
\label{m38711*eip792}\noindent{}\textbf{Materials:}
\textsl{Reaction 1:} \\
3 beakers; lead (II) nitrate; sodium iodide; mass meter \\
\textsl{Reaction 2:} \\
 hydrochloric acid; bromothymol blue; sodium hydroxide solution; mass meter \\
 \textsl{Reaction 3:} \\
any effervescant tablet (e.g. Cal-C-Vita tablet), ballon; rubber band; mass meter; test tube; beaker
\par 
\label{m38711*eip-153}
\Warning{Always be careful when handling chemicals (particularly strong acids like hydrochloric acid) as you can burn yourself badly. }
	\par
      \label{m38711*id72432}\noindent
\textbf{Method:} \\
\textsl{Reaction 1} \\
\label{m38711*id6342}\begin{enumerate}[noitemsep, label=\textbf{\arabic*}. ] 
            \item Solution 1: Dissolve $5~\mathsf{g}$ of silver nitrate in $100~\mathsf{ml}$ of water.
\item Solution 2: Dissolve $4.5~\mathsf{g}$ of sodium iodide in $100~\mathsf{ml}$ of water.
\item Determine the mass of the reactants.
\item Add solution 1 to solution 2. What do you observe? Has a chemical reaction taken place? 
\item Determine the mass of the products. 
\item What do you notice about the masses?
\item Write a balanced equation for this reaction.
\end{enumerate}
\textsl{Reaction 2:}\\
\label{m38711*id63452}\begin{enumerate}[noitemsep, label=\textbf{\arabic*}. ] 
\item Solution 1: Dissolve $0.4~\mathsf{g}$ of sodium hydroxide in $100~\mathsf{ml}$ of water. Add a few drops of bromothymol blue indicator to the solution. 
\item Solution 2: Measure $100~\mathsf{ml}$ of $0,1~\mathsf{M}$ hydrochloric acid solution into a beaker.
\item Determine the mass of the reactants.
\item Add small quantities of solution 2 to solution 1 (you can use a plastic pipette for this) until a colour change has taken place. Has a chemical reaction taken place?  
\item Determine the mass of hydrochloric acid added. (You do this by weighing the remaining solution, and subtracting this from the starting mass)
\item Compare the mass before the reaction to the total mass after the reaction. What do you notice?
\item Write a balanced equation for this reaction.
\end{enumerate}
\textbf{Reaction 3}
\label{m38711*id634223}\begin{enumerate}[noitemsep, label=\textbf{\arabic*}. ] 
\item Half fill a large test tube with water.
\item Determine the mass of the test tube and water.
\item Break an effervescant tablet in two or three pieces and place them in a ballon.
\item Determine the mass of the ballon and tablet.
\item Fit the ballon tightly to the test tube, being careful to not drop the contents into the water. You can stand the test tube in a beaker to help you do this.
\item Determine the total mass of the test tube and ballon.
\item Lift the ballon so that the table goes into the water. What do you observe? Has a chemical reaction taken place?
\item Determine the mass of the test tube balloon combination.
\item What do you observe about the masses before and after the reaction?
\end{enumerate}
        \par \label{m38711*eip-768}\noindent{}\textbf{Results:}Fill in the following table for the total mass of reactants (starting materials) and products (ending materials).  \par 
    % \textbf{m38711*eip-581}\par
          \begin{table}[H]
    % \begin{table}[H]
    % \\ '' '0'
        \begin{center}
      \label{m38711*eip-581}
      \begin{tabular}{|l|l|l|l|}\hline
         &
        Reaction 1 &
        Reaction 2 &
        Reaction 3 \\ \hline
        Reactants &
         &
         &
        \\ \hline
        Products &
         &
         &
        \\ \hline
    \end{tabular}
      \end{center}
\end{table}
    \par
  \label{m38711*eip-634}Add the masses for the reactants for each reaction. Do the same for the products. For each reaction compare the mass of the reactants to the mass of the products. What do you notice? Is the mass conserved?\par \label{m38711*eip-65}In the experiment above you should have found that the total mass at the start of the reaction is the same as the mass at the end of the reaction. Mass does not appear or disappear in chemical reactions. Mass is conserved, in other words, the total mass you start with is the total mass you will end with.  \par
\end{i_experiment} 
    \label{m38711*cid6}
            \section{Law of constant composition}
            \nopagebreak
      \label{m38711*id65065}In any given chemical compound, the elements always combine in the same proportion with each other. This is the \textbf{law of constant proportion}.\par 
      \label{m38711*id65075}The \textbf{law of constant composition} says that, in any particular chemical compound, all samples of that compound will be made up of the same elements in the same proportion or ratio. For example, any water molecule is always made up of two hydrogen atoms and one oxygen atom in a 2:1 ratio. If we look at the relative masses of oxygen and hydrogen in a water molecule, we see that 94\% of the mass of a water molecule is accounted for by oxygen and the remaining 6\% is the mass of hydrogen. This mass proportion will be the same for any water molecule.\par 
      \label{m38711*id65089}This does not mean that hydrogen and oxygen always combine in a 2:1 ratio to form $\mathsf{H}{}_{2}\mathsf{O}$. Multiple proportions are possible. For example, hydrogen and oxygen may combine in different proportions to form $\mathsf{H}{}_{2}\mathsf{O}{}_{2}$ rather than $\mathsf{H}{}_{2}\mathsf{O}$. In $\mathsf{H}{}_{2}\mathsf{O}{}_{2}$, the H:O ratio is 1:1 and the mass ratio of hydrogen to oxygen is 1:16. This will be the same for any molecule of hydrogen peroxide.\par 
    \label{m38711*cid7}
            \subsection*{Volume relationships in gases}
            \nopagebreak
      \label{m38711*id65179}In a chemical reaction between gases, the relative volumes of the gases in the reaction are present in a ratio of small whole numbers if all the gases are at the same temperature and pressure. This relationship is also known as \textbf{Gay-Lussac's Law}.\par 
      \label{m38711*id65189}For example, in the reaction between hydrogen and oxygen to produce water, two volumes of $\mathsf{H}{}_{2}$ react with 1 volume of $\mathsf{O}{}_{2}$ to produce 2 volumes of $\mathsf{H}{}_{2}\mathsf{O}$.\par 
      \label{m38711*id65237}$2\mathsf{H}_{2}\mathsf{g} +\mathsf{O}_{2} \mathsf{(g)} \to 2\mathsf{H}_{2}\mathsf{O} \mathsf{\ell}$\par 
      \label{m38711*id65282}In the reaction to produce ammonia, one volume of nitrogen gas reacts with three volumes of hydrogen gas to produce two volumes of ammonia gas.\par 
      \label{m38711*id65286}$\mathsf{N}_{2} \mathsf{(g)}+3\mathsf{H}_{2} \mathsf{(g)} \to 2\mathsf{NH}_{3} \mathsf{(g)}$
      \par  
    \label{m38711*cid8}
            \section{Summary}
            \nopagebreak
\label{m38711*id972312}
The following video provides a summary of the concepts covered in this chapter.
    \setcounter{subfigure}{0}
	\begin{figure}[H] % horizontal\label{m38711*summary}
    \textnormal{Physical and chemical change}\vspace{.1in} \nopagebreak
  \label{m38711*yt-media1}\label{m38711*yt-video1}
            \raisebox{-5 pt}{ \includegraphics[width=0.5cm]{col11305.imgs/summary_www.png}} { (Video:  P10058 )}
      \vspace{2pt}
    \vspace{.1in}
 \end{figure}       
\par 
      \label{m38711*id65342}\begin{enumerate}[noitemsep, label=\textbf{\arabic*}. ] 
            \label{m38711*uid40}\item Matter does not stay the same. It may undergo physical or chemical changes.
\label{m38711*uid41}\item A \textbf{physical change} means that the form of matter may change, but not its identity. For example, when water evaporates, the energy and the arrangement of water molecules will change, but not the structure of the water molecules themselves.
\label{m38711*uid42}\item During a physical change, the \textbf{arrangement of particles} may change but the mass, number of atoms and number of molecules will stay the same.
\label{m38711*uid43}\item Physical changes involve small changes in \textbf{energy} and are easily reversible.
\label{m38711*uid44}\item A chemical change occurs when one or more substances change into other materials. A chemical reaction involves the formation of new substances with \textbf{different properties}. For example, magnesium and oxygen react to form magnesium oxide ($\mathsf{MgO}$) \label{m38711*uid45}\item A chemical change may involve a \textbf{decomposition} or \textbf{synthesis} reaction. During chemical change, the mass and number of atoms is conserved, but the number of molecules is not always the same.
\label{m38711*uid46}\item Chemical reactions involve larger changes in energy. During a reaction, energy is needed to break bonds in the reactants and energy is released when new products form. Chemical reactions are not easily reversible.
\label{m38711*uid48}\item The \textbf{law of conservation of mass} states that the total mass of all the substances taking part in a chemical reaction is conserved and the number of atoms of each element in the reaction does not change when a new product is formed.
\label{m38711*uid49}\item The \textbf{conservation of energy principle} states that energy cannot be created or destroyed, it can only change from one form to another.
\label{m38711*uid50}\item The \textbf{law of constant composition} states that in any particular compound, all samples of that compound will be made up of the same elements in the same proportion or ratio.
\label{m38711*uid51}\item \textbf{Gay-Lussac's Law} states that in a chemical reaction between gases, the relative volumes of the gases in the reaction are present in a ratio of small whole numbers if all the gases are at the same temperature and pressure.
\end{enumerate}
\label{m38711*secfhsst!!!underscore!!!id584}
            \begin{eocexercises}{Physical and chemical change}
            \nopagebreak
      \label{m38711*id65631}\begin{enumerate}[noitemsep, label=\textbf{\arabic*}. ] 
            \label{m38711*uid6234}\item For each of the following definitions give one word or term:
\label{m38711*id632243}\begin{enumerate}[noitemsep, label=\textbf{\alph*}. ] 
            \item A change that can be seen or felt, where the particles involved are not broken up in any way\item The formation of new substances in a chemical reaction\item A reaction where a new product is formed from elements or smaller compounds\end{enumerate}
\label{m38711*id63272}\item State the conservation of energy principle.\newline
\label{m38711*id6244}\item Explain how a chemical change differs from a physical change.\newline
\label{m38711*uid52}\item Complete the following table by saying whether each of the descriptions is an example of a physical or chemical change:
    % \textbf{m38711*id65648}\par
          \begin{table}[H]
    % \begin{table}[H]
    % \\ 'id2931526' '1'
        \begin{center}
      \label{m38711*id65648}
    \noindent
      \begin{tabular}{|l|l|}\hline
        \textbf{Description} &
        \textbf{Physical or chemical} \\ \hline
        hot and cold water mix together &
     \\ \hline
        milk turns sour &
     \\ \hline
        a car starts to rust &
       \\ \hline
        food digests in the stomach &
       \\ \hlien
        alcohol disappears when it is placed on your skin &
       \\ \hline
        warming food in a microwave &
     \\ \hline
        separating sand and gravel &
       \\ \hline
        fireworks exploding &
       \\ \hline
    \end{tabular}
      \end{center}
\end{table}
    \par
          \label{m38711*uid53}\item For each of the following reactions, say whether it is an example of a synthesis or decomposition reaction:
\label{m38711*id65862}\begin{enumerate}[noitemsep, label=\textbf{\alph*}. ] 
            \label{m38711*uid54}\item 
$\left({\mathsf{NH}}_{4}{\right)}_{2}{\mathsf{CO}}_{3}\to {\mathsf{NH}}_{3}+{\mathsf{CO}}_{2}+\mathsf{H}{}_{2}\mathsf{O}$
\label{m38711*uid56}\item ${\mathsf{N}}_{2}\left(\mathsf{g}\right)+3{\mathsf{H}}_{2}\left(\mathsf{g}\right)\to 2{\mathsf{NH}}_{3}$\label{m38711*uid57}\item 
${\mathsf{CaCO}}_{3}\left(\mathsf{s}\right)\to \mathsf{CaO}+{\mathsf{CO}}_{2}$\end{enumerate}
                \label{m38711*uid58}\item For the following equation:
${\mathsf{CaCO}}_{3}\left(\mathsf{s}\right)\to \mathsf{CaO}+{\mathsf{CO}}_{2}$
show that the 'law of conservation of mass' applies. Draw sub-microscopic diagrams to represent this reaction.\newline
        \end{enumerate}
  \label{m38711**end}
  \label{324e353f2415b0f24a8077f8f18039bb**end}
\par \raisebox{-5 pt}{\includegraphics[width=0.5cm]{col11305.imgs/summary_www.png}} Find the answers with the shortcodes:
 \par \begin{tabular}[h]{cccccc}
 (1.) l2z  &  (2.) l2u  &  (3.) l2J  &  (4.) l3q  &  (5.) l3l  &  (6.) l3i  & \end{tabular}
\end{eocexercises}
