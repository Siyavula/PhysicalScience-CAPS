\chapter{Exercise Solutions}

\section{1. Science skills}

\subsection{Exercise 1-1} 
\begin{multicols}{4}
\begin{enumerate}[noitemsep, label=\textbf{\arabic*}. ] 
\item %Carry out the following calculations
  \begin{enumerate}[itemsep=5pt, label=\textbf{(\alph*)} ] 
    \item $3,63 \times 10^6$
    \item $37,83$
    \item $6,3 \times 10^{−4}$
    \end{enumerate}
\item %Write the following in scientific notation using Table 1.3 as a reference.
    \begin{enumerate}[itemsep=5pt, label=\textbf{(\alph*)} ] 
    \item $5,11 \times 10^{5} \text{ V}$
    \item $1,0 \times 10^{-1} \ell$
    \item $5 \times 10^{-7} \text{ m}$
    \item $2,50 \times 10^{-7} \text{ m}$
    \item $3,5 \times 10^{-2} \text{ g}$
    \end{enumerate}
 \item %Write the following using the prefixes in Table 1.3.
    \begin{enumerate}[itemsep=5pt, label=\textbf{(\alph*)} ] 
    \item $0,1602 \text{ aC}$
    \item $1,992 \text{ MJ}$
    \item $59,8 \text{ kN}$
    \item $2,5 \text{ mA}$
    \item $7,5 \text{ km}$
    \end{enumerate}
\end{enumerate}
\end{multicols}
\subsection{Exercise 1-2} %p. 9-12
% \begin{exercises}{}
% {
% Use your calculator to write the following in decimal form, round off to $3$ decimal places:
\begin{enumerate}[itemsep=5pt, label=\textbf{\arabic*}. ]
 \item %
\item %
\item %
\item
\item %
\item %
\end{enumerate}
}
\end{exercises}

\section {2. Classification of matter}
\subsection{Exercise 2-1} %p. 50-51
%\begin{exercises}{}
% {
% Solve the following equations (assume all denominators are non-zero): \\

\begin{multicols}{2}
\begin{enumerate}[itemsep=6pt, label=\textbf{\arabic*}. ] 
\item %  $2y-3=7$
\item %  $-3y=0$        
\item %  $16y+4=-10$        
\item %  $12y+0=144$
\item %  $7+5y=62$       
\item % $55=5x+\dfrac{3}{4}$ 
\item %  $5x=2x+45$        
\item % $23x-12=6+3x$
\item %  $12-6x+34x=2x-24-64$
\item %  $6x+3x=4-5(2x-3)$
\item %  $18-2p=p+9$   
\item %  $\dfrac{4}{p}=\dfrac{16}{24}$
\item %  $-(-16-p)=13p-1$
\item %  $3f-10=10$
\item %  $3f+16=4f-10$
\item %  $10f+5=-2f-3f+80$
\item %  $8(f-4)=5(f-4)$
\item % $6=6(f+7)+5f$      
\item %$(a-1)^{2} - 2a = (a+3)(a-2) - 3$
\item %$-7x = x+8(1-x)$ 
\item %$5-\dfrac{7}{b} = \dfrac{2(b+4)}{b}$
\item %$\dfrac{x+2}{4} - \dfrac{x-6}{3} = \dfrac{1}{2}$
\item %$ 3 - \dfrac{y-2}{4} = 4$
\item %$ \dfrac{a+1}{a+2} = \dfrac{a-3}{a+1}$
  
\end{enumerate}
\end{multicols}

% }
% \end{exercises}

\subsection{Exercise 2-2} %p. 54
% \begin{exercises}{ }
% {
% Solve the following equations:
\begin{multicols}{2}
\begin{enumerate}[itemsep=5pt, label=\textbf{\arabic*}. ] 
\item % $(3x+2)(3x-4)=0$
\item % $(5x-9)(x+6)=0$
\item % $(2y+3)(2y-3)=0$ 
\item % $(2x+1)(2x-9)=0$    
\item % $(4x)(x-3)=-9$       
\item % $20m+25{m}^{2}=0$
\item % $2{x}^{2}-5x-12=0$  
\item % $-75{x}^{2}+290x=240$
\item % $2x=\frac{1}{3}{x}^{2}-3x+14\frac{2}{3}$
\item % ${x}^{2}-4x=-4$      
\item % $-{x}^{2}+4x-6=4{x}^{2}-5x+3$       
\item % ${t}^{2}=3t$  
\item % ${x}^{2}-10x=-25$      
\item % ${x}^{2}=18$
\item % ${p}^{2}-6p=7$
\item % $4{x}^{2}-17x-77=0$
\item % $14{x}^{2}+5x=6$
\item % $2{x}^{2}-2x=12$              
\end{enumerate}
\end{multicols}

% }
% \end{exercises}

\subsection{Exercise 2-3} %p. 63
% \begin{exercises}{}
% {
\begin{enumerate}[noitemsep, label=\textbf{\arabic*}. ] 

\item %Solve algebraically: 
\begin{enumerate}[noitemsep, label=\textbf{(\alph*)} ] 
\item %$3x-14y=0$, $x-4y+1=0$
\item %$x+y=8$, $3x + 2y = 21$
\item %$y=2x+1$, $x + 2y + 3 = 0$
\end{enumerate}

\item %Solve graphically and check your answer algebraically:

\begin{enumerate}[noitemsep, label=\textbf{(\alph*)} ] 

\item % $x+2y=1$, $\frac{x}{3} + \frac{y}{2} = 1$ NEEDS GRAPHIC
\begin{pspicture}(-3,-2)(4,2)
% \psgrid[gridcolor=lightgray,gridlabels=0,gridwidth=0.5pt]
\psaxes[dx=0.5,Dx=1,arrows=<->](0,0)(-3,-5)(6,3)
\psplot[xunit=0.5,plotstyle=curve,arrows=<->]{-2}{10.5}{x 0.5 neg mul 0.5 add}
\psplot[xunit=0.5,plotstyle=curve,arrows=<->]{-1}{10.5}{x 0.67 neg mul 2 add}
\uput[l](2.75,1.5){$y=-\frac{2}{3}x+2$}
\uput[l](3,-2.5){$y=-\frac{1}{2}x+\frac{1}{2}$}
\uput[l](0.1,-0.2){$0$}
\uput[r](6,0.1){$x$}
\uput[r](0,3){$y$}
\psdot(4.5,-4)
\psline[linestyle=dotted](4.5,-4)(4.5,0)
\psline[linestyle=dotted](0,-4)(4.5,-4)
% \psline[arrows=<->](-4,1)(4,1)
\end{pspicture}
\item %$5= x+y$, $-x = y-2$
\begin{pspicture}(-3,-2)(4,2)
% \psgrid[gridcolor=lightgray,gridlabels=0,gridwidth=0.5pt]
\psaxes[dx=1,Dx=1,dy=1, Dy=1,arrows=<->](0,0)(-3,-5)(8,6)
\psplot[xunit=1,plotstyle=curve,arrows=<->]{-1}{6}{x 1 neg mul 5 add}
\psplot[xunit=1,plotstyle=curve,arrows=<->]{-1}{6}{x 2 sub}
\uput[l](2.7,4.4){$y=-x+5$}
\uput[l](2.7, -1){$y=x-2$}
\uput[l](0.1,-0.2){$0$}
\uput[r](8,0.1){$x$}
\uput[r](0,6){$y$}
\psdot(3.5,1.5)
\psline[linestyle=dotted](3.5,1.5)(3.5,0)
\psline[linestyle=dotted](0,1.5)(3.5,1.5)
% \psline[arrows=<->](-4,1)(4,1)
\end{pspicture}
\item %$3x - 2y = 0$, $x - 4y + 1 = 0$

\end{enumerate}
\end{enumerate}

% }
% \end{exercises}

\subsection{End of chapter exercises} %p.78-79
% \begin{eocexercises}{}

 
\begin{enumerate}[itemsep=5pt, label=\textbf{\arabic*}. ] 
 \item %
% Solve:
\begin{enumerate}[itemsep=6pt,label=\textbf{(\alph*)}]
\begin{multicols}{2} 
\item %$2(p-1) = 3(p+2)$
\item %$3-6k = 2k-1$
\item %$m + 6(-m+1) + 5m = 0$
\item %$2k + 3 = 2-3(k+3)$
\item %$5t-1=t^{2}-(t+2)(t-2)$\
\item %$3+\dfrac{q}{5} = \dfrac{q}{2}$ 
\item %$5-\dfrac{2(m+4)}{m} = \dfrac{7}{m}$
\item %$\dfrac{2}{t} - 2 - \dfrac{1}{2} = \dfrac{1}{2}(1+\dfrac{2}{t})$
\item %$x^{2} - 3x + 2=0$
\item %$y^{2} + y = 6$
\item %$0=2x^{2} - 5x - 18$
\item %$(d+4)(d-3)-d=(3d-2)^{2} - 8d(d-1)$
\item %$5x+2\leq4(2x-1)$
\item %$\dfrac{4x-2}{6} > 2x+1$
\item %$\dfrac{x}{3} - 14 > 14 - \dfrac{x}{7}$
\item %$\dfrac{1-a}{2} - \dfrac{2-a}{3} \geq 1$
\item %$-5 \leq 2k + 1 < 5$
\item %$x-1=\dfrac{42}{x}$  
\end{multicols}
\end{enumerate}

\item %Consider the following literal equations:
\begin{enumerate}[itemsep=6pt,label=\textbf{(\alph*)}]
% \setcounter{enumi}{18}
\item %Solve for $i$: $P = VI$
\item %Solve for $m$: $E=mc^{2}$
\item %Solve for $t$: $v = u + at$
\item %Solve for $f$: $\dfrac{1}{u} + \dfrac{1}{v} = \dfrac{1}{f}$
\item %Solve for $C$: $F=\frac{9}{5}C + 32$
\item %Solve for $y$: $m = \dfrac{y-c}{x}$
\end{enumerate}

\item %Solve the following simultaneous equations:
\begin{enumerate}[itemsep=5pt,label=\textbf{(\alph*)}]
% \setcounter{enumi}{24}
\item %$7x+3y=13$ and $2x-3y=-4$  
\item %$10=2x+y$ and $y=x-2$
\item %$7x-41=3y$ and $17=3x-y$
\item %$2y=x+8$ and $4y=2x-44$
\end{enumerate}

\item %Find the solutions to the following word problems:
\begin{enumerate}[itemsep=5pt,label=\textbf{(\alph*)}]
% \setcounter{enumi}{28}
\item %$\frac{7}{8}$ of a certain number is $5$ more than of $\frac{1}{3}$ of the number. Find the number.
\item %Three rulers and two pens have a total cost of R $21,00$. One ruler and one pen have a total cost of R $8,00$. How much does a ruler cost and how much does a pen cost? 
\item %A man runs to the bus stop and back in $15$ minutes. His speed on the way to the bus stop is $5$ km/h and his speed on the way back is $4$ km/h. Find the distance to the bus stop.
\item %Zanele and Piet skate towards each other on a straight path. They set off $20$ km apart. Zanele skates at $15$ km/h and Piet at $10$ km/h. How far will Piet have skated when they reach each other?
\item %When the price of chocolates is increased by R $10$, we can buy five fewer chocolates for R $300$. What was the price of each chocolate before the price was increased?
   

\end{enumerate}
\end{enumerate}

% \end{eocexercises}


\section {3. States of matter and the KMT}
\subsection{End of chapter exercises} %p.96
% \begin{eocexercises}{}
\begin{enumerate}[label=\textbf{\arabic*}., itemsep=5pt]
\item %Simplify:
\begin{multicols}{2}
\begin{enumerate}[label=\textbf{(\alph*)}, itemsep=7pt]
\item %$ t^3 \times 2t^0 $
\item %$ 5^{2x+y} . 5^{3(x+z)} $
\item %$ (b^{k+1})^k $
\item %$ \dfrac{6^{5p}}{9^p} $
\item %$ m^{-2t} \times (3m^t)^3 $
\item %$\dfrac{3{x}^{-3}}{{(3x)}^{2}}$
\item %$\dfrac{{5}^{b-3}}{{5}^{b+1}}$
\item %$\dfrac{{2}^{a-2}.{3}^{a+3}}{{6}^{a}}$
\item %$\dfrac{{3}^{n}  .{9}^{n-3}}{{27}^{n-1}}$
\item %${(\dfrac{2{x}^{2a}}{{y}^{-b}})}^{3}$
\item %$\dfrac{{2}^{3x-1}  .{8}^{x+1}}{{4}^{2x-2}}$
\item %$\dfrac{{6}^{2x}  .{11}^{2x}}{{22}^{2x-1}  .{3}^{2x}}$
\item %$\dfrac{{(-3)}^{-3}  .{(-3)}^{2}}{{(-3)}^{-4}}$
\item %${({3}^{-1}+{2}^{-1})}^{-1}$
\item %$\dfrac{{9}^{n-1}  .{27}^{3-2n}}{{81}^{2-n}}$
\item %$\dfrac{{2}^{3n+2}  .{8}^{n-3}}{{4}^{3n-2}}$
\item %$\dfrac{3^{t+3} + 3^t}{2 . 3^t} $
\item %$\dfrac{2^{3p} +1}{2^p + 1} $
\end{enumerate}
\end{multicols}


\item %Solve:
\begin{multicols}{2}
\begin{enumerate}[label=\textbf{(\alph*)}, itemsep=7pt]
\setcounter{enumi}{18}
\item %$ 3^x = \dfrac{1}{27} $
\item %$ 5^{t-1} = 1 $
\item %$ 2 . 7^{3x} = 98 $
\item %$ 2^{m+1} = (0.5)^{m-2}$
\item %$ 3^{y+1} = 5^{y+1} $
\item %$ z^{^3/_2} = 64 $
\item %$ 16x^{\frac{1}{2}} - 4 = 0 $
\item %$ m^0 + m^{-1} = 0 $
\item %$ t^{\frac{1}{2}} - 3t^{\frac{1}{4}} + 2 = 0 $
\item %$ 3^p + 3^p + 3^p = 27 $
\item %$ k^{-1} - 7x^{-\frac{1}{2}} -18 = 0 $
\item %$ x^{\frac{1}{2}}+3x^{\frac{1}{4}}-18 = 0 $
\end{enumerate}
\end{multicols}


% \end{eocexercises}

\section {4. The atom}
\subsection{Exercise 4-1} %p. 103
% \begin{exercises}{}
% { 

\begin{enumerate}[noitemsep, label=\textbf{\arabic*}. ] 
\item %Write down the next three terms in each of the following sequences:

  \begin{enumerate} [noitemsep, label=\textbf{(\alph*)} ]
  \item %$5;~15;~25\ldots$
  \item %$-8;-3;~ 2 \ldots$
  \item %$30;~27;~24\ldots$
  \end{enumerate}
 \item %The general term is given for each sequence below. Calculate the missing terms.
  \begin{enumerate} [noitemsep, label=\textbf{(\alph*)} ]
  \item %$0;3;\ldots;~15;~24$\hspace{2.2cm}$T_{n}={n}^{2}-1$
  \item %$3;2;~1;~0;\ldots;-2$\hspace{2cm}$T_{n}=-n+4$
  \item %$-11;\ldots;-7;\ldots;-3$\hspace{1.5cm}$T_{n}=-13+2n$
  \end{enumerate}
     
\item %Find the general formula for the following sequences and then find ${T}_{10}$, ${T}_{50}$ and ${T}_{100}$
  \begin{enumerate}[noitemsep, label=\textbf{(\alph*)} ]
  \item %$2;~5;~8;~11;~14;\ldots$
  \item %$0;~4;~8;~12;~16;\ldots$
  \item %$2;-1;-4;-7;-10;\ldots$
  \end{enumerate}
\end{enumerate}
}%\End of exercise
% \end{exercises}
\subsection{Exercise 4-2}
\subsection{Exercise 4-3}
\subsection{Exercise 4-4}
\subsection{Exercise 4-5}
\subsection{End of chapter exercises} %p.107-108
% \begin{eocexercises}{}
\begin{enumerate}[noitemsep, label=\textbf{\arabic*}. ] 
\item %Find the $6^{th}$ term in each of the following sequences:
    \begin{enumerate}[noitemsep, label=\textbf{(\alph*)} ]
    \item %$4;~13;~22;~31;\ldots$
    \item %$~5;~2;~-1;~-4;\ldots$
    \item %$7,4;~ 9,7; ~12; ~14,3; \ldots$
    \end{enumerate}


\item %Find the general term of the following sequences:
    \begin{enumerate}[noitemsep, label=\textbf{(\alph*)} ]
    \item %$3;~7;~11;~15;\ldots$
    \item %$-2;~1;~4;~7;\ldots$
    \item %$11;~15;~19;~23;\ldots$
    \item %$\dfrac{1}{3};~ \dfrac{2}{3};~ 1; ~1\dfrac{1}{3};\ldots$
    \end{enumerate}

\item %The seating of a sports stadium is arranged so that the first row has $15$ seats, the second row has $19$ seats, the third row has $23$ seats and so on. Calculate how many seats are in the twenty-fifth row.
\item %A single square is made from $4$ matchsticks. Two squares in a row need $7$ matchsticks and three squares in a row need $10$ matchsticks. For this sequence determine:
    \begin{enumerate}[noitemsep, label=\textbf{(\alph*)} ]
    \item %the first term
    \item %the common difference
    \item %the general formula
    \item %how many matchsticks there are in a row of twenty-five squares
    \end{enumerate}
  

\item %You would like to start saving some money, but because you have never tried to save money before, you decide to start slowly. At the end of the first week you deposit R$~5$ into your bank account. Then at the end of the second week you deposit R$~10$ and at the end of the third week, R$~15$. After how many weeks will you deposit R$~50$ into your bank account?

\item %A horizontal line intersects a piece of string at $4$ points and divides it into five parts, as shown below. 
%      
% If the piece of string is intersected in this way by $19$ parallel lines, each of which intersects it at $4$ points, determine the number
% of parts into which the string will be divided.
 
\item %Consider the following pattern: 
%   \begin{equation*}
%     \begin{array}{ccl}\hfill 9+16&=& 25\\ 9+28 &=& 37\\9+43&=& 52\end{array}
%   \end{equation} 

  \begin{enumerate}[noitemsep, label=\textbf{(\alph*)} ]
  \item %What pattern do you see?
  \item %Make a conjecture and express it in words.
  \item %Generalise your conjecture algebraically.
  \item %Prove that your conjecture is true.
  \end{enumerate}

\end{enumerate}




% \end{eocexercises}

\section {5. The periodic table}
\subsection{Exercise 5-1} %p. 112-113
% \begin{exercises}{}
% {
\begin{enumerate}[noitemsep, label=\textbf{\arabic*}. ] 


\item %Write the following in set notation:
\begin{enumerate}[noitemsep, label=\textbf{(\alph*)} ] 
 \item %$(- \infty; 7]$
\item %$[13;4)$
\item %$(35; \infty)$
\item %$[\frac{3}{4}; 21)$
\item %$[-\frac{1}{2}; \frac{1}{2}]$
\item %$(-\sqrt{3}; \infty)$
\end{enumerate}

\item %Write the following in interval notation:
\begin{enumerate}[noitemsep, label=\textbf{(\alph*)} ] 
% \setcounter{enumi}{6}
 \item %$\{p: p \in \mathbb{R},~ p \leq 6\}$
 \item %$\{k: k \in \mathbb{R},~ -5 < k < 5\}$
 \item %$\{x: x \in \mathbb{R},~ x > \frac{1}{5}\}$
 \item %$\{z: z \in \mathbb{R},~ 21 \leq z < 41\}$
\end{enumerate}
\end{enumerate}
% } 
% \end{exercises}
\subsection{Exercise 5-2} %p. 120-121
% \begin{exercises}{}
% {
\begin{enumerate}[noitemsep, label=\textbf{\arabic*}. ] 

\item %List the $x$- and $y$-intercepts for the following straight line graphs. Indicate whether the graph is increasing or decreasing:
      \begin{enumerate}[noitemsep, label=\textbf{(\alph*)} ] 
      \item %$y=x+1$
      \item %$y=x-1$
      \item %$h(x)=2x-1$
      \item %$y+1=2x$
      \item %$3y-2x=6$
      \item %$k(x)=-3$
      \item %$x=3y$
      \item %$\frac{x}{2} - \frac{y}{3} = 1$
      \end{enumerate}


\item %For the functions in the diagram below, give the equation and domain and range:
    \begin{enumerate}[noitemsep, label=\textbf{(\alph*)} ]  

    \item %$a(x)$
\item %$b(x)$
\item %$c(x)$
\item %$d(x)$

    \end{enumerate}  


\item %Sketch the following functions on the same set of axes, using the dual intercept method. Clearly indicate the intercepts and the point of intersection of the two graphs: $x+2y-5=0$ and $3x-y-1=0$
\item %On the same system of axes, draw the graphs of $f(x)=3-3x$ and $g(x)=\frac{1}{3}x+1$ using the gradient-intercept method.
\end{enumerate}

% }
% \end{exercises}

\subsection{End of chapter exercises} %p.182-186
% \begin{eocexercises}{}
    \begin{enumerate}[itemsep=6pt, label=\textbf{\arabic*}. ] 
\item %Sketch the graphs of the following: 
 \begin{enumerate}[noitemsep, label=\textbf{(\alph*)} ]
    \item %$y=2x+4$ 
    \item %$y-3x=0$ 
    \item %$2y=4-x$
    \end{enumerate}
\item %Sketch the following functions: 
 \begin{enumerate}[noitemsep, label=\textbf{(\alph*)} ]
% \setcounter{enumi}{3} 
    \item %$y=x^{2}+3$ 
    \item %$y=\frac{1}{2}x^{2}+4$
    \item %$y=2x^{2}-4$
    \end{enumerate}
\item %Sketch the following functions and identify the asymptotes: 
 \begin{enumerate}[noitemsep, label=\textbf{(\alph*)} ] 
% \setcounter{enumi}{6} 
    \item %$y=3^{x}+2$ 
    \item %$y=-4.2^{x}$ 
    \item %$y=\Big(\dfrac{1}{3}\Big)^{x}-2$ 
    \end{enumerate}
\item %Sketch the following functions and identify the asymptotes: 
 \begin{enumerate}[noitemsep, label=\textbf{(\alph*)} ]
% \setcounter{enumi}{9} 
    \item %$y=\dfrac{3}{x}+4$ 
    \item %$y=\dfrac{1}{x}$ 
    \item %$y=\dfrac{2}{x}-2$ 
    \end{enumerate}
\item %Determine whether the following statements are true or false. If the statement is false, give reasons why:
  \begin{enumerate}[noitemsep, label=\textbf{(\alph*)} ]
% \setcounter{enumi}{12} 
    \item %The given or chosen $y$-value is known as the independent variable.
    \item %A graph is said to be congruent if there are no breaks in the graph.
    \item %Functions of the form $y=ax+q$ are straight lines.
    \item %Functions of the form $y=\frac{a}{x}+q$ are exponential functions.
    \item % An asymptote is a straight line which a graph will intersect at least once.
    \item %Given a function of the form $y=ax+q$, to find the $y$-intercept let $x=0$ and solve for $y$.
    \end{enumerate}
\item %Given the functions $f(x)=-2{x}^{2}-18$ and $g(x)=-2x+6$
 \begin{enumerate}[noitemsep, label=\textbf{(\alph*)} ]
% \setcounter{enumi}{18} 
    \item %Draw $f$ and $g$ on the same set of axes.
    \item %Calculate the points of intersection of $f$ and $g$.
\end{enumerate}
\item %Hence use your graphs and the points of intersection to solve for $x$ when:
	 \begin{enumerate}[noitemsep, label=\textbf{(\alph*)} ]
% \setcounter{enumi}{20} 
	\item %$f(x)>0$
	\item %$\dfrac{f(x)}{g(x)}\le 0$
\item %Give the equation of the reflection of $f$ in the $x$-axis.
\end{enumerate}
	\item %After a ball is dropped, the rebound height of each bounce decreases. The equation $y=5{(0,8)}^{x}$ shows the relationship between the number of bounces ($x$) and the height of the bounce ($y$) for a certain ball. 

% What is the approximate height of the fifth bounce of this ball to the nearest tenth of a unit ?


\item %Mark had $15$ coins in R$~5$ and R$~2$ pieces. He had $3$ more R$~2$ coins than R$~5$ coins. He wrote a system of equations to represent this situation, letting $x$ represent the number of R$~5$ coins and $y$ represent the number of R$~2$ coins. Then he solved the system by graphing.
	 \begin{enumerate}[noitemsep, label=\textbf{(\alph*)} ]
% \setcounter{enumi}{24} 
    \item %Write down the system of equations.
    \item %Draw their graphs on the same set of axes.
    \item %What is the solution?
    \end{enumerate}
\item %Sketch graphs of the following trigonometric functions for $\theta \in[0;360]$ (show intercepts and asymptotes):
	 \begin{enumerate}[noitemsep, label=\textbf{(\alph*)} ] 
% \setcounter{enumi}{27} 
    \item %$y=-4~cos~\theta$
    \item %$y=sin~\theta -2$
\item %$y=-2~sin~\theta +1$
\item %$y=tan~\theta+2$
\item %$y=\dfrac{cos~\theta}{2}$
    \end{enumerate}

\item %Given the general equation, determine the values of $a$ and $q$ for each of the following graphs:\vspace{20pt}\\

\item %$y=2^x$ and $y=-2^x$ are sketched below. Answer the questions that follow:\\

 \begin{enumerate}[noitemsep, label=\textbf{(\alph*)} ]
% \setcounter{enumi}{39}
\item %Calculate the coordinates of $M$ and $N$.
\item %Calculate the length of $MN$.
\item %Calculate length $PQ$ if $OR=1$ unit.
\item %Give the equation of $y=2^x$ reflected about the $y$-axis.
\item %Give the range of both graphs.
\end{enumerate}

\item %$f(x)=4^x$ and $g(x)=4x^2+q$ and sketched below. The points $A(0;1)$, $B(1;4)$ and $C(1,-3)$ are given. Answer the questions that follow:\\


 \begin{enumerate}[noitemsep, label=\textbf{(\alph*)} ]
% \setcounter{enumi}{44}
\item %Calculate the coordinates of $M$ and $N$.
\item %Calculate the length of $MN$.
\item %Calculate length $PQ$ if $OR=1$ unit.
\item %Give the equation of $y=2^x$ reflected about the $y$-axis.
\item %Give the range of both graphs.
\end{enumerate}

\item %Sketch the graphs $h(x)=x^2-4$ and $k(x)=-x^2+4$ on the same set of axes and answer the questions that follow: 
 \begin{enumerate}[noitemsep, label=\textbf{(\alph*)} ]
% \setcounter{enumi}{49}
\item %Describe the relationship between $h$ and $k$.
\item %Give the equation of $k(x)$ reflected about the line $y=4$.
\item %Give the domain and range of $h$.
\end{enumerate}

\item %Sketch the graphs $f(\theta)=2~sin~\theta$ and $g(\theta)=cos~\theta-1$ on the same set of axes. Use your sketch to determine:
 \begin{enumerate}[noitemsep, label=\textbf{(\alph*)} ]
% \setcounter{enumi}{52}
\item %$f(180^{\circ})$
\item %$g(180^{\circ})$
\item %$g(270^{\circ}) -f(270^{\circ})$
\item %The domain and range of $g$.
\item %The amplitude and period of $f$.
\end{enumerate}
\end{enumerate}

% \end{eocexercises}

\section {6. Chemical bonding}
\subsection{Exercise 6-1} %p. 193
% \begin{exercises}{}{
    \begin{enumerate}[itemsep=6pt, label=\textbf{\arabic*}.]
	\item %An amount of R~$3~500$ is invested in a savings account which pays simple interest at a rate of $7,5\%$ per annum. Calculate the balance accumulated by the end of 2 years.

	\item %Calculate the accumulated amount in the following situations:
	\begin{enumerate}[noitemsep, label=\textbf{(\alph*)} ]
	    \item %A loan of R~$300$ at a rate of $8\%$ for 1 year.

	    \item %An investment of R~$2~250$ at a rate of $12,5\%$p.a. for 6 years.
	\end{enumerate}

	\item %Sally wanted to calculate the number of years she needed to invest R~$1~000$ for in order to accumulate R~$2~500$. She has been offered a simple interest rate of $8,2\%$ p.a. How many years will it take for the money to grow to R~$2~500$?

	\item %Joseph made a deposit of R~$5~000$ in the bank for his 5 year old son’s 21st birthday. He has given his son the amount of R~$18~000$ on his birthday. At what rate was the money invested, if simple interest was calculated?\\
    \end{enumerate}

% }
% \end{exercises}
\subsection{Exercise 6-2} %p. 197
% \begin{exercises}{}{
    \begin{enumerate}[label=\textbf{\arabic*}.]
	\item %Vanessa wants to buy a fridge on a hire purchase agreement. The cash price of the fridge is R~$4~500$. She is required to pay a deposit of $15\%$ and pay the remaining loan amount off over 24 months at an interest rate of $12\%$ p.a.
	\begin{enumerate}[noitemsep, label=\textbf{(\alph*)} ]
	    \item %What is the principal loan amount?
	    \item %What is the accumulated loan amount?
	    \item %What are Vanessa’s monthly repayments?
	    \item %What is the total amount she has paid for the fridge?
	\end{enumerate}


	\item %Bongani buys a dining room table costing R~$8~500$ on a hire purchase agreement. He is charged an interest rate of $17,5\%$p.a. over 3 years.
	\begin{enumerate}[noitemsep, label=\textbf{(\alph*)} ]
	    \item %How much will Bongani pay in total?
	    \item %How much interest does he pay?
	    \item %What is his monthly instalment?
	\end{enumerate}

	\item %A lounge suite is advertised on TV, to be paid off over 36 months at R~$150$ a month.
	\begin{enumerate}[noitemsep, label=\textbf{(\alph*)} ]
	    \item %Assuming that no deposit is needed, how much will the buyer pay for the lounge suite once it has been paid off?
	    \item %If the interest rate is $9\%$ p.a, what is the cash price of the suite?\\
	\end{enumerate}
    \end{enumerate}
% }
% \end{exercises}
\subsection{Exercise 6-3} %p. 204
% \begin{exercises}{}{
    \begin{enumerate}[label=\textbf{\arabic*}.]
	\item %An amount of R~$3~500$ is invested in a savings account which pays a compound interest rate of $7,5\%$ p.a. Calculate the balance accumulated by the end of 2 years.

	\item %Morgan invests R~$5~000$ into an account which pays out a lump sum at the end of 5 years. If he gets R~$7~500$ at the end of the period, what compound interest rate did the bank offer him?

	\item %Nicola wants to invest some money at a compound interest rate of $11\%$ p.a. How much money (to the nearest Rand) should be invested if she wants to reach a sum of R~$100~000$ in five years time?\\
    \end{enumerate}

% }
% \end{exercises}
\subsection{Exercise 6-4} %p.207
% \begin{exercises}{}{
    \begin{enumerate}[label=\textbf{\arabic*}.]
	\item %If the average rate of inflation for the past few years was $7,3\%$ p.a. and your water and electricity account is R~$1~425$ on average, what would you expect to pay in 6 years time?

	\item %The price of popcorn and a coke at the movies is now R~$60$. If the average rate of inflation is $9,2\%$ p.a., what was the price of popcorn and coke 5 years ago?

	\item %A small town in Ohio, USA is experience a huge increase in births. If the average growth rate of the population is $16\%$ p.a, how many babies will be born to the 1600 residents in the next 2 years?\\
    \end{enumerate}

% }
% \end{exercises}
\subsection{Exercise 6-5} %p.209
% \begin{exercises}{}
% {
    \begin{enumerate}[itemsep=6pt, label=\textbf{\arabic*}.]
	\item %Bridget wants to buy an iPod that costs £~$100$, with the exchange rate currently at £~$1$ $=$ R~$14$. She estimates that the exchange rate will drop to R~$12$ in a month.
	\begin{enumerate}[noitemsep, label=\textbf{(\alph*)} ]
	    \item %How much will the MP3 player cost in Rands, if she buys it now?
	    \item %How much will she save if the exchange rate drops to R~$12$?
	    \item %How much will she lose if the exchange rate moves to R~$15$?
	\end{enumerate}

	\item %Study the following exchange rate table:\\
	
	\begin{enumerate}[noitemsep, label=\textbf{(\alph*)} ]
	    \item %In South Africa the cost of a new Honda Civic is R~$173~400$. In England the same vehicle costs £~$12~200$ and in the USA \$~$21~900$. In which country is the car the cheapest?

	    \item %Sollie and Arinda are waiters in a South African restaurant attracting many tourists from abroad. Sollie gets a £~$6$ tip from a tourist and Arinda gets \$~$12$. Who got the better tip?
	\end{enumerate}
    \end{enumerate}

% }
% \end{exercises}

\subsection{End of chapter exercises} %p.210-211
% \begin{eocexercises}{}
    \begin{enumerate}[label=\textbf{\arabic*}.]
	\item %Alison is going on holiday to Europe. Her hotel will cost €~$200$ per night. How much will she need in Rands to cover her hotel bill, if the exchange rate is €~$1$ = R~$9,20$?

	\item %Calculate how much you will earn if you invested R~$500$ for 1 year at the following interest rates:
	\begin{enumerate}[noitemsep, label=\textbf{(\alph*)} ]
	    \item %$6,85\%$ simple interest
	    \item %$4,00\%$ compound interest
	\end{enumerate}

	\item %Bianca has R~$1~450$ to invest for 3 years. Bank A offers a savings account which pays simple interest at a rate of $11\%$ per annum, whereas Bank B offers a savings account paying compound interest at a rate of $10,5\%$ per annum. Which account would leave Bianca with the highest accumulated balance at the end of the 3 year period?

	\item %How much simple interest is payable on a loan of R~$2~000$ for a year, if the interest rate is $10\%$ p.a.?

	\item %How much compound interest is payable on a loan of R~$2~000$ for a year, if the interest rate is $10\%$ p.a.?

	\item %Discuss:
	\begin{enumerate}[noitemsep, label=\textbf{(\alph*)} ]
	    \item %Which type of interest would you like to use if you are the borrower?

	    \item %Which type of interest would you like to use if you were the banker?
	\end{enumerate}

	\item %Calculate the compound interest for the following problems.
	\begin{enumerate}[noitemsep, label=\textbf{(\alph*)} ]
	    \item %A R~$2~000$ loan for 2 years at $5\%$ p.a..
	    \item %A R~$1~500$ investment for 3 years at $6\%$ p.a..
	    \item %A R~$800$ loan for 1 year at $16\%$ p.a..
	\end{enumerate}

	\item %If the exchange rate for ¥~$100$ = R~$6,2287$ (¥ = Yen) and 1 Australian Dollar (AUD) = R~$5,1094$, determine the exchange rate between the Australian Dollar and the Japanese Yen.

	\item %Bonnie bought a stove for R~$3~750$. After 3 years she had finished paying for it and the R~$956,25$ interest that was charged for hire purchase. Determine the rate of simple interest that was charged.
    \end{enumerate}

% \end{eocexercises}

\section {7. Transverse pulses}
\subsection{Exercise 7-1} %p. 218-219
% \begin{exercises}{}
% {
\begin{enumerate}[itemsep=5pt, label=\textbf{\arabic*}. ]
\item %In each of the following triangles, state whether $a$, $b$ and $c$ are the hypotenuse, opposite or adjacent sides of the triangle with respect to $\theta$. 
% \setcounter{subfigure}{0}
\begin{enumerate}[noitemsep, label=\textbf{(\alph*)} ]
 \item 
\item 
\item 
\item 
\item 
\item 
\end{enumerate}



\item %Use your calculator to determine the value of the following (correct to $2$ decimal places):
\begin{enumerate}[noitemsep, label=\textbf{(\alph*)} ]
\begin{multicols}{2} 
% \setcounter{enumi}{6} 
\item %$tan~65^{\circ}$
\item %$sin~38^{\circ}$
\item %$cos~74^{\circ}$
\item %$sin~12^{\circ}$
\item %$cos~26^{\circ}$
\item %$tan~49^{\circ}$
\item %$\frac{1}{4}~cos~20^{\circ}$
\item %$3~tan~40^{\circ}$
\item %$\frac{2}{3}~sin~90^{\circ}$
\end{multicols}
\end{enumerate}

\item %If $x=39^{\circ}$ and $y=21^{\circ}$ use a calculator to determine whether the following statements are true or false:
\begin{enumerate}[noitemsep, label=\textbf{(\alph*)} ]
\begin{multicols}{2} 
% \setcounter{enumi}{15} 
\item %$cos~x + 2~cos~x=3~cos~x$
\item %$cos~2y = cos~y+cos~y$
\item %$tan~x=\dfrac{sin~x}{cos~x}$
\item %$cos~(x+y) = cos~x+cos~y$
\end{multicols}
\end{enumerate}


\item %Complete each of the following (the first example has been done):



\begin{enumerate}[noitemsep, label=\textbf{(\alph*)} ]
\begin{multicols}{2}
% \setcounter{enumi}{19}
\item %$sin~\hat{A} = \frac{\mbox{opposite}}{\mbox{hypotenuse}}=\dfrac{CB}{AC}$
\item %$cos~\hat{A} = $
\item %$tan~\hat{A}= $
\item %$sin~\hat{C}= $
\item %$cos~\hat{C}= $
\item %$tan~\hat{C}= $
\end{multicols}
\end{enumerate}

\item %Use the triangle below to complete the following:

\begin{enumerate}[noitemsep, label=\textbf{(\alph*)} ]
\begin{multicols}{2}
% \setcounter{enumi}{25}
\item %$sin~60^{\circ} = $
\item %$cos~60^{\circ} = $
\item %$tan~60^{\circ}= $
\item %$sin~30^{\circ}= $
\item %$cos~30^{\circ}= $
\item %$tan~30^{\circ}= $
\end{multicols}
\end{enumerate}


\item %Use the triangle below to complete the following:


\begin{enumerate}[noitemsep, label=\textbf{(\alph*)} ]

% \setcounter{enumi}{31}
\item %$sin~45^{\circ} = $
\item %$cos~45^{\circ} = $
\item %$tan~45^{\circ}= $

\end{enumerate}
\end{enumerate}
% }
% \end{exercises}

\subsection{Exercise 7-2} %p. 220
% \begin{exercises}{}
% {
\begin{enumerate}[itemsep=6pt, label=\textbf{\arabic*}. ] 
\item %Calculate the value of the following without using a calculator:
\begin{enumerate}[noitemsep, label=\textbf{(\alph*)} ]
\item %$sin~45^{\circ} \times cos~45^{\circ}$
\item %$cos~60^{\circ} + tan~45^{\circ}$
\item %$sin~60^{\circ} - cos~60^{\circ}$
\end{enumerate}

\item %Use the table to  show that:
\begin{enumerate}[itemsep=5pt, label=\textbf{(\alph*)} ]
% \setcounter{enumi}{3}
\item %$\dfrac{sin~60^{\circ}}{cos~60^{\circ}} = tan~60^{\circ} $
\item %$sin^{2}~45^{\circ}+ cos^{2}~45^{\circ} =1 $
\item %$cos~30^{\circ} =\sqrt{1- sin^{2}~30^{\circ}$
\end{enumerate}

\item %Use the definitions of the trigonometric rations to answer the following questions:
\begin{enumerate}[noitemsep, label=\textbf{(\alph*)} ]
% \setcounter{enumi}{6}
\item %Explain why $sin~\alpha \leq 1$ for all values of $\alpha$.
\item %Explain why $cos~\beta$ has a maximum value of $1$.
\item %Is there a maximum value for $tan~\gamma$ ?
\end{enumerate}
\end{enumerate}
% }
% \end{exercises}

\subsection{End of chapter exercises} %p.240-242
% \begin{eocexercises}{}

\begin{enumerate}[itemsep=6pt, label=\textbf{\arabic*}. ] 
\item %Without using a calculator determine the value of 
% \begin{equation*}
% sin~60^{\circ}~cos~30^{\circ}-cos~60^{\circ}sin~30^{\circ} + tan~45^{\circ}
% \end{equation*}
\item %If $3~tan~\alpha = -5$ and $0^{\circ} < \alpha < 270^{\circ}$, use a sketch to determine:
    \begin{enumerate}[noitemsep, label=\textbf{(\alph*)} ]
    \item %$cos~\alpha$
    \item %$tan^{2}~\alpha - sec^{2}~\alpha$
    \end{enumerate}
\item %Solve for $\theta$ if $\theta$ is a postive, acute angle:
    \begin{enumerate}[noitemsep, label=\textbf{(\alph*)} ]
    \item %$2~sin~\theta = 1,34$
    \item %$1 - tan~\theta = -1$
    \item %$cos~2\theta = sin~40^{\circ}$ 
    \item %$\dfrac{sin~\theta}{cos~\theta}= 1$
    \end{enumerate}


\item %Calculate the unknown lengths in the diagrams below:

\item %Determine $\alpha$ in the following right-angled triangles:
    \begin{enumerate}[noitemsep, label=\textbf{(\alph*)} ]
       \item 
\item
\item
\item
\item
\item
\item
      \end{enumerate}
\item %In $\triangle PQR$, $PR=20$ cm, $QR=22$ cm and $P\hat{R}Q = 30^{\circ}$. The perpendicular line from $P$ to $QR$ intersects $QR$ at $X$. Calculate 
\begin{enumerate}[noitemsep, label=\textbf{(\alph*)} ]
\item %the length $XR$ 
\item %the length $PX$
\item %the angle $Q\hat{P}X$ 
\end{enumerate} 
\item %A ladder of length $15$ m is resting against a wall, the base of the ladder is $5$ m from the wall. Find the angle between the wall and the ladder. 
\item %In the following triangle find the angle $A\hat{B}C$:

\item %In the following triangle find the length of side $CD$:
 
\item %$A(5;0)$ and $B(11;4)$. Find the angle between the line through $A$ and $B$ and the $x$-axis. 
\item %$C(0;-13)$ and $D(-12;14)$. Find the angle between the line through $C$ and $D$ and the $y$-axis. 


\item %A right-angled triangle has hypotenuse $13$ mm. Find the length of the other two sides if one of the angles of the triangle is $50^{\circ}$.
\item %One of the angles of a rhombus with perimeter $20$ cm is $30^{\circ}$. 
\begin{enumerate}[noitemsep, label=\textbf{(\alph*)} ]
\item %Find the sides of the rhombus. 
\item %Find the length of both diagonals. 
\end{enumerate} 
\item %Captain Jack was sailing towards a cliff with a height of $10$ m. 
\begin{enumerate}[noitemsep, label=\textbf{(\alph*)} ] 
\item %The distance from the boat to the top of the cliff is $30$ m, calculate the angle of elevation from the boat to the top of the cliff (correct to the nearest integer).
\item %If the boat sails $7$ m closer to the cliff, what is the new angle of elevation from the boat to the top of the cliff? 
\end{enumerate} 
\item %Given the points: $E(5;0)$; $F(6;2)$ and $G(8;-2)$. Find the angle $F\hat{E}G$. 
\item % A triangle with angles $40^{\circ}$; $40^{\circ}$ and $100^{\circ}$ has a perimeter of $20$ cm. Find the length of each side of the triangle. 

\end{enumerate}
% \end{eocexercises}


\section {8. Transverse waves}
\subsection{Exercise 8-1} %p. 248
% \begin{exercises}{}{
\begin{enumerate}[label=\textbf{\arabic*}.]
\item %Find the length of $AB$ if:
 \begin{enumerate}[noitemsep, label=\textbf{(\alph*)} ] 
\item %$A(2;7)$ and $B(-3;5)$.
\item %$A(-3;5)$ and $B(-9;1)$.
\item %$A(x;y)$ and $B(x+4;y-1)$.
\end{enumerate}

\item %The length of $CD=5$. Find the missing coordinate if:
 \begin{enumerate}[noitemsep, label=\textbf{(\alph*)} ] 
\item %$C(6;-2)$ and $D(x;2)$.
\item %$C(4;y)$ and $D(1;-1)$.
\end{enumerate}
\end{enumerate}

% \end{exercises}

\subsection{End of chapter exercises} %p.272-274

% \begin{eocexercises}{}
\begin{enumerate}[noitemsep, label=\textbf{\arabic*}. ] 
\item %Represent the following figures on the Cartesian plane: 
 \begin{enumerate}[noitemsep, label=\textbf{(\alph*)} ]
\item %Triangle $DEF$ with $D(1;2)$, $E(3;2)$ and $F(2;4)$ 
\item %Quadrilateral $GHIJ$ with $G(2;-1)$, $H(0;2)$, $I(-2;-2)$ and $J(1;-3)$
\item %Quadrilateral $MNOP$ with $M(1;1)$, $N(-1;3)$, $O(-2;3)$ and $P(-4;1)$ 
\item %Quadrilateral $WXYZ$ with $W(1;-2)$, $X(-1;-3)$, $Y(2;-4)$ and $Z(3;-2)$
\end{enumerate}
\item %In the diagram below the vertices of a quadrilateral are $F(2;0)$, $G(1;5)$, $H(3;7)$ and $I(7;2)$ are given.
 \begin{enumerate}[noitemsep, label=\textbf{(\alph*)} ]
\item %What are the lengths of the opposite sides of $FGHI$?
\item %Are the opposite sides of $FGHI$ parallel?
\item % Do the diagonals of $FGHI$ bisect each other?
\item % Can you state what type of quadrilateral $FGHI$ is? Give reasons for your answer.
\end{enumerate}
\item %A quadrialteral $ABCD$ with vertices $A(3;2)$, $B(1;7)$, $C(4;5)$ and $D(1;3)$ is given.
 \begin{enumerate}[noitemsep, label=\textbf{(\alph*)} ]
\item % Draw the quadrilateral.
\item % Find the lengths of the sides of the quadrilateral.
\end{enumerate}
\item %$ABCD$ is a quadrilateral with verticies $A(0;3)$, $B(4;3)$, $C(5;-1)$ and $D(-1;-1)$.
 \begin{enumerate}[noitemsep, label=\textbf{(\alph*)} ]
\item %Show that:
\begin{enumerate}[noitemsep, label=\textbf{\roman*}. ] 
\item %$AD = BC$
\item %$AB \parallel DC$
\end{enumerate}
\item %What type of quadrilateral is $ABCD$?
\item %Show that the diagonals $AC$ and $BD$ do not bisect each other.
\end{enumerate}
\item %$P$, $Q$, $R$ and $S$ are the points $(-2;0)$, $(2;3)$, $(5;3)$, $(-3;-3)$ respectively.
 \begin{enumerate}[noitemsep, label=\textbf{(\alph*)} ]
\item %Show that:
\begin{enumerate}[noitemsep, label=\textbf{\roman*}. ] 
\item %$SR = 2PQ$
\item %$SR \parallel PQ$
\end{enumerate}
\item %Calculate:
\begin{enumerate}[noitemsep, label=\textbf{\roman*}. ] 
\item %$PS$
\item %$QR$
\end{enumerate}
\item %What kind of a quadrilateral is $PQRS$? Give reasons for your answers.
\end{enumerate}
\item %$EFGH$ is a parallelogram with verticies $E(-1;2)$, $F(-2;-1)$ and $G(2;0)$. Find the coordinates of $H$ by using the fact that the diagonals of a parallelogram bisect each other.
\item % $PQRS$ is a quadrilateral with points $P(0;-3)$ ; $Q(-2;5)$ ; $R(3;2)$ and $S(3;-2)$  in the Cartesian plane.
 \begin{enumerate}[noitemsep, label=\textbf{(\alph*)} ]
\item %Find the length of $QR$.
\item %Find the gradient of $PS$.
\item %Find the mid-point of $PR$.
\item %Is $PQRS$ a parallelogram?  Give reasons for your answer.
 \end{enumerate}
\item %$A(-2;3)$ and $B(2;6)$ are points in the Cartesian plane. $C(a;b)$ is the mid-point of $AB$. Find the values of $a$ and $b$.
\item %Consider triangle $ABC$ with vertices $A(1; 3)$ , $B(4;1)$ and $C (6; 4)$:
 \begin{enumerate}[noitemsep, label=\textbf{(\alph*)} ]
\item %Sketch triangle $ABC$ on the Cartesian plane. 
\item %Show that $ABC$ is an isoceles triangle.
\item %Determine the coordinates of $M$, the mid-point of $AC$.
\item %Determine the gradient of $AB$.
\item %Show that the following points are collinear: $A$, $B$ and $D(7;-1)$
\end{enumerate}
\item %In the diagram, $A$ is the point $(-6;1)$ and $B$ is the point $(0;3)$

 \begin{enumerate}[noitemsep, label=\textbf{(\alph*)} ]
\item %Find the equation of line $AB$ 
\item %Calculate the length of $AB$
% \item % $A'$ is the image of $A$ and $B'$ is the image of $B$. Both these images are obtain by applying the transformation: $(x;y)\to(x - 4;y - 1)$. Give the coordinates of both $A'$ and $B'$
% \item %Find the equation of $A'B'$
% \item %Calculate the length of $A'B'$
% \item %Can you state with certainty that $AA'B'B$ is a parallelogram? Justify your answer.
\end{enumerate}
\end{enumerate}

% \end{eocexercises}

\section {9. Longitudinal waves}

\subsection{End of chapter exercises} %p.307-309
% \begin{eocexercises}{}
  \begin{enumerate}[itemsep=6pt, label=\textbf{\arabic*}.]

  \item %
%   In a park, the tallest $7$ trees in a park have heights in metres of
%     $41$; $60$; $47$; $42$; $44$; $42$; and $47$. Find the median of
%     their heights.

  \item %The students in Ndeme's class have the following ages: $5$;
%     $6$; $7$; $5$; $4$; $6$; $6$; $6$; $7$; $4$. Find the mode of
%     their ages.

  \item %An engineering company has designed two different types of
%     engines for motorbikes. The two different motorbikes are tested
%     for the time (in seconds) it takes for them to accelerate from $0$
%     km/h to $60$ km/h.

    
\begin{enumerate}[noitemsep, label=\textbf{(\alph*)} ]
    \item %What measure of central tendency should be used for this
%       information?
    \item %Calculate the measure of central tendency that you chose in
%       the previous question, for each motorbike.
    \item %Which motorbike would you choose based on this information?
%       Take note of the accuracy of the numbers from each set of tests.
    \end{enumerate}

  \item %In a traffic survey, a random sample of $50$ motorists were
%     asked the distance they drove to work daily. This information is
%     shown in the table below.\\

     \begin{enumerate}[noitemsep, label=\textbf{(\alph*)} ]
    \item %Find the approximate mean of the data.
    \item %What percentage of samples had a speed of
      \begin{enumerate}[noitemsep, label=\textbf{\roman*}. ]
      \item %less than $16$ km?
      \item %more than $30$ km?
      \item %between $16$ km and $30$ km daily?
      \end{enumerate}
\item %Draw a histogram to represent the data
    \end{enumerate}

  \item %A company wanted to evaluate the training programme in its
%     factory. They gave the same task to trained and untrained
%     employees and timed each one in seconds.

    \begin{enumerate}[noitemsep, label=\textbf{(\alph*)} ]
    \item %Find the medians and quartiles for both sets of data
    \item %Find the interquartile range for both sets of data
    \item %Comment on the results
\item %Draw a box-and-whisker diagram to illustrate the five number summary
    \end{enumerate}

  \item %A small firm employs nine people. The annual salaries of the employers are:

    \begin{enumerate}[noitemsep, label=\textbf{(\alph*)} ]
    \item %Find the mean of these salaries
    \item %Find the mode
    \item %Find the median
    \item %Of these three figures, which would you use for
%       negotiating salary increases if you were a trade union
%       official? Why?
    \end{enumerate}

  \end{enumerate}
% \end{eocexercises}


\section {10. Sound}
\subsection{Exercise 10-1} %p. 316
% \begin{exercises}{}
% {
  \begin{enumerate}[itemsep=5pt, label=\textbf{\arabic*}. ]
  \item %
% A bag contains $6$ red, $3$ blue, $2$ green and $1$ white
%     balls. A ball is picked at random. Determine the probablity that it
%     is:
    \begin{enumerate}[noitemsep, label=\textbf{(\alph*)} ]
    \item %red
    \item %blue or white
    \item %not green
    \item %not green or red
    \end{enumerate}
  \item %
% A playing card is selected randomly from a pack of $52$
%     cards. Determine the probability that it is:
    \begin{enumerate}[noitemsep, label=\textbf{(\alph*)} ]
% \setcounter{enumi}{4}
    \item %the $2$ of hearts
    \item %a red card
    \item %a picture card
    \item %an ace
    \item %a number less than $4$?
    \end{enumerate}
\item %Even numbers in the range $2$--$100$ are written on cards. 
%What is
%     the probability of selecting a multiple of $5$, if a card is drawn
%     at random?

\end{enumerate}
% }
% \end{exercises}

\subsection{End of chapter exercises} %p.333-335

% \begin{eocexercises}{}
  \begin{enumerate}[itemsep=5pt, label=\textbf{\arabic*}. ]
  \item %A group of $45$ children were asked if they eat Frosties and/or
%     Strawberry Pops. $31$ eat both and $6$ eat only Frosties. What is the
%     probability that a child chosen at random will eat only Strawberry
%     Pops?
  \item %In a group of $42$ pupils, all but $3$ had a packet of chips
%     or a Fanta or both. If $23$ had a packet of chips and $7$ of these
%     also had a Fanta, what is the probability that one pupil chosen at
%     random has:
    \begin{enumerate}[noitemsep, label=\textbf{(\alph*)} ]
    \item %both chips and Fanta
    \item %only Fanta?
    \end{enumerate}
  \item %Use a Venn diagram to work out the following probabilities
%     from a die be rolled:
    \begin{enumerate}[noitemsep, label=\textbf{(\alph*)} ]
    \item %a multiple of $5$ and an odd number
    \item %a number that is neither a multiple of $5$ nor an odd
%       number
    \item %a number which is not a multiple of $5$, but is odd
    \end{enumerate}
  \item %A packet has yellow and pink sweets. The probability of taking
%     out a pink sweet is $\frac{7}{12}$.
    \begin{enumerate}[noitemsep, label=\textbf{(\alph*)} ]
    \item %What is the probability of taking out a yellow sweet
    \item %If $44$ if the sweets are yellow, how many sweets are pink?
    \end{enumerate}
  \item %In a car park with $300$ cars, there are $190$ Opels. What is the
%     probability that the first car to leave the car park is:
    \begin{enumerate}[noitemsep, label=\textbf{(\alph*)} ]
    \item %an Opel
    \item %not an Opel
    \end{enumerate}
  \item %Tamara has $18$ loose socks in a drawer. Eight of these are
%     orange and two are pink. Calculate the probability that the first
%     sock taken out at random is:
    \begin{enumerate}[noitemsep, label=\textbf{(\alph*)} ]
    \item %orange
    \item %not orange
    \item %pink
    \item %not pink
    \item %orange or pink
    \item %neither orange nor pink
    \end{enumerate}
  \item %A plate contains $9$ shortbread cookies, $4$ ginger biscuits,
%     $11$ chocolate chip cookies and $18$ Jambos. If a biscuit is
%     selected at random, what is the probability that:
    \begin{enumerate}[noitemsep, label=\textbf{(\alph*)} ]
    \item %it is either a ginger biscuit of a Jambo
    \item %it is not a shortbread cookie
    \end{enumerate}
  \item %$280$ tickets were sold at a raffle. Ingrid bought $15$
%     tickets. What is the probability that Ingrid:
    \begin{enumerate}[noitemsep, label=\textbf{(\alph*)} ]
    \item %wins the prize
    \item %does not win the prize
    \end{enumerate}
  \item %The children in a nursery school were classified by hair and
%     eye colour. $44$ had red hair and not brown eyes, $14$ had brown eyes
%     and red hair, $5$ had brown eyes but not red hair and $40$ did not
%     have brown eyes or red hair.
    \begin{enumerate}[noitemsep, label=\textbf{(\alph*)} ]
    \item %how many children were in the schoo?
    \item %What is the probility that a child chosen at random has:
      \begin{enumerate}
      \item %brown eyes
      \item %red hair
      \end{enumerate} 
    \item %A child with brown eyes is chosen randomly. What is the
%       probability that this child will have red hair?
    \end{enumerate}
  \item %A jar has purple, blue and black sweets in it. The probability
%     that a sweet chosen at random will be purple is $\frac{1}{7}$
%     and the probability that it will be black is $\frac{3}{5}$.
    \begin{enumerate}[noitemsep, label=\textbf{(\alph*)} ]
\item %If I choose a sweet at random what
%       is the probability that it will be:
      \begin{enumerate}
      \item %purple or blue
      \item %black
      \item %purple
      \end{enumerate}
    \item %If there are $70$ sweets in the jar how many purple ones are       there?
    \item %$\frac{1}{4}$ of the purple sweets in b) have streaks on
%       them and the rest do not. How many purple sweets have streaks?
    \end{enumerate}
\item %For each of the following, draw a Venn diagram to represent
%     the situation and find an example to illustrate the situation.
    \begin{enumerate}[noitemsep, label=\textbf{(\alph*)} ]
    \item %a sample space in which there are two events that are not
%       mutually exclusive
    \item %a sample space in which there are two events that are
%       complementary
    \end{enumerate}
\item %Use a Venn diagram to prove that the probability of either
%     event $A$ or $B$ occuring is given by: ($A$ and $B$ are not
%     exclusive)
%     \[P(A \cup B) = P(A) + P(B) - P(A \cap B)\]
\item %All the clubs are taken out of a pack of cards. The remaining
%     cards are then shuffled and one card chosen. After being chosen,
%     the card is replaced before the next card is chosen.
    \begin{enumerate}[noitemsep, label=\textbf{(\alph*)} ]
    \item %What is the sample space?
    \item %Find a set to represent the event, $P$, of drawing a picture
%       card.
    \item %Find a set for the event, $N$, of drawing a numbered card.
    \item %Represent the above events in a Venn diagram.
    \item %What description of the sets $P$ and $N$ is suitable?
%       (Hint: Find any elements of $P$ in $N$ and of $N$ in $P$.)
    \end{enumerate}
\item %Thuli has a bag containing five orange, three purple and seven
%     pink blocks. The bag is shaken and a block is withdrawn. The
%     colour of the block is noted and the block is replaced.
    \begin{enumerate}[noitemsep, label=\textbf{(\alph*)} ]
    \item %What is the sample space for this experiment?
    \item %What is the set describing the event of drawing a pink
%       block, $P$?
    \item %Write down a set, $O$ or $B$, to represent the event of
%       drawing either a orange or a purple block.
    \item %Draw a Venn diagram to show the above information.
    \end{enumerate}
  \end{enumerate}

% \end{eocexercises}


\section {11. EM radiation}
\subsection{Exercise 11-1} %p. 342-344
% \begin{exercises}{}
% {
       
\begin{enumerate}[label=\textbf{\arabic*}.]
\item %Use adjacent, corresponding, co-interior and alternate angles to fill in all the angles labeled with letters in the diagram:\\

\item %Find all the unknown angles in the figure: \\

 
\item %Find the value of $x$ in the figure: \\


\item %Determine whether the pairs of lines in the following figures are parallel:
\begin{enumerate}[itemsep=10pt, label=\textbf{(\alph*)} ] 
            \item %

\item %

    \item %

    \end{enumerate}
\item %If $AB$ is parallel to $CD$ and $AB$ is parallel to $EF$, explain why $CD$ must be parallel to $EF$.\vspace{8pt}\\
\end{enumerate}
% }
% \end{exercises}
\subsection{Exercise 11-2} %p. 351-352
% \begin{exercises}{}{
 \begin{enumerate}[itemsep=5pt, label=\textbf{\arabic*}. ]
  \item %Calculate the unknown variables in each of the following figures.
\begin{enumerate}[noitemsep, label=\textbf{(\alph*)} ]
\item
\item
\item
\item
\item
\item
\item
\item
\end{enumerate}
\item %State whether the following pairs of triangles are congruent or not. Give reasons for your answers. If there is not enough information to make a descision, explain why.
\begin{enumerate}[noitemsep, label=\textbf{(\alph*)} ]

\item
\item
\item
\item
\item
\end{enumerate}

\end{enumerate}
% }
% \end{exercises}
\subsection{Exercise 11-3}
  \begin{enumerate}[itemsep=5pt, label=\textbf{\arabic*}. ]
 \item %Prove that the diagonals of the parallelogram $MNRS$ bisect one another at $P$. \\
\end{enumerate}

\subsection{End of chapter exercises} %p.374-379
% \begin{eocexercises}{}

\begin{enumerate}[itemsep=20pt, label=\textbf{\arabic*}.]
%First question
\item %Identify the types of angles shown below:\\
\begin{enumerate}[noitemsep, label=\textbf{(\alph*)} ]
\item
\item
\item
\item
\item
\item
\item
\item
\end{enumerate}

\item %Assess whether the following statements are true or false. If the
% statement is false, explain why:
   \begin{enumerate}[noitemsep, label=\textbf{(\alph*)} ]
% \setcounter{enumi}{8}
\item % A trapezium is a quadrilateral with two pairs of opposite sides that are parallel.
\item % Both diagonals of a parallelogram bisect each other.
\item % A rectangle is a parallelogram that has one corner angles equal to $90^{\circ}$.
\item % Two adjacent sides of a rhombus have different lengths.
\item % The diagonals of a kite intersect at right angles.
\item %All squares are parallelograms.
\item %A rhombus is a kite with a pair of equal, opposite sides.
\item %The diagonals of a parallelogram are axes of symmetry.
\item %The diagonals of a rhombus are equal in length.
\item %Both diagonals of a kite bisect the interior angles.
\end{enumerate}
%Third question
\item %Calculate the size of the third angle ($x$) in each of the diagrams below:\\
\begin{enumerate}[noitemsep, label=\textbf{(\alph*)} ]
\item
\item
\item
\item
\item
\item
\end{enumerate}

%Question 4
\item %Find all the pairs of parallel lines in the following figures, giving reasons in each case.
\begin{enumerate}[noitemsep, label=\textbf{(\alph*)} ]
\item
\item
\item
\end{enumerate}
% Question 5

\item %Find angles $a$, $b$, $c$ and $d$ in each case, giving reasons:\\
\begin{enumerate}[noitemsep, label=\textbf{(\alph*)} ]
\item
\item
\item
\end{enumerate}

%Question 6
\item %Say which of the following pairs of triangles are congruent with reasons. 

  \begin{enumerate}[itemsep=6pt, label=\textbf{(\alph*)} ]
% \setcounter{enumi}{30}
\item %

\item %

\item %
\
\item %

\end{enumerate}

%Question 7 
\item %Using Pythagoras' theorem for right-angled triangles, calculate the length $x$:
   \begin{enumerate}[itemsep=8pt, label=\textbf{(\alph*)} ]
% \setcounter{enumi}{34}
\item %

\item %

\item %

\item %

\end{enumerate}
%Question 8
\item %Consider the diagram below. Is $\triangle ABC ||| \triangle DEF$? Give reasons for your answer. \\



%Question 9
\item %Explain why $\triangle PQR$ is similar to $\triangle TRS$ and calculate the values of $x$ and $y$.\\


\item %Calculate $a$ and $b$:\\


\item

% $ABCD$ is a parallelogram with diagonal $AC$.\\
% Given that $AF=HC$, show that:
   \begin{enumerate}[noitemsep, label=\textbf{(\alph*)} ]
 \item %$\triangle AFD \equiv \triangle CHB$
\item %$DF\parallel HB$
\item %$DFBH$ is a parallelogram
\end{enumerate}



\item % $\triangle PQR$ and $\triangle PSR$ are equilateral triangles. Prove that $PQRS$ is a rhombus:\\

\item %Given parallelogram $ABCD$ with $AE$ and $FC$, $AE$ bisects $\hat{A}$ and $FC$ bisects $\hat{C}$.
   \begin{enumerate}[noitemsep, label=\textbf{(\alph*)} ]
 \item %Write all interior angles in terms of $y$.
\item %Prove that $AFCE$ is a parallelogram.
\end{enumerate}

\item %Given that $WZ=ZY=YX$ and $WZ \parallel ZY$, prove that:
   \begin{enumerate}[noitemsep, label=\textbf{(\alph*)} ]
\item %$XZ$ bisects $\hat{X}$
\item %$WY=YZ$
% \begin{figure}[H]

\end{enumerate}

\item %$LMAO$ is a quadrilateral with $LM=LO$ and diagonals that intersect at $S$ such that $MS=SO$. Prove that:
   \begin{enumerate}[noitemsep, label=\textbf{(\alph*)} ]
% \setcounter{enumi}{49}
 \item %$M\hat{L}S = S\hat{L}O$
\item %$\triangle LOA \equiv \triangle LMA$
\item %$MO \perp LA$
% \begin{figure}[H]

% \end{figure}
\end{enumerate}


%Challenge
\item %\textbf{Challenge problem:} Using the figure below, show that the sum of the three angles in a triangle is 180$^{\circ }$. Line $DE$ is parallel to $BC$.\\

\end{enumerate}

% \end{eocexercises}
\section {12. The particles that substances are made of}
\subsection{End of chapter exercises} %p.426-427
% \begin{eocexercises}{}

\begin{enumerate}[itemsep=6pt, label=\textbf{\arabic*}. ] 
\item %Consider the solids below and answer the questions that follow (correct to one decimal place, if necessary):

    \begin{enumerate}[noitemsep, label=\textbf{(\alph*)} ]
  \item %Calculate the surface area of each solid.
\item %Calculate volume of each solid.
\item %If each dimension of the solid is increased by a factor of $3$, calculate the new surface area of each solid.
\item %If each dimension of the solid is increased by a factor of $3$, calculate the new volume of each solid.
 \end{enumerate}
\item %Consider the solids below:

    \begin{enumerate}[noitemsep, label=\textbf{(\alph*)} ]
% \setcounter{enumi}{4}
 \item %Calculate the surface area of each solid.
\item %Calculate the volume of each solid.

\end{enumerate}

% \setcounter{enumi}{6}
\item %Calculate the volume and surface area of the solid below (correct to $1$ decimal place):
\end{enumerate}

% }
% \end{eocexercises}
\section{13. Physical and chemical change}
\subsection{Exercise 1-1}
\subsection{Exercise 1-2}
\subsection{End of chapter exercises}
\section{14. Representing chemical change}
\subsection{Exercise 1-1}
\subsection{Exercise 1-2}
\subsection{End of chapter exercises}
\section{15. Magnetism}
\subsection{End of chapter exercises}
\section{16. Electrostatics}
\subsection{End of chapter exercises}
\section{17. Electric circuits}
\subsection{Exercise 1-1}
\subsection{End of chapter exercises}
\section{18. Reactions in aqueous solutions}
\subsection{Exercise 1-1}
\subsection{Exercise 1-2}
\subsection{End of chapter exercises}
\section{19. Quantitative aspects of chemical change}
\subsection{Exercise 1-1}
\subsection{Exercise 1-2}
\subsection{Exercise 1-3}
\subsection{Exercise 1-4}
\subsection{Exercise 1-5}
\subsection{Exercise 1-6}
\subsection{Exercise 1-7}
\subsection{End of chapter exercises}
\section{20. Vectors}
\subsection{Exercise 1-1}
\subsection{Exercise 1-2}
\subsection{Exercise 1-3}
\subsection{Exercise 1-4}
\subsection{Exercise 1-5}
\subsection{Exercise 1-6}
\subsection{End of chapter exercises}
\section{21. Motion in one dimension}
\subsection{Exercise 1-1}
\subsection{Exercise 1-2}
\subsection{Exercise 1-3}
\subsection{Exercise 1-4}
\subsection{Exercise 1-5}
\subsection{Exercise 1-6}
\subsection{Exercise 1-7}
\subsection{End of chapter exercises}
\section{22. Mechanical energy}
\subsection{Exercise 1-1}
\subsection{Exercise 1-2}
\subsection{Exercise 1-3}
\subsection{End of chapter exercises}
\section{22. The hydrosphere}
\subsection{End of chapter exercises}