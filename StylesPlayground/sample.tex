         \chapter{Electric circuits}
    \setcounter{figure}{1}
    \setcounter{subfigure}{1}
    \label{f13bac5321b85aca0e213ebdf4f72465}
         \section{ Introduction and key concepts}
    \nopagebreak
            \label{m38771} $ \hspace{-5pt}\begin{array}{cccccccccccc}   \includegraphics[width=0.75cm]{col11305.imgs/summary_fullmarks.png} &   \includegraphics[width=0.75cm]{col11305.imgs/summary_simulation.png} &   \end{array} $ \hspace{2 pt}\raisebox{-5 pt}{} {(section shortcode: P10074 )} \par 
    \label{m38771*cid2}
            \subsection{ Electric Circuits}
            \nopagebreak
      \label{m38771*id62184}People all over the world depend on electricity to provide power for most appliances in the home and at work. For example, fluorescent lights, electric heating and cooking (on electric stoves), all depend on electricity to work.
To realise just how big an impact electricity has on our daily lives, just think about what happens when there is a power
failure or load shedding.\par 
\label{m38771*secfhsst!!!underscore!!!id72}
            \subsubsection{  Discussion : Uses of electricity }
            \nopagebreak
      \label{m38771*id62198}With a partner, take the following topics and, for each topic, write down at least 5 items/appliances/machines which need
electricity to work. Try not to use the same item more than once.\par 
      \label{m38771*id62544}\begin{itemize}[noitemsep]
            \label{m38771*uid1}\item At home
\label{m38771*uid2}\item At school
\label{m38771*uid3}\item At the hospital
\label{m38771*uid4}\item In the city
\end{itemize}
      \label{m38771*id62592}Once you have finished making your lists, compare with the lists of other people in your class. (Save your lists somewhere safe for later because there will be another activity for which you'll need them.)\par 
      \label{m38771*id62597}When you start comparing, you should notice that there are many different items which we use in our daily lives which rely on electricity to work!
 \par 
\label{m38771*notfhsst!!!underscore!!!id89}
\begin{tabular}{cc}
	   \hspace*{-50pt}\raisebox{-8 mm}{ \includegraphics[width=0.5in]{col11305.imgs/pstip2.png}  }& 
	\begin{minipage}{0.85\textwidth}
	\begin{note}
      {tip: }\textbf{Safety Warning:}
We believe in experimenting and learning about physics at every opportunity, BUT
playing with electricity and electrical appliances can be \textbf{EXTREMELY DANGEROUS}! Do not try to build
homemade circuits alone. Make sure you have someone with you who knows if what you are doing is safe.
Normal electrical outlets are dangerous. Treat electricity with respect in your everyday life. Do not touch exposed wires and do not approach downed power lines.
	\end{note}
	\end{minipage}
	\end{tabular}
	\par
      \label{m38771*uid5}
            \subsubsection{ Closed circuits}
            \nopagebreak
        \label{m38771*id62637}In the following activity we will investigate what is needed to cause charge to flow in an electric circuit.\par 
\label{m38771*secfhsst!!!underscore!!!id97}
            \subsubsection{ Experiment : Closed circuits }
            \nopagebreak
            \label{m38771*id62648}\noindent{}\textbf{Aim:}
To determine what is required to make electrical charges flow.
In this experiment, we will use a lightbulb to check whether electrical charge is flowing in the circuit or not. If charge is flowing, the lightbulb should glow. On the other hand, if no charge is flowing, the lightbulb will not glow.\par 
        \label{m38771*id62665}\noindent{}\textbf{Apparatus:}
        You will need a small lightbulb which is attached to a metal conductor (e.g. a bulb from a school electrical kit), some connecting wires and a battery.\par 
        \label{m38771*id62679}\noindent{}\textbf{Method:}
        Take the apparatus items and try to connect them in a way that you cause the light bulb to glow (i.e. charge flows in the circuit).\par 
        \label{m38771*id62694}\noindent{}\textbf{Questions:}
        \label{m38771*id62702}\begin{enumerate}[noitemsep, label=\textbf{\arabic*}. ] 
            \label{m38771*uid6}\item Once you have arranged your circuit elements to make the lightbulb glow, draw your circuit.
\label{m38771*uid7}\item What can you say about how the battery is connected? (i.e. does it have one or two connecting leads attached? Where are they attached?)
\label{m38771*uid8}\item What can you say about how the light bulb is connected in your circuit? (i.e. does it connect to one or two connecting leads, and where are they attached?)
\label{m38771*uid9}\item Are there any items in your circuit which are not attached to something? In other words, are there any gaps in your circuit?
\end{enumerate}
        \par 
        \label{m38771*id62757}Write down your conclusion about what is needed to make an electric circuit work and charge to flow.
 \par 
        \label{m38771*id62768}In the experiment above, you will have seen that the light bulb only glows when there is a \textsl{closed} circuit i.e. there are no gaps in the circuit and all the circuit elements are connected in a \textsl{closed loop}. Therefore, in order for charges to flow, a closed circuit and an energy source (in this case the battery) are needed. (Note: you do not have to have a lightbulb in the circuit! We used this as a check that charge was flowing.)\par 
\label{m38771*fhsst!!!underscore!!!id128}\begin{definition}
	  \begin{tabular*}{15 cm}{m{15 mm}m{}}
	\hspace*{-50pt}  \includegraphics[width=0.5in]{col11305.imgs/psflag2.png}   & \Definition{   \label{id2477990}\textbf{ Electric circuit }} { \label{m38771*meaningfhsst!!!underscore!!!id128}
        \label{m38771*id62792}An electric circuit is a closed path (with no breaks or gaps) along which electrical charges (electrons) flow powered by an energy source. \par 
         } 
      \end{tabular*}
      \end{definition}
      \label{m38771*uid10}
            \subsubsection{ Representing electric circuits}
            \nopagebreak
        \label{m38771*uid11}
            \subsubsection{ Components of electrical circuits}
            \nopagebreak
          \label{m38771*id62821}Some common elements (components) which can be found in electrical circuits include light bulbs, batteries, connecting leads, switches, resistors, voltmeters and ammeters. You will learn more about these items in later sections, but it is important to know what their symbols are and how to represent them in circuit diagrams. Below is a table with the items and their symbols:\par 
    % \textbf{m38516*uid12}\par
          \begin{table}[H]
    % \begin{table}[H]
    % \\ '' '0'
        \begin{center}
      \label{m38773*id67892}
    \noindent
    \tabletail{%
        \hline
        \multicolumn{3}{|p{\mytableboxwidth}|}{\raggedleft \small \sl continued on next page}\\
        \hline
      }
      \tablelasttail{}
      \begin{xtabular}[t]{|l|l|l|}\hline
                  \textbf{Instrument}
                 &
                  \textbf{Measured Quantity}
                 &
                  \textbf{Proper Connection}
                % make-rowspan-placeholders
     \tabularnewline\cline{1-1}\cline{2-2}\cline{3-3}
      %--------------------------------------------------------------------
        Voltmeter &
        Voltage &
        In Parallel% make-rowspan-placeholders
     \tabularnewline\cline{1-1}\cline{2-2}\cline{3-3}
      %--------------------------------------------------------------------
        Ammeter &
        Current &
        In Series% make-rowspan-placeholders
     \tabularnewline\cline{1-1}\cline{2-2}\cline{3-3}
      %--------------------------------------------------------------------
    \end{xtabular}
      \end{center}
    \begin{center}{\small\bfseries Table 16.2}\end{center}
    \begin{caption}{\small\bfseries Table 16.2}\end{caption}
\end{table}
    \par
  \label{m38773**end}
         \section{ Resistance}
    \nopagebreak
            \label{m38776} $ \hspace{-5pt}\begin{array}{cccccccccccc}   \includegraphics[width=0.75cm]{col11305.imgs/summary_fullmarks.png} &   \includegraphics[width=0.75cm]{col11305.imgs/summary_video.png} &   \end{array} $ \hspace{2 pt}\raisebox{-5 pt}{} {(section shortcode: P10077 )} \par 
    \label{m38776*cid5}
            \subsection{ Resistance}
            \nopagebreak
      \label{m38776*sb1235}
	The resistance of a circuit element can be thought of as how much it opposes the flow of electric current in the circuit. 
	\vspace{\rubberspace}\par
        \label{m38776*fhsst!!!underscore!!!id1729}\begin{definition}
	  \begin{tabular*}{15 cm}{m{15 mm}m{}}
	\hspace*{-50pt}  \includegraphics[width=0.5in]{col11305.imgs/psflag2.png}   & \Definition{   \label{id2485077}\textbf{ Resistance }} { \label{m38776*meaningfhsst!!!underscore!!!id1729}
        \label{m38776*id67260} The resistance of a conductor is defined as the potential difference across it divided by the current flowing though it. We use the symbol \textbf{R} to show resistance and it is measured in units called \textbf{Ohms} with the symbol $\mathrm{\Omega }$.\par 
        \label{m38776*id67288}\nopagebreak\noindent{}
          
    \begin{equation}
    1\phantom{\rule{4pt}{0ex}}\mathrm{Ohm}=1\frac{\mathrm{Volt}}{\mathrm{Ampere}}.\tag{16.30}
      \end{equation}
         } 
      \end{tabular*}
      \end{definition}
	\par 
      \label{m38776*uid62}
            \subsubsection{ What causes resistance?}
            \nopagebreak
        \label{m38776*id67246}We have spoken about resistors that reduce the flow of charge
in a conductor. On a microscopic level, electrons moving through
the conductor collide with the particles of which the conductor
(metal) is made. When they collide, they transfer kinetic energy.
The electrons lose kinetic energy and slow down. This leads to
resistance. The transferred energy causes the resistor to heat up.
You can feel this directly if you touch a cellphone charger when you are charging a cell phone - the charger gets warm because its circuits have some resistors in them!\par 
        \label{m38776*id67330}\textsl{All} conductors have some resistance. For example, a piece of wire
has less resistance than a light bulb, but both have resistance. A lightbulb is a very thin wire surrounded by a glass housing The high resistance of the filament (small wire) in a lightbulb causes the electrons to
transfer a lot of their kinetic energy in the form of heat\label{m38776*uid63}\footnote{Flourescent lightbulbs do not use thin wires; they use the fact that certain gases glow when a current flows through them. They are much more efficient (much less resistance) than lightbulbs.}. The heat energy is enough
to cause the filament to glow white-hot which produces light. The wires
connecting the lamp to the cell or battery hardly even get warm while
conducting the same amount of current. This is because of their
much lower resistance due to their larger cross-section (they are thicker).\par 
        \label{m38776*id67354}An important effect of a resistor is that it \textsl{converts} electrical
energy into \textbf{heat} energy. \textbf{Light} is a by-product of the heat that is produced.\par 
\label{m38776*notfhsst!!!underscore!!!id1758}
\begin{tabular}{cc}
	\hspace*{-50pt}\raisebox{-8 mm}{\hspace{-0.2in}\includegraphics[width=0.75in]{col11305.imgs/psfact2.png} } & 
	\begin{minipage}{0.85\textwidth}
	\begin{note}
      {note: } There is a special type of conductor,
called a \textbf{superconductor} that has no resistance, but the
materials that make up all known superconductors only start superconducting
at very low temperatures (approximately -170${}^{\circ }$C).
	\end{note}
	\end{minipage}
	\end{tabular}
	\par
 \label{m38776*uid64}
            \subsubsection{ Why do batteries go flat?}
            \nopagebreak
          \label{m38776*id67413}A battery stores chemical potential energy. When it is connected in a circuit, a chemical reaction takes place inside the battery which converts chemical potential energy to electrical energy which powers the charges (electrons) to move through the circuit. All the circuit elements (such as the conducting leads, resistors and lightbulbs) have some resistance to the flow of charge and convert the electrical energy to heat and, in the case of the lightbulb, heat and light.
Since energy is always conserved, the battery goes flat when all its chemical potential energy has been converted into other forms of energy.\par 
      \label{m38776*uid65}
            \subsubsection{ Resistors in electric circuits}
            \nopagebreak
        \label{m38776*id67431}It is important to understand what effect adding resistors to a circuit has on the \textsl{total} resistance of a circuit and on the current that can flow in the circuit.\par 
        \label{m38776*uid66}
            \subsubsection{ Resistors in series}
            \nopagebreak
          \label{m38776*id67450}When we add resistors in series to a circuit, we \textsl{increase} the resistance to the flow of current. There is only \textbf{one path} along which the current can flow and the current is the same at all places in the series circuit. Take a look at the diagram below: On the left there is a circuit with a single resistor and a battery. No matter where we measure the current, it is the same in a series circuit. On the right, we have added a second resistor in series to the circuit. The \textsl{total} resistance of the circuit has \textsl{increased} and you can see from the reading on the ammeter that the current in the circuit has \textsl{decreased} and is still the same everywhere in the circuit.\par 
          \label{m38776*id67481}
    \setcounter{subfigure}{0}
	\begin{figure}[H] % horizontal\label{m38776*id67485}
    \begin{center}
    \label{m38776*id67485!!!underscore!!!media}\label{m38776*id67485!!!underscore!!!printimage}\includegraphics{col11305.imgs/m38776_PG10C9_032.png} % m38776;PG10C9\_032.png;;;6.0;8.5;
      \vspace{2pt}
    \vspace{.1in}
    \end{center}
 \end{figure}       
          \par 
\label{m38776*id64070}\noindent{}\textbf{Potential difference and resistors in series}When resistors are in series, one after the other, there is a potential difference across each resistor. The total potential difference across a set of resistors in series is the sum of the potential differences across each of the resistors in the set. This is the same as falling a large distance under gravity or falling that same distance (difference) in many smaller steps. The total distance (difference) is the same.\par 
        \label{m38776*id64076}Look at the circuits below. If we measured the potential difference between the black dots in all of these circuits it would be the same; it is just the potential difference across the battery which is the same as the potential difference across the rest of the circuit. So we now know the total potential difference is the same across one, two or three resistors. We also know that some work is required to make charge flow through each one. Each is a step down in potential energy. These steps add up to the total voltage drop which we know is the difference between the two dots.
The sum of the potential differences across each individual resistor is equal to the potential difference measured across all of them together. For this reason, series circuits are sometimes called \textbf{voltage dividers}.\par 
        \label{m38776*id64084}
    \setcounter{subfigure}{0}
	\begin{figure}[H] % horizontal\label{m38776*id64087}
    \begin{center}
    \label{m38776*id64087!!!underscore!!!media}\label{m38776*id64087!!!underscore!!!printimage}\includegraphics[width=\columnwidth]{col11305.imgs/m38776_PG10C9_018.png} % m38776;PG10C9\_018.png;;;6.0;8.5;
      \vspace{2pt}
    \vspace{.1in}
    \end{center}
 \end{figure}       
        \par 
        \label{m38776*id64094}Let us look at this in a bit more detail. In the picture below you can see what the different measurements for 3 identical resistors in series could look like. The total voltage across all three resistors is the sum of the voltages across the individual resistors. \par 
        \label{m38776*id64099}
    \setcounter{subfigure}{0}
	\begin{figure}[H] % horizontal\label{m38776*id64102}
    \begin{center}
    \label{m38776*id64102!!!underscore!!!media}\label{m38776*id64102!!!underscore!!!printimage}\includegraphics{col11305.imgs/m38776_PG10C9_019.png} % m38776;PG10C9\_019.png;;;6.0;8.5;
      \vspace{2pt}
    \vspace{.1in}
    \end{center}
 \end{figure}       
        \par 
\label{m38776*uid25342}
            \subsubsection{ Equivalent Series Resistance}
            \nopagebreak
            \label{m38776*id63926}When there is more than one resistor in a circuit, we are usually able to calculate the total combined resitance of all the resistors. The resistance of the single resistor is known as \textsl{equivalent resistance} or total resistance.
Consider a circuit consisting of three resistors and a single cell connected in series.\par 
          \label{m38776*id63930}
    \setcounter{subfigure}{0}
	\begin{figure}[H] % horizontal\label{m38776*id63934}
    \begin{center}
    \label{m38776*id63934!!!underscore!!!media}\label{m38776*id63934!!!underscore!!!printimage}\includegraphics[width=0.4\columnwidth]{col11305.imgs/m38776_PG11C9_007.png} % m38776;PG11C9\_007.png;;;6.0;8.5;
      \vspace{2pt}
    \vspace{.1in}
    \end{center}
 \end{figure}       
          \par 
\label{m38776*eip-546}We can define the total resistance in a series circuit as:\par \label{m38776*fhsst!!!underscore!!!id788}\begin{definition}
	  \begin{tabular*}{15 cm}{m{15 mm}m{}}
	\hspace*{-50pt}  \includegraphics[width=0.5in]{col11305.imgs/psflag2.png}   & \Definition{   \label{id2485652}\textbf{ Equivalent resistance in a series circuit, ${R}_{s}$ }} { \label{m38776*meaningfhsst!!!underscore!!!id788}
          \label{m38776*id64628}For $n$ resistors in series the equivalent resistance is:\par 
          \label{m38776*uid2532}\nopagebreak\noindent{}
            
    \begin{equation}
    {R}_{s}={R}_{1}+{R}_{2}+{R}_{3}+\cdots +{R}_{n}\tag{16.31}
      \end{equation}
           } 
      \end{tabular*}
      \end{definition}
          \label{m38776*id64719}The more resistors we add in series, the higher the equivalent resistance in the circuit. Since the resistors act as obstacles to the flow of charge through the circuit, the current in the circuit is reduced. Therefore, the \textsl{higher} the resistance in the circuit, the \textsl{lower} the current through the battery and the circuit. We say that the current in the battery is inversely proportional to the resistance in the circuit. 
Let us apply the rule of equivalent resistance in a series circuit to the following circuit.\par 
          \label{m38776*id64722}
    \setcounter{subfigure}{0}
	\begin{figure}[H] % horizontal\label{m38776*id64726}
    \begin{center}
    \label{m38776*id64726!!!underscore!!!media}\label{m38776*id64726!!!underscore!!!printimage}\includegraphics[width=0.4\columnwidth]{col11305.imgs/m38776_PG11C9_008.png} % m38776;PG11C9\_008.png;;;6.0;8.5;
      \vspace{2pt}
    \vspace{.1in}
    \end{center}
 \end{figure}       
          \par 
          \label{m38776*id64732}The resistors are in series, therefore:\par 
          \label{m38776*id64736}\nopagebreak\noindent{}
    \begin{equation}
    \begin{array}{ccc}\hfill {R}_{s}& =& {R}_{1}+{R}_{2}+{R}_{3}\hfill \\ & =& 3\phantom{\rule{0.166667em}{0ex}}\mathrm{\Omega }+10\phantom{\rule{0.166667em}{0ex}}\mathrm{\Omega }+5\phantom{\rule{0.166667em}{0ex}}\mathrm{\Omega }\hfill \\ & =& 18\phantom{\rule{0.166667em}{0ex}}\mathrm{\Omega }\hfill \end{array}\tag{16.32}
      \end{equation}
\label{m38776*eip-186}
            \subsubsection{ Experiment : Current in Series Circuits}
            \nopagebreak
            \label{m38776*id66871}\noindent{}\textbf{Aim:}
          To determine the effect of multiple resistors on current in a circuit\par 
        \label{m38776*id66886}\noindent{}\textbf{Apparatus:}
        \label{m38776*id66895}\begin{itemize}[noitemsep]
            \label{m38776*uid49}\item Battery
\label{m38776*uid50}\item Resistors
\label{m38776*uid51}\item Light bulb
\label{m38776*uid52}\item Wires
\end{itemize}
        \par 
        \label{m38776*id66948}\noindent{}\textbf{Method:}
        \label{m38776*id66957}\begin{enumerate}[noitemsep, label=\textbf{\arabic*}. ] 
            \label{m38776*uid53}\item Construct the following circuits
    \setcounter{subfigure}{0}
	\begin{figure}[H] % horizontal\label{m38776*id66976}
    \begin{center}
    \label{m38776*id66976!!!underscore!!!media}\label{m38776*id66976!!!underscore!!!printimage}\includegraphics[width=\columnwidth]{col11305.imgs/m38776_PG10C9_027.png} % m38776;PG10C9\_027.png;;;6.0;8.5;
      \vspace{2pt}
    \vspace{.1in}
    \end{center}
 \end{figure}       \label{m38776*uid54}\item Rank the three circuits in terms of the brightness of the bulb.
\end{enumerate}
        \par 
        \label{m38776*id66996}\noindent{}\textbf{Conclusions:}
        The brightness of the bulb is an indicator of how much current is flowing. If the bulb gets brighter because of a change then more current is flowing. If the bulb gets dimmer less current is flowing.
You will find that the more resistors you have the dimmer the bulb.
 \par 
\label{m38776*secfhsst!!!underscore!!!id903}\vspace{.5cm} 
      \noindent
      \hspace*{-30pt}\includegraphics[width=0.5in]{col11305.imgs/pspencil2.png}   \raisebox{25mm}{   
      \begin{mdframed}[linewidth=4, leftmargin=40, rightmargin=40]  
      \begin{exercise}
    \noindent\textbf{Exercise 16.11:  Equivalent series resistance I }
          \label{m38776*probfhsst!!!underscore!!!id904}
          \label{m38776*id64862}Two 10 k$\Omega $ resistors are connected in series. Calculate the equivalent resistance. \par 
          \vspace{5pt}
          \label{m38776*solfhsst!!!underscore!!!id907}\noindent\textbf{Solution to Exercise } \label{m38776*listfhsst!!!underscore!!!id907}\begin{enumerate}[noitemsep, label=\textbf{Step} \textbf{\arabic*}. ] 
            \leftskip=20pt\rightskip=\leftskip\item  
          \label{m38776*id64896}Since the resistors are in series we can use:\par 
          \label{m38776*id64899}\nopagebreak\noindent{}
            
    \begin{equation}
    {R}_{s}={R}_{1}+{R}_{2}\tag{16.33}
      \end{equation}
          \item  
          \label{m38776*id64940}\nopagebreak\noindent{}
            
    \begin{equation}
    \begin{array}{ccc}\hfill {R}_{s}& =& {R}_{1}+{R}_{2}\hfill \\ & =& 10\phantom{\rule{0.166667em}{0ex}}\mathrm{k}\phantom{\rule{0.166667em}{0ex}}\Omega +10\phantom{\rule{0.166667em}{0ex}}\mathrm{k}\phantom{\rule{0.166667em}{0ex}}\Omega \hfill \\ & =& 20\phantom{\rule{0.166667em}{0ex}}\mathrm{k}\phantom{\rule{0.166667em}{0ex}}\Omega \hfill \end{array}\tag{16.34}
      \end{equation}
          \item  
          \label{m38776*id65061}The equivalent resistance of two 10 k$\Omega $ resistors connected in series is 20 k$\Omega $. \par 
          \end{enumerate}
    \end{exercise}
    \end{mdframed}
    }
    \noindent
\label{m38776*secfhsst!!!underscore!!!id1001}\vspace{.5cm} 
      \noindent
      \hspace*{-30pt}\includegraphics[width=0.5in]{col11305.imgs/pspencil2.png}   \raisebox{25mm}{   
      \begin{mdframed}[linewidth=4, leftmargin=40, rightmargin=40]  
      \begin{exercise}
    \noindent\textbf{Exercise 16.12:  Equivalent series resistance II }
          \label{m38776*probfhsst!!!underscore!!!id1002}
          \label{m38776*id65108}Two resistors are connected in series. The equivalent resistance is 100 $\Omega $. If one resistor is 10 $\Omega $, calculate the value of the second resistor. \par 
          \vspace{5pt}
          \label{m38776*solfhsst!!!underscore!!!id1005}\noindent\textbf{Solution to Exercise } \label{m38776*listfhsst!!!underscore!!!id1005}\begin{enumerate}[noitemsep, label=\textbf{Step} \textbf{\arabic*}. ] 
            \leftskip=20pt\rightskip=\leftskip\item  
          \label{m38776*id65152}Since the resistors are in series we can use:\par 
          \label{m38776*id65155}\nopagebreak\noindent{}
            
    \begin{equation}
    {R}_{s}={R}_{1}+{R}_{2}\tag{16.35}
      \end{equation}
          \label{m38776*id65192}We are given the value of ${R}_{s}$ and ${R}_{1}$.\par 
          \item  
          \label{m38776*id65230}\nopagebreak\noindent{}
            
    \begin{equation}
    \begin{array}{ccc}\hfill {R}_{s}& =& {R}_{1}+{R}_{2}\hfill \\ \hfill \therefore {R}_{2}& =& {R}_{s}-{R}_{1}\hfill \\ & =& 100\phantom{\rule{0.166667em}{0ex}}\Omega -10\phantom{\rule{0.166667em}{0ex}}\Omega \hfill \\ & =& 90\phantom{\rule{0.166667em}{0ex}}\Omega \hfill \end{array}\tag{16.36}
      \end{equation}
          \item  
          \label{m38776*id65371}The second resistor has a resistance of 90 $\Omega $. \par 
          \end{enumerate}
    \end{exercise}
    \end{mdframed}
    }
    \noindent
    \setcounter{subfigure}{0}
	\begin{figure}[H] % horizontal\label{m38776*circuits-2}
    \textnormal{Khan academy video on circuits - 2}\vspace{.1in} \nopagebreak
  \label{m38776*yt-media2}\label{m38776*yt-video2}
            \raisebox{-5 pt}{ \includegraphics[width=0.5cm]{col11305.imgs/summary_www.png}} { (Video:  P10078 )}
      \vspace{2pt}
    \vspace{.1in}
 \end{figure}       
        \label{m38776*uid67}
            \subsubsection{ Resistors in parallel}
            \nopagebreak
          \label{m38776*id67501}In contrast to the series case, when we add resistors in parallel, we create \textbf{more paths} along which current can flow. By doing this we \textsl{decrease} the total resistance of the circuit!\par 
          \label{m38776*id67518}Take a look at the diagram below. On the left we have the same circuit as shown on the left in Figure~16.25 with a battery and a resistor. The ammeter shows a current of 1 ampere. On the right we have added a second resistor in parallel to the first resistor. This has increased the number of paths (branches) the charge can take through the circuit - the total resistance has decreased. You can see that the current in the circuit has increased. Also notice that the current in the different branches can be different (in this case 1 A and 2 A) but must add up to the current through the battery (3 A). Since the total current in the circuit is equal to the sum of the currents in the parallel branches, a parallel circuit is sometimes called a \textbf{current divider}.\par 
          \label{m38776*id67525}
    \setcounter{subfigure}{0}
	\begin{figure}[H] % horizontal\label{m38776*id67528}
    \begin{center}
    \label{m38776*id67528!!!underscore!!!media}\label{m38776*id67528!!!underscore!!!printimage}\includegraphics{col11305.imgs/m38776_PG10C9_033.png} % m38776;PG10C9\_033.png;;;6.0;8.5;
      \vspace{2pt}
    \vspace{.1in}
    \end{center}
 \end{figure}       
          \par 
        \label{m38776*id64009}\noindent{}\textbf{Potential difference and parallel resistors}When resistors are connected in parallel the start and end points for all the resistors are the same. These points have the same potential energy and so the potential difference between them is the same no matter what is put in between them. You can have one, two or many resistors between the two points, the potential difference will not change. You can ignore whatever components are between two points in a circuit when calculating the difference between the two points.\par 
        \label{m38776*id64017}Look at the following circuit diagrams. The battery is the same in all cases. All that changes is that more resistors are added between the points marked by the black dots. If we were to measure the potential difference between the two dots in these circuits we would get the same answer for all three cases.\par 
        \label{m38776*id64023}
    \setcounter{subfigure}{0}
	\begin{figure}[H] % horizontal\label{m38776*id64026}
    \begin{center}
    \label{m38776*id64026!!!underscore!!!media}\label{m38776*id64026!!!underscore!!!printimage}\includegraphics[width=\columnwidth]{col11305.imgs/m38776_PG10C9_016.png} % m38776;PG10C9\_016.png;;;6.0;8.5;
      \vspace{2pt}
    \vspace{.1in}
    \end{center}
 \end{figure}       
        \par 
        \label{m38776*id64033}Let's look at two resistors in parallel more closely. When you construct a circuit you use wires and you might think that measuring the voltage in different places on the wires will make a difference. This is not true. The potential difference or voltage measurement will only be different if you measure a different set of components. All points on the wires that have no circuit components between them will give you the same measurements.\par 
        \label{m38776*id64040}All three of the measurements shown in the picture below (i.e. A--B, C--D and E--F) will give you the same voltage. The different measurement points on the left (i.e. A, E, C) have no components between them so there is no change in potential energy.
Exactly the same applies to the different points on the right (i.e. B, F, D). When you measure the potential difference between the points on the left and right you will get the same answer.\par 
        \label{m38776*id64049}
    \setcounter{subfigure}{0}
	\begin{figure}[H] % horizontal\label{m38776*id64052}
    \begin{center}
    \label{m38776*id64052!!!underscore!!!media}\label{m38776*id64052!!!underscore!!!printimage}\includegraphics{col11305.imgs/m38776_PG10C9_017.png} % m38776;PG10C9\_017.png;;;6.0;8.5;
      \vspace{2pt}
    \vspace{.1in}
    \end{center}
 \end{figure}       
        \par 
        \label{m38776*eip-684}\label{m38776*eip-id1170813612376}\begin{definition}
	  \begin{tabular*}{15 cm}{m{15 mm}m{}}
	\hspace*{-50pt}  \includegraphics[width=0.5in]{col11305.imgs/psflag2.png}   & \Definition{   \label{id2487246}\textbf{ Equivalent resistance of two parallel resistor, ${R}_{p}$ }} { \label{m38776*eip-id1170826978594}
          \label{m38776*eip-id1170816625786}For $2$ resistors in parallel with resistances ${R}_{1}$ and ${R}_{2}$, the equivalent resistance is:\par 
          \label{m38776*eip-id1170814512227}\nopagebreak\noindent{}
    \begin{equation}
    {R}_{p}=\frac{{R}_{1}{R}_{2}}{{R}_{1}+{R}_{2}}\tag{16.37}
      \end{equation}
           } 
      \end{tabular*}
      \end{definition}
\par \label{m38776*uid2446}
            \subsubsection{ Equivalent parallel resistance}
            \nopagebreak
            \label{m38776*id65406}Consider a circuit consisting of a single cell and three resistors that are connected in parallel.\par 
          \label{m38776*id65410}
    \setcounter{subfigure}{0}
	\begin{figure}[H] % horizontal\label{m38776*id65414}
    \begin{center}
    \label{m38776*id65414!!!underscore!!!media}\label{m38776*id65414!!!underscore!!!printimage}\includegraphics[width=300px]{col11305.imgs/m38776_PG11C9_009.png} % m38776;PG11C9\_009.png;;;6.0;8.5;
      \vspace{2pt}
    \vspace{.1in}
    \end{center}
 \end{figure}       
          \par 
\label{m38776*eip-754}Using what we know about voltage and current in parallel circuits we can define the equivalent resistance of several resistors in parallel as:\par \label{m38776*fhsst!!!underscore!!!id1576}\begin{definition}
	  \begin{tabular*}{15 cm}{m{15 mm}m{}}
	\hspace*{-50pt}  \includegraphics[width=0.5in]{col11305.imgs/psflag2.png}   & \Definition{   \label{id2487427}\textbf{ Equivalent resistance in a parallel circuit, ${R}_{p}$ }} { \label{m38776*meaningfhsst!!!underscore!!!id1576}
          \label{m38776*id6613206}For $n$ resistors in parallel, the equivalent resistance is:\par 
          \label{m38776*uid2944}\nopagebreak\noindent{}
            
    \begin{equation}
    \frac{1}{{R}_{p}}=\left(\frac{1}{{R}_{1}}+\frac{1}{{R}_{2}}+\frac{1}{{R}_{3}}+\cdots +\frac{1}{{R}_{n}}\right)\tag{16.38}
      \end{equation}
           } 
      \end{tabular*}
      \end{definition}
          \label{m38776*id66324}Let us apply this formula to the following circuit.\par 
          \label{m38776*id66328}
    \setcounter{subfigure}{0}
	\begin{figure}[H] % horizontal\label{m38776*id66331}
    \begin{center}
    \label{m38776*id66331!!!underscore!!!media}\label{m38776*id66331!!!underscore!!!printimage}\includegraphics[width=300px]{col11305.imgs/m38776_PG11C9_010.png} % m38776;PG11C9\_010.png;;;6.0;8.5;
      \vspace{2pt}
    \vspace{.1in}
    \end{center}
 \end{figure}       
          \par 
          \label{m38776*id663318}What is the total resistance in the circuit?\par 
          \label{m38776*id66342}\nopagebreak\noindent{}
            
    \begin{equation}
    \begin{array}{ccc}\hfill \frac{1}{{R}_{p}}& =& \left(\frac{1}{{R}_{1}}+\frac{1}{{R}_{2}}+\frac{1}{{R}_{3}}\right)\hfill \\ & =& \left(\frac{1}{10\phantom{\rule{0.166667em}{0ex}}\Omega }+\frac{1}{2\phantom{\rule{0.166667em}{0ex}}\Omega }+\frac{1}{1\phantom{\rule{0.166667em}{0ex}}\Omega }\right)\hfill \\ & =& \left(\frac{1+5+10}{10}\right)\hfill \\ & =& \left(\frac{16}{10}\right)\hfill \\ \hfill \therefore {R}_{p}& =& 0,625\phantom{\rule{0.166667em}{0ex}}\Omega \hfill \end{array}\tag{16.39}
      \end{equation}
          \label{m38776*eip-234}
            \subsubsection{ Experiment : Current in Parallel Circuits }
            \nopagebreak
            \label{m38776*id67061}\noindent{}\textbf{Aim:}
          To determine the effect of multiple resistors on current in a circuit\par 
        \label{m38776*id67076}\noindent{}\textbf{Apparatus:}
        \label{m38776*id67085}\begin{itemize}[noitemsep]
            \label{m38776*uid56}\item Battery
\label{m38776*uid57}\item Resistors
\label{m38776*uid58}\item Light bulb
\label{m38776*uid59}\item Wires
\end{itemize}
        \par 
        \label{m38776*id67138}\noindent{}\textbf{Method:}
        \label{m38776*id67147}\begin{enumerate}[noitemsep, label=\textbf{\arabic*}. ] 
            \label{m38776*uid60}\item Construct the following circuits
    \setcounter{subfigure}{0}
	\begin{figure}[H] % horizontal\label{m38776*id67166}
    \begin{center}
    \label{m38776*id67166!!!underscore!!!media}\label{m38776*id67166!!!underscore!!!printimage}\includegraphics[width=\columnwidth]{col11305.imgs/m38776_PG10C9_030.png} % m38776;PG10C9\_030.png;;;6.0;8.5;
      \vspace{2pt}
    \vspace{.1in}
    \end{center}
 \end{figure}       \label{m38776*uid61}\item Rank the three circuits in terms of the brightness of the bulb.
\end{enumerate}
        \par 
        \label{m38776*id67186}\noindent{}\textbf{Conclusions:}
        The brightness of the bulb is an indicator of how much current is flowing. If the bulb gets brighter because of a change then more current is flowing. If the bulb gets dimmer less current is flowing.
You will find that the more resistors you have the brighter the bulb.
 \par \label{m38776*eip-828}Why is this the case? Why do more resistors make it easier for charge to flow in the circuit? It is because they are in parallel so there are more paths for charge to take to move. You can think of it like a highway with more lanes, or the tube of marbles splitting into multiple parallel tubes. The more branches there are, the easier it is for charge to flow. You will learn more about the total resistance of parallel resistors later but always remember that more resistors in parallel mean more pathways. In series the pathways come one after the other so it does not make it easier for charge to flow.\par \label{m38776*eip-291}\vspace{3.5cm} 
\vspace{\rubberspace}      
\hspace*{-30pt}\includegraphics[width=0.5in]{col11305.imgs/pspencil2.png}   \raisebox{25mm}{   
      \begin{mdframed}[linewidth=4, leftmargin=40, rightmargin=40]  
      \begin{exercise}
    \noindent\textbf{Exercise 16.13}\label{m38776*probfhsst!!!underscore!!!id9084}
          \label{m38776*id648622}Two $8\phantom{\rule{2pt}{0ex}}\mathrm{k}\Omega $ resistors are connected in parallel. Calculate the equivalent resistance. \par 
          \vspace{5pt}
          \label{m38776*solfhsst!!!underscore!!!id9073}\noindent\textbf{Solution to Exercise } \label{m38776*listfhsst!!!underscore!!!id9074}\begin{enumerate}[noitemsep, label=\textbf{Step} \textbf{\arabic*}. ] 
            \leftskip=20pt\rightskip=\leftskip\item  
          \label{m38776*id648926}Since the resistors are in parallel we can use:\par 
          \label{m38776*id64849}\nopagebreak\noindent{}
            
    \begin{equation}
    \frac{1}{{R}_{p}}=\frac{1}{{R}_{1}}+\frac{1}{{R}_{2}}\tag{16.40}
      \end{equation}
          \item  
          \label{m38776*id6494034}\nopagebreak\noindent{}
            
    \begin{equation}
    \begin{array}{ccc}\hfill \frac{1}{{R}_{p}}& =& \frac{1}{{R}_{1}}+\frac{1}{{R}_{2}}\hfill \\ & =& \frac{1}{8\phantom{\rule{0.166667em}{0ex}}\mathrm{k}\phantom{\rule{0.166667em}{0ex}}\Omega }+\frac{1}{10\phantom{\rule{0.166667em}{0ex}}\mathrm{k}\phantom{\rule{0.166667em}{0ex}}\Omega }\hfill \\ \hfill {R}_{p}& =& \frac{2}{8}\hfill \\ & =& 4\phantom{\rule{0.166667em}{0ex}}\mathrm{k}\phantom{\rule{0.166667em}{0ex}}\Omega \hfill \end{array}\tag{16.41}
      \end{equation}
          \item  
          \label{m38776*id650261}The equivalent resistance of two $8\phantom{\rule{2pt}{0ex}}\mathrm{k}\Omega $ resistors connected in parallel is $4\phantom{\rule{2pt}{0ex}}\mathrm{k}\Omega $. \par 
          \end{enumerate}
    \end{exercise}
    \end{mdframed}
    }
    \noindent
  \label{m38776*eip-994}\vspace{.5cm} 
     \vspace{\rubberspace} 
     \vspace{\rubberspace}
     \hspace*{-30pt}\includegraphics[width=0.5in]{col11305.imgs/pspencil2.png}   \raisebox{25mm}{   
      \begin{mdframed}[linewidth=4, leftmargin=40, rightmargin=40]  
      \begin{exercise}
    \noindent\textbf{Exercise 16.14}\label{m38776*probfhsst!!!underscore!!!id10102}
          \label{m38776*id651098}Two resistors are connected in parallel. The equivalent resistance is $100\phantom{\rule{2pt}{0ex}}\Omega $. If one resistor is  $150\phantom{\rule{2pt}{0ex}}\Omega $, calculate the value of the second resistor. \par 
          \vspace{5pt}
          \label{m38776*solfhsst!!!underscore!!!id11005}\noindent\textbf{Solution to Exercise } \label{m38776*listfhsst!!!underscore!!!id10905}\begin{enumerate}[noitemsep, label=\textbf{Step} \textbf{\arabic*}. ] 
            \leftskip=20pt\rightskip=\leftskip\item  
          \label{m38776*id6512152}Since the resistors are in parallel we can use:\par 
          \label{m38776*id651555}\nopagebreak\noindent{}
            
    \begin{equation}
    \frac{1}{{R}_{p}}=\frac{1}{{R}_{1}}+\frac{1}{{R}_{2}}\tag{16.42}
      \end{equation}
          \label{m38776*id6519212}We are given the value of ${R}_{p}$ and ${R}_{1}$.\par 
          \item  
          \label{m38776*id6523230}\nopagebreak\noindent{}
            
    \begin{equation}
    \begin{array}{ccc}\hfill \frac{1}{{R}_{p}}& =& \frac{1}{{R}_{1}}+\frac{1}{{R}_{2}}\hfill \\ \hfill \therefore \frac{1}{{R}_{2}}& =& \frac{1}{{R}_{p}}-\frac{1}{{R}_{1}}\hfill \\ & =& \frac{1}{100\phantom{\rule{0.166667em}{0ex}}\Omega }-\frac{1}{150\phantom{\rule{0.166667em}{0ex}}\Omega }\hfill \\ & =& \frac{3-2}{300}\hfill \\ & =& \frac{1}{300}\hfill \\ \hfill {R}_{2}& =& 300\phantom{\rule{0.166667em}{0ex}}\Omega \hfill \end{array}\tag{16.43}
      \end{equation}
          \item  
          \label{m38776*id6537341}The second resistor has a resistance of $300\phantom{\rule{2pt}{0ex}}\Omega $. \par 
          \end{enumerate}
    \end{exercise}
    \end{mdframed}
    }
    \noindent
    \setcounter{subfigure}{0}
	\begin{figure}[H] % horizontal\label{m38776*circuits-3}
    \textnormal{Khan academy video on circuits - 3}\vspace{.1in} \nopagebreak
  \label{m38776*yt-media3}\label{m38776*yt-video3}
            \raisebox{-5 pt}{ \includegraphics[width=0.5cm]{col11305.imgs/summary_www.png}} { (Video:  P10079 )}
      \vspace{2pt}
    \vspace{.1in}
 \end{figure}       
\label{m38776*secfhsst!!!underscore!!!id1795}
            \subsubsection{  Resistance }
            \nopagebreak
          \label{m38776*id67542}\begin{enumerate}[noitemsep, label=\textbf{\arabic*}. ] 
            \label{m38776*uid68}\item What is the unit of resistance called and what is its symbol?         
\label{m38776*uid69}\item Explain what happens to the total resistance of a circuit when resistors are added in series?         
\label{m38776*uid70}\item Explain what happens to the total resistance of a circuit when resistors are added in parallel?         
\label{m38776*uid71}\item Why do batteries go flat?         
\end{enumerate}
   \label{m38776*sb9871}
\par \raisebox{-5 pt}{\includegraphics[width=0.5cm]{col11305.imgs/summary_www.png}} Find the answers with the shortcodes:
 \par \begin{tabular}[h]{cccccc}
 (1.) lqk  &  (2.) lq0  &  (3.) lq8  &  (4.) lq9  & \end{tabular}
            \subsection{ }
            \nopagebreak
    \setcounter{subfigure}{0}
	\begin{figure}[H] % horizontal\label{m38776*circuits-4}
    \textnormal{Khan academy video on circuits - 4}\vspace{.1in} \nopagebreak
  \label{m38776*yt-media4}\label{m38776*yt-video4}
            \raisebox{-5 pt}{ \includegraphics[width=0.5cm]{col11305.imgs/summary_www.png}} { (Video:  P10080 )}
      \vspace{2pt}
    \vspace{.1in}
 \end{figure}       
    \label{m38776*eip-872}The following presentation summarizes the concepts covered in this chapter. 
    \setcounter{subfigure}{0}
	\begin{figure}[H] % horizontal\label{m38776*slidesharefigure}
        \label{m38776*slidesharemedia}\label{m38776*slideshareflash}\raisebox{-5 pt}{ \includegraphics[width=0.5cm]{col11305.imgs/summary_www.png}} { (Presentation:  P10081 )}
      \vspace{2pt}
    \vspace{.1in}
 \end{figure}       
    \label{m38776*cid7}
            \subsection{ Exercises - Electric circuits}
            \nopagebreak
      \label{m38776*id68040}\begin{enumerate}[noitemsep, label=\textbf{\arabic*}. ] 
            \label{m38776*uid79}\item  Write definitions for each of the following:
\label{m38776*id68056}\begin{enumerate}[noitemsep, label=\textbf{\alph*}. ] 
            \label{m38776*uid80}\item resistor
\label{m38776*uid81}\item coulomb
\label{m38776*uid82}\item voltmeter
\end{enumerate}
                  \label{m38776*uid83}\item  Draw a circuit diagram which consists of the following components:
\label{m38776*id68109}\begin{enumerate}[noitemsep, label=\textbf{\alph*}. ] 
            \label{m38776*uid84}\item 2 batteries in parallel
\label{m38776*uid85}\item an open switch
\label{m38776*uid86}\item 2 resistors in parallel
\label{m38776*uid87}\item an ammeter measuring total current
\label{m38776*uid88}\item a voltmeter measuring potential difference across one of the parallel resistors
\end{enumerate}
                  \label{m38776*uid89}\item  Complete the table below:
    % \textbf{m38776*id68187}\par
          \begin{table}[H]
    % \begin{table}[H]
    % \\ 'id2959178' '1'
        \begin{center}
      \label{m38776*id68187}
    \noindent
    \tabletail{%
        \hline
        \multicolumn{4}{|p{\mytableboxwidth}|}{\raggedleft \small \sl continued on next page}\\
        \hline
      }
      \tablelasttail{}
      \begin{xtabular}[t]{|l|l|l|l|}\hline
        \textbf{Quantity} &
        \textbf{Symbol} &
        \textbf{Unit of meaurement} &
        \textbf{Symbol of unit}% make-rowspan-placeholders
     \tabularnewline\cline{1-1}\cline{2-2}\cline{3-3}\cline{4-4}
      %--------------------------------------------------------------------
        e.g. Distance &
        e.g. d &
        e.g. kilometer &
        e.g. km% make-rowspan-placeholders
     \tabularnewline\cline{1-1}\cline{2-2}\cline{3-3}\cline{4-4}
      %--------------------------------------------------------------------
        Resistance &
         &
         &
        % make-rowspan-placeholders
     \tabularnewline\cline{1-1}\cline{2-2}\cline{3-3}\cline{4-4}
      %--------------------------------------------------------------------
        Current &
         &
         &
        % make-rowspan-placeholders
     \tabularnewline\cline{1-1}\cline{2-2}\cline{3-3}\cline{4-4}
      %--------------------------------------------------------------------
        Potential difference &
         &
         &
        % make-rowspan-placeholders
     \tabularnewline\cline{1-1}\cline{2-2}\cline{3-3}\cline{4-4}
      %--------------------------------------------------------------------
    \end{xtabular}
      \end{center}
    \begin{center}{\small\bfseries Table 16.3}\end{center}
    \begin{caption}{\small\bfseries Table 16.3}\end{caption}
\end{table}
    \par
            \item Draw a diagram of a circuit which contains a battery connected to a lightbulb and a resistor all in series. \label{m38776*id6742}\begin{enumerate}[noitemsep, label=\textbf{\alph*}. ] 
            \item  Also include in the diagram where you would place an ammeter if you wanted to measure the current through the lightbulb.\item Draw where and how you would place a voltmeter in the circuit to measure the potential difference across the resistor.\end{enumerate}
                  \item Thandi wants to measure the current through the resistor in the circuit shown below and sets up the circuit as shown below. What is wrong with her circuit setup? 
    \setcounter{subfigure}{0}
	\begin{figure}[H] % horizontal\label{m38776*id6854}
    \begin{center}
    \label{m38776*id6854!!!underscore!!!media}\label{m38776*id6854!!!underscore!!!printimage}\includegraphics[width=0.4\columnwidth]{col11305.imgs/m38776_circuit1.png} % m38776;circuit1.png;;;6.0;8.5;
      \vspace{2pt}
    \vspace{.1in}
    \end{center}
 \end{figure}                
\label{m38776*uid90}\item (SC 2003/11) The emf of a battery can best be explained as the $\cdots $\label{m38776*id68404}\begin{enumerate}[noitemsep, label=\textbf{\alph*}. ] 
            \label{m38776*uid91}\item rate of energy delivered per unit current
\label{m38776*uid92}\item rate at which charge is delivered
\label{m38776*uid93}\item rate at which energy is delivered
\label{m38776*uid94}\item charge per unit of energy delivered by the battery
\end{enumerate}
                  \label{m38776*uid95}\item (IEB 2002/11 HG1) Which of the following is the correct definition of the emf of a battery?
\label{m38776*id68470}\begin{enumerate}[noitemsep, label=\textbf{\alph*}. ] 
            \label{m38776*uid96}\item It is the product of current and the external resistance of the circuit.
\label{m38776*uid97}\item It is a measure of the cell's ability to conduct an electric current.
\label{m38776*uid98}\item It is equal to the ``lost volts'' in the internal resistance of the circuit.
\label{m38776*uid99}\item It is the power supplied by the battery per unit current passing through the battery.
\end{enumerate}
                  \label{m38776*uid100}\item (IEB 2005/11 HG) Three identical light bulbs A, B and C are connected in an electric circuit as shown in the diagram below.
    \setcounter{subfigure}{0}
	\begin{figure}[H] % horizontal\label{m38776*id68542}
    \begin{center}
    \label{m38776*id68542!!!underscore!!!media}\label{m38776*id68542!!!underscore!!!printimage}\includegraphics[width=300px]{col11305.imgs/m38776_PG10C9_037.png} % m38776;PG10C9\_037.png;;;6.0;8.5;
      \vspace{2pt}
    \vspace{.1in}
    \end{center}
 \end{figure}       \label{m38776*id68549}\begin{enumerate}[noitemsep, label=\textbf{\alph*}. ] 
            \label{m38776*uid101}\item How bright is bulb A compared to B and C?
\label{m38776*uid102}\item How bright are the bulbs after switch S has been opened?
\label{m38776*uid103}\item How do the currents in bulbs A and B change when switch S is opened?
    % \textbf{m38776*id68590}\par
          \begin{table}[H]
    % \begin{table}[H]
    % \\ 'id2959512' '1'
        \begin{center}
      \label{m38776*id68590}
    \noindent
    \tabletail{%
        \hline
        \multicolumn{3}{|p{\mytableboxwidth}|}{\raggedleft \small \sl continued on next page}\\
        \hline
      }
      \tablelasttail{}
      \begin{xtabular}[t]{|l|l|l|}\hline
         &
        \textbf{Current in A} &
        \textbf{Current in B}% make-rowspan-placeholders
     \tabularnewline\cline{1-1}\cline{2-2}\cline{3-3}
      %--------------------------------------------------------------------
        (a) &
        decreases &
        increases% make-rowspan-placeholders
     \tabularnewline\cline{1-1}\cline{2-2}\cline{3-3}
      %--------------------------------------------------------------------
        (b) &
        decreases &
        decreases% make-rowspan-placeholders
     \tabularnewline\cline{1-1}\cline{2-2}\cline{3-3}
      %--------------------------------------------------------------------
        (c) &
        increases &
        increases% make-rowspan-placeholders
     \tabularnewline\cline{1-1}\cline{2-2}\cline{3-3}
      %--------------------------------------------------------------------
        (d) &
        increases &
        decreases% make-rowspan-placeholders
     \tabularnewline\cline{1-1}\cline{2-2}\cline{3-3}
      %--------------------------------------------------------------------
    \end{xtabular}
      \end{center}
    \begin{center}{\small\bfseries Table 16.4}\end{center}
    \begin{caption}{\small\bfseries Table 16.4}\end{caption}
\end{table}
    \par
  \end{enumerate}
                  \label{m38776*uid104}\item (IEB 2004/11 HG1) When a current $I$ is maintained in a conductor for a time of $t$, how many electrons with charge e pass any cross-section of the conductor per second?
\label{m38776*id68784}\begin{enumerate}[noitemsep, label=\textbf{\alph*}. ] 
            \label{m38776*uid105}\item It
\label{m38776*uid106}\item It/e
\label{m38776*uid107}\item Ite
\label{m38776*uid108}\item e/It
\end{enumerate}
                  \end{enumerate}
  \label{m38776**end}
  \label{f13bac5321b85aca0e213ebdf4f72465**end}
\par \raisebox{-5 pt}{\includegraphics[width=0.5cm]{col11305.imgs/summary_www.png}} Find the answers with the shortcodes:
 \par \begin{tabular}[h]{cccccc}
 (1.) lqX  &  (2.) lqI  &  (3.) lq5  &  (4.) lTw  &  (5.) lTv  &  (6.) lqn  &  (7.) lqR  &  (8.) lqN  &  (9.) lqQ  & \end{tabular}
         \chapter{The atom}
    \setcounter{figure}{1}
    \setcounter{subfigure}{1}
    \label{ea1c9e59656f96ee804546971cf6dee6}
         \section{ Introduction and models}
    \nopagebreak
            \label{m38756} $ \hspace{-5pt}\begin{array}{cccccccccccc}   \includegraphics[width=0.75cm]{col11305.imgs/summary_video.png} &   \end{array} $ \hspace{2 pt}\raisebox{-5 pt}{} {(section shortcode: P10019 )} \par 
    \label{m38756*cid1}
            \subsection{ Introduction}
            \nopagebreak
      \label{m38756*eip-794}The following video covers some of the properties of an atom.
    \setcounter{subfigure}{0}
	\begin{figure}[H] % horizontal\label{m38756*the-atom-1}
    \textnormal{Veritasium video on the atom - 1}\vspace{.1in} \nopagebreak
  \label{m38756*yt-media10}\label{m38756*yt-video10}
            \raisebox{-5 pt}{ \includegraphics[width=0.5cm]{col11305.imgs/summary_www.png}} { (Video:  P10020 )}
      \vspace{2pt}
    \vspace{.1in}
 \end{figure}       \par \label{m38756*id254141}We have now looked at many examples of the types of matter and materials that exist around us and we have investigated some of the ways that materials are classified. But what is it that makes up these materials? And what makes one material different from another? In order to understand this, we need to take a closer look at the building block of matter - the \textbf{atom}. Atoms are the basis of all the structures and organisms in the universe. The planets, sun, grass, trees, air we breathe and people are all made up of different combinations of atoms.\par 
    \label{m38756*eip-613}
            \subsection{ Project: Models of the atom}
            \nopagebreak
            \label{m38756*eip-3}
Our current understanding of the atom came about over a long period of time, with many different people playing a role. Conduct some research into the development of the different ideas of the atom and the people who contributed to it. Some suggested people to look at are: JJ Thomson, Ernest Rutherford, Marie Curie, JC Maxwell, Max Planck, Albert Einstein, Niels Bohr, Lucretius, LV de Broglie, CJ Davisson, LH Germer, Chadwick, Werner Heisenberg, Max Born, Erwin Schrodinger, John Dalton, Empedocles, Leucippus, Democritus, Epicurus, Zosimos, Maria the Jewess, Geber, Rhazes, Robert Boyle, Henry Cavendish, A Lavoisier and H Becquerel. You do not need to find information on all these people, but try to find information about as many of them as possible.
\par 
\label{m38756*id7342}Make a list of the key contributions to a model of the atom that each of these people made and then make a timeline of this information. (You can use an online tool such as Dipity\footnote{http://www.dipity.com/}
         to make a timeline.) Try to get a feel for how it all eventually fit together into the modern understanding of the atom. 
\par \label{m38756*cid2}
            \subsection{ Models of the Atom}
            \nopagebreak
      \label{m38756*id254164}It is important to realise that a lot of what we know about the structure of atoms has been developed over a long period of time. This is often how scientific knowledge develops, with one person building on the ideas of someone else. We are going to look at how our modern understanding of the atom has evolved over time.\par 
      \label{m38756*id254508}The idea of atoms was invented by two Greek philosophers, Democritus and Leucippus in the fifth century BC. The Greek word $\mathrm{\alpha \tau o\mu o\nu }$ \hspace{1ex} (atom) means \textbf{indivisible} because they believed that atoms could not be broken into smaller pieces.\par 
      \label{m38756*id254540}Nowadays, we know that atoms are made up of a \textbf{positively charged nucleus} in the centre
surrounded by \textbf{negatively charged electrons}. However, in the past, before the structure of the atom was properly understood, scientists came up with lots of different \textbf{models} or \textbf{pictures} to describe what atoms look like.\par 
\label{m38756*fhsst!!!underscore!!!id72}\begin{definition}
	  \begin{tabular*}{15 cm}{m{15 mm}m{}}
	\hspace*{-50pt}  \includegraphics[width=0.5in]{col11305.imgs/psflag2.png}   & \Definition{   \label{id2414493}\textbf{ Model }} { \label{m38756*meaningfhsst!!!underscore!!!id72}
      \label{m38756*id254584}A model is a representation of a system in the real world. Models help us to understand systems and their properties. For example, an \textsl{atomic model} represents what the structure of an atom \textsl{could} look like, based on what we know about how atoms behave. It is not necessarily a true picture of the exact structure of an atom. \par 
       } 
      \end{tabular*}
      \end{definition}
      \label{m38756*uid1}
            \subsubsection{ The Plum Pudding Model}
            \nopagebreak
        \label{m38756*id254616}After the electron was discovered by J.J. Thomson in 1897, people realised that atoms were made up of even smaller particles than they had previously thought. However, the atomic nucleus had not been discovered yet and so the 'plum pudding model' was put forward in 1904. In this model, the atom is made up of negative electrons that float in a soup of positive charge, much like plums in a pudding or raisins in a fruit cake (Figure~3.2). In 1906, Thomson was awarded the Nobel Prize for his work in this field. However, even with the Plum Pudding Model, there was still no understanding of how these electrons in the atom were arranged.\par 
    \setcounter{subfigure}{0}
	\begin{figure}[H] % horizontal\label{m38756*uid2}
    \begin{center}
    \rule[.1in]{\figurerulewidth}{.005in} \\
        \label{m38756*uid2!!!underscore!!!media}\label{m38756*uid2!!!underscore!!!printimage}\includegraphics[width=9cm]{col11305.imgs/m38756_CG10C3_001.png} % m38756;CG10C3\_001.png;;;6.0;8.5;
      \vspace{2pt}
    \vspace{\rubberspace}\par \begin{cnxcaption}
	  \small \textbf{Figure 3.2: }A schematic diagram to show what the atom looks like according to the Plum Pudding model
	\end{cnxcaption}
    \vspace{.1in}
    \rule[.1in]{\figurerulewidth}{.005in} \\
    \end{center}
 \end{figure}       
        \label{m38756*id254642}The discovery of \textbf{radiation} was the next step along the path to building an accurate picture of atomic structure. In the early twentieth century, Marie Curie and her husband Pierre,  discovered that some elements (the \textsl{radioactive} elements) emit particles, which are able to pass through matter in a similar way to X-rays (read more about this in Grade 11). It was Ernest Rutherford who, in 1911, used this discovery to revise the model of the atom.\par 
      \label{m38756*eip-956}
\begin{tabular}{cc}
	\hspace*{-50pt}\raisebox{-8 mm}{\hspace{-0.2in}\includegraphics[width=0.75in]{col11305.imgs/psfact2.png} } & 
	\begin{minipage}{0.85\textwidth}
	\begin{note}
      {note: }Two other models proposed for the atom were the cubic model and the Saturnian model. In the cubic model, the electrons were imagined to lie at the corners of a cube. In the Saturnian model, the electrons were imagined to orbit a very big, heavy nucleus.
	\end{note}
	\end{minipage}
	\end{tabular}
	\par
      \label{m38756*uid3}
            \subsubsection{ Rutherford's model of the atom}
            \nopagebreak
            \label{m38756*id254751}Rutherford carried out some experiments which led to a change in ideas around the atom. His new model described the atom as a tiny, dense, positively charged core called a nucleus surrounded by lighter, negatively charged electrons. Another way of thinking about this model was that the atom was seen to be like a mini solar system where the electrons orbit the nucleus like planets orbiting around the sun. A simplified picture of this is shown in Figure~3.3. This model is sometimes known as the planetary model of the atom.\par 
    \setcounter{subfigure}{0}
	\begin{figure}[H] % horizontal\label{m38756*uid5}
    \begin{center}
    \rule[.1in]{\figurerulewidth}{.005in} \\
        \label{m38756*uid5!!!underscore!!!media}\label{m38756*uid5!!!underscore!!!printimage}\includegraphics[width=8cm]{col11305.imgs/m38756_CG10C3_003.png} % m38756;CG10C3\_003.png;;;6.0;8.5;
      \vspace{2pt}
    \vspace{\rubberspace}\par \begin{cnxcaption}
	  \small \textbf{Figure 3.3: }Rutherford's model of the atom
	\end{cnxcaption}
    \vspace{.1in}
    \rule[.1in]{\figurerulewidth}{.005in} \\
    \end{center}
 \end{figure}       
      \label{m38756*uid6}
            \subsubsection{ The Bohr Model}
            \nopagebreak
        \label{m38756*id254784}There were, however, some problems with this model: for example it could not explain the very interesting
observation that atoms only emit light at certain wavelengths or frequencies. Niels Bohr solved
this problem by proposing that the electrons could only orbit the nucleus in certain special orbits
at different energy levels around the nucleus. The exact energies of the orbitals in each energy level depends on
the type of atom. Helium for example, has different energy levels to Carbon. If an electron jumps down
from a higher energy level to a lower energy level, then light is emitted from
the atom. The energy of the light emitted is the same as the gap in the energy between the two
energy levels. You can read more about this in "Energy quantisation and electron configuration". The distance between the nucleus and the electron in the lowest energy level of a hydrogen atom is known as the \textbf{Bohr radius}.\par 
\label{m38756*notfhsst!!!underscore!!!id119}
\begin{tabular}{cc}
	\hspace*{-50pt}\raisebox{-8 mm}{\hspace{-0.2in}\includegraphics[width=0.75in]{col11305.imgs/psfact2.png} } & 
	\begin{minipage}{0.85\textwidth}
	\begin{note}
      {note: }
        \label{m38756*id254816}Light has the properties of both a particle \textbf{and} a wave! Einstein discovered
that light comes in energy packets which are called \textbf{photons}. When an electron in an atom
changes energy levels, a photon of light is emitted. This photon has the same energy as
the difference between the two electron energy levels.\par 
	\end{note}
	\end{minipage}
	\end{tabular}
	\par
      \label{m38756*eip-279}
            \subsubsection{ Other models of the atom}
            \nopagebreak
            \label{m38756*eip-993}
Although the most common model of the atom is the Bohr model, scientists have not stopped thinking about other ways to describe atoms. One of the most important contributions to atomic theory (the field of science that looks at atoms) was the development of quantum theory. Schrodinger, Heisenberg, Born and many others have had a role in developing quantum theory. The description of an atom by quantum theory is very complex and is only covered at university level.  
\par \label{m38756*eip-179}
            \subsubsection{ Models of the atom}
            \nopagebreak
            \label{m38756*eip-786}Match the information in column A, with the key discoverer in column B.
    % \textbf{m38756*eip-551}\par
          \begin{table}[H]
    % \begin{table}[H]
    % \\ '' '0'
        \begin{center}
      \label{m38756*eip-551}
    \noindent
    \tabletail{%
        \hline
        \multicolumn{2}{|p{\mytableboxwidth}|}{\raggedleft \small \sl continued on next page}\\
        \hline
      }
      \tablelasttail{}
      \begin{xtabular}[t]{|l|l|}\hline
        Column A &
        Column B% make-rowspan-placeholders
     \tabularnewline\cline{1-1}\cline{2-2}
      %--------------------------------------------------------------------
        Discovery of electrons and the plum pudding model &
        Niels Bohr% make-rowspan-placeholders
     \tabularnewline\cline{1-1}\cline{2-2}
      %--------------------------------------------------------------------
        Arrangement of electrons &
        Marie Curie and her husband, Pierre% make-rowspan-placeholders
     \tabularnewline\cline{1-1}\cline{2-2}
      %--------------------------------------------------------------------
        Atoms as the smallest building block of matter &
        Ancient Greeks% make-rowspan-placeholders
     \tabularnewline\cline{1-1}\cline{2-2}
      %--------------------------------------------------------------------
        Discovery of the nucleus &
        JJ Thomson% make-rowspan-placeholders
     \tabularnewline\cline{1-1}\cline{2-2}
      %--------------------------------------------------------------------
        Discovery of radiation &
        Rutherford% make-rowspan-placeholders
     \tabularnewline\cline{1-1}\cline{2-2}
      %--------------------------------------------------------------------
    \end{xtabular}
      \end{center}
    \begin{center}{\small\bfseries Table 3.1}\end{center}
    \begin{caption}{\small\bfseries Table 3.1}\end{caption}
\end{table}
    \par
        \par 
    \label{m38756*cid3}
\par \raisebox{-5 pt}{\includegraphics[width=0.5cm]{col11305.imgs/summary_www.png}} Find the answers with the shortcodes:
 \par \begin{tabular}[h]{cccccc}
 (1.) l4g  & \end{tabular}
            \subsection{ Atomic mass and diameter}
            \nopagebreak
            \label{m38756*id254850}It is difficult sometimes to imagine the size of an atom, or its mass, because we cannot see an atom and also because we are not used to working with such small measurements.\par 
      \label{m38756*uid7}
            \subsubsection{ How heavy is an atom?}
            \nopagebreak
        \label{m38756*id254863}It is possible to determine the mass of a single atom in kilograms. But to do this, you would need very modern \textsl{mass spectrometers} and the values you would get would be very clumsy and difficult to use. The mass of a carbon atom, for example, is about $1,99\ensuremath{\times}{10}^{-26}\phantom{\rule{2pt}{0ex}}\mathrm{kg}$, while the mass of an atom of hydrogen is about $1,67\ensuremath{\times}{10}^{-27}\phantom{\rule{2pt}{0ex}}\mathrm{kg}$. Looking at these very small numbers makes it difficult to compare how much bigger the mass of one atom is when compared to another.\par 
        \label{m38756*id254908}To make the situation simpler, scientists use a different unit of mass when they are describing the mass of an atom. This unit is called the \textbf{atomic mass unit} (amu). We can abbreviate (shorten) this unit to just 'u'. Scientists use the \textbf{carbon standard} to determine amu. The carbon standard assigns carbon an atomic mass of 12 u. Using the carbon standard the mass of an atom of hydrogen will be 1 u. You can check this by dividing the mass of a carbon atom in kilograms (see above) by the mass of a hydrogen atom in kilograms (you will need to use a calculator for this!). If you do this calculation, you will see that the mass of a carbon atom is twelve times greater than the mass of a hydrogen atom. When we use atomic mass units instead of kilograms, it becomes easier to see this. Atomic mass units are therefore not giving us the \textsl{actual} mass of an atom, but rather its mass \textsl{relative} to the mass of one (carefully chosen) atom in the Periodic Table. Although carbon is the usual element to compare other elements to, oxygen and hydrogen have also been used. The important thing to remember here is that the atomic mass unit is relative to one (carefully chosen) element. The atomic masses of some elements are shown in the  table  (Table 3.2) below.\par 
    % \textbf{m38756*uid8}\par
          \begin{table}[H]
    % \begin{table}[H]
    % \\ '' '0'
        \begin{center}
      \label{m38756*uid8}
    \noindent
    \tabletail{%
        \hline
        \multicolumn{2}{|p{\mytableboxwidth}|}{\raggedleft \small \sl continued on next page}\\
        \hline
      }
      \tablelasttail{}
      \begin{xtabular}[t]{|l|l|}\hline
                  \textbf{Element}
                 &
                  \textbf{Atomic mass (u)}
                % make-rowspan-placeholders
     \tabularnewline\cline{1-1}\cline{2-2}
      %--------------------------------------------------------------------
        Carbon ($\mathrm{C}$) &
        12% make-rowspan-placeholders
     \tabularnewline\cline{1-1}\cline{2-2}
      %--------------------------------------------------------------------
        Nitrogen ($\mathrm{N}$) &
        14% make-rowspan-placeholders
     \tabularnewline\cline{1-1}\cline{2-2}
      %--------------------------------------------------------------------
        Bromine ($\mathrm{Br}$) &
        80% make-rowspan-placeholders
     \tabularnewline\cline{1-1}\cline{2-2}
      %--------------------------------------------------------------------
        Magnesium ($\mathrm{Mg}$) &
        24% make-rowspan-placeholders
     \tabularnewline\cline{1-1}\cline{2-2}
      %--------------------------------------------------------------------
        Potassium ($\mathrm{K}$) &
        39% make-rowspan-placeholders
     \tabularnewline\cline{1-1}\cline{2-2}
      %--------------------------------------------------------------------
        Calcium ($\mathrm{Ca}$) &
        40% make-rowspan-placeholders
     \tabularnewline\cline{1-1}\cline{2-2}
      %--------------------------------------------------------------------
        Oxygen ($\mathrm{O}$) &
        16% make-rowspan-placeholders
     \tabularnewline\cline{1-1}\cline{2-2}
      %--------------------------------------------------------------------
    \end{xtabular}
      \end{center}
    \begin{center}{\small\bfseries Table 3.2}: The atomic mass number of some of the elements\end{center}
    \begin{caption}{\small\bfseries Table 3.2}: The atomic mass number of some of the elements\end{caption}
\end{table}
    \par
        \label{m38756*id255096}The actual value of 1 atomic mass unit is $1,67\ensuremath{\times}{10}^{-24}\phantom{\rule{2pt}{0ex}}\mathrm{g}$ or $1,67\ensuremath{\times}{10}^{-27}\phantom{\rule{2pt}{0ex}}\mathrm{kg}$. This is a very tiny mass!\par 
      \label{m38756*uid9}
            \subsubsection{ How big is an atom?}
            \nopagebreak
\label{m38756*notfhsst!!!underscore!!!id177}
\begin{tabular}{cc}
	   \hspace*{-50pt}\raisebox{-8 mm}{ \includegraphics[width=0.5in]{col11305.imgs/pstip2.png}  }& 
	\begin{minipage}{0.85\textwidth}
	\begin{note}
      {tip: }\textsl{pm} stands for \textsl{picometres}. $1\phantom{\rule{2pt}{0ex}}\mathrm{pm}={10}^{-12}\phantom{\rule{2pt}{0ex}}\mathrm{m}$
	\end{note}
	\end{minipage}
	\end{tabular}
	\par
        \label{m38756*id255173}Atomic radius also varies depending on the element. On average, the radius of an atom ranges from $32\phantom{\rule{2pt}{0ex}}\mathrm{pm}$ (Helium) to $225\phantom{\rule{2pt}{0ex}}\mathrm{pm}$ (Caesium). Using different units, $100\phantom{\rule{2pt}{0ex}}\mathrm{pm}=1\phantom{\rule{2pt}{0ex}}\mathrm{Angstrom}$, and $1\phantom{\rule{2pt}{0ex}}\mathrm{Angstrom}={10}^{-10}\phantom{\rule{2pt}{0ex}}\mathrm{m}$. That is the same as saying that $1\phantom{\rule{2pt}{0ex}}\mathrm{Angstrom}=0,0000000010\phantom{\rule{2pt}{0ex}}\mathrm{m}$ or that $100\phantom{\rule{2pt}{0ex}}\mathrm{pm}=0,0000000010\phantom{\rule{2pt}{0ex}}\mathrm{m}$! In other words, the diameter of an atom ranges from $0,0000000010\phantom{\rule{2pt}{0ex}}m$ to $0,0000000067\phantom{\rule{2pt}{0ex}}m$. This is very small indeed.\par \label{m38756*eip-30}The atomic radii given above are for the whole atom (nucleus and electrons). The nucleus itself is even smaller than this by a factor of about 23 000 in uranium and 145 000 in hydrogen. If the nucleus were the size of a golf ball, then the nearest electrons would be about one kilometer away! This should help you realise that the atom is mostly made up of empty space. \par \label{m38756*eip-320}
            \subsubsection{ Rutherfords alpha-particle scattering experiment}
            \nopagebreak
            \label{m38756*id254668}Radioactive elements emit different types of particles. Some of these are positively charged alpha ($\alpha $) particles.
Rutherford carried out a series of experiments where he bombarded sheets of gold foil with these particles, to try to get a better understanding of where the positive charge in the atom was. A simplified diagram of his experiment is shown in Figure~3.4.\par 
    \setcounter{subfigure}{0}
	\begin{figure}[H] % horizontal\label{m38756*uid4}
    \begin{center}
    \rule[.1in]{\figurerulewidth}{.005in} \\
        \label{m38756*uid4!!!underscore!!!media}\label{m38756*uid4!!!underscore!!!printimage}\includegraphics[width=300px]{col11305.imgs/m38756_CG10C3_002.png} % m38756;CG10C3\_002.png;;;6.0;8.5;
      \vspace{2pt}
    \vspace{\rubberspace}\par \begin{cnxcaption}
	  \small \textbf{Figure 3.4: }Rutherford's gold foil experiment. Figure (a) shows the path of the $\alpha $ particles after they hit the gold sheet. Figure (b) shows the arrangement of atoms in the gold sheets and the path of the $\alpha $ particles in relation to this.
	\end{cnxcaption}
    \vspace{.1in}
    \rule[.1in]{\figurerulewidth}{.005in} \\
    \end{center}
 \end{figure}       
        \label{m38756*id254715}Rutherford set up his experiment so that a beam of alpha particles was directed at the gold sheets. Behind the gold sheets was a screen made of zinc sulphide. This screen allowed Rutherford to see where the alpha particles were landing. Rutherford knew that the \textsl{electrons} in the gold atoms would not really affect the path of the alpha particles, because the mass of an electron is so much smaller than that of a proton. He reasoned that the positively charged \textsl{protons} would be the ones to \textsl{repel} the positively charged alpha particles and alter their path.\par 
        \label{m38756*id254738}What he discovered was that most of the alpha particles passed through the foil undisturbed and could be detected on the screen directly behind the foil (A). Some of the particles ended up being slightly deflected onto other parts of the screen (B). But what was even more interesting was that some of the particles were deflected straight back in the direction from where they had come (C)! These were the particles that had been repelled by the positive protons in the gold atoms. If the Plum Pudding model of the atom were true then Rutherford would have expected much more repulsion, since the positive charge according to that model is distributed throughout the atom. But this was not the case. The fact that most particles passed straight through suggested that the positive charge was concentrated in one part of the atom only.\par 
      \label{m38756*eip-491}
            \subsubsection{ Relative atomic mass}
            \nopagebreak
            \par
            \label{m38756*eip-890}\begin{definition}
	  \begin{tabular*}{15 cm}{m{15 mm}m{}}
	\hspace*{-50pt}  \includegraphics[width=0.5in]{col11305.imgs/psflag2.png}   & \Definition{   \label{id2415700}\textbf{ Relative atomic mass }} { \label{m38756*meaningfhsst!!!underscore!!!id598}
        \label{m38756*id258580}Relative atomic mass is the average mass of one atom of all the naturally occurring isotopes of a particular chemical element, expressed in atomic mass units.
 \par 
         } 
      \end{tabular*}
      \end{definition}
\label{m38756*eip-933}The relative atomic mass of an element is the number you will find on the periodic table. \par 
  \label{m38756**end}
         \section{ Structure}
    \nopagebreak
            \label{m38745} $ \hspace{-5pt}\begin{array}{cccccccccccc}   \includegraphics[width=0.75cm]{col11305.imgs/summary_fullmarks.png} &   \end{array} $ \hspace{2 pt}\raisebox{-5 pt}{} {(section shortcode: P10021 )} \par 
    \label{m38745*cid4}
            \subsection{ Structure of the atom}
            \nopagebreak
      \label{m38745*id255206}As a result of the work done by previous scientists on atomic models (that we discussed in "Models of the Atom"), scientists now have a good idea of what an atom looks like. This knowledge is important because it helps us to understand why materials have different properties and why some materials bond with others. Let us now take a closer look at the microscopic structure of the atom.\par 
      \label{m38745*id255216}So far, we have discussed that atoms are made up of a positively charged \textbf{nucleus} surrounded by
one or more negatively charged \textbf{electrons}. These electrons orbit the nucleus.\par 
      \label{m38745*eip-577}Before we look at some useful concepts we first need to understand what electrons, protons and neutrons are.\par \label{m38745*uid10}
            \subsubsection{ The Electron}
            \nopagebreak
        \label{m38745*id255241}The electron is a very light particle. It has a mass of $9,11\ensuremath{\times}{10}^{-31}\phantom{\rule{2pt}{0ex}}\mathrm{kg}$.
Scientists believe that the electron can be treated as a \textbf{point particle}
or \textbf{elementary particle}
meaning that it can't be broken down into anything smaller. The electron also carries one unit
of \textbf{negative} electric charge which is the same as $1,6\ensuremath{\times}{10}^{-19}\phantom{\rule{2pt}{0ex}}\mathrm{C}$ (Coulombs).\par \label{m38745*eip-222}The electrons determine the charge on an atom. If the number of electrons is the same as the number of protons then the atom will be neutral. If the number of electrons is greater than the number of protons then the atom will be negatively charged. If the number of electrons is less than the number of protons then the atom will be positively charged. Atoms that are not neutral are called ions. Ions will be covered in more detail in a later chapter. For now all you need to know is that for each electron you remove from an atom you loose $-1$ of charge and for each electron that you add to an atom you gain $+1$ of charge. For example, the charge on an atom of sodium after removing one electron is  $-1$.\par 
      \label{m38745*uid11}
            \subsubsection{ The Nucleus}
            \nopagebreak
        \label{m38745*id255305}Unlike the electron, the nucleus \textbf{can} be broken up into smaller building
blocks called \textbf{protons} and \textbf{neutrons}. Together, the protons and
neutrons are called \textbf{nucleons}.\par 
        \label{m38745*uid12}
            \subsubsection{ The Proton}
            \nopagebreak
          \label{m38745*id255338}Each proton carries one unit of \textbf{positive} electric charge.
Since we know that atoms are
\textbf{electrically neutral}, i.e. do not carry any extra charge, then the number
of protons in an atom has to be the same as the number of electrons to balance
out the positive and negative charge to zero. The total positive charge of a
nucleus is equal to the number of protons in the nucleus. The proton is much heavier
than the electron (10 000 times heavier!) and has a mass of $1,6726\ensuremath{\times}{10}^{-27}\phantom{\rule{2pt}{0ex}}\mathrm{kg}$. When we talk about the atomic mass of an atom, we are mostly referring to the combined mass of the protons and neutrons, i.e. the nucleons.\par 
        \label{m38745*uid13}
            \subsubsection{ The Neutron}
            \nopagebreak
          \label{m38745*id254468}The neutron is electrically neutral i.e. it carries no charge at all.
Like the proton, it is much heavier than the electron and its mass is $1,6749\ensuremath{\times}{10}^{-27}\phantom{\rule{2pt}{0ex}}\mathrm{kg}$ (slightly heavier than the proton).\par 
\label{m38745*notfhsst!!!underscore!!!id214}
\begin{tabular}{cc}
	\hspace*{-50pt}\raisebox{-8 mm}{\hspace{-0.2in}\includegraphics[width=0.75in]{col11305.imgs/psfact2.png} } & 
	\begin{minipage}{0.85\textwidth}
	\begin{note}
      {note: }
          \label{m38745*id254497}Rutherford predicted (in 1920) that another kind of particle must be
present in the nucleus along with the proton. He predicted this because
if there were only positively charged protons in the nucleus, then it should break
into bits because of the repulsive forces between the like-charged protons! Also,
if protons were the only particles
in the nucleus, then a helium nucleus (atomic number 2) would have
two protons and therefore only twice the mass of hydrogen. However,
it is actually \textbf{four} times heavier than hydrogen. This suggested that there
must be something else inside the nucleus as well as the protons.
To make sure that the atom stays electrically neutral, this particle would have to be neutral itself. In 1932 James Chadwick discovered the neutron and measured
its mass.\par 
	\end{note}
	\end{minipage}
	\end{tabular}
	\par
    % \textbf{m38745*uid14}\par
          \begin{table}[H]
    % \begin{table}[H]
    % \\ '' '0'
        \begin{center}
      \label{m38745*uid14}
    \noindent
    \tabletail{%
        \hline
        \multicolumn{4}{|p{\mytableboxwidth}|}{\raggedleft \small \sl continued on next page}\\
        \hline
      }
      \tablelasttail{}
      \begin{xtabular}[t]{|l|l|l|l|}\hline
         &
                    \textbf{proton}
                   &
                    \textbf{neutron}
                   &
                    \textbf{electron}
                  % make-rowspan-placeholders
     \tabularnewline\cline{1-1}\cline{2-2}\cline{3-3}\cline{4-4}
      %--------------------------------------------------------------------
                    \textbf{Mass (kg)}
                   &
        $1,6726\ensuremath{\times}{10}^{-27}$ &
        $1,6749\ensuremath{\times}{10}^{-27}$ &
        $9,11\ensuremath{\times}{10}^{-31}$% make-rowspan-placeholders
     \tabularnewline\cline{1-1}\cline{2-2}\cline{3-3}\cline{4-4}
      %--------------------------------------------------------------------
                    \textbf{Units of charge}
                   &
        $+1$ &
        $0$ &
        $-1$% make-rowspan-placeholders
     \tabularnewline\cline{1-1}\cline{2-2}\cline{3-3}\cline{4-4}
      %--------------------------------------------------------------------
                    \textbf{Charge (C)}
                   &
        $1,6\ensuremath{\times}{10}^{-19}$ &
        $0$ &
        $-1,6\ensuremath{\times}{10}^{-19}$% make-rowspan-placeholders
     \tabularnewline\cline{1-1}\cline{2-2}\cline{3-3}\cline{4-4}
      %--------------------------------------------------------------------
    \end{xtabular}
      \end{center}
    \begin{center}{\small\bfseries Table 3.3}: Summary of the particles inside the atom\end{center}
    \begin{caption}{\small\bfseries Table 3.3}: Summary of the particles inside the atom\end{caption}
\end{table}
    \par
    \label{m38745*cid5}
            \subsection{ Atomic number and atomic mass number}
            \nopagebreak
      \label{m38745*id255805}The chemical properties of an element are determined by the charge of
its nucleus, i.e. by the \textbf{number of protons}. This number is
called the \textbf{atomic number} and is denoted by the letter \textbf{Z}.\par 
\label{m38745*fhsst!!!underscore!!!id284}\begin{definition}
	  \begin{tabular*}{15 cm}{m{15 mm}m{}}
	\hspace*{-50pt}  \includegraphics[width=0.5in]{col11305.imgs/psflag2.png}   & \Definition{   \label{id2416514}\textbf{ Atomic number (Z) }} { \label{m38745*meaningfhsst!!!underscore!!!id284}
      \label{m38745*id255833}The number of protons in an atom \par 
       } 
      \end{tabular*}
      \end{definition}
      \label{m38745*eip-164}You can find the atomic number on the periodic table. The atomic number is an integer and ranges from 1 to about 118.\par \label{m38745*id255845}The mass of an atom depends on how many nucleons its nucleus contains.
The number of nucleons, i.e. the total number of protons \textbf{plus} neutrons,
is called the
\textbf{atomic mass number} and is denoted by the letter \textbf{A}.\par 
\label{m38745*fhsst!!!underscore!!!id291}\begin{definition}
	  \begin{tabular*}{15 cm}{m{15 mm}m{}}
	\hspace*{-50pt}  \includegraphics[width=0.5in]{col11305.imgs/psflag2.png}   & \Definition{   \label{id2416571}\textbf{ Atomic mass number (A) }} { \label{m38745*meaningfhsst!!!underscore!!!id291}
      \label{m38745*id255874}The number of protons and neutrons in the nucleus of an atom \par 
       } 
      \end{tabular*}
      \end{definition}
\label{m38745*notfhsst!!!underscore!!!id321}
\begin{tabular}{cc}
	   \hspace*{-50pt}\raisebox{-8 mm}{ \includegraphics[width=0.5in]{col11305.imgs/pstip2.png}  }& 
	\begin{minipage}{0.85\textwidth}
	\begin{note}
      {tip: }Don't confuse the notation we have used above with the way this information appears on the Periodic Table. On the Periodic Table, the atomic number usually appears in the top lefthand corner of the block or immediately above the element's symbol. The number below the element's symbol is its \textbf{relative atomic mass}. This is not exactly the same as the atomic mass number. This will be explained in "Isotopes". The example of iron is shown below.
      \label{m38745*id256180}
    \setcounter{subfigure}{0}
	\begin{figure}[H] % horizontal\label{m38745*id256183}
    \begin{center}
    \label{m38745*id256183!!!underscore!!!media}\label{m38745*id256183!!!underscore!!!printimage}\includegraphics[width=2cm]{col11305.imgs/m38745_CG10C3_004.png} % m38745;CG10C3\_004.png;;;6.0;8.5;
      \vspace{2pt}
    \vspace{.1in}
    \end{center}
 \end{figure}       
      \par 
	\end{note}
	\end{minipage}
	\end{tabular}
	\par
      \label{m38745*id256193}You will notice in the example of iron that the atomic mass number is more or less the same as its atomic mass. Generally, an atom that contains \textsl{n} nucleons (protons and neutrons), will have a mass approximately equal to $n$u. For example the mass of a $\mathrm{C}-12$ atom which has 6 protons, 6 neutrons and 6 electrons is 12u, where the protons and neutrons have about the same mass and the electron mass is negligible. \par \pagebreak
\label{m38745*eip-950}\vspace{.5cm} 
      \noindent
      \hspace*{-30pt}\includegraphics[width=0.5in]{col11305.imgs/pspencil2.png}   \raisebox{25mm}{   
      \begin{mdframed}[linewidth=4, leftmargin=40, rightmargin=40]  
      \begin{exercise}
    \noindent\textbf{Exercise 3.1}\label{m38745*eip-686}
  \label{m38745*eip-509}Use standard notation to represent sodium and give the number of protons, neutrons and electrons in the element.
  \par 
\vspace{5pt}
\label{m38745*eip-600}\noindent\textbf{Solution to Exercise }
  \label{m38745*eip-374}\label{m38745*listfhsst!!!underscore!!!id431}\begin{enumerate}[noitemsep, label=\textbf{Step} \textbf{\arabic*}. ] 
            \leftskip=20pt\rightskip=\leftskip\item Sodium is given by $Na$\item Sodium has 11 protons, so we have: ${}_{11}Na$\item Sodium has 12 neutrons.\item $A=N+Z=12+11=23$\item In standard notation sodium is given by: $_{11}^{23}Na$. The number of protons is 11, the number of neutrons is 12 and the number of electrons is 11.\end{enumerate}
  \par 
    \end{exercise}
    \end{mdframed}
    } \vspace{-.5cm}
    \noindent
  \label{m38745*secfhsst!!!underscore!!!id332}
            \subsubsection{  The structure of the atom
      }
            \nopagebreak
      \label{m38745*id256225}\begin{enumerate}[noitemsep, label=\textbf{\arabic*}. ] 
            \label{m38745*uid15}\item Explain the meaning of each of the following terms:
\label{m38745*id256240}\begin{enumerate}[noitemsep, label=\textbf{\alph*}. ] 
            \label{m38745*uid16}\item nucleus
\label{m38745*uid17}\item electron
\label{m38745*uid18}\item atomic mass
\end{enumerate}
                \label{m38745*uid19}\item Complete the following table: (Note: You will see that the atomic masses on the Periodic Table are not \textsl{whole numbers}. This will be explained later. For now, you can round off to the nearest whole number.)
    % \textbf{m38745*id256298}\par
          \begin{table}[H]
    % \begin{table}[H]
    % \\ 'id2886115' '1'
        \begin{center}
      \label{m38745*id256298}
    \noindent
    \tabletail{%
        \hline
        \multicolumn{6}{|p{\mytableboxwidth}|}{\raggedleft \small \sl continued on next page}\\
        \hline
      }
      \tablelasttail{}
      \begin{xtabular}[t]{|l|l|l|l|l|l|}\hline
        \textbf{Element} &
        \textbf{Atomic mass} &
        \textbf{Atomic number} &
        \textbf{Number of protons} &
        \textbf{Number of electrons} &
        \textbf{Number of neutrons}% make-rowspan-placeholders
     \tabularnewline\cline{1-1}\cline{2-2}\cline{3-3}\cline{4-4}\cline{5-5}\cline{6-6}
      %--------------------------------------------------------------------
        $Mg$ &
        24 &
        12 &
         &
         &
        % make-rowspan-placeholders
     \tabularnewline\cline{1-1}\cline{2-2}\cline{3-3}\cline{4-4}\cline{5-5}\cline{6-6}
      %--------------------------------------------------------------------
        $\mathrm{O}$ &
         &
         &
        8 &
         &
        % make-rowspan-placeholders
     \tabularnewline\cline{1-1}\cline{2-2}\cline{3-3}\cline{4-4}\cline{5-5}\cline{6-6}
      %--------------------------------------------------------------------
         &
         &
        17 &
         &
         &
        % make-rowspan-placeholders
     \tabularnewline\cline{1-1}\cline{2-2}\cline{3-3}\cline{4-4}\cline{5-5}\cline{6-6}
      %--------------------------------------------------------------------
        $Ni$ &
         &
         &
         &
        28 &
        % make-rowspan-placeholders
     \tabularnewline\cline{1-1}\cline{2-2}\cline{3-3}\cline{4-4}\cline{5-5}\cline{6-6}
      %--------------------------------------------------------------------
         &
        40 &
         &
         &
         &
        20% make-rowspan-placeholders
     \tabularnewline\cline{1-1}\cline{2-2}\cline{3-3}\cline{4-4}\cline{5-5}\cline{6-6}
      %--------------------------------------------------------------------
        $Zn$ &
         &
         &
         &
         &
        % make-rowspan-placeholders
     \tabularnewline\cline{1-1}\cline{2-2}\cline{3-3}\cline{4-4}\cline{5-5}\cline{6-6}
      %--------------------------------------------------------------------
         &
         &
         &
         &
         &
        0% make-rowspan-placeholders
     \tabularnewline\cline{1-1}\cline{2-2}\cline{3-3}\cline{4-4}\cline{5-5}\cline{6-6}
      %--------------------------------------------------------------------
        $\mathrm{C}$ &
        12 &
         &
         &
        6 &
        % make-rowspan-placeholders
     \tabularnewline\cline{1-1}\cline{2-2}\cline{3-3}\cline{4-4}\cline{5-5}\cline{6-6}
      %--------------------------------------------------------------------
    \end{xtabular}
      \end{center}
    \begin{center}{\small\bfseries Table 3.4}\end{center}
    \begin{caption}{\small\bfseries Table 3.4}\end{caption}
\end{table}
    \par
          \label{m38745*uid20}\item Use standard notation to represent the following elements:
\label{m38745*id256772}\begin{enumerate}[noitemsep, label=\textbf{\alph*}. ] 
            \label{m38745*uid21}\item potassium
\label{m38745*uid22}\item copper
\label{m38745*uid23}\item chlorine
\end{enumerate}
                \label{m38745*uid24}\item 
For the element $_{17}^{35}Cl$, give the number of ...
\label{m38745*id256843}\begin{enumerate}[noitemsep, label=\textbf{\alph*}. ] 
            \label{m38745*uid25}\item protons
\label{m38745*uid26}\item neutrons
\label{m38745*uid27}\item electrons
\end{enumerate}
... in the atom.\newline
\label{m38745*uid28}\item Which of the following atoms has 7 electrons?
\label{m38745*id256898}\begin{enumerate}[noitemsep, label=\textbf{\alph*}. ] 
            \label{m38745*uid29}\item $_{2}^{5}He$
\label{m38745*uid30}\item $_{6}^{13}\mathrm{C}$
\label{m38745*uid31}\item $_{3}^{7}Li$
\label{m38745*uid32}\item $_{7}^{15}\mathrm{N}$
\end{enumerate}
                \label{m38745*uid33}\item 
In each of the following cases, give the number or the element symbol represented by 'X'.
\label{m38745*id257023}\begin{enumerate}[noitemsep, label=\textbf{\alph*}. ] 
            \label{m38745*uid34}\item $_{18}^{40}\mathrm{X}$
\label{m38745*uid35}\item $_{20}^{x}\mathrm{Ca}$
\label{m38745*uid36}\item $_{x}^{31}\mathrm{P}$
\end{enumerate}
                \label{m38745*uid37}\item 
Complete the following table:
    % \textbf{m38745*id257121}\par
          \begin{table}[H]
    % \begin{table}[H]
    % \\ 'id2886554' '1'
        \begin{center}
      \label{m38745*id257121}
    \noindent
    \tabletail{%
        \hline
        \multicolumn{4}{|p{\mytableboxwidth}|}{\raggedleft \small \sl continued on next page}\\
        \hline
      }
      \tablelasttail{}
      \begin{xtabular}[t]{|l|l|l|l|}\hline
         &
        \textbf{A} &
        \textbf{Z} &
        \textbf{N}% make-rowspan-placeholders
     \tabularnewline\cline{1-1}\cline{2-2}\cline{3-3}\cline{4-4}
      %--------------------------------------------------------------------
        $_{92}^{235}\mathrm{U}$ &
         &
         &
        % make-rowspan-placeholders
     \tabularnewline\cline{1-1}\cline{2-2}\cline{3-3}\cline{4-4}
      %--------------------------------------------------------------------
        $_{92}^{238}\mathrm{U}$ &
         &
         &
        % make-rowspan-placeholders
     \tabularnewline\cline{1-1}\cline{2-2}\cline{3-3}\cline{4-4}
      %--------------------------------------------------------------------
    \end{xtabular}
      \end{center}
    \begin{center}{\small\bfseries Table 3.5}\end{center}
    \begin{caption}{\small\bfseries Table 3.5}\end{caption}
\end{table}
    \par
In these two different forms of Uranium...
\label{m38745*id257277}\begin{enumerate}[noitemsep, label=\textbf{\alph*}. ] 
            \label{m38745*uid38}\item What is the \textsl{same}?
\label{m38745*uid39}\item What is \textsl{different}?
\end{enumerate}
Uranium can occur in different forms, called \textsl{isotopes}. You will learn more about isotopes in "Isotopes".\newline
\end{enumerate}
  \label{m38745**end}
\par \raisebox{-5 pt}{\includegraphics[width=0.5cm]{col11305.imgs/summary_www.png}} Find the answers with the shortcodes:
 \par \begin{tabular}[h]{cccccc}
 (1.) ll0  &  (2.) ll8  &  (3.) ll9  &  (4.) llX  &  (5.) llk  &  (6.) llK  &  (7.) llB  & \end{tabular}
         \section{ Isotopes}
    \nopagebreak
            \label{m38753} $ \hspace{-5pt}\begin{array}{cccccccccccc}   \includegraphics[width=0.75cm]{col11305.imgs/summary_fullmarks.png} &   \includegraphics[width=0.75cm]{col11305.imgs/summary_simulation.png} &   \end{array} $ \hspace{2 pt}\raisebox{-5 pt}{} {(section shortcode: P10022 )} \par 
    \label{m38753*cid6}
            \subsection{ Isotopes}
            \nopagebreak
            \label{m38753*uid40}
            \subsubsection{ What is an isotope?}
            \nopagebreak
            \label{m38753*id257359}The chemical properties of an element depend on the number of protons and electrons inside the atom. So if a neutron or two is added or removed from the nucleus, then the chemical properties will not change. This means that such an atom would remain in the same place in the Periodic Table. For example, no matter how many neutrons we add or subtract from a nucleus with 6 protons, that element will \textbf{always} be called carbon and have the
element symbol $\mathrm{C}$ (see the Table of Elements). Atoms which have the same number of protons, but a different number of neutrons, are called \textbf{isotopes}.\par 
\label{m38753*fhsst!!!underscore!!!id386}\begin{definition}
	  \begin{tabular*}{15 cm}{m{15 mm}m{}}
	\hspace*{-50pt}  \includegraphics[width=0.5in]{col11305.imgs/psflag2.png}   & \Definition{   \label{id2417667}\textbf{ Isotope }} { \label{m38753*meaningfhsst!!!underscore!!!id386}
        \label{m38753*id257386}The \textbf{isotope} of a particular element is made up of atoms which have the same number of protons as the atoms in the original element, but a different number of neutrons.  \par 
         } 
      \end{tabular*}
      \end{definition}
        \label{m38753*id257405}The different isotopes of an element have the same atomic
number $Z$ but different mass numbers $A$ because they have a different
number of neutrons $N$. The chemical properties of the different
isotopes of an element are the same, but they might vary in how stable their nucleus is. Note that we can also write elements as $\mathrm{X\; -\; A}$ where the X is the element symbol and the A is the atomic mass of that element. For example, $\mathrm{C-}12$ has an atomic mass of 12 and $\mathrm{Cl-}35$ has an atomic mass of 35 u, while $\mathrm{Cl-}37$ has an atomic mass of 37 u.\par 
\label{m38753*notfhsst!!!underscore!!!id393}
\begin{tabular}{cc}
	\hspace*{-50pt}\raisebox{-8 mm}{\hspace{-0.2in}\includegraphics[width=0.75in]{col11305.imgs/psfact2.png} } & 
	\begin{minipage}{0.85\textwidth}
	\begin{note}
      {note: }
        \label{m38753*id257445}In Greek, ``same place'' reads as
$\stackrel{`}{\iota }\sigma o\varsigma $$\tau \stackrel{`}{o}\pi o\varsigma $\hspace{1ex}
(isos topos). This is why atoms which have the same number of protons, but
different numbers of neutrons, are called \textsl{isotopes}. They are in the same place on the Periodic Table!\par 
	\end{note}
	\end{minipage}
	\end{tabular}
	\par
\label{m38753*id248557}It is important to realise that the atomic mass of isotopes of the same element will be different because they have a different number of nucleons. Chlorine, for example, has two common isotopes which are chlorine-35 and chlorine-37. Chlorine-35 has an atomic mass of 35 u, while chlorine-37 has an atomic mass of 37 u. In the world around us, both of these isotopes occur naturally. It doesn't make sense to say that the element chlorine has an atomic mass of 35 u, or that it has an atomic mass of 37 u. Neither of these are absolutely true since the mass varies depending on the form in which the element occurs. We need to look at how much more common one is than the other in order to calculate the \textbf{relative atomic mass} for the element chlorine. This is the number that you find on the Periodic Table.\par 
\label{m38753*eip-8228}
\begin{tabular}{cc}
	\hspace*{-50pt}\raisebox{-8 mm}{\hspace{-0.2in}\includegraphics[width=0.75in]{col11305.imgs/psfact2.png} } & 
	\begin{minipage}{0.85\textwidth}
	\begin{note}
      {note: }\label{m38753*id9742331}The relative atomic mass of some elements depends on where on Earth the element is found. This is because the isotopes can be found in varying ratios depending on certain factors such as geological composition, etc. The International Union of Pure and Applied Chemistry (IUPAC) has decided to give the relative atomic mass of some elements as a range to better represent the varying isotope ratios on the Earth. For the calculations that you will do at high school, it is enough to simply use one number without worrying about these ranges.\par 
	\end{note}
	\end{minipage}
	\end{tabular}
	\par \pagebreak
      \label{m38753*secfhsst!!!underscore!!!id2602}\vspace{.5cm} 
      \noindent
      \hspace*{-30pt}\includegraphics[width=0.5in]{col11305.imgs/pspencil2.png}   \raisebox{25mm}{   
      \begin{mdframed}[linewidth=4, leftmargin=40, rightmargin=40]  
      \begin{exercise}
    \noindent\textbf{Exercise 3.2:  The relative atomic mass of an isotopic element }
        \label{m38753*probfhsst!!!underscore!!!id6203}
        \label{m38753*id2586016}The element chlorine has two isotopes, chlorine-35 and chlorine-37. The abundance of these isotopes when they occur naturally is 75\% chlorine-35 and 25\% chlorine-37. Calculate the \textsl{average} relative atomic mass for chlorine.
 \par 
        \vspace{5pt}
        \label{m38753*solfhsst!!!underscore!!!id6207}\noindent\textbf{Solution to Exercise } \label{m38753*listfhsst!!!underscore!!!id6107}\begin{enumerate}[noitemsep, label=\textbf{Step} \textbf{\arabic*}. ] 
            \leftskip=20pt\rightskip=\leftskip\item  
        \label{m38753*id2582637}Contribution of $\mathrm{Cl-}35=\left(\frac{75}{100}\ensuremath{\times}35\right)=26,25\phantom{\rule{2pt}{0ex}}\mathrm{u}$
 \par 
        \item  
        \label{m38753*id2528657}Contribution of $\mathrm{Cl-}37=\left(\frac{25}{100}\ensuremath{\times}37\right)=9,25\phantom{\rule{2pt}{0ex}}\mathrm{u}$\par 
        \item  
        \label{m38753*id2586730}$\mathrm{Relative\; atomic\; mass\; of\; chlorine}=26,25\phantom{\rule{2pt}{0ex}}\mathrm{u}+9,25\phantom{\rule{2pt}{0ex}}\mathrm{u}=35,5\phantom{\rule{2pt}{0ex}}\mathrm{u}$\par 
        \label{m38753*id2586714}If you look on the periodic table, the average relative atomic mass for chlorine is $35,5\phantom{\rule{2pt}{0ex}}\mathrm{u}$. You will notice that for many elements, the relative atomic mass that is shown is not a whole number. You should now understand that this number is the \textsl{average} relative atomic mass for those elements that have naturally occurring isotopes.\par 
\end{enumerate}
    \end{exercise}
    \end{mdframed}
    }
    \noindent
This simulation allows you to see how isotopes and relative atomic mass are inter related.
    \setcounter{subfigure}{0}
	\begin{figure}[H] % horizontal\label{m38806*transverse-waves}
    \textnormal{Phet simulation for Isotopes}\vspace{.1in} \nopagebreak
  \label{m38806*phet!!!underscore!!!sim}\label{m38806*phet-simulation}
            \raisebox{-5 pt}{ \includegraphics[width=0.5cm]{col11305.imgs/summary_www.png}} { (Simulation:  lbT )}
      \vspace{2pt}
    \vspace{.1in}
 \end{figure}           \par \label{m38753*secfhsst!!!underscore!!!id6128}
            \subsubsection{  Isotopes }
            \nopagebreak
        \label{m38753*id25862299}\begin{enumerate}[noitemsep, label=\textbf{\arabic*}. ] 
            \label{m38753*uid619}\item Complete the table below:
    % \textbf{m38753*id25871115}\par
          \begin{table}[H]
    % \begin{table}[H]
    % \\ 'id2887415' '1'
        \begin{center}
      \label{m38753*id25871115}
    \noindent
    \tabletail{%
        \hline
        \multicolumn{6}{|p{\mytableboxwidth}|}{\raggedleft \small \sl continued on next page}\\
        \hline
      }
      \tablelasttail{}
      \begin{xtabular}[t]{|l|l|l|l|l|l|}\hline
        \textbf{Isotope} &
        \textbf{Z} &
        \textbf{A} &
        \textbf{Protons} &
        \textbf{Neutrons} &
        \textbf{Electrons}% make-rowspan-placeholders
     \tabularnewline\cline{1-1}\cline{2-2}\cline{3-3}\cline{4-4}\cline{5-5}\cline{6-6}
      %--------------------------------------------------------------------
        Carbon-12 &
         &
         &
         &
         &
        % make-rowspan-placeholders
     \tabularnewline\cline{1-1}\cline{2-2}\cline{3-3}\cline{4-4}\cline{5-5}\cline{6-6}
      %--------------------------------------------------------------------
        Carbon-14 &
         &
         &
         &
         &
        % make-rowspan-placeholders
     \tabularnewline\cline{1-1}\cline{2-2}\cline{3-3}\cline{4-4}\cline{5-5}\cline{6-6}
      %--------------------------------------------------------------------
        Chlorine-35 &
         &
         &
         &
         &
        % make-rowspan-placeholders
     \tabularnewline\cline{1-1}\cline{2-2}\cline{3-3}\cline{4-4}\cline{5-5}\cline{6-6}
      %--------------------------------------------------------------------
        Chlorine-37 &
         &
         &
         &
         &
        % make-rowspan-placeholders
     \tabularnewline\cline{1-1}\cline{2-2}\cline{3-3}\cline{4-4}\cline{5-5}\cline{6-6}
      %--------------------------------------------------------------------
    \end{xtabular}
      \end{center}
    \begin{center}{\small\bfseries Table 3.6}\end{center}
    \begin{caption}{\small\bfseries Table 3.6}\end{caption}
\end{table}
    \par
          \label{m38753*uid7170}\item If a sample contains 90\% carbon-12 and 10\% carbon-14, calculate the relative atomic mass of an atom in that sample.
\hspace{1ex}        \label{m38753*uid7100}\item If a sample contains 22,5\% Cl-37 and 77,5\% Cl-35, calculate the relative atomic mass of an atom in that sample.\hspace{1ex}        
\end{enumerate}
      \label{m38753*id255886}Standard notation shows the chemical symbol, the atomic mass number
and the atomic number of an element as follows:\par 
      \label{m38753*id255890}
    \setcounter{subfigure}{0}
	\begin{figure}[H] % horizontal\label{m38753*id255893}
    \begin{center}
    \label{m38753*id255893!!!underscore!!!media}\label{m38753*id255893!!!underscore!!!printimage}\includegraphics{col11305.imgs/m38753_atom_sym.png} % m38753;atom\_sym.png;;;6.0;8.5;
      \vspace{2pt}
    \vspace{.1in}
    \end{center}
 \end{figure}       
      \par 
      \label{m38753*eip-741}
\begin{tabular}{cc}
	\hspace*{-50pt}\raisebox{-8 mm}{\hspace{-0.2in}\includegraphics[width=0.75in]{col11305.imgs/psfact2.png} } & 
	\begin{minipage}{0.85\textwidth}
	\begin{note}
      {note: }A nuclide is a distinct kind of atom or nucleus characterized by the number of protons and neutrons in the atom. To be absolutely correct, when we represent atoms like we do here, then we should call them nuclides. 
	\end{note}
	\end{minipage}
	\end{tabular}
	\par
      \label{m38753*id255900}For example, the iron nucleus which has 26 protons and 30 neutrons, is
denoted as:\par 
      \label{m38753*id255904}\nopagebreak\noindent{}
        
    \begin{equation}
    _{26}^{56}\mathrm{Fe}\phantom{\rule{4pt}{0ex}}\tag{3.1}
      \end{equation}
      \label{m38753*id255929}where the atomic number is $Z=26$\hspace{1ex} and the mass number $A=56$.
The number of neutrons is simply the difference $N=A-Z$.\par 
        \label{m38753*id257512}The following worked examples will help you to understand the concept of an isotope better.\par \pagebreak
\label{m38753*secfhsst!!!underscore!!!id400}\vspace{.5cm} 
      \noindent
      \hspace*{-30pt}\includegraphics[width=0.5in]{col11305.imgs/pspencil2.png}   \raisebox{25mm}{   
      \begin{mdframed}[linewidth=4, leftmargin=40, rightmargin=40]  
      \begin{exercise}
    \noindent\textbf{Exercise 3.3:  Isotopes  }
        \label{m38753*probfhsst!!!underscore!!!id401}
        \label{m38753*id257529}For the element $_{92}^{234}\mathrm{U}$ (uranium), use standard notation to describe:\par 
        \label{m38753*id257558}\begin{enumerate}[noitemsep, label=\textbf{\arabic*}. ] 
            \leftskip=20pt\rightskip=\leftskip\label{m38753*uid41}\item the isotope with 2 fewer neutrons
\label{m38753*uid42}\item the isotope with 4 more neutrons
\end{enumerate}
        \vspace{5pt}
        \label{m38753*solfhsst!!!underscore!!!id411}\noindent\textbf{Solution to Exercise } \label{m38753*listfhsst!!!underscore!!!id411}\begin{enumerate}[noitemsep, label=\textbf{Step} \textbf{\arabic*}. ] 
            \leftskip=20pt\rightskip=\leftskip\item  
        \label{m38753*id257607}We know that isotopes of any element have the \textbf{same} number
of protons (same atomic number)
in each atom, which means that they have the same chemical symbol. However, they have a different number of neutrons, and therefore a different mass number.\par 
        \item  
        \label{m38753*id257640}Therefore, any isotope of uranium will have the symbol:\par 
        \label{m38753*id257644}\nopagebreak\noindent{}
          
    \begin{equation}
    \mathrm{U}\tag{3.2}
      \end{equation}
        \label{m38753*id257657}Also, since the number of protons in uranium isotopes is always the same, we can write
down the atomic number:\par 
        \label{m38753*id257664}\nopagebreak\noindent{}
          
    \begin{equation}
    {}_{92}\mathrm{U}\tag{3.3}
      \end{equation}
        \label{m38753*id257684}Now, if the isotope we want has 2 fewer neutrons than $_{92}^{234}\mathrm{U}$,
then we take the original mass number and subtract 2, which gives:\par 
        \label{m38753*id257713}\nopagebreak\noindent{}
          
    \begin{equation}
    _{92}^{232}\mathrm{U}\tag{3.4}
      \end{equation}
        \label{m38753*id257736}Following the steps above, we can write the isotope with 4 more neutrons as:\par 
        \label{m38753*id257740}\nopagebreak\noindent{}
          
    \begin{equation}
    _{92}^{238}\mathrm{U}\tag{3.5}
      \end{equation}
 \end{enumerate}
    \end{exercise}
    \end{mdframed}
    }
    \noindent
\par
            \label{m38753*secfhsst!!!underscore!!!id466}\vspace{.5cm} 
      \noindent
      \hspace*{-30pt}\includegraphics[width=0.5in]{col11305.imgs/pspencil2.png}   \raisebox{25mm}{   
      \begin{mdframed}[linewidth=4, leftmargin=40, rightmargin=40]  
      \begin{exercise}
    \noindent\textbf{Exercise 3.4:  Isotopes }
        \label{m38753*probfhsst!!!underscore!!!id467}
        \label{m38753*id257776}Which of the following are isotopes of $_{20}^{40}\mathrm{Ca}$?\par 
        \label{m38753*id257802}\begin{itemize}[noitemsep]
            \leftskip=20pt\rightskip=\leftskip\label{m38753*uid43}\item 
            $_{19}^{40}\mathrm{K}$
          \label{m38753*uid44}\item 
            $_{20}^{42}\mathrm{Ca}$
          \label{m38753*uid45}\item 
            $_{18}^{40}\mathrm{Ar}$
          \end{itemize}
        \vspace{5pt}
        \label{m38753*solfhsst!!!underscore!!!id509}\noindent\textbf{Solution to Exercise } \label{m38753*listfhsst!!!underscore!!!id509}\begin{enumerate}[noitemsep, label=\textbf{Step} \textbf{\arabic*}. ] 
            \leftskip=20pt\rightskip=\leftskip\item  
        \label{m38753*id257917}We know that isotopes have the same atomic number but different mass numbers.\par 
        \item  
        \label{m38753*id257943}You need to look for the element that has the same atomic number but a different atomic mass number. The only element is $_{20}^{42}Ca$. What is different is that there are 2 more neutrons than in the original element.\par 
\end{enumerate}
    \end{exercise}
    \end{mdframed}
    }
    \noindent
\label{m38753*secfhsst!!!underscore!!!id517}\vspace{.5cm} 
      \noindent
      \hspace*{-30pt}\includegraphics[width=0.5in]{col11305.imgs/pspencil2.png}   \raisebox{25mm}{   
      \begin{mdframed}[linewidth=4, leftmargin=40, rightmargin=40]  
      \begin{exercise}
    \noindent\textbf{Exercise 3.5:  Isotopes }
        \label{m38753*probfhsst!!!underscore!!!id518}
        \label{m38753*id257977}For the sulphur isotope $_{16}^{33}\mathrm{S}$, give the number of...\par 
        \label{m38753*id258000}\begin{enumerate}[noitemsep, label=\textbf{\alph*}. ] 
            \leftskip=20pt\rightskip=\leftskip\label{m38753*uid46}\item protons
\label{m38753*uid47}\item nucleons
\label{m38753*uid48}\item electrons
\label{m38753*uid49}\item neutrons
\end{enumerate}
        \vspace{5pt}
        \label{m38753*solfhsst!!!underscore!!!id532}\noindent\textbf{Solution to Exercise } \label{m38753*listfhsst!!!underscore!!!id532}\begin{enumerate}[noitemsep, label=\textbf{Step} \textbf{\arabic*}. ] 
            \leftskip=20pt\rightskip=\leftskip\item  
        \label{m38753*id258074}$Z=16$, therefore the number of protons is 16 (answer to (a)).
 \par 
        \item  
        \label{m38753*id258094}$A=33$, therefore the number of nucleons is 33 (answer to (b)).\par 
        \item  
        \label{m38753*id258102}The atom is neutral, and therefore the number of electrons is the same as the number of protons. The number of electrons is 16 (answer to (c)).\par 
        \item  
        \label{m38753*id258110}\nopagebreak\noindent{}
          
    \begin{equation}
    \begin{array}{c}\hfill N=A-Z=33-16=17\end{array}\tag{3.6}
      \end{equation}
        \label{m38753*id258151}The number of neutrons is 17 (answer to (d)).\par 
\end{enumerate}
    \end{exercise}
    \end{mdframed}
    }
    \noindent
% \label{m38753*secfhsst!!!underscore!!!id567}
% \par \raisebox{-5 pt}{\includegraphics[width=0.5cm]{col11305.imgs/summary_www.png}} Find the answers with the shortcodes:
%  \par \begin{tabular}[h]{cccccc}
%  (1.) llj  &  (2.) llb  &  (3.) llT  & \end{tabular}
            \subsubsection{  Isotopes }
            \nopagebreak
        \label{m38753*id258162}\begin{enumerate}[noitemsep, label=\textbf{\arabic*}. ] 
            \label{m38753*uid50}\item Atom A has 5 protons and 5 neutrons, and atom B has 6 protons and 5 neutrons. These atoms are...
\label{m38753*id258178}\begin{enumerate}[noitemsep, label=\textbf{\alph*}. ] 
            \label{m38753*uid51}\item allotropes
\label{m38753*uid52}\item isotopes
\label{m38753*uid53}\item isomers
\label{m38753*uid54}\item atoms of different elements
\end{enumerate}
                \label{m38753*uid55}\item For the sulphur isotopes, $_{16}^{32}\mathrm{S}$ and $_{16}^{34}\mathrm{S}$, give the number of...
\label{m38753*id258277}\begin{enumerate}[noitemsep, label=\textbf{\alph*}. ] 
            \label{m38753*uid56}\item protons
\label{m38753*uid57}\item nucleons
\label{m38753*uid58}\item electrons
\label{m38753*uid59}\item neutrons
\end{enumerate}
                \label{m38753*uid60}\item Which of the following are isotopes of $_{17}^{35}\mathrm{Cl}$?
\label{m38753*id258355}\begin{enumerate}[noitemsep, label=\textbf{\alph*}. ] 
            \label{m38753*uid61}\item $_{35}^{17}\mathrm{Cl}$
\label{m38753*uid62}\item $_{17}^{35}\mathrm{Cl}$
\label{m38753*uid63}\item $_{17}^{37}\mathrm{Cl}$
\end{enumerate}
                \label{m38753*uid64}\item Which of the following are isotopes of $\mathrm{U-}235$? (X represents an element symbol)
\label{m38753*id258452}\begin{enumerate}[noitemsep, label=\textbf{\alph*}. ] 
            \label{m38753*uid65}\item $_{92}^{238}\mathrm{X}$
\label{m38753*uid66}\item $_{90}^{238}\mathrm{X}$
\label{m38753*uid67}\item $_{92}^{235}\mathrm{X}$
\end{enumerate}
                \end{enumerate}
      \label{m38753*uid68}
\par \raisebox{-5 pt}{\includegraphics[width=0.5cm]{col11305.imgs/summary_www.png}} Find the answers with the shortcodes:
 \par \begin{tabular}[h]{cccccc}
 (1.) ll4  &  (2.) llZ  &  (3.) llW  &  (4.) llD  & \end{tabular}
            \subsubsection{ Relative atomic mass}
            \nopagebreak
            \label{m38753*id258557}It is important to realise that the atomic mass of isotopes of the same element will be different because they have a different number of nucleons. Chlorine, for example, has two common isotopes which are chlorine-35 and chlorine-37. Chlorine-35 has an atomic mass of 35 u, while chlorine-37 has an atomic mass of 37 u. In the world around us, both of these isotopes occur naturally. It doesn't make sense to say that the element chlorine has an atomic mass of 35 u, or that it has an atomic mass of 37 u. Neither of these are absolutely true since the mass varies depending on the form in which the element occurs. We need to look at how much more common one is than the other in order to calculate the \textbf{relative atomic mass} for the element chlorine. This is the number that you find on the Periodic Table.\par 
\label{m38753*eip-828}
\begin{tabular}{cc}
	\hspace*{-50pt}\raisebox{-8 mm}{\hspace{-0.2in}\includegraphics[width=0.75in]{col11305.imgs/psfact2.png} } & 
	\begin{minipage}{0.85\textwidth}
	\begin{note}
      {note: }\label{m38753*id97421}The relative atomic mass of some elements depends on where on Earth the element is found. This is because the isotopes can be found in varying ratios depending on certain factors such as geological composition, etc. The International Union of Pure and Applied Chemistry (IUPAC) has decided to give the relative atomic mass of some elements as a range to better represent the varying isotope ratios on the Earth. For the calculations that you will do at high school, it is enough to simply use one number without worrying about these ranges.\par 
	\end{note}
	\end{minipage}
	\end{tabular}
	\par
      \label{m38753*secfhsst!!!underscore!!!id602}\vspace{.5cm} 
      \noindent
      \hspace*{-30pt}\includegraphics[width=0.5in]{col11305.imgs/pspencil2.png}   \raisebox{25mm}{   
      \begin{mdframed}[linewidth=4, leftmargin=40, rightmargin=40]  
      \begin{exercise}
    \noindent\textbf{Exercise 3.6:  The relative atomic mass of an isotopic element }
        \label{m38753*probfhsst!!!underscore!!!id603}
        \label{m38753*id258606}The element chlorine has two isotopes, chlorine-35 and chlorine-37. The abundance of these isotopes when they occur naturally is 75\% chlorine-35 and 25\% chlorine-37. Calculate the \textsl{average} relative atomic mass for chlorine.
 \par 
        \vspace{5pt}
        \label{m38753*solfhsst!!!underscore!!!id607}\noindent\textbf{Solution to Exercise } \label{m38753*listfhsst!!!underscore!!!id607}\begin{enumerate}[noitemsep, label=\textbf{Step} \textbf{\arabic*}. ] 
            \leftskip=20pt\rightskip=\leftskip\item  
        \label{m38753*id258637}Contribution of $\mathrm{Cl-}35=\left(\frac{75}{100}\ensuremath{\times}35\right)=26,25\phantom{\rule{2pt}{0ex}}\mathrm{u}$
 \par 
        \item  
        \label{m38753*id258657}Contribution of $\mathrm{Cl-}37=\left(\frac{25}{100}\ensuremath{\times}37\right)=9,25\phantom{\rule{2pt}{0ex}}\mathrm{u}$\par 
        \item  
        \label{m38753*id258670}$\mathrm{Relative\; atomic\; mass\; of\; chlorine}=26,25\phantom{\rule{2pt}{0ex}}\mathrm{u}+9,25\phantom{\rule{2pt}{0ex}}\mathrm{u}=35,5\phantom{\rule{2pt}{0ex}}\mathrm{u}$\par 
        \label{m38753*id258674}If you look on the periodic table, the average relative atomic mass for chlorine is $35,5\phantom{\rule{2pt}{0ex}}\mathrm{u}$. You will notice that for many elements, the relative atomic mass that is shown is not a whole number. You should now understand that this number is the \textsl{average} relative atomic mass for those elements that have naturally occurring isotopes.\par 
\end{enumerate}
    \end{exercise}
    \end{mdframed}
    }
    \noindent
        \par \label{m38753*secfhsst!!!underscore!!!id618}
            \subsubsection{  Isotopes }
            \nopagebreak
        \label{m38753*id258699}\begin{enumerate}[noitemsep, label=\textbf{\arabic*}. ] 
            \label{m38753*uid69}\item Complete the table below:
    % \textbf{m38753*id258715}\par
          \begin{table}[H]
    % \begin{table}[H]
    % \\ 'id2889162' '1'
        \begin{center}
      \label{m38753*id258715}
    \noindent
    \tabletail{%
        \hline
        \multicolumn{6}{|p{\mytableboxwidth}|}{\raggedleft \small \sl continued on next page}\\
        \hline
      }
      \tablelasttail{}
      \begin{xtabular}[t]{|l|l|l|l|l|l|}\hline
        \textbf{Isotope} &
        \textbf{Z} &
        \textbf{A} &
        \textbf{Protons} &
        \textbf{Neutrons} &
        \textbf{Electrons}% make-rowspan-placeholders
     \tabularnewline\cline{1-1}\cline{2-2}\cline{3-3}\cline{4-4}\cline{5-5}\cline{6-6}
      %--------------------------------------------------------------------
        Carbon-12 &
         &
         &
         &
         &
        % make-rowspan-placeholders
     \tabularnewline\cline{1-1}\cline{2-2}\cline{3-3}\cline{4-4}\cline{5-5}\cline{6-6}
      %--------------------------------------------------------------------
        Carbon-14 &
         &
         &
         &
         &
        % make-rowspan-placeholders
     \tabularnewline\cline{1-1}\cline{2-2}\cline{3-3}\cline{4-4}\cline{5-5}\cline{6-6}
      %--------------------------------------------------------------------
        Chlorine-35 &
         &
         &
         &
         &
        % make-rowspan-placeholders
     \tabularnewline\cline{1-1}\cline{2-2}\cline{3-3}\cline{4-4}\cline{5-5}\cline{6-6}
      %--------------------------------------------------------------------
        Chlorine-37 &
         &
         &
         &
         &
        % make-rowspan-placeholders
     \tabularnewline\cline{1-1}\cline{2-2}\cline{3-3}\cline{4-4}\cline{5-5}\cline{6-6}
      %--------------------------------------------------------------------
    \end{xtabular}
      \end{center}
    \begin{center}{\small\bfseries Table 3.7}\end{center}
    \begin{caption}{\small\bfseries Table 3.7}\end{caption}
\end{table}
    \par
          \label{m38753*uid70}\item If a sample contains 90\% carbon-12 and 10\% carbon-14, calculate the relative atomic mass of an atom in that sample.
\hspace{1ex}        \label{m38753*uid71}\item If a sample contains 22,5\% Cl-37 and 77,5\% Cl-35, calculate the relative atomic mass of an atom in that sample.\hspace{1ex}        
\end{enumerate}
  \label{m38753**end}
\par \raisebox{-5 pt}{\includegraphics[width=0.5cm]{col11305.imgs/summary_www.png}} Find the answers with the shortcodes:
 \par \begin{tabular}[h]{cccccc}
 (1.) llj  &  (2.) llb  &  (3.) llT  & \end{tabular}
         \section{ Electronic structure}
    \nopagebreak
            \label{m38741} $ \hspace{-5pt}\begin{array}{cccccccccccc}   \includegraphics[width=0.75cm]{col11305.imgs/summary_fullmarks.png} &   \includegraphics[width=0.75cm]{col11305.imgs/summary_simulation.png} &   \includegraphics[width=0.75cm]{col11305.imgs/summary_video.png} &   \includegraphics[width=0.75cm]{col11305.imgs/summary_presentation.png} &   \end{array} $ \hspace{2 pt}\raisebox{-5 pt}{} {(section shortcode: P10023 )} \par 
    \label{m38741*cid7}
            \subsection{ Electron configuration}
            \nopagebreak
      \label{m38741*uid79}
            \subsubsection{ The energy of electrons}
            \nopagebreak
        \label{m38741*id259210}You will remember from our earlier discussions that an atom is made up of a central nucleus, which contains protons and neutrons and that this nucleus is surrounded by electrons. Although these electrons all have the same charge and the same mass, each electron in an atom has a different amount of \textsl{energy}. Electrons that have the \textsl{lowest} energy are found closest to the nucleus where the attractive force of the positively charged nucleus is the greatest. Those electrons that have \textsl{higher} energy, and which are able to overcome the attractive force of the nucleus, are found further away.\par 
      \label{m38741*uid81}
            \subsubsection{ Electron configuration}
            \nopagebreak
            \label{m38741*id9722401}We will start with a very simple view of the arrangement or configuration of electrons around an atom. This view simply states that electrons are arranged in energy levels (or shells) around the nucleus of an atom. These energy levels are numbered 1, 2, 3, etc. Electrons that are in the first energy level (energy level 1) are closest to the nucleus and will have the lowest energy. Electrons further away from the nucleus will have a higher energy. \par 
\label{m38741*id259357}In the following examples, the energy levels are shown as concentric circles around the central nucleus. The important thing to know for these diagrams is that the first energy level can hold 2 electrons, the second energy level can hold 8 electrons and the third energy level can hold 8 electrons.\par 
        \label{m38741*id259361}\begin{enumerate}[noitemsep, label=\textbf{\arabic*}. ] 
            \label{m38741*uid86}\item \textbf{Lithium}
Lithium (Li) has an atomic number of 3, meaning that in a neutral atom, the number of electrons will also be 3. The first two electrons are found in the first energy level, while the third electron is found in the second energy level (Figure~3.9).
    \setcounter{subfigure}{0}
	\begin{figure}[H] % horizontal\label{m38741*uid87}
    \begin{center}
    \label{m38741*uid87!!!underscore!!!media}\label{m38741*uid87!!!underscore!!!printimage}\includegraphics[width=300px]{col11305.imgs/m38741_CG10C3_005.png} % m38741;CG10C3\_005.png;;;6.0;8.5;
      \vspace{2pt}
    \vspace{\rubberspace}\par \begin{cnxcaption}
	  \small \textbf{Figure 3.9: }The arrangement of electrons in a lithium atom.
	\end{cnxcaption}
    \vspace{.1in}
    \end{center}
 \end{figure}       \label{m38741*uid88}\item \textbf{Fluorine}
Fluorine ($\mathrm{F}$) has an atomic number of 9, meaning that a neutral atom also has 9 electrons. The first 2 electrons are found in the first energy level, while the other 7 are found in the second energy level (Figure~3.10).
    \setcounter{subfigure}{0}
	\begin{figure}[H] % horizontal\label{m38741*uid89}
    \begin{center}
    \label{m38741*uid89!!!underscore!!!media}\label{m38741*uid89!!!underscore!!!printimage}\includegraphics[width=2cm]{col11305.imgs/m38741_CG10C3_006.png} % m38741;CG10C3\_006.png;;;6.0;8.5;
      \vspace{2pt}
    \vspace{\rubberspace}\par \begin{cnxcaption}
	  \small \textbf{Figure 3.10: }The arrangement of electrons in a fluorine atom.
	\end{cnxcaption}
    \vspace{.1in}
    \end{center}
 \end{figure}       \label{m38741*uid90}\item \textbf{Argon}
Argon has an atomic number of 18, meaning that a neutral atom also has 18 electrons. The first 2 electrons are found in the first energy level, the next 8 are found in the second energy level, and the last 8 are found in the third energy level (Figure~3.11).
    \setcounter{subfigure}{0}
	\begin{figure}[H] % horizontal\label{m38741*uid91}
    \begin{center}
    \label{m38741*uid91!!!underscore!!!media}\label{m38741*uid91!!!underscore!!!printimage}\includegraphics[width=2cm]{col11305.imgs/m38741_CG10C3_007.png} % m38741;CG10C3\_007.png;;;6.0;8.5;
      \vspace{2pt}
    \vspace{\rubberspace}\par \begin{cnxcaption}
	  \small \textbf{Figure 3.11: }The arrangement of electrons in an argon atom.
	\end{cnxcaption}
    \vspace{.1in}
    \end{center}
 \end{figure}       \end{enumerate}
\label{m38741*id259478}But the situation is slightly more complicated than this. Within each energy level, the electrons move in \textbf{orbitals}. An orbital defines the spaces or regions where electrons move.\par 
\label{m38741*fhsst!!!underscore!!!id687}\begin{definition}
	  \begin{tabular*}{15 cm}{m{15 mm}m{}}
	\hspace*{-50pt}  \includegraphics[width=0.5in]{col11305.imgs/psflag2.png}   & \Definition{   \label{id2420554}\textbf{ Atomic orbital }} { \label{m38741*meaningfhsst!!!underscore!!!id687}
        \label{m38741*id259495}An atomic orbital is the region in which an electron may be found around a single atom.
 \par 
         } 
      \end{tabular*}
      \end{definition}
\label{m38741*id6732}
The first energy level contains only one 's' orbital, the second energy level contains one 's' orbital and three 'p' orbitals and the third energy level contains one 's' orbital and three 'p' orbitals (as well as 5 'd' orbitals). Within each energy level, the 's' orbital is at a lower energy than the 'p' orbitals. This arrangement is shown in Figure~3.12.\par 
    \setcounter{subfigure}{0}
	\begin{figure}[H] % horizontal\label{m38741*uid92}
    \begin{center}
    \rule[.1in]{\figurerulewidth}{.005in} \\
        \label{m38741*uid92!!!underscore!!!media}\label{m38741*uid92!!!underscore!!!printimage}\includegraphics[height=300px]{col11305.imgs/m38741_CG10C3_008.png} % m38741;CG10C3\_008.png;;;6.0;8.5;
      \vspace{2pt}
    \vspace{\rubberspace}\par \begin{cnxcaption}
	  \small \textbf{Figure 3.12: }The positions of the first ten orbitals of an atom on an energy diagram. Note that each block is able to hold two electrons.
	\end{cnxcaption}
    \vspace{.1in}
    \rule[.1in]{\figurerulewidth}{.005in} \\
    \end{center}
 \end{figure}       
        \label{m38741*eip-752}This diagram also helps us when we are working out the electron configuration of an element. The electron configuration of an element is the arrangement of the electrons in the shells and subshells. There are a few guidelines for working out the electron configuration. These are:
\par \label{m38741*id259303}\begin{itemize}[noitemsep]
            \label{m38741*uid93}\item Each orbital can only hold \textbf{two electrons}. Electrons that occur together in an orbital are called an \textbf{electron pair}.
\label{m38741*uid94}\item An electron will always try to enter an orbital with the lowest possible energy.
\label{m38741*uid95}\item An electron will occupy an orbital on its own, rather than share an orbital with another electron. An electron would also rather occupy a lower energy orbital \textsl{with} another electron, before occupying a higher energy orbital. In other words, within one energy level, electrons will fill an 's' orbital before starting to fill 'p' orbitals.
\label{m38741*uid83}\item The s subshell can hold 2 electrons
\label{m38741*uid84}\item The p subshell can hold 6 electrons
\end{itemize}
        \label{m38741*eip-15}In the examples you will cover, you will mainly be filling the s and p subshells. Occasionally you may get an example that has the d subshell. The f subshell is more complex and is not covered at this level.
\par 
        \label{m38741*id259599}The way that electrons are arranged in an atom is called its \textbf{electron configuration}.\par 
\label{m38741*fhsst!!!underscore!!!id709}\begin{definition}
	  \begin{tabular*}{15 cm}{m{15 mm}m{}}
	\hspace*{-50pt}  \includegraphics[width=0.5in]{col11305.imgs/psflag2.png}   & \Definition{   \label{id2420737}\textbf{ Electron configuration }} { \label{m38741*meaningfhsst!!!underscore!!!id709}
        \label{m38741*id259615}Electron configuration is the arrangement of electrons in an atom, molecule or other physical structure. \par 
         } 
      \end{tabular*}
      \end{definition}
        \label{m38741*id259628}An element's electron configuration can be represented using \textbf{Aufbau diagrams} or energy level diagrams. An Aufbau diagram uses arrows to represent electrons. You can use the following steps to help you to draw an Aufbau diagram:\par 
        \label{m38741*id259639}\begin{enumerate}[noitemsep, label=\textbf{\arabic*}. ] 
            \label{m38741*uid96}\item Determine the number of electrons that the atom has.
\label{m38741*uid97}\item Fill the 's' orbital in the first energy level (the $1\mathrm{s}$ orbital) with the first two electrons.
\label{m38741*uid98}\item Fill the 's' orbital in the second energy level (the $2\mathrm{s}$ orbital) with the second two electrons.
\label{m38741*uid99}\item Put one electron in each of the three 'p' orbitals in the second energy level (the $2\mathrm{p}$ orbitals) and then if there are still electrons remaining, go back and place a second electron in each of the $2\mathrm{p}$ orbitals to complete the electron pairs.
\label{m38741*uid100}\item Carry on in this way through each of the successive energy levels until all the electrons have been drawn.
\end{enumerate}
\label{m38741*notfhsst!!!underscore!!!id725}
\begin{tabular}{cc}
	   \hspace*{-50pt}\raisebox{-8 mm}{ \includegraphics[width=0.5in]{col11305.imgs/pstip2.png}  }& 
	\begin{minipage}{0.85\textwidth}
	\begin{note}
      {tip: }When there are two electrons in an orbital, the electrons are called an \textbf{electron pair}. If the orbital only has one electron, this electron is said to be an \textbf{unpaired electron}. Electron pairs are shown with arrows pointing in opposite directions.
	\end{note}
	\end{minipage}
	\end{tabular}
	\par
        \label{m38741*eip-770}
\begin{tabular}{cc}
	\hspace*{-50pt}\raisebox{-8 mm}{\hspace{-0.2in}\includegraphics[width=0.75in]{col11305.imgs/psfact2.png} } & 
	\begin{minipage}{0.85\textwidth}
	\begin{note}
      {note: }Aufbau is the German word for 'building up'. Scientists used this term since this is exactly what we are doing when we work out electron configuration, we are building up the atoms structure.
	\end{note}
	\end{minipage}
	\end{tabular}
	\par
      \label{m38741*eip-873}You can think of Aufbau diagrams as being similar to people getting on a bus or a train. People will first sit in empty seats with empty seats between them and the other people (unless they know the people and then they will sit next to them). When all the seats are filled like this, any more people that get on will be forced to sit next to someone or stand. As the bus or train fills even more the people have to stand to fit on. \par \label{m38741*id259728}An Aufbau diagram for the element Lithium is shown in Figure~3.13.\par 
    \setcounter{subfigure}{0}
	\begin{figure}[H] % horizontal\label{m38741*uid101}
    \begin{center}
    \rule[.1in]{\figurerulewidth}{.005in} \\
        \label{m38741*uid101!!!underscore!!!media}\label{m38741*uid101!!!underscore!!!printimage}\includegraphics[width=.9cm]{col11305.imgs/m38741_CG10C3_009.png} % m38741;CG10C3\_009.png;;;6.0;8.5;
      \vspace{2pt}
    \vspace{\rubberspace}\par \begin{cnxcaption}
	  \small \textbf{Figure 3.13: }The electron configuration of Lithium, shown on an Aufbau diagram
	\end{cnxcaption}
    \vspace{.1in}
    \rule[.1in]{\figurerulewidth}{.005in} \\
    \end{center}
 \end{figure}       
        \label{m38741*id259749}A special type of notation is used to show an atom's electron configuration. The notation describes the energy levels, orbitals and the number of electrons in each. For example, the electron configuration of lithium is ${1\mathrm{s}}^{2}{2\mathrm{s}}^{1}$. The number and letter describe the energy level and orbital and the number above the orbital shows how many electrons are in that orbital.\par 
        \label{m38741*id259782}Aufbau diagrams for the elements fluorine and argon are shown in Figure~3.14 and Figure~3.15 respectively. Using standard notation, the electron configuration of fluorine is ${1\mathrm{s}}^{2}{2\mathrm{s}}^{2}{2\mathrm{p}}^{5}$ and the electron configuration of argon is ${1\mathrm{s}}^{2}{2\mathrm{s}}^{2}{2\mathrm{p}}^{6}$.\par 
    \setcounter{subfigure}{0}
	\begin{figure}[H] % horizontal\label{m38741*uid102}
    \begin{center}
    \rule[.1in]{\figurerulewidth}{.005in} \\
        \label{m38741*uid102!!!underscore!!!media}\label{m38741*uid102!!!underscore!!!printimage}\includegraphics[height=300px]{col11305.imgs/m38741_CG10C3_010.png} % m38741;CG10C3\_010.png;;;6.0;8.5;
      \vspace{2pt}
    \vspace{\rubberspace}\par \begin{cnxcaption}
	  \small \textbf{Figure 3.14: }An Aufbau diagram showing the electron configuration of fluorine
	\end{cnxcaption}
    \vspace{.1in}
    \rule[.1in]{\figurerulewidth}{.005in} \\
    \end{center}
 \end{figure}       
    \setcounter{subfigure}{0}
	\begin{figure}[H] % horizontal\label{m38741*uid103}
    \begin{center}
    \rule[.1in]{\figurerulewidth}{.005in} \\
        \label{m38741*uid103!!!underscore!!!media}\label{m38741*uid103!!!underscore!!!printimage}\includegraphics[height=300px]{col11305.imgs/m38741_CG10C3_011.png} % m38741;CG10C3\_011.png;;;6.0;8.5;
      \vspace{2pt}
    \vspace{\rubberspace}\par \begin{cnxcaption}
	  \small \textbf{Figure 3.15: }An Aufbau diagram showing the electron configuration of argon
	\end{cnxcaption}
    \vspace{.1in}
    \rule[.1in]{\figurerulewidth}{.005in} \\
    \end{center}
 \end{figure}       
      \par
            \label{m38741*eip-138}\vspace{.5cm} 
      \noindent
      \hspace*{-30pt}\includegraphics[width=0.5in]{col11305.imgs/pspencil2.png}   \raisebox{25mm}{   
      \begin{mdframed}[linewidth=4, leftmargin=40, rightmargin=40]  
      \begin{exercise}
    \noindent\textbf{Exercise 3.7: Aufbau diagrams}\label{m38741*eip-483}
  \label{m38741*eip-491}Give the electron configuration for sodium ($Na$) and draw an aufbau diagram.
  \par 
\vspace{5pt}
\label{m38741*eip-119}\noindent\textbf{Solution to Exercise }
\label{m38741*listfhsst!!!underscore!!!id667}\begin{enumerate}[noitemsep, label=\textbf{Step} \textbf{\arabic*}. ] 
            \leftskip=20pt\rightskip=\leftskip\item Sodium has 11 electrons.\item We start by placing two electrons in the $1s$ orbital: ${1s}^{2}$. Now we have 9 electrons left to place in orbitals, so we put two in the $2s$ orbital: ${2s}^{2}$. There are now 7 electrons to place in orbitals so we place 6 of them in the $2p$ orbital: ${2p}^{6}$. The last electron goes into the $3s$ orbital: ${3s}^{1}$.\item The electron configuration is: ${1s}^{2}{2s}^{2}{2p}^{6}{3s}^{1}$\item Using the electron configuration we get the following diagram:
    \setcounter{subfigure}{0}
	\begin{figure}[H] % horizontal\label{m38741*uid847}
    \begin{center}
    \label{m38741*uid847!!!underscore!!!media}\label{m38741*uid847!!!underscore!!!printimage}\includegraphics[width=3cm]{col11305.imgs/m38126_wexaufbau.png} % ;wexaufbau.png;;;6.0;8.5;
      \vspace{2pt}
    \vspace{.1in}
    \end{center}
 \end{figure}       \end{enumerate}
    \end{exercise}
    \end{mdframed}
    }
    \noindent
  \label{m38741*eip-793}There are different orbital shapes, but we will be mainly dealing with only two. These are the 's' and 'p' orbitals (there are also 'd' and 'f' orbitals). The 's' orbitals are spherical and the 'p' orbitals are dumbbell shaped. 
    \setcounter{subfigure}{0}
	\begin{figure}[H] % horizontal\label{m38741*id8245}
    \begin{center}
    \label{m38741*uid8934!!!underscore!!!media}\label{m38741*uid8934!!!underscore!!!printimage}\includegraphics[width=5cm]{col11305.imgs/m38741_orbitals.png} % m38741;orbitals.png;;;6.0;8.5;
      \vspace{2pt}
    \vspace{\rubberspace}\par \begin{cnxcaption}
	  \small \textbf{Figure 3.17: }The shapes of orbitals. a) shows an 's' orbital, b) shows a single 'p' orbital and c) shows the three 'p' orbitals.
	\end{cnxcaption}
    \vspace{.1in}
    \end{center}
 \end{figure}       \par \label{m38741*eip-581}
            \subsubsection{ Hund's rule and Pauli's principle}
            \nopagebreak
            \label{m38741*eip-188}
Sometimes people refer to Hund's rule for electron configuration. This rule simply says that electrons would rather be in a subshell on their own than share a subshell. This is why when you are filling the subshells you put one electron in each subshell and then go back and fill the subshell, before moving onto the next energy level.
\par 
\label{m38741*eip-id1167385514309}
Pauli's exclusion principle simply states that electrons have a property known as spin and that two electrons in a subshell will not spin the same way. This is why we draw electrons as one arrow pointing up and one arrow pointing down.
\par \label{m38741*uid104}
            \subsubsection{ Core and valence electrons}
            \nopagebreak
        \label{m38741*id259935}Electrons in the outermost energy level of an atom are called \textbf{valence electrons}. The electrons that are in the energy shells closer to the nucleus are called \textbf{core electrons}. Core electrons are all the electrons in an atom, excluding the valence electrons. An element that has its valence energy level full is \textsl{more stable} and \textsl{less likely to react} than other elements with a valence energy level that is not full.\par 
\label{m38741*fhsst!!!underscore!!!id755}\begin{definition}
	  \begin{tabular*}{15 cm}{m{15 mm}m{}}
	\hspace*{-50pt}  \includegraphics[width=0.5in]{col11305.imgs/psflag2.png}   & \Definition{   \label{id2421518}\textbf{ Valence electrons }} { \label{m38741*meaningfhsst!!!underscore!!!id755}
        \label{m38741*id259971}The electrons in the outer energy level of an atom \par 
         } 
      \end{tabular*}
      \end{definition}
\label{m38741*fhsst!!!underscore!!!id758}\begin{definition}
	  \begin{tabular*}{15 cm}{m{15 mm}m{}}
	\hspace*{-50pt}  \includegraphics[width=0.5in]{col11305.imgs/psflag2.png}   & \Definition{   \label{id2421542}\textbf{ Core electrons }} { \label{m38741*meaningfhsst!!!underscore!!!id758}
        \label{m38741*id259989}All the electrons in an atom, excluding the valence electrons \par 
         } 
      \end{tabular*}
      \end{definition}
      \label{m38741*uid105}
            \subsubsection{ The importance of understanding electron configuration}
            \nopagebreak
        \label{m38741*id260011}By this stage, you may well be wondering why it is important for you to understand how electrons are arranged around the nucleus of an atom. Remember that during chemical reactions, when atoms come into contact with one another, it is the \textsl{electrons} of these atoms that will interact first. More specifically, it is the \textbf{valence electrons} of the atoms that will determine how they react with one another.\par 
        \label{m38741*id260029}To take this a step further, an atom is at its most stable (and therefore \textsl{unreactive}) when all its orbitals are full. On the other hand, an atom is least stable (and therefore most \textsl{reactive}) when its valence electron orbitals are not full. This will make more sense when we go on to look at chemical bonding in a later chapter. To put it simply, the valence electrons are largely responsible for an element's chemical behaviour and elements that have the same number of valence electrons often have similar chemical properties.\par 
\label{m38741*eip-106}One final point to note about electron configurations is stability. Which configurations are stable and which are not? Very simply, the most stable configurations are the ones that have full energy levels. These configurations occur in the noble gases. The noble gases are very stable elements that do not react easily (if at all) with any other elements. This is due to the full energy levels. All elements would like to reach the most stable electron configurations, i.e. all elements want to be noble gases. This principle of stability is sometimes referred to as the octet rule. An octet is a set of 8, and the number of electrons in a full energy level is 8. \par \label{m38741*eip-739}
            \subsubsection{ Experiment: Flame tests}
            \nopagebreak
            \label{m38741*eip-699}\noindent{}\textbf{Aim:}\newline
    To determine what colour a metal cation will cause a flame to be.
\par 
\label{m38741*eip-6991}\noindent{}\textbf{Apparatus:}\newline
    Watch glass, bunsen burner, methanol, bamboo sticks, metal salts (e.g. $\mathrm{NaCl}$, ${\mathrm{CuCl}}_{2}$, ${\mathrm{CaCl}}_{2}$, $\mathrm{KCl}$, etc. ) and metal powders (e.g. copper, magnesium, zinc, iron, etc.)
\par 
\label{m38741*eip-479}
\begin{tabular}{cc}
	\hspace*{-50pt}\raisebox{-8 mm}{\hspace{-0.2in}\includegraphics[width=0.75in]{col11305.imgs/psfact2.png} } & 
	\begin{minipage}{0.85\textwidth}
	\begin{note}
      {warning: }Be careful when working with bunsen burners as you can easily burn yourself. Make sure all scarves/loose clothing is securely tucked in and long hair is tied back. Ensure that you work in a well-ventilated space and that there is nothing flammable near the open flame.
	\end{note}
	\end{minipage}
	\end{tabular}
	\par
      \label{m38741*eip-6992}\noindent{}\textbf{Method:}\newline
    For each salt or powder do the following: \label{m38741*id7092}\begin{enumerate}[noitemsep, label=\textbf{\arabic*}. ] 
            \item Dip a clean bamboo stick into the methanol\item Dip the stick into the salt or powder\item Wave the stick through the flame from the bunsen burner. DO NOT hold the stick in the flame, but rather wave it back and forth through the flame.\item Observe what happens\end{enumerate}
\par 
\label{m38741*eip-6993}\noindent{}\textbf{Results:}\newline
    Record your results in a table, listing the metal salt and the colour of the flame.
\par 
\label{m38741*eip-6994}\noindent{}\textbf{Conclusion:}\newline
    You should have observed different colours for each of the metal salts and powders that you tested.\par \label{m38741*eip-378}The above experiment on flame tests relates to the line emission spectra of the metals. These line emission spectra are a direct result of the arrangement of the electrons in metals.\par \label{m38741*secfhsst!!!underscore!!!id766}
            \subsubsection{  Energy diagrams and electrons
        }
            \nopagebreak
        \label{m38741*id260063}\begin{enumerate}[noitemsep, label=\textbf{\arabic*}. ] 
            \label{m38741*uid106}\item Draw Aufbau diagrams to show the electron configuration of each of the following elements:
\label{m38741*id260079}\begin{enumerate}[noitemsep, label=\textbf{\alph*}. ] 
            \label{m38741*uid107}\item magnesium
\label{m38741*uid108}\item potassium
\label{m38741*uid109}\item sulphur
\label{m38741*uid110}\item neon
\label{m38741*uid111}\item nitrogen
\end{enumerate}
        \label{m38741*uid112}\item Use the Aufbau diagrams you drew to help you complete the following table:
    % \textbf{m38741*id260157}\par
          \begin{table}[H]
    % \begin{table}[H]
    % \\ 'id2891050' '1'
        \begin{center}
      \label{m38741*id260157}
    \noindent
    \tabletail{%
        \hline
        \multicolumn{5}{|p{\mytableboxwidth}|}{\raggedleft \small \sl continued on next page}\\
        \hline
      }
      \tablelasttail{}
      \begin{xtabular}[t]{|l|l|l|l|l|}\hline
        \textbf{Element} &
        \textbf{No. of energy levels} &
        \textbf{No. of core electrons} &
        \textbf{No. of valence electrons} &
        \textbf{Electron configuration (standard notation)}% make-rowspan-placeholders
     \tabularnewline\cline{1-1}\cline{2-2}\cline{3-3}\cline{4-4}\cline{5-5}
      %--------------------------------------------------------------------
        $\mathrm{Mg}$ &
         &
         &
         &
        % make-rowspan-placeholders
     \tabularnewline\cline{1-1}\cline{2-2}\cline{3-3}\cline{4-4}\cline{5-5}
      %--------------------------------------------------------------------
        $\mathrm{K}$ &
         &
         &
         &
        % make-rowspan-placeholders
     \tabularnewline\cline{1-1}\cline{2-2}\cline{3-3}\cline{4-4}\cline{5-5}
      %--------------------------------------------------------------------
        $\mathrm{S}$ &
         &
         &
         &
        % make-rowspan-placeholders
     \tabularnewline\cline{1-1}\cline{2-2}\cline{3-3}\cline{4-4}\cline{5-5}
      %--------------------------------------------------------------------
        $\mathrm{Ne}$ &
         &
         &
         &
        % make-rowspan-placeholders
     \tabularnewline\cline{1-1}\cline{2-2}\cline{3-3}\cline{4-4}\cline{5-5}
      %--------------------------------------------------------------------
        $\mathrm{N}$ &
         &
         &
         &
        % make-rowspan-placeholders
     \tabularnewline\cline{1-1}\cline{2-2}\cline{3-3}\cline{4-4}\cline{5-5}
      %--------------------------------------------------------------------
    \end{xtabular}
      \end{center}
    \begin{center}{\small\bfseries Table 3.8}\end{center}
    \begin{caption}{\small\bfseries Table 3.8}\end{caption}
\end{table}
    \par
  \label{m38741*uid113}\item Rank the elements used above in order of \textsl{increasing reactivity}. Give reasons for the order you give.
 Click here for the answer\footnote{http://www.fhsst.org/ll2}
        \end{enumerate}
\label{m38741*secfhsst!!!underscore!!!id783}
            \subsubsection{ Group work : Building a model of an atom         }
            \nopagebreak
            \label{m38741*id260472}Earlier in this chapter, we talked about different 'models' of the atom. In science, one of the uses of models is that they can help us to understand the structure of something that we can't see. In the case of the atom, models help us to build a picture in our heads of what the atom looks like.\par 
        \label{m38741*id260480}Models are often simplified. The small toy cars that you may have played with as a child are models. They give you a good idea of what a real car looks like, but they are much smaller and much simpler. A model cannot always be absolutely accurate and it is important that we realise this so that we don't build up a false idea about something.\par 
        \label{m38741*id260488}In groups of 4-5, you are going to build a model of an atom. Before you start, think about these questions:\par 
        \label{m38741*id260495}\begin{itemize}[noitemsep]
            \label{m38741*uid114}\item What information do I know about the structure of the atom? (e.g. what parts make it up? how big is it?)
\label{m38741*uid115}\item What materials can I use to represent these parts of the atom as accurately as I can?
\label{m38741*uid116}\item How will I put all these different parts together in my model?
\end{itemize}
        \label{m38741*id260537}As a group, share your ideas and then plan how you will build your model. Once you have built your model, discuss the following questions:\par 
        \label{m38741*id260542}\begin{itemize}[noitemsep]
            \label{m38741*uid117}\item Does our model give a good idea of what the atom actually looks like?
\label{m38741*uid118}\item In what ways is our model \textsl{inaccurate}? For example, we know that electrons \textsl{move} around the atom's nucleus, but in your model, it might not have been possible for you to show this.
\label{m38741*uid119}\item Are there any ways in which our model could be improved?
\end{itemize}
        \label{m38741*id260596}Now look at what other groups have done. Discuss the same questions for each of the models you see and record your answers. \par 
      \label{m38741*eip-816}The following simulation allows you to build an atom\newline
    \setcounter{subfigure}{0}
	\begin{figure}[H] % horizontal\label{m38806*transverse-waves}
    \textnormal{Phet simulation for building atoms}\vspace{.1in} \nopagebreak
  \label{m38806*phet!!!underscore!!!sim}\label{m38806*phet-simulation}
            \raisebox{-5 pt}{ \includegraphics[width=0.5cm]{col11305.imgs/summary_www.png}} { (Simulation:  lbb )}
      \vspace{2pt}
    \vspace{.1in}
 \end{figure}       
\par \label{m38741*eip-583}This is another simulation that allows you to build an atom. This simulation also provides a summary of what you have learnt so far. \newline
    \setcounter{subfigure}{0}
	\begin{figure}[H] % horizontal\label{m38806*transverse-waves}
    \textnormal{Simulation for building atoms 2}\vspace{.1in} \nopagebreak
  \label{m38806*phet!!!underscore!!!sim}\label{m38806*phet-simulation}
            \raisebox{-5 pt}{ \includegraphics[width=0.5cm]{col11305.imgs/summary_www.png}} { (Simulation:  lbj )}
      \vspace{2pt}
    \vspace{.1in}
 \end{figure}       
    \par 
    \label{m38741*cid10}
            \subsection{ Summary}
            \nopagebreak
      \label{m38741*id262657}\begin{itemize}[noitemsep]
            \label{m38741*uid169}\item Much of what we know today about the atom, has been the result of the work of a number of scientists who have added to each other's work to give us a good understanding of atomic structure.
\label{m38741*uid170}\item Some of the important scientific contributors include \textbf{J.J.Thomson} (discovery of the electron, which led to the Plum Pudding Model of the atom), \textbf{Ernest Rutherford} (discovery that positive charge is concentrated in the centre of the atom) and \textbf{Niels Bohr} (the arrangement of electrons around the nucleus in energy levels).
\label{m38741*uid171}\item Because of the very small mass of atoms, their mass is measured in \textbf{atomic mass units} (u). $1\phantom{\rule{2pt}{0ex}}u=1,67\ensuremath{\times}10{}^{-24}\phantom{\rule{2pt}{0ex}}g$.
\label{m38741*uid172}\item An atom is made up of a central \textbf{nucleus} (containing \textbf{protons} and \textbf{neutrons}), surrounded by \textbf{electrons}.
\label{m38741*uid173}\item The \textbf{atomic number} (Z) is the number of protons in an atom.
\label{m38741*uid174}\item The \textbf{atomic mass number} (A) is the number of protons and neutrons in the nucleus of an atom.
\label{m38741*uid175}\item The \textbf{standard notation} that is used to write an element, is $_{Z}^{A}\mathrm{X}$, where X is the element symbol, A is the atomic mass number and Z is the atomic number.
\label{m38741*uid176}\item The \textbf{isotope} of a particular element is made up of atoms which have the same number of protons as the atoms in the original element, but a different number of neutrons. This means that not all atoms of an element will have the same atomic mass.
\label{m38741*uid177}\item The \textbf{relative atomic mass} of an element is the average mass of one atom of all the naturally occurring isotopes of a particular chemical element, expressed in atomic mass units. The relative atomic mass is written under the elements' symbol on the Periodic Table.
\label{m38741*uid178}\item The energy of electrons in an atom is \textbf{quantised}. Electrons occur in specific energy levels around an atom's nucleus.
\label{m38741*uid179}\item Within each energy level, an electron may move within a particular shape of \textbf{orbital}. An orbital defines the space in which an electron is most likely to be found. There are different orbital shapes, including s, p, d and f orbitals.
\label{m38741*uid180}\item Energy diagrams such as \textbf{Aufbau diagrams} are used to show the electron configuration of atoms.
\label{m38741*uid181}\item The electrons in the outermost energy level are called \textbf{valence electrons}.
\label{m38741*uid182}\item The electrons that are not valence electrons are called \textbf{core electrons}.
\label{m38741*uid183}\item Atoms whose outermost energy level is full, are less chemically reactive and therefore more stable, than those atoms whose outer energy level is not full.
\end{itemize}
\label{m38741*eip-867}
    \setcounter{subfigure}{0}
	\begin{figure}[H] % horizontal\label{m38741*slidesharefigure2}
    \label{m38741*slidesharemedia2}\label{m38741*slideshareflash2}
            \raisebox{-5 pt}{ \includegraphics[width=0.5cm]{col11305.imgs/summary_www.png}} { (Presentation:  P10024 )}
      \vspace{2pt}
    \vspace{.1in}
 \end{figure}       \par \label{m38741*secfhsst!!!underscore!!!id1140}
            \subsubsection{ End of chapter exercises}
            \nopagebreak
      \label{m38741*id263110}\begin{enumerate}[noitemsep, label=\textbf{\arabic*}. ] 
            \label{m38741*uid189}\item Write down only the word/term for each of the following descriptions.
\label{m38741*id263126}\begin{enumerate}[noitemsep, label=\textbf{\alph*}. ] 
            \label{m38741*uid190}\item The sum of the number of protons and neutrons in an atom
\label{m38741*uid191}\item The defined space around an atom's nucleus, where an electron is most likely to be found
\end{enumerate}
                \label{m38741*uid192}\item For each of the following, say whether the statement is True or False. If it is False, re-write the statement correctly.
\label{m38741*id263169}\begin{enumerate}[noitemsep, label=\textbf{\alph*}. ] 
            \label{m38741*uid193}\item $_{10}^{20}\mathrm{Ne}$ and $_{10}^{22}\mathrm{Ne}$ each have 10 protons, 12 electrons and 12 neutrons.
\label{m38741*uid194}\item The atomic mass of any atom of a particular element is always the same.
\label{m38741*uid195}\item It is safer to use helium gas rather than hydrogen gas in balloons.
\label{m38741*uid196}\item Group 1 elements readily form negative ions.
\end{enumerate}
                \label{m38741*uid197}\item Multiple choice questions: In each of the following, choose the \textbf{one} correct answer.
\label{m38741*id263273}\begin{enumerate}[noitemsep, label=\textbf{\alph*}. ] 
            \label{m38741*uid198}\item The three basic components of an atom are:
\label{m38741*id263289}\begin{enumerate}[noitemsep, label=\textbf{\alph*}. ] 
            \label{m38741*uid199}\item protons, neutrons, and ions
\label{m38741*uid200}\item protons, neutrons, and electrons
\label{m38741*uid201}\item protons, neutrinos, and ions
\label{m38741*uid202}\item protium, deuterium, and tritium
\end{enumerate}
                \label{m38741*uid203}\item The charge of an atom is...
\label{m38741*id263355}\begin{enumerate}[noitemsep, label=\textbf{\alph*}. ] 
            \label{m38741*uid204}\item positive
\label{m38741*uid205}\item neutral
\label{m38741*uid206}\item negative
\end{enumerate}
                \label{m38741*uid207}\item If Rutherford had used neutrons instead of alpha particles in his scattering experiment, the neutrons would...
\label{m38741*id263410}\begin{enumerate}[noitemsep, label=\textbf{\alph*}. ] 
            \label{m38741*uid208}\item not deflect because they have no charge
\label{m38741*uid209}\item have deflected more often
\label{m38741*uid210}\item have been attracted to the nucleus easily
\label{m38741*uid211}\item have given the same results
\end{enumerate}
                \label{m38741*uid212}\item Consider the isotope $_{92}^{234}\mathrm{U}$. Which of the following statements is \textsl{true}?
\label{m38741*id263500}\begin{enumerate}[noitemsep, label=\textbf{\alph*}. ] 
            \label{m38741*uid213}\item The element is an isotope of $_{94}^{234}\mathrm{Pu}$
\label{m38741*uid214}\item The element contains 234 neutrons
\label{m38741*uid215}\item The element has the same electron configuration as $_{92}^{238}\mathrm{U}$
\label{m38741*uid216}\item The element has an atomic mass number of 92
\end{enumerate}
                \label{m38741*uid217}\item The electron configuration of an atom of chlorine can be represented using the following notation:
\label{m38741*id263598}\begin{enumerate}[noitemsep, label=\textbf{\alph*}. ] 
            \label{m38741*uid218}\item  ${1\mathrm{s}}^{2}{2\mathrm{s}}^{8}{3\mathrm{s}}^{7}$\label{m38741*uid219}\item 
${1\mathrm{s}}^{2}{2\mathrm{s}}^{2}{2\mathrm{p}}^{6}{3\mathrm{s}}^{2}{3\mathrm{p}}^{5}$
\label{m38741*uid220}\item 
${1\mathrm{s}}^{2}{2\mathrm{s}}^{2}{2\mathrm{p}}^{6}{3\mathrm{s}}^{2}{3\mathrm{p}}^{6}$\label{m38741*uid221}\item 
${1\mathrm{s}}^{2}{2\mathrm{s}}^{2}{2\mathrm{p}}^{5}$\end{enumerate}
                \end{enumerate}
        \item Give the standard notation for the following elements:
\label{m38741*id8223}\begin{enumerate}[noitemsep, label=\textbf{\alph*}. ] 
            \item beryllium\item carbon-12\item titanium-48\item fluorine\end{enumerate}
\item Give the electron configurations and aufbau diagrams for the following elements:\label{m38741*id7624}\begin{enumerate}[noitemsep, label=\textbf{\alph*}. ] 
            \item aluminium\item phosphorus\item carbon\end{enumerate}
\item Use standard notation to represent the following elements: \label{m38741*id74324}\begin{enumerate}[noitemsep, label=\textbf{\alph*}. ] 
            \item argon\item calcium\item silver-107\item bromine-79\end{enumerate}
\item For each of the following elements give the number of protons, neutrons and electrons in the element: \label{m38741*id74374}\begin{enumerate}[noitemsep, label=\textbf{\alph*}. ] 
            \item $_{78}^{195}\mathrm{Pt}$\item $_{18}^{40}\mathrm{Ar}$\item $_{27}^{59}\mathrm{Co}$\item $_{3}^{7}\mathrm{Li}$\item $_{5}^{11}\mathrm{B}$\end{enumerate}
\item For each of the following elements give the element or number represented by 'x': \label{m38741*id7434324}\begin{enumerate}[noitemsep, label=\textbf{\alph*}. ] 
            \item $_{45}^{103}\mathrm{X}$\item $_{x}^{35}\mathrm{Cl}$\item $_{4}^{x}\mathrm{Be}$\end{enumerate}
\item Which of the following are isotopes of $_{12}^{24}\mathrm{Mg}$: \label{m38741*id743234}\begin{enumerate}[noitemsep, label=\textbf{\alph*}. ] 
            \item $_{25}^{12}\mathrm{Mg}$\item $_{12}^{26}\mathrm{Mg}$\item $_{13}^{24}\mathrm{Al}$\end{enumerate}
\item If a sample contains 69\% of copper-63 and 31\% of copper-65, calculate the relative atomic mass of an atom in that sample.\newline
            \item Complete the following table:
    % \textbf{m38741*eip-282}\par
          \begin{table}[H]
    % \begin{table}[H]
    % \\ 'id2892806' '1'
        \begin{center}
      \label{m38741*eip-282}
    \noindent
    \tabletail{%
        \hline
        \multicolumn{4}{|p{\mytableboxwidth}|}{\raggedleft \small \sl continued on next page}\\
        \hline
      }
      \tablelasttail{}
      \begin{xtabular}[t]{|l|l|l|l|}\hline
        Element &
        Electron configuration &
        Core electrons &
        Valence electrons% make-rowspan-placeholders
     \tabularnewline\cline{1-1}\cline{2-2}\cline{3-3}\cline{4-4}
      %--------------------------------------------------------------------
        Boron (B) &
         &
         &
        % make-rowspan-placeholders
     \tabularnewline\cline{1-1}\cline{2-2}\cline{3-3}\cline{4-4}
      %--------------------------------------------------------------------
        Calcium (Ca) &
         &
         &
        % make-rowspan-placeholders
     \tabularnewline\cline{1-1}\cline{2-2}\cline{3-3}\cline{4-4}
      %--------------------------------------------------------------------
        Silicon (Si) &
         &
         &
        % make-rowspan-placeholders
     \tabularnewline\cline{1-1}\cline{2-2}\cline{3-3}\cline{4-4}
      %--------------------------------------------------------------------
        Lithium (Li) &
         &
         &
        % make-rowspan-placeholders
     \tabularnewline\cline{1-1}\cline{2-2}\cline{3-3}\cline{4-4}
      %--------------------------------------------------------------------
        Neon (Ne) &
         &
         &
        % make-rowspan-placeholders
     \tabularnewline\cline{1-1}\cline{2-2}\cline{3-3}\cline{4-4}
      %--------------------------------------------------------------------
    \end{xtabular}
      \end{center}
    \begin{center}{\small\bfseries Table 3.9}\end{center}
    \begin{caption}{\small\bfseries Table 3.9}\end{caption}
\end{table}
    \par
\item Draw aufbau diagrams for the following elements:\label{m38741*id78624}\begin{enumerate}[noitemsep, label=\textbf{\alph*}. ] 
            \item beryllium\item sulphur\item argon\end{enumerate}
\end{enumerate}
  \label{m38741**end}
  \label{ea1c9e59656f96ee804546971cf6dee6**end}
\par \raisebox{-5 pt}{\includegraphics[width=0.5cm]{col11305.imgs/summary_www.png}} Find the answers with the shortcodes:
 \par \begin{tabular}[h]{cccccc}
 (1.) lif  &  (2.) liG  &  (3.) li7  &  (4.) liA  &  (5.) lio  &  (6.) lis  &  (7.) liH  &  (8.) lg7  &  (9.) lgG  &  (10.) l44  &  (11.) l42  &  (12.) l4T  &  (13.) l4b  &  (14.) l4j  &  (15.) l4D  &  (16.) lgW  & \end{tabular}
         \chapter{Transverse pulses}
    \setcounter{figure}{1}
    \setcounter{subfigure}{1}
    \label{21d48a6f8839b4b265192acd9ea3d978}
         \section{ Introduction and key concepts}
    \nopagebreak
            \label{m38801} $ \hspace{-5pt}\begin{array}{cccccccccccc}   \includegraphics[width=0.75cm]{col11305.imgs/summary_fullmarks.png} &   \end{array} $ \hspace{2 pt}\raisebox{-5 pt}{} {(section shortcode: P10037 )} \par 
    \label{m38801*cid2}
            \subsection{ Introduction}
            \nopagebreak
      \label{m38801*id312450}This chapter forms the basis of the discussion into mechanical waves. Waves are all around us, even though most of us are not aware of it. The most common waves are waves in the sea, but waves can be created in any container of water, ranging from an ocean to a tea-cup. Waves do not only occur in water, they occur in any kind of medium. Earthquakes generate waves that travel through the rock of the Earth. When your friend speaks to you he produces \textsl{sound waves} that travel through the air to your ears. Light is made up of electromagnetic waves. A wave is simply moving energy.\par 
    \label{m38801*cid3}
            \subsection{ What is a \textsl{medium}?}
            \nopagebreak
      \label{m38801*id312816}In this chapter, as well as in the following chapters, we will speak about waves moving in a medium. A medium is just the substance or material through which waves move. In other words the medium carries the wave from one place to another. The medium does not create the wave and the medium is not the wave. Therefore the medium does not travel with the wave as the wave propagates through it. Air is a medium for sound waves, water is a medium for water waves and rock is a medium for earthquakes (which are also a type of wave). Air, water and rock are therefore examples of media (media is the plural of medium).\par 
\label{m38801*fhsst!!!underscore!!!id51}\begin{definition}
	  \begin{tabular*}{15 cm}{m{15 mm}m{}}
	\hspace*{-50pt}  \includegraphics[width=0.5in]{col11305.imgs/psflag2.png}   & \Definition{   \label{id2434692}\textbf{ Medium }} { \label{m38801*meaningfhsst!!!underscore!!!id51}
      \label{m38801*id312830}A medium is the substance or material in which a wave will move. \par 
       } 
      \end{tabular*}
      \end{definition}
      \label{m38801*id312841}In each medium, the atoms that make up the medium are moved \textsl{temporarily} from their rest position. In order for a wave to travel, the different parts of the medium must be able to interact with each other.\par 
    \label{m38801*cid4}
            \subsection{ What is a \textsl{pulse}?}
            \nopagebreak
            \label{m38801*secfhsst!!!underscore!!!id58}
            \subsubsection{  Investigation : Observation of Pulses }
            \nopagebreak
      \label{m38801*id312873}Take a heavy rope. Have two people hold the rope stretched out horizontally. Flick the rope at one end only once.\par 
      \label{m38801*id312879}
    \setcounter{subfigure}{0}
	\begin{figure}[H] % horizontal\label{m38801*id312882}
    \begin{center}
    \label{m38801*id312882!!!underscore!!!media}\label{m38801*id312882!!!underscore!!!printimage}\includegraphics[width=300px]{col11305.imgs/m38801_PG10C4_001.png} % m38801;PG10C4\_001.png;;;6.0;8.5;
      \vspace{2pt}
    \vspace{.1in}
    \end{center}
 \end{figure}       
      \par 
      \label{m38801*id312888}What happens to the disturbance that you created in the rope? Does it stay at the place where it was created or does it move down the length of the rope? \par 
      \label{m38801*id312898}In the activity, we created a \textsl{pulse}. A pulse is a \textsl{single} disturbance that moves through a medium. In a transverse pulse the displacement of the medium is perpendicular to the direction of motion of the pulse. Figure~6.2 shows an example of a transverse pulse. In the activity, the rope or spring was held horizontally and the pulse moved the rope up and down. This was an example of a transverse pulse.\par 
\label{m38801*fhsst!!!underscore!!!id71}\begin{definition}
	  \begin{tabular*}{15 cm}{m{15 mm}m{}}
	\hspace*{-50pt}  \includegraphics[width=0.5in]{col11305.imgs/psflag2.png}   & \Definition{   \label{id2434846}\textbf{ Pulse }} { \label{m38801*meaningfhsst!!!underscore!!!id71}
      \label{m38801*id312926}A pulse is a single disturbance that moves through a medium. \par 
       } 
      \end{tabular*}
      \end{definition}
\label{m38801*fhsst!!!underscore!!!id72}\begin{definition}
	  \begin{tabular*}{15 cm}{m{15 mm}m{}}
	\hspace*{-50pt}  \includegraphics[width=0.5in]{col11305.imgs/psflag2.png}   & \Definition{   \label{id2434870}\textbf{ Transverse Pulse }} { \label{m38801*meaningfhsst!!!underscore!!!id72}
      \label{m38801*id3129262}A pulse where all of the particles disturbed by the pulse move perpendicular (at a right angle) to the direction in which the pulse is moving. \par 
       } 
      \end{tabular*}
      \end{definition}
      \label{m38801*uid1}
            \subsubsection{ Pulse Length and Amplitude}
            \nopagebreak
        \label{m38801*id312946}The amplitude of a pulse is a measurement of how far the medium is displaced momentarily from a position of rest. The pulse length is a measurement of how long the pulse is. Both these quantities are shown in Figure~6.2.\par 
\label{m38801*fhsst!!!underscore!!!id77}\begin{definition}
	  \begin{tabular*}{15 cm}{m{15 mm}m{}}
	\hspace*{-50pt}  \includegraphics[width=0.5in]{col11305.imgs/psflag2.png}   & \Definition{   \label{id2434920}\textbf{ Amplitude }} { \label{m38801*meaningfhsst!!!underscore!!!id77}
        \label{m38801*id312963}The amplitude of a pulse is a measurement of how far the medium is displaced from rest. \par 
         } 
      \end{tabular*}
      \end{definition}
    \setcounter{subfigure}{0}
	\begin{figure}[H] % horizontal\label{m38801*uid2}
    \begin{center}
    \rule[.1in]{\figurerulewidth}{.005in} \\
        \label{m38801*uid2!!!underscore!!!media}\label{m38801*uid2!!!underscore!!!printimage}\includegraphics[width=300px]{col11305.imgs/m38801_PG10C4_002.png} % m38801;PG10C4\_002.png;;;6.0;8.5;
      \vspace{2pt}
    \vspace{\rubberspace}\par \begin{cnxcaption}
	  \small \textbf{Figure 6.2: }Example of a transverse pulse
	\end{cnxcaption}
    \vspace{.1in}
    \rule[.1in]{\figurerulewidth}{.005in} \\
    \end{center}
 \end{figure}       
\label{m38801*eip-400}The position of rest is the position the medium would be in if it were undisturbed. This is also called the equilibrium position. Sometimes people will use rest and sometimes equilibrium but they will also use to the two in the same discussion to mean the same thing.\par \label{m38801*secfhsst!!!underscore!!!id87}
            \subsubsection{  Investigation : Pulse Length and Amplitude }
            \nopagebreak
        \label{m38801*id312993}The graphs below show the positions of a pulse at different times.\par 
        \label{m38801*id312998}
    \setcounter{subfigure}{0}
	\begin{figure}[H] % horizontal\label{m38801*id313002}
    \begin{center}
    \label{m38801*id313002!!!underscore!!!media}\label{m38801*id313002!!!underscore!!!printimage}\includegraphics[width=300px]{col11305.imgs/m38801_PG10C4_003.png} % m38801;PG10C4\_003.png;;;6.0;8.5;
      \vspace{2pt}
    \vspace{.1in}
    \end{center}
 \end{figure}       
        \par 
        \label{m38801*id313008}Use your ruler to measure the lengths of $a$ and $p$. Fill your answers in the table.\par 
    % \textbf{m38801*id313027}\par
          \begin{table}[H]
    % \begin{table}[H]
    % \\ '' '0'
        \begin{center}
      \label{m38801*id313027}
    \noindent
    \tabletail{%
        \hline
        \multicolumn{3}{|p{\mytableboxwidth}|}{\raggedleft \small \sl continued on next page}\\
        \hline
      }
      \tablelasttail{}
      \begin{xtabular}[t]{|l|l|l|}\hline
        Time &
                  $a$
                 &
                  $p$
                % make-rowspan-placeholders
     \tabularnewline\cline{1-1}\cline{2-2}\cline{3-3}
      %--------------------------------------------------------------------
        $t=0$~s &
         &
        % make-rowspan-placeholders
     \tabularnewline\cline{1-1}\cline{2-2}\cline{3-3}
      %--------------------------------------------------------------------
        $t=1$~s &
         &
        % make-rowspan-placeholders
     \tabularnewline\cline{1-1}\cline{2-2}\cline{3-3}
      %--------------------------------------------------------------------
        $t=2$~s &
         &
        % make-rowspan-placeholders
     \tabularnewline\cline{1-1}\cline{2-2}\cline{3-3}
      %--------------------------------------------------------------------
        $t=3$~s &
         &
        % make-rowspan-placeholders
     \tabularnewline\cline{1-1}\cline{2-2}\cline{3-3}
      %--------------------------------------------------------------------
    \end{xtabular}
      \end{center}
    \begin{center}{\small\bfseries Table 6.1}\end{center}
    \begin{caption}{\small\bfseries Table 6.1}\end{caption}
\end{table}
    \par
        \label{m38801*id313222}What do you notice about the values of $a$ and $p$?
 \par 
        \label{m38801*id313246}In the activity, we found that the values for how high the pulse ($a$) is and how wide the pulse ($p$) is the same at different times. \textsl{Pulse length} and \textsl{amplitude} are two important quantities of a pulse.\par 
      \label{m38801*uid3}
            \subsubsection{ Pulse Speed}
            \nopagebreak
\par
            \label{m38801*fhsst!!!underscore!!!id145}\begin{definition}
	  \begin{tabular*}{15 cm}{m{15 mm}m{}}
	\hspace*{-50pt}  \includegraphics[width=0.5in]{col11305.imgs/psflag2.png}   & \Definition{   \label{id2435275}\textbf{ Pulse Speed }} { \label{m38801*meaningfhsst!!!underscore!!!id145}
        \label{m38801*id313292}Pulse speed is the distance a pulse travels per unit time. \par 
         } 
      \end{tabular*}
      \end{definition}
        \label{m38801*id313303}In Motion in one dimension\footnote{\raggedright{}"Motion in One Dimension - Grade 10 [CAPS]" <http://http://cnx.org/content/m37923/latest/>} we saw that speed was defined as the distance traveled per unit time. We can use the same definition of speed to calculate how fast a pulse travels. If the pulse travels a distance $D$ in a time $t$, then the pulse speed $v$ is:\par 
        \label{m38801*uid4}\nopagebreak\noindent{}
    \begin{equation}
    v=\frac{D}{t}\tag{6.1}
      \end{equation}
\par
            \label{m38801*secfhsst!!!underscore!!!id161}\vspace{.5cm} 
      \noindent
      \hspace*{-30pt}\includegraphics[width=0.5in]{col11305.imgs/pspencil2.png}   \raisebox{25mm}{   
      \begin{mdframed}[linewidth=4, leftmargin=40, rightmargin=40]  
      \begin{exercise}
    \noindent\textbf{Exercise 6.1:  Pulse Speed }
        \label{m38801*probfhsst!!!underscore!!!id162}
        \label{m38801*id313369}A pulse covers a distance of $2\phantom{\rule{3pt}{0ex}}\mathrm{m}$ in $4\phantom{\rule{3pt}{0ex}}\mathrm{s}$ on a heavy rope. Calculate the pulse speed. \par 
        \vspace{5pt}
        \label{m38801*solfhsst!!!underscore!!!id165}\noindent\textbf{Solution to Exercise } \label{m38801*listfhsst!!!underscore!!!id165}\begin{enumerate}[noitemsep, label=\textbf{Step} \textbf{\arabic*}. ] 
            \leftskip=20pt\rightskip=\leftskip\item  
        \label{m38801*id313393}We are given:\par 
        \label{m38801*id313396}\begin{itemize}[noitemsep]
            \leftskip=20pt\rightskip=\leftskip\label{m38801*uid5}\item the distance travelled by the pulse: $D=2\phantom{\rule{2pt}{0ex}}\mathrm{m}$
\label{m38801*uid6}\item the time taken to travel $2\phantom{\rule{2pt}{0ex}}\mathrm{m}$: $t=4\phantom{\rule{2pt}{0ex}}\mathrm{s}$
\end{itemize}
        \label{m38801*id313439}We are required to calculate the speed of the pulse.\par 
        \item  
        \label{m38801*id313447}We can use:\par 
        \label{m38801*id313450}\nopagebreak\noindent{}
    \begin{equation}
    v=\frac{D}{t}\tag{6.2}
      \end{equation}
        \label{m38801*id313470}to calculate the speed of the pulse.\par 
        \item  
        \label{m38801*id313479}\nopagebreak\noindent{}
    \begin{equation}
    \begin{array}{ccc}\hfill v& =& \frac{D}{t}\hfill \\ & =& \frac{2\phantom{\rule{0.166667em}{0ex}}\mathrm{m}}{4\phantom{\rule{0.166667em}{0ex}}\mathrm{s}}\hfill \\ & =& 0,5\phantom{\rule{0.166667em}{0ex}}\mathrm{m}\ensuremath{\cdot}{\mathrm{s}}^{-1}\hfill \end{array}\tag{6.3}
      \end{equation}
        \item  
        \label{m38801*id313763}The pulse speed is $0,5\phantom{\rule{2pt}{0ex}}\mathrm{m}\ensuremath{\cdot}\mathrm{s}{}^{-1}$.
 \par 
        \end{enumerate}
    \end{exercise}
    \end{mdframed}
    }
    \noindent
\label{m38801*notfhsst!!!underscore!!!id259}
\begin{tabular}{cc}
	   \hspace*{-50pt}\raisebox{-8 mm}{ \includegraphics[width=0.5in]{col11305.imgs/pstip2.png}  }& 
	\begin{minipage}{0.85\textwidth}
	\begin{note}
      {tip: }The pulse speed depends on the properties of the medium and not on the amplitude or pulse length of the pulse.
	\end{note}
	\end{minipage}
	\end{tabular}
	\par
\label{m38801*secfhsst!!!underscore!!!id260}
            \subsubsection{  Pulse Speed }
            \nopagebreak
        \label{m38801*id313813}\begin{enumerate}[noitemsep, label=\textbf{\arabic*}. ] 
            \label{m38801*uid7}\item A pulse covers a distance of $5\phantom{\rule{2pt}{0ex}}\mathrm{m}$ in $15\phantom{\rule{2pt}{0ex}}\mathrm{s}$. Calculate the speed of the pulse.\newline
\label{m38801*uid8}\item A pulse has a speed of $5\phantom{\rule{2pt}{0ex}}\mathrm{cm}\ensuremath{\cdot}\mathrm{s}{}^{-1}$. How far does it travel in $2,5\phantom{\rule{2pt}{0ex}}\mathrm{s}$?\newline
\label{m38801*uid9}\item A pulse has a speed of $0,5\phantom{\rule{2pt}{0ex}}\mathrm{m}\ensuremath{\cdot}\mathrm{s}{}^{-1}$. How long does it take to cover a distance of $25\phantom{\rule{2pt}{0ex}}\mathrm{cm}$?\newline
\label{m38801*uid10}\item How long will it take a pulse moving at $0,25\phantom{\rule{2pt}{0ex}}\mathrm{m}\ensuremath{\cdot}\mathrm{s}{}^{-1}$ to travel a distance of $20\phantom{\rule{2pt}{0ex}}\mathrm{m}$?\newline
\label{m38801*uid11}\item The diagram shows two pulses in the same medium. Which has the higher speed? Explain your answer.
    \setcounter{subfigure}{0}
	\begin{figure}[H] % horizontal\label{m38801*id313945}
    \begin{center}
    \label{m38801*id313945!!!underscore!!!media}\label{m38801*id313945!!!underscore!!!printimage}\includegraphics[width=300px]{col11305.imgs/m38801_PG10C4_004.png} % m38801;PG10C4\_004.png;;;6.0;8.5;
      \vspace{2pt}
    \vspace{.1in}
    \end{center}
 \end{figure}               \end{enumerate}
  \label{m38801**end}
\par \raisebox{-5 pt}{\includegraphics[width=0.5cm]{col11305.imgs/summary_www.png}} Find the answers with the shortcodes:
 \par \begin{tabular}[h]{cccccc}
 (1.) l1f  &  (2.) l1G  &  (3.) l17  &  (4.) l1A  &  (5.) l1o  & \end{tabular}
         \section{ Superposition of pulses}
    \nopagebreak
            \label{m38802} $ \hspace{-5pt}\begin{array}{cccccccccccc}   \includegraphics[width=0.75cm]{col11305.imgs/summary_fullmarks.png} &   \includegraphics[width=0.75cm]{col11305.imgs/summary_video.png} &   \end{array} $ \hspace{2 pt}\raisebox{-5 pt}{} {(section shortcode: P10038 )} \par 
\label{m38802*fs-id1166232432154}
            \subsection{ Superposition of Pulses}
            \nopagebreak
      \label{m38802*id316136}Two or more pulses can pass through the same medium at that same time in the same place. When they do they interact with each other to form a different disturbance at that point. The resulting pulse is obtained by using the \textsl{principle of superposition}. The principle of superposition states that the effect of the different pulses is the sum of their individual effects. After pulses pass through each other, each pulse continues along its original direction of travel, and their original amplitudes remain unchanged.\par 
      \label{m38802*id316148}Constructive interference takes place when two pulses meet each other to create a larger pulse. The amplitude of the resulting pulse is the sum of the amplitudes of the two initial pulses. This is shown in Figure~6.5.\par 
\label{m38802*fhsst!!!underscore!!!id567}\begin{definition}
	  \begin{tabular*}{15 cm}{m{15 mm}m{}}
	\hspace*{-50pt}  \includegraphics[width=0.5in]{col11305.imgs/psflag2.png}   & \Definition{   \label{id2436281}\textbf{ Constructive interference}} { \label{m38802*meaningfhsst!!!underscore!!!id567}
      Constructive interference is when two pulses meet, resulting in a bigger pulse. 
       } 
      \end{tabular*}
      \end{definition}
    \setcounter{subfigure}{0}
	\begin{figure}[H] % horizontal\label{m38802*uid53}
    \begin{center}
    \rule[.1in]{\figurerulewidth}{.005in} \\
        \label{m38802*uid53!!!underscore!!!media}\label{m38802*uid53!!!underscore!!!printimage}\includegraphics[width=0.3\columnwidth]{col11305.imgs/m38802_PG10C4_018.png} % m38802;PG10C4\_018.png;;;6.0;8.5;
      \vspace{2pt}
    \vspace{\rubberspace}\par \begin{cnxcaption}
	  \small \textbf{Figure 6.5: }Superposition of two pulses: constructive interference.
	\end{cnxcaption}
    \vspace{.1in}
    \rule[.1in]{\figurerulewidth}{.005in} \\
    \end{center}
 \end{figure}       
      \label{m38802*id316190}Destructive interference takes place when two pulses meet and cancel each other. The amplitude of the resulting pulse is the sum of the amplitudes of the two initial pulses, but the one amplitude will be a negative number. This is shown in Figure~6.6. In general, amplitudes of individual pulses add together to give the amplitude of the resultant pulse.\par 
\label{m38802*fhsst!!!underscore!!!id578}\begin{definition}
	  \begin{tabular*}{15 cm}{m{15 mm}m{}}
	\hspace*{-50pt}  \includegraphics[width=0.5in]{col11305.imgs/psflag2.png}   & \Definition{   \label{id2436357}\textbf{ Destructive interference}} { \label{m38802*meaningfhsst!!!underscore!!!id578}
      Destructive interference is when two pulses meet, resulting in a smaller pulse. 
       } 
      \end{tabular*}
      \end{definition}
    \setcounter{subfigure}{0}
	\begin{figure}[H] % horizontal\label{m38802*uid54}
    \begin{center}
    \rule[.1in]{\figurerulewidth}{.005in} \\
        \label{m38802*uid54!!!underscore!!!media}\label{m38802*uid54!!!underscore!!!printimage}\includegraphics[width=300px]{col11305.imgs/m38802_PG10C4_019.png} % m38802;PG10C4\_019.png;;;6.0;8.5;
      \vspace{2pt}
    \vspace{\rubberspace}\par \begin{cnxcaption}
	  \small \textbf{Figure 6.6: }Superposition of two pulses. The left-hand series of images demonstrates destructive interference, since the pulses cancel each other. The right-hand series of images demonstrate a partial cancelation of two pulses, as their amplitudes are not the same in magnitude.
	\end{cnxcaption}
    \vspace{.1in}
    \rule[.1in]{\figurerulewidth}{.005in} \\
    \end{center}
 \end{figure}       
\par
            \label{m38802*secfhsst!!!underscore!!!id588}\vspace{.5cm} 
      \noindent
      \hspace*{-30pt}\includegraphics[width=0.5in]{col11305.imgs/pspencil2.png}   \raisebox{25mm}{   
      \begin{mdframed}[linewidth=4, leftmargin=40, rightmargin=40]  
      \begin{exercise}
    \noindent\textbf{Exercise 6.2:  Superposition of Pulses }
      \label{m38802*probfhsst!!!underscore!!!id589}
      \label{m38802*id316249}The two pulses shown below approach each other at $1\phantom{\rule{2pt}{0ex}}\mathrm{m}\ensuremath{\cdot}\mathrm{s}{}^{-1}$. Draw what the waveform would look like after $1\phantom{\rule{2pt}{0ex}}\mathrm{s}$, $2\phantom{\rule{2pt}{0ex}}\mathrm{s}$ and $5\phantom{\rule{2pt}{0ex}}\mathrm{s}$.\par 
      \label{m38802*id316283}
    \setcounter{subfigure}{0}
	\begin{figure}[H] % horizontal\label{m38802*id316286}
    \begin{center}
    \label{m38802*id316286!!!underscore!!!media}\label{m38802*id316286!!!underscore!!!printimage}\includegraphics[width=300px]{col11305.imgs/m38802_PG10C4_020.png} % m38802;PG10C4\_020.png;;;6.0;8.5;
      \vspace{2pt}
    \vspace{.1in}
    \end{center}
 \end{figure}       
      \par 
      \vspace{5pt}
      \label{m38802*solfhsst!!!underscore!!!id601}\noindent\textbf{Solution to Exercise } \label{m38802*listfhsst!!!underscore!!!id601}\begin{enumerate}[noitemsep, label=\textbf{Step} \textbf{\arabic*}. ] 
            \leftskip=20pt\rightskip=\leftskip\item  
      \label{m38802*id316313}After $1\phantom{\rule{2pt}{0ex}}\mathrm{s}$, pulse A has moved $1\phantom{\rule{2pt}{0ex}}\mathrm{m}$ to the right and pulse B has moved $1\phantom{\rule{2pt}{0ex}}\mathrm{m}$ to the left.\par 
      \label{m38802*id316318}
    \setcounter{subfigure}{0}
	\begin{figure}[H] % horizontal\label{m38802*id316321}
    \begin{center}
    \label{m38802*id316321!!!underscore!!!media}\label{m38802*id316321!!!underscore!!!printimage}\includegraphics{col11305.imgs/m38802_PG10C4_021.png} % ;PG10C4\_021.png;;;6.0;8.5;
      \vspace{2pt}
    \vspace{.1in}
    \end{center}
 \end{figure}       
      \par 
      \item  
      \label{m38802*id316332}After $1\phantom{\rule{2pt}{0ex}}\mathrm{s}$ more, pulse A has moved $1\phantom{\rule{2pt}{0ex}}\mathrm{m}$ to the right and pulse B has moved $1\phantom{\rule{2pt}{0ex}}\mathrm{m}$ to the left.\par 
      \label{m38802*id316337}
    \setcounter{subfigure}{0}
	\begin{figure}[H] % horizontal\label{m38802*id316340}
    \begin{center}
    \label{m38802*id316340!!!underscore!!!media}\label{m38802*id316340!!!underscore!!!printimage}\includegraphics{col11305.imgs/m38802_PG10C4_022.png} % ;PG10C4\_022.png;;;6.0;8.5;
      \vspace{2pt}
    \vspace{.1in}
    \end{center}
 \end{figure}       
      \par 
      \item  
      \label{m38802*id316351}After $5\phantom{\rule{2pt}{0ex}}\mathrm{s}$, pulse A has moved $5\phantom{\rule{2pt}{0ex}}\mathrm{m}$ to the right and pulse B has moved $5\phantom{\rule{2pt}{0ex}}\mathrm{m}$ to the left.\par 
      \label{m38802*id316356}
    \setcounter{subfigure}{0}
	\begin{figure}[H] % horizontal\label{m38802*id316360}
    \begin{center}
    \label{m38802*id316360!!!underscore!!!media}\label{m38802*id316360!!!underscore!!!printimage}\includegraphics{col11305.imgs/m38802_PG10C4_023.png} % ;PG10C4\_023.png;;;6.0;8.5;
      \vspace{2pt}
    \vspace{.1in}
    \end{center}
 \end{figure}       
      \par 
      \end{enumerate}
    \end{exercise}
    \end{mdframed}
    }
    \noindent
\label{m38802*notfhsst!!!underscore!!!id635}
\begin{tabular}{cc}
	   \hspace*{-50pt}\raisebox{-8 mm}{ \includegraphics[width=0.5in]{col11305.imgs/pstip2.png}  }& 
	\begin{minipage}{0.85\textwidth}
	\begin{note}
      {tip: }The idea of superposition is one that occurs often in physics. You will see \textsl{much, much more} of superposition!
	\end{note}
	\end{minipage}
	\end{tabular}
	\par
\label{m38802*eip-791}
            \subsubsection{ Experiment: Constructive and destructive interference}
            \nopagebreak
            \label{m38802*eip-260}\noindent{}\textbf{Aim}
To demonstrate constructive and destructive interference
\par 
\label{m38802*eip7241}\noindent{}\textbf{Apparatus} 
Ripple tank apparatus
    \setcounter{subfigure}{0}
	\begin{figure}[H] % horizontal\label{m38802*id63458}
    \begin{center}
    \label{m38802*id63458!!!underscore!!!media}\label{m38802*id63458!!!underscore!!!printimage}\includegraphics[width=0.8\columnwidth]{col11305.imgs/m38802_rippletray.png} % m38802;rippletray.png;;;6.0;8.5;
      \vspace{2pt}
    \vspace{.1in}
    \end{center}
 \end{figure}       \par 
\label{m38802*eip7474}\noindent{}\textbf{Method}
\label{m38802*id6242}\begin{enumerate}[noitemsep, label=\textbf{\arabic*}. ] 
            \item Set up the ripple tank\item Produce a single pulse and observe what happens\item Produce two pulses simultaneously and observe what happens\item Produce two pulses at slightly different times and observe what happens\end{enumerate}
\par 
\label{m38802*id614134}\noindent{}\textbf{Results and conclusion}
You should observe that when you produce two pulses simultaneously you see them interfere constructively and when you produce two pulses at slightly different times you see them interfere destructively.
\par \label{m38802*secfhsst!!!underscore!!!id636}
            \subsubsection{ Problems Involving Superposition of Pulses }
            \nopagebreak
            \label{m38802*id316401}\begin{enumerate}[noitemsep, label=\textbf{\arabic*}. ] 
            \label{m38802*uid55}\item For the following pulse, draw the resulting wave forms after $1\phantom{\rule{2pt}{0ex}}\mathrm{s}$, $2\phantom{\rule{2pt}{0ex}}\mathrm{s}$, $3\phantom{\rule{2pt}{0ex}}\mathrm{s}$, $4\phantom{\rule{2pt}{0ex}}\mathrm{s}$ and $5\phantom{\rule{2pt}{0ex}}\mathrm{s}$. Each pulse is travelling at $1\phantom{\rule{2pt}{0ex}}\mathrm{m}\ensuremath{\cdot}\mathrm{s}{}^{-1}$. Each block represents $1\phantom{\rule{2pt}{0ex}}\mathrm{m}$. The pulses are shown as thick black lines and the undisplaced medium as dashed lines.
    \setcounter{subfigure}{0}
	\begin{figure}[H] % horizontal\label{m38802*id316460}
    \begin{center}
    \label{m38802*id316460!!!underscore!!!media}\label{m38802*id316460!!!underscore!!!printimage}\includegraphics[width=300px]{col11305.imgs/m38802_PG10C4_024.png} % m38802;PG10C4\_024.png;;;6.0;8.5;
      \vspace{2pt}
    \vspace{.1in}
    \end{center}
 \end{figure}               \label{m38802*uid57}\item For the following pulse, draw the resulting wave forms after $1\phantom{\rule{2pt}{0ex}}\mathrm{s}$, $2\phantom{\rule{2pt}{0ex}}\mathrm{s}$, $3\phantom{\rule{2pt}{0ex}}\mathrm{s}$, $4\phantom{\rule{2pt}{0ex}}\mathrm{s}$ and $5\phantom{\rule{2pt}{0ex}}\mathrm{s}$. Each pulse is travelling at $1\phantom{\rule{2pt}{0ex}}\mathrm{m}\ensuremath{\cdot}\mathrm{s}{}^{-1}$. Each block represents $1\phantom{\rule{2pt}{0ex}}\mathrm{m}$. The pulses are shown as thick black lines and the undisplaced medium as dashed lines.
    \setcounter{subfigure}{0}
	\begin{figure}[H] % horizontal\label{m38802*id316477}
    \begin{center}
    \label{m38802*id316477!!!underscore!!!media}\label{m38802*id316477!!!underscore!!!printimage}\includegraphics[width=300px]{col11305.imgs/m38802_PG10C4_025.png} % m38802;PG10C4\_025.png;;;6.0;8.5;
      \vspace{2pt}
    \vspace{.1in}
    \end{center}
 \end{figure}               \label{m38802*uid58}\item For the following pulse, draw the resulting wave forms after $1\phantom{\rule{2pt}{0ex}}\mathrm{s}$, $2\phantom{\rule{2pt}{0ex}}\mathrm{s}$, $3\phantom{\rule{2pt}{0ex}}\mathrm{s}$, $4\phantom{\rule{2pt}{0ex}}\mathrm{s}$ and $5\phantom{\rule{2pt}{0ex}}\mathrm{s}$. Each pulse is travelling at $1\phantom{\rule{2pt}{0ex}}\mathrm{m}\ensuremath{\cdot}\mathrm{s}{}^{-1}$. Each block represents $1\phantom{\rule{2pt}{0ex}}\mathrm{m}$. The pulses are shown as thick black lines and the undisplaced medium as dashed lines.
    \setcounter{subfigure}{0}
	\begin{figure}[H] % horizontal\label{m38802*id316495}
    \begin{center}
    \label{m38802*id316495!!!underscore!!!media}\label{m38802*id316495!!!underscore!!!printimage}\includegraphics[width=300px]{col11305.imgs/m38802_PG10C4_026.png} % m38802;PG10C4\_026.png;;;6.0;8.5;
      \vspace{2pt}
    \vspace{.1in}
    \end{center}
 \end{figure}               \label{m38802*uid59}\item For the following pulse, draw the resulting wave forms after $1\phantom{\rule{2pt}{0ex}}\mathrm{s}$, $2\phantom{\rule{2pt}{0ex}}\mathrm{s}$, $3\phantom{\rule{2pt}{0ex}}\mathrm{s}$, $4\phantom{\rule{2pt}{0ex}}\mathrm{s}$ and $5\phantom{\rule{2pt}{0ex}}\mathrm{s}$. Each pulse is travelling at $1\phantom{\rule{2pt}{0ex}}\mathrm{m}\ensuremath{\cdot}\mathrm{s}{}^{-1}$. Each block represents $1\phantom{\rule{2pt}{0ex}}\mathrm{m}$. The pulses are shown as thick black lines and the undisplaced medium as dashed lines.
    \setcounter{subfigure}{0}
	\begin{figure}[H] % horizontal\label{m38802*id316512}
    \begin{center}
    \label{m38802*id316512!!!underscore!!!media}\label{m38802*id316512!!!underscore!!!printimage}\includegraphics[width=300px]{col11305.imgs/m38802_PG10C4_027.png} % m38802;PG10C4\_027.png;;;6.0;8.5;
      \vspace{2pt}
    \vspace{.1in}
    \end{center}
 \end{figure}               \label{m38802*uid60}\item For the following pulse, draw the resulting wave forms after $1\phantom{\rule{2pt}{0ex}}\mathrm{s}$, $2\phantom{\rule{2pt}{0ex}}\mathrm{s}$, $3\phantom{\rule{2pt}{0ex}}\mathrm{s}$, $4\phantom{\rule{2pt}{0ex}}\mathrm{s}$ and $5\phantom{\rule{2pt}{0ex}}\mathrm{s}$. Each pulse is travelling at $1\phantom{\rule{2pt}{0ex}}\mathrm{m}\ensuremath{\cdot}\mathrm{s}{}^{-1}$. Each block represents $1\phantom{\rule{2pt}{0ex}}\mathrm{m}$. The pulses are shown as thick black lines and the undisplaced medium as dashed lines.
    \setcounter{subfigure}{0}
	\begin{figure}[H] % horizontal\label{m38802*id316530}
    \begin{center}
    \label{m38802*id316530!!!underscore!!!media}\label{m38802*id316530!!!underscore!!!printimage}\includegraphics[width=300px]{col11305.imgs/m38802_PG10C4_028.png} % m38802;PG10C4\_028.png;;;6.0;8.5;
      \vspace{2pt}
    \vspace{.1in}
    \end{center}
 \end{figure}               \label{m38802*uid61}\item For the following pulse, draw the resulting wave forms after $1\phantom{\rule{2pt}{0ex}}\mathrm{s}$, $2\phantom{\rule{2pt}{0ex}}\mathrm{s}$, $3\phantom{\rule{2pt}{0ex}}\mathrm{s}$, $4\phantom{\rule{2pt}{0ex}}\mathrm{s}$ and $5\phantom{\rule{2pt}{0ex}}\mathrm{s}$. Each pulse is travelling at $1\phantom{\rule{2pt}{0ex}}\mathrm{m}\ensuremath{\cdot}\mathrm{s}{}^{-1}$. Each block represents $1\phantom{\rule{2pt}{0ex}}\mathrm{m}$. The pulses are shown as thick black lines and the undisplaced medium as dashed lines.
    \setcounter{subfigure}{0}
	\begin{figure}[H] % horizontal\label{m38802*id316547}
    \begin{center}
    \label{m38802*id316547!!!underscore!!!media}\label{m38802*id316547!!!underscore!!!printimage}\includegraphics[width=300px]{col11305.imgs/m38802_PG10C4_029.png} % m38802;PG10C4\_029.png;;;6.0;8.5;
      \vspace{2pt}
    \vspace{.1in}
    \end{center}
 \end{figure}               \label{m38802*uid62}\item 
          What is superposition of waves?\newline
\label{m38802*uid64}\item What is constructive interference?\newline
\label{m38802*uid65}\item What is destructive interference?\newline
        \end{enumerate}
\label{m38802*fs-id1165499443114} The following presentation provides a summary of the work covered in this chapter. Although the presentation is titled waves, the presentation covers pulses only.
    \setcounter{subfigure}{0}
	\begin{figure}[H] % horizontal\label{m38802*slidesharefigure}
    \label{m38802*slidesharemedia}\label{m38802*slideshareflash}\raisebox{-5 pt}{ \includegraphics[width=0.5cm]{col11305.imgs/summary_www.png}} { (Presentation:  P10039 )}
      \vspace{2pt}
    \vspace{.1in}
 \end{figure}       
\par 
  \label{m38802*eip-812}
\par \raisebox{-5 pt}{\includegraphics[width=0.5cm]{col11305.imgs/summary_www.png}} Find the answers with the shortcodes:
 \par \begin{tabular}[h]{cccccc}
 (1.) l1M  &  (2.) l1e  &  (3.) l1t  &  (4.) l1z  &  (5.) l1u  &  (6.) l1J  &  (7.) l1S  &  (8.) l1h  &  (9.) lrg  & \end{tabular}
            \subsection{ Summary}
            \nopagebreak
            \label{m38802*eip-404}\begin{itemize}[noitemsep]
            \item A medium is the substance or material in which a wave will move\item A pulse is a single disturbance that moves through a medium\item The amplitude of a pules is a measurement of how far the medium is displaced from rest\item Pulse speed is the distance a pulse travels per unit time\item Constructive interference is when two pulses meet and result in a bigger pulse\item Destructive interference is when two pulses meet and and result in a smaller pulse\item We can draw graphs to show the motion of a particle in the medium or to show the motion of a pulse through the medium\item When a pulse moves from a thin rope to a thick rope, the speed and pulse length decrease. The pulse will be reflected and inverted in the thin rope. The reflected pulse has the same length and speed, but a different amplitude\item When a pulse moves from a thick rope to a thin rope, the speed and pulse length increase. The pulse will be reflected in the thick rope. The reflected pulse has the same length and speed, but a different amplitude\item A pulse reaching a free end will be reflected but not inverted. A pulse reaching a fixed end will be reflected and inverted\end{itemize}
        \label{m38802*cid9}
            \subsection{ Exercises - Transverse Pulses}
            \nopagebreak
      \label{m38802*id316647}\begin{enumerate}[noitemsep, label=\textbf{\arabic*}. ] 
            \label{m38802*uid66}\item A heavy rope is flicked upwards, creating a single pulse in the rope. Make a drawing of the rope and indicate the following in your drawing:
\label{m38802*id316663}\begin{enumerate}[noitemsep, label=\textbf{\alph*}. ] 
            \label{m38802*uid67}\item The direction of motion of the pulse
\label{m38802*uid68}\item Amplitude
\label{m38802*uid69}\item Pulse length
\label{m38802*uid70}\item Position of rest
\end{enumerate}
                \label{m38802*uid71}\item A pulse has a speed of $2,5\phantom{\rule{2pt}{0ex}}\mathrm{m}\ensuremath{\cdot}\mathrm{s}{}^{-1}$. How far will it have travelled in $6\phantom{\rule{2pt}{0ex}}\mathrm{s}$?\newline
\label{m38802*uid72}\item A pulse covers a distance of $75\phantom{\rule{2pt}{0ex}}\mathrm{cm}$ in $2,5\phantom{\rule{2pt}{0ex}}\mathrm{s}$. What is the speed of the pulse?\newline
\label{m38802*uid73}\item How long does it take a pulse to cover a distance of $200\phantom{\rule{2pt}{0ex}}\mathrm{mm}$ if its speed is $4\phantom{\rule{2pt}{0ex}}\mathrm{m}\ensuremath{\cdot}\mathrm{s}{}^{-1}$?\newline
\label{m38802*uid74}\item The following position-time graph for a pulse in a slinky spring is given. Draw an accurate sketch graph of the velocity of the pulse against time.
    \setcounter{subfigure}{0}
	\begin{figure}[H] % horizontal\label{m38802*id316803}
    \begin{center}
    \label{m38802*id316803!!!underscore!!!media}\label{m38802*id316803!!!underscore!!!printimage}\includegraphics[width=0.5\columnwidth]{col11305.imgs/m38802_PG10C4_030.png} % m38802;PG10C4\_030.png;;;6.0;8.5;
      \vspace{2pt}
    \vspace{.1in}
    \end{center}
 \end{figure}               \label{m38802*uid75}\item The following velocity-time graph for a particle in a medium is given. Draw an accurate sketch graph of the position of the particle vs. time.
    \setcounter{subfigure}{0}
	\begin{figure}[H] % horizontal\label{m38802*id316826}
    \begin{center}
    \label{m38802*id316826!!!underscore!!!media}\label{m38802*id316826!!!underscore!!!printimage}\includegraphics[width=0.5\columnwidth]{col11305.imgs/m38802_PG10C4_031.png} % m38802;PG10C4\_031.png;;;6.0;8.5;
      \vspace{2pt}
    \vspace{.1in}
    \end{center}
 \end{figure}               \label{m38802*uid76}\item Describe what happens to a pulse in a slinky spring when:
\label{m38802*id316845}\begin{enumerate}[noitemsep, label=\textbf{\alph*}. ] 
            \label{m38802*uid77}\item the slinky spring is tied to a wall.
\label{m38802*uid78}\item the slinky spring is loose, i.e. not tied to a wall.
\end{enumerate}
(Draw diagrams to explain your answers.)\newline
\label{m38802*uid79}\item The following diagrams each show two approaching pulses. Redraw the diagrams to show what type of interference takes place, and label the type of interference.
\label{m38802*id316891}\begin{enumerate}[noitemsep, label=\textbf{\alph*}. ] 
            \label{m38802*uid80}\item 
    \setcounter{subfigure}{0}
	\begin{figure}[H] % horizontal\label{m38802*id316905}
    \begin{center}
    \label{m38802*id316905!!!underscore!!!media}\label{m38802*id316905!!!underscore!!!printimage}\includegraphics[width=0.5\columnwidth]{col11305.imgs/m38802_PG10C4_032.png} % m38802;PG10C4\_032.png;;;6.0;8.5;
      \vspace{2pt}
    \vspace{.1in}
    \end{center}
 \end{figure}       \label{m38802*uid81}\item 
    \setcounter{subfigure}{0}
	\begin{figure}[H] % horizontal\label{m38802*id316923}
    \begin{center}
    \label{m38802*id316923!!!underscore!!!media}\label{m38802*id316923!!!underscore!!!printimage}\includegraphics[width=0.5\columnwidth]{col11305.imgs/m38802_PG10C4_033.png} % m38802;PG10C4\_033.png;;;6.0;8.5;
      \vspace{2pt}
    \vspace{.1in}
    \end{center}
 \end{figure}       \end{enumerate}
                \label{m38802*uid82}\item Two pulses, A and B, of identical shape and amplitude are simultaneously generated in two identical wires of equal mass and length. Wire A is, however, pulled tighter than wire B. Which pulse will arrive at the other end first, or will they both arrive at the same time?\newline
\end{enumerate}
  \label{m38802**end}
  \label{21d48a6f8839b4b265192acd9ea3d978**end}
\par \raisebox{-5 pt}{\includegraphics[width=0.5cm]{col11305.imgs/summary_www.png}} Find the answers with the shortcodes:
 \par \begin{tabular}[h]{cccccc}
 (1.) lrl  &  (2.) lri  &  (3.) lr3  &  (4.) lrO  &  (5.) lrc  &  (6.) lrx  &  (7.) lra  &  (8.) lrC  &  (9.) lr1  & \end{tabular}
