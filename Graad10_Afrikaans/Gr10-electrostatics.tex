 \chapter{Elektrostatika}
\fancyfoot[LO,RE]{Fisika: Elektrisiteit and Magnetisme}
\label{464e844ca5615087ea89d9d95dd9a43a}
 

\section{Inleiding en sleutelkonsepte}
    \nopagebreak
    \label{m38780*cid2}
       
Elektrostatika is die studie van elektriese lading wat staties is (dit beweeg nie). In hierdie hoofstuk kyk ons na die basiese beginsels van elektrostatika sowel as die beginsel van behoud van lading.

\chapterstartvideo{VPfmg}

\section{Twee soorte lading}
            \nopagebreak

Alle voorwerpe om ons (mense insluitend!) bevat groot hoeveelhede elektriese lading. Daar is twee tipes elektriese lading: \textbf{positiewe} and \textbf{negatiewe} lading. As dieselfde hoeveelhede positiewe en negatiewe ladings in 'n voorwerp bevat word, is die \textbf{netto lading nul} en die voorwerp is \textbf{neutraal}. As daar meer van een soort lading as die ander is, is die voorwerp elektries \textbf{gelaai}. Die diagram hieronder wys hoe die verspreiding van ladings vir 'n neutraal, positief en negatief gelaaide voorwerp. \par

      \label{m38780*id200640}
	\begin{figure}[H] % horizontal\label{m38780*id200643}
    \begin{center}
    \begin{pspicture}(0,-2.3017187)(13.100625,2.3017187)
\pscircle[linewidth=0.04,dimen=outer](1.6696875,-0.48828125){0.92}
\pscircle[linewidth=0.04,dimen=outer](6.2496877,-0.48828125){0.92}
\pscircle[linewidth=0.04,dimen=outer](10.789687,-0.46828124){0.92}
\rput(1.364375,0.10171875){+}
\rput(2.124375,-0.09828125){+}
\rput(2.184375,-0.87828124){+}
\rput(1.504375,-0.49828124){+}
\rput(1.364375,-1.1182812){+}
\rput(1.004375,-0.31828126){+}
\rput(1.1751562,-0.17828125){-}
\rput(1.5751562,0.04171875){-}
\rput(2.2551563,-0.19828124){-}
\rput(1.7351563,-0.45828125){-}
\rput(1.9951563,-0.97828126){-}
\rput(1.1751562,-0.91828126){-}
\rput(5.964375,0.08171875){+}
\rput(6.884375,-0.45828125){+}
\rput(5.624375,-0.35828125){+}
\rput(6.064375,-1.1782813){+}
\rput(6.404375,-0.35828125){+}
\rput(6.584375,-0.8382813){+}
\rput(5.784375,-0.8382813){+}
\rput(6.384375,0.00171875){+}
\rput(5.755156,0.00171875){-}
\rput(5.8351564,-0.41828126){-}
\rput(5.715156,-0.97828126){-}
\rput(6.215156,-1.0582813){-}
\rput(6.8951564,-0.61828125){-}
\rput(6.615156,0.04171875){-}
\rput(10.7951565,0.26171875){-}
\rput(11.084375,0.12171875){+}
\rput(11.384375,-0.43828124){+}
\rput(10.944375,-0.25828126){+}
\rput(10.784375,-1.1382812){+}
\rput(10.264375,-0.8382813){+}
\rput(10.364375,-0.21828125){+}
\rput(10.895156,-0.39828125){-}
\rput(11.395156,-0.6982812){-}
\rput(11.075156,-1.1182812){-}
\rput(10.395156,-0.9582813){-}
\rput(10.115156,-0.21828125){-}
\rput(10.595157,-0.59828126){-}
\rput(10.955156,-0.79828125){-}
\rput(10.475156,0.14171875){-}
\rput(1.019375,2.1217186){Daar is:}
\rput(1.7278125,1.4117187){\small 6 positiewe ladings en}
\rput(1.4625,1.0517187){\small 6 negatiewe ladings}
\rput(1.66125,0.63171875){\small 6 + (-6) = 0}
\rput(1.6478125,-1.6882813){\small Daar is nul netto lading:}
\rput(1.6682812,-2.0882812){\small Die voorwerp is neutraal}
\rput(6.284844,1.4117187){\small 8 positiewe ladings en}
\rput(6.0225,1.0517187){\small 6 negatiewe ladings}
\rput(6.257344,0.63171875){\small 8 + (-6) = 2}
\rput(6.2725,-1.6882813){\small Die netto lading is +2}
\rput(6.33,-2.0882812){\small Die voorwerp is positief gelaai}
\rput(10.787812,1.4117187){\small 6 positiewe ladings en}
\rput(10.523125,1.0517187){\small 9 negatiewe ladings}
\rput(10.813281,0.63171875){\small 6 + (-9) = -3}
\rput(10.745469,-1.6882813){\small Die netto lading is -3}
\rput(10.87,-2.0882812){\small Die voorwerp is negatief gelaai}
\end{pspicture}
\end{center}
 \end{figure}       
      

\par 

Positiewe lading in materie word deur protone gedra en negatiewe lading deur elektrone. Die verandering in die algehele lading van 'n voorwerp is gewoonlik vanwe\"e die verandering in die hoeveelheid elektrone in die voorwerp.

Om 'n voorwerp meer
      \begin{itemize}
        \item \textbf{Positief gelaai te maak}: elektrone moet weggeneem word, en die voorwerp word elektron \textvf{arm}
        \item \textbf{Negatief gelaai te maak}: elektrone word bygevoeg en die voorwerp word \textbf{ryk} aan elektrone.
      \end{itemize}

So in die praktyk bly die hoeveelheid positiewe ladings (protone) dieselfde en die hoeveelheid elektrone verander.

\begin{figure}[H] % horizontal\label{m38780*id200643}
    \begin{center}
    \begin{pspicture}(0,-2.3017187)(13.100625,2.3017187)
\pscircle[linewidth=0.04,dimen=outer](1.6696875,-0.48828125){0.92}
\pscircle[linewidth=0.04,dimen=outer](6.2496877,-0.48828125){0.92}
\pscircle[linewidth=0.04,dimen=outer](10.789687,-0.46828124){0.92}
\rput(1.364375,0.10171875){+}
\rput(2.124375,-0.09828125){+}
\rput(2.184375,-0.87828124){+}
\rput(1.504375,-0.49828124){+}
\rput(1.364375,-1.1182812){+}
\rput(1.004375,-0.31828126){+}
\rput(1.1751562,-0.17828125){-}
\rput(1.5751562,0.04171875){-}
\rput(2.2551563,-0.19828124){-}
\rput(1.7351563,-0.45828125){-}
\rput(1.9951563,-0.97828126){-}
\rput(1.1751562,-0.91828126){-}
\rput(5.964375,0.08171875){+}
\rput(6.884375,-0.45828125){+}
\rput(5.624375,-0.35828125){+}
\rput(6.064375,-1.1782813){+}
\rput(6.404375,-0.35828125){+}
\rput(6.584375,-0.8382813){+}
% \rput(5.784375,-0.8382813){+}
% \rput(6.384375,0.00171875){+}
% \rput(5.755156,0.00171875){-}
% \rput(5.8351564,-0.41828126){-}
\rput(5.715156,-0.97828126){-}
\rput(6.215156,-1.0582813){-}
\rput(6.8951564,-0.61828125){-}
\rput(6.615156,0.04171875){-}
\rput(10.7951565,0.26171875){-}
\rput(11.084375,0.12171875){+}
\rput(11.384375,-0.43828124){+}
\rput(10.944375,-0.25828126){+}
\rput(10.784375,-1.1382812){+}
\rput(10.264375,-0.8382813){+}
\rput(10.364375,-0.21828125){+}
\rput(10.895156,-0.39828125){-}
\rput(11.395156,-0.6982812){-}
\rput(11.075156,-1.1182812){-}
\rput(10.395156,-0.9582813){-}
\rput(10.115156,-0.21828125){-}
\rput(10.595157,-0.59828126){-}
\rput(10.955156,-0.79828125){-}
\rput(10.475156,0.14171875){-}
\rput(1.019375,2.1217186){Daar is:}
\rput(1.7278125,1.4117187){\small \textbf{6} positiewe ladings and}
\rput(1.4625,1.0517187){\small 6 negatiewe ladings}
\rput(1.66125,0.63171875){\small 6 + (-6) = 0}
\rput(1.6478125,-1.6882813){\small Die netto lading is nul:}
\rput(1.6682812,-2.0882812){\small Die voorwerp is neutraal}
\rput(6.284844,1.4117187){\small \textbf{6} positiewe ladings en}
\rput(6.0225,1.0517187){\small 4 negatiewe ladings}
\rput(6.257344,0.63171875){\small 6 + (-4) = 2}
\rput(6.2725,-1.6882813){\small Die netto lading is +2}
\rput(6.33,-2.0882812){\small Die voorwerp is positief gelaai}
\rput(10.787812,1.4117187){\small \textbf{6} positiewe ladings end}
\rput(10.523125,1.0517187){\small 9 negatiewe ladings}
\rput(10.813281,0.63171875){\small 6 + (-9) = -3}
\rput(10.745469,-1.6882813){\small Die netto lading is -3}
\rput(10.87,-2.0882812){\small Die voorwerp is negatief gelaai}
\end{pspicture}
\end{center}
 \end{figure}       



\subsection{Tribo-elektriese belading}
\nopagebreak
Voorwerpe kan op vele maniere gelaai word, insluitend deur kontak met of wrywing teen ander voorwerpe. Dit beteken dat hulle negatiewe lading kan bykry of verloor. Byvoorbeeld, belading kan gebeur as jy jou voete skuur teen die tapyt. Waneer jy dan aan metaal of 'n ander persoon raak, kan jy 'n skok voel wanneer die oortollig lading wat jy opgebou het \textbf{ontlaai}.\par

\Tip{Lading, soos energie, kan nie geskep of vernietig word nie. Ons s\^e lading word \textbf{behou}.}

Wanneer jy jou voete teen die tapyt vryf, word negatiewe lading na jou toe oorgedra vanaf die tapyt. Die tapyt word dus positief belaai met \textbf{dieselfde hoeveelheid}. \par
      

\mindsetvid{Behaviour of charged particles}{VPfns}
'n Ander voorbeeld is om twee neutrale voorwerpe te vat, soos 'n plastiese liniaal en 'n katoen lap (sakdoek). In die begin is altwee voorwerpe neutraal.\par
      

\begin{figure}[H] % horizontal\label{m38780*id200777}
    \begin{center}
    \begin{pspicture}(0,-1.9746875)(14.900937,1.9746875)
\definecolor{color1034}{rgb}{0.6,0.6,0.6}
\psline[linewidth=0.04cm,linecolor=color1034](0.234375,0.67875)(0.234375,0.41875)
\psline[linewidth=0.04cm,linecolor=color1034](0.314375,0.69875)(0.314375,0.53875)
\psline[linewidth=0.04cm,linecolor=color1034](0.394375,0.69875)(0.394375,0.53875)
\psline[linewidth=0.04cm,linecolor=color1034](0.454375,0.69875)(0.454375,0.53875)
\psline[linewidth=0.04cm,linecolor=color1034](0.534375,0.69875)(0.534375,0.53875)
\psline[linewidth=0.04cm,linecolor=color1034](0.614375,0.67875)(0.614375,0.41875)
\psline[linewidth=0.04cm,linecolor=color1034](0.694375,0.69875)(0.694375,0.53875)
\psline[linewidth=0.04cm,linecolor=color1034](0.774375,0.69875)(0.774375,0.53875)
\psline[linewidth=0.04cm,linecolor=color1034](0.834375,0.69875)(0.834375,0.53875)
\psline[linewidth=0.04cm,linecolor=color1034](0.914375,0.69875)(0.914375,0.53875)
\psline[linewidth=0.04cm,linecolor=color1034](0.974375,0.67875)(0.974375,0.41875)
\psline[linewidth=0.04cm,linecolor=color1034](1.054375,0.69875)(1.054375,0.53875)
\psline[linewidth=0.04cm,linecolor=color1034](1.134375,0.69875)(1.134375,0.53875)
\psline[linewidth=0.04cm,linecolor=color1034](1.194375,0.69875)(1.194375,0.53875)
\psline[linewidth=0.04cm,linecolor=color1034](1.274375,0.69875)(1.274375,0.53875)
\psline[linewidth=0.04cm,linecolor=color1034](1.354375,0.67875)(1.354375,0.41875)
\psline[linewidth=0.04cm,linecolor=color1034](1.434375,0.69875)(1.434375,0.53875)
\psline[linewidth=0.04cm,linecolor=color1034](1.514375,0.69875)(1.514375,0.53875)
\psline[linewidth=0.04cm,linecolor=color1034](1.574375,0.69875)(1.574375,0.53875)
\psline[linewidth=0.04cm,linecolor=color1034](1.654375,0.69875)(1.654375,0.53875)
\psline[linewidth=0.04cm,linecolor=color1034](1.734375,0.67875)(1.734375,0.41875)
\psline[linewidth=0.04cm,linecolor=color1034](1.814375,0.69875)(1.814375,0.53875)
\psline[linewidth=0.04cm,linecolor=color1034](1.894375,0.69875)(1.894375,0.53875)
\psline[linewidth=0.04cm,linecolor=color1034](1.954375,0.69875)(1.954375,0.53875)
\psline[linewidth=0.04cm,linecolor=color1034](2.034375,0.69875)(2.034375,0.53875)
\psline[linewidth=0.04cm,linecolor=color1034](2.114375,0.67875)(2.114375,0.41875)
\psline[linewidth=0.04cm,linecolor=color1034](2.194375,0.69875)(2.194375,0.53875)
\psline[linewidth=0.04cm,linecolor=color1034](2.274375,0.69875)(2.274375,0.53875)
\psline[linewidth=0.04cm,linecolor=color1034](2.334375,0.69875)(2.334375,0.53875)
\psline[linewidth=0.04cm,linecolor=color1034](2.414375,0.69875)(2.414375,0.53875)
\psline[linewidth=0.04cm,linecolor=color1034](2.494375,0.67875)(2.494375,0.41875)
\psline[linewidth=0.04cm,linecolor=color1034](2.574375,0.69875)(2.574375,0.53875)
\psline[linewidth=0.04cm,linecolor=color1034](2.654375,0.69875)(2.654375,0.53875)
\psline[linewidth=0.04cm,linecolor=color1034](2.714375,0.69875)(2.714375,0.53875)
\psline[linewidth=0.04cm,linecolor=color1034](2.794375,0.69875)(2.794375,0.53875)
\psline[linewidth=0.04cm,linecolor=color1034](2.874375,0.67875)(2.874375,0.41875)
\psline[linewidth=0.04cm,linecolor=color1034](2.954375,0.69875)(2.954375,0.53875)
\psline[linewidth=0.04cm,linecolor=color1034](3.034375,0.69875)(3.034375,0.53875)
\psline[linewidth=0.04cm,linecolor=color1034](3.094375,0.69875)(3.094375,0.53875)
\psline[linewidth=0.04cm,linecolor=color1034](3.174375,0.69875)(3.174375,0.53875)
\psline[linewidth=0.04cm,linecolor=color1034](3.254375,0.67875)(3.254375,0.41875)
\psline[linewidth=0.04cm,linecolor=color1034](3.334375,0.69875)(3.334375,0.53875)
\psline[linewidth=0.04cm,linecolor=color1034](3.414375,0.69875)(3.414375,0.53875)
\psline[linewidth=0.04cm,linecolor=color1034](3.474375,0.69875)(3.474375,0.53875)
\psline[linewidth=0.04cm,linecolor=color1034](3.554375,0.69875)(3.554375,0.53875)
\psline[linewidth=0.04cm,linecolor=color1034](3.634375,0.67875)(3.634375,0.41875)
\psline[linewidth=0.04cm,linecolor=color1034](3.714375,0.69875)(3.714375,0.53875)
\psline[linewidth=0.04cm,linecolor=color1034](3.794375,0.69875)(3.794375,0.53875)
\psline[linewidth=0.04cm,linecolor=color1034](3.854375,0.69875)(3.854375,0.53875)
\psline[linewidth=0.04cm,linecolor=color1034](3.934375,0.69875)(3.934375,0.53875)
\psline[linewidth=0.04cm,linecolor=color1034](4.014375,0.67875)(4.014375,0.41875)
\psline[linewidth=0.04cm,linecolor=color1034](4.094375,0.69875)(4.094375,0.53875)
\psline[linewidth=0.04cm,linecolor=color1034](4.174375,0.69875)(4.174375,0.53875)
\psline[linewidth=0.04cm,linecolor=color1034](4.234375,0.69875)(4.234375,0.53875)
\psline[linewidth=0.04cm,linecolor=color1034](4.314375,0.69875)(4.314375,0.53875)
\psline[linewidth=0.04cm,linecolor=color1034](4.394375,0.67875)(4.394375,0.41875)
\psline[linewidth=0.04cm,linecolor=color1034](4.474375,0.69875)(4.474375,0.53875)
\psline[linewidth=0.04cm,linecolor=color1034](4.554375,0.69875)(4.554375,0.53875)
\psline[linewidth=0.04cm,linecolor=color1034](4.614375,0.69875)(4.614375,0.53875)
\psline[linewidth=0.04cm,linecolor=color1034](4.694375,0.69875)(4.694375,0.53875)
\psline[linewidth=0.04cm,linecolor=color1034](4.794375,0.69875)(4.794375,0.43875)
\psframe[linewidth=0.04,dimen=outer](5.234375,0.69875)(0.054375,-0.02125)
\rput(4.9290624,0.24875){+}
\rput(4.4090624,0.24875){+}
\rput(3.8690624,0.22875){+}
\rput(3.3290625,0.22875){+}
\rput(2.7090626,0.24875){+}
\rput(2.1890626,0.24875){+}
\rput(1.7090625,0.22875){+}
\rput(1.1490625,0.22875){+}
\rput(0.5090625,0.20875){+}
\rput(0.21984375,0.24875){-}
\rput(0.81984377,0.20875){-}
\rput(1.3998437,0.24875){-}
\rput(1.9198438,0.28875){-}
\rput(2.4198437,0.28875){-}
\rput(2.9598436,0.24875){-}
\rput(3.6198437,0.24875){-}
\rput(4.119844,0.28875){-}
\rput(4.639844,0.26875){-}
\psbezier[linewidth=0.04](6.042201,1.0521983)(5.814375,1.19875)(6.6262674,-0.91064626)(6.6262674,-0.53135264)(6.6262674,-0.152059)(8.434375,-1.02125)(8.401829,-0.5882467)(8.369282,-0.15524337)(7.636984,0.64375)(7.86481,0.8406855)(8.092636,1.037621)(6.270027,0.9056466)(6.042201,1.0521983)
\rput(6.6290627,0.62875){+}
\rput(7.4490623,0.30875){+}
\rput(6.6690626,-0.09125){+}
\rput(7.6490626,-0.37125){+}
\rput(7.3090625,0.76875){+}
\rput(6.4598436,0.48875){-}
\rput(6.8798437,0.08875){-}
\rput(7.9598436,-0.37125){-}
\rput(7.5398436,0.12875){-}
\rput(7.099844,0.66875){-}
\rput(2.4446876,-0.34125){\small Die liniaal het 9 positiewe ladings en}
\rput(7.749375,-1.12125){\small die katoen lap het}
\rput(7.7117186,-1.44125){\small 5 positiewe ladings and}
\rput(7.726406,-1.76125){\small 5 negatiewe ladings}
\rput(12.6725,0.91875){\small Die totale lading is:}
\rput(12.3175,0.49875){\small (9+5)=14 positiewe ladings}
\rput(12.3375,0.11875){\small (9+5)=14 negatiewe ladings}
\rput(1.3078125,-0.68125){\small 9 negatiewe ladings}
\rput(1.2048438,1.79875){\small VOOR wrywing:}
\end{pspicture}\end{center}
 \end{figure}       
      \par 


As die katoen lap nou gebruik word om die liniaal te vryf, sal negatiewe lading van die lap na die liniaal oorgedra word. Die liniaal is nou \textsl{negatief} gelaai (dit het nou oortollige elektrone) en die lap is \textsl{positief} gelaai (dit is arm in elektrone).

As jy al die positiewe en negatiewe ladings tel voor en na die eksperiment, sal jy dieselfde hoeveelheid tel, die totale lading is \textsl{behou}!. \par     

\begin{figure}[H] % horizontal\label{m38780*id200819}
    \begin{center}
    \begin{pspicture}(0,-2.1746874)(15.098437,2.1746874)
\definecolor{color1034}{rgb}{0.6,0.6,0.6}
\psline[linewidth=0.04cm,linecolor=color1034](0.2371875,0.87875)(0.2371875,0.61875)
\psline[linewidth=0.04cm,linecolor=color1034](0.3171875,0.89875)(0.3171875,0.73875)
\psline[linewidth=0.04cm,linecolor=color1034](0.3971875,0.89875)(0.3971875,0.73875)
\psline[linewidth=0.04cm,linecolor=color1034](0.4571875,0.89875)(0.4571875,0.73875)
\psline[linewidth=0.04cm,linecolor=color1034](0.5371875,0.89875)(0.5371875,0.73875)
\psline[linewidth=0.04cm,linecolor=color1034](0.6171875,0.87875)(0.6171875,0.61875)
\psline[linewidth=0.04cm,linecolor=color1034](0.6971875,0.89875)(0.6971875,0.73875)
\psline[linewidth=0.04cm,linecolor=color1034](0.7771875,0.89875)(0.7771875,0.73875)
\psline[linewidth=0.04cm,linecolor=color1034](0.8371875,0.89875)(0.8371875,0.73875)
\psline[linewidth=0.04cm,linecolor=color1034](0.9171875,0.89875)(0.9171875,0.73875)
\psline[linewidth=0.04cm,linecolor=color1034](0.9771875,0.87875)(0.9771875,0.61875)
\psline[linewidth=0.04cm,linecolor=color1034](1.0571876,0.89875)(1.0571876,0.73875)
\psline[linewidth=0.04cm,linecolor=color1034](1.1371875,0.89875)(1.1371875,0.73875)
\psline[linewidth=0.04cm,linecolor=color1034](1.1971875,0.89875)(1.1971875,0.73875)
\psline[linewidth=0.04cm,linecolor=color1034](1.2771875,0.89875)(1.2771875,0.73875)
\psline[linewidth=0.04cm,linecolor=color1034](1.3571875,0.87875)(1.3571875,0.61875)
\psline[linewidth=0.04cm,linecolor=color1034](1.4371876,0.89875)(1.4371876,0.73875)
\psline[linewidth=0.04cm,linecolor=color1034](1.5171875,0.89875)(1.5171875,0.73875)
\psline[linewidth=0.04cm,linecolor=color1034](1.5771875,0.89875)(1.5771875,0.73875)
\psline[linewidth=0.04cm,linecolor=color1034](1.6571875,0.89875)(1.6571875,0.73875)
\psline[linewidth=0.04cm,linecolor=color1034](1.7371875,0.87875)(1.7371875,0.61875)
\psline[linewidth=0.04cm,linecolor=color1034](1.8171875,0.89875)(1.8171875,0.73875)
\psline[linewidth=0.04cm,linecolor=color1034](1.8971875,0.89875)(1.8971875,0.73875)
\psline[linewidth=0.04cm,linecolor=color1034](1.9571875,0.89875)(1.9571875,0.73875)
\psline[linewidth=0.04cm,linecolor=color1034](2.0371876,0.89875)(2.0371876,0.73875)
\psline[linewidth=0.04cm,linecolor=color1034](2.1171875,0.87875)(2.1171875,0.61875)
\psline[linewidth=0.04cm,linecolor=color1034](2.1971874,0.89875)(2.1971874,0.73875)
\psline[linewidth=0.04cm,linecolor=color1034](2.2771876,0.89875)(2.2771876,0.73875)
\psline[linewidth=0.04cm,linecolor=color1034](2.3371875,0.89875)(2.3371875,0.73875)
\psline[linewidth=0.04cm,linecolor=color1034](2.4171875,0.89875)(2.4171875,0.73875)
\psline[linewidth=0.04cm,linecolor=color1034](2.4971876,0.87875)(2.4971876,0.61875)
\psline[linewidth=0.04cm,linecolor=color1034](2.5771875,0.89875)(2.5771875,0.73875)
\psline[linewidth=0.04cm,linecolor=color1034](2.6571875,0.89875)(2.6571875,0.73875)
\psline[linewidth=0.04cm,linecolor=color1034](2.7171874,0.89875)(2.7171874,0.73875)
\psline[linewidth=0.04cm,linecolor=color1034](2.7971876,0.89875)(2.7971876,0.73875)
\psline[linewidth=0.04cm,linecolor=color1034](2.8771875,0.87875)(2.8771875,0.61875)
\psline[linewidth=0.04cm,linecolor=color1034](2.9571874,0.89875)(2.9571874,0.73875)
\psline[linewidth=0.04cm,linecolor=color1034](3.0371876,0.89875)(3.0371876,0.73875)
\psline[linewidth=0.04cm,linecolor=color1034](3.0971875,0.89875)(3.0971875,0.73875)
\psline[linewidth=0.04cm,linecolor=color1034](3.1771874,0.89875)(3.1771874,0.73875)
\psline[linewidth=0.04cm,linecolor=color1034](3.2571876,0.87875)(3.2571876,0.61875)
\psline[linewidth=0.04cm,linecolor=color1034](3.3371875,0.89875)(3.3371875,0.73875)
\psline[linewidth=0.04cm,linecolor=color1034](3.4171875,0.89875)(3.4171875,0.73875)
\psline[linewidth=0.04cm,linecolor=color1034](3.4771874,0.89875)(3.4771874,0.73875)
\psline[linewidth=0.04cm,linecolor=color1034](3.5571876,0.89875)(3.5571876,0.73875)
\psline[linewidth=0.04cm,linecolor=color1034](3.6371875,0.87875)(3.6371875,0.61875)
\psline[linewidth=0.04cm,linecolor=color1034](3.7171874,0.89875)(3.7171874,0.73875)
\psline[linewidth=0.04cm,linecolor=color1034](3.7971876,0.89875)(3.7971876,0.73875)
\psline[linewidth=0.04cm,linecolor=color1034](3.8571875,0.89875)(3.8571875,0.73875)
\psline[linewidth=0.04cm,linecolor=color1034](3.9371874,0.89875)(3.9371874,0.73875)
\psline[linewidth=0.04cm,linecolor=color1034](4.0171876,0.87875)(4.0171876,0.61875)
\psline[linewidth=0.04cm,linecolor=color1034](4.0971875,0.89875)(4.0971875,0.73875)
\psline[linewidth=0.04cm,linecolor=color1034](4.1771874,0.89875)(4.1771874,0.73875)
\psline[linewidth=0.04cm,linecolor=color1034](4.2371874,0.89875)(4.2371874,0.73875)
\psline[linewidth=0.04cm,linecolor=color1034](4.3171873,0.89875)(4.3171873,0.73875)
\psline[linewidth=0.04cm,linecolor=color1034](4.3971877,0.87875)(4.3971877,0.61875)
\psline[linewidth=0.04cm,linecolor=color1034](4.4771876,0.89875)(4.4771876,0.73875)
\psline[linewidth=0.04cm,linecolor=color1034](4.5571876,0.89875)(4.5571876,0.73875)
\psline[linewidth=0.04cm,linecolor=color1034](4.6171875,0.89875)(4.6171875,0.73875)
\psline[linewidth=0.04cm,linecolor=color1034](4.6971874,0.89875)(4.6971874,0.73875)
\psline[linewidth=0.04cm,linecolor=color1034](4.7971873,0.89875)(4.7971873,0.63875)
\psframe[linewidth=0.04,dimen=outer](5.2371874,0.89875)(0.0571875,0.17875)
\rput(4.931875,0.44875){+}
\rput(4.411875,0.44875){+}
\rput(3.871875,0.42875){+}
\rput(3.331875,0.42875){+}
\rput(2.711875,0.44875){+}
\rput(2.191875,0.44875){+}
\rput(1.711875,0.42875){+}
\rput(1.151875,0.42875){+}
\rput(0.511875,0.40875){+}
\rput(0.22265625,0.44875){-}
\rput(0.8226563,0.40875){-}
\rput(1.4026562,0.44875){-}
\rput(1.9226563,0.48875){-}
\rput(2.4226563,0.48875){-}
\rput(2.9626563,0.44875){-}
\rput(3.6226563,0.44875){-}
\rput(4.1226563,0.48875){-}
\rput(4.6426563,0.46875){-}
\psbezier[linewidth=0.04](6.0450134,1.2521983)(5.8171873,1.39875)(6.62908,-0.7106463)(6.62908,-0.33135265)(6.62908,0.047941)(8.437187,-0.82125)(8.404641,-0.3882467)(8.372094,0.04475663)(7.6397963,0.84375)(7.8676224,1.0406855)(8.0954485,1.237621)(6.2728395,1.1056466)(6.0450134,1.2521983)
\rput(6.631875,0.82875){+}
\rput(7.451875,0.50875){+}
\rput(6.671875,0.10875){+}
\rput(7.651875,-0.17125){+}
\rput(7.311875,0.96875){+}
\rput(4.9826565,0.72875){-}
\rput(5.0426564,0.34875){-}
\rput(7.962656,-0.17125){-}
\rput(7.5426564,0.32875){-}
\rput(4.6226563,0.62875){-}
\rput(2.4475,-0.14125){\small Die liniaal het 9 positiewe ladings en}
\rput(7.4821873,-0.92125){\small Die katoen lap het}
\rput(7.614531,-1.24125){\small 5 positiewe ladings en}
\rput(7.3940625,-1.58125){\small 2 negatiewe ladings.}
\rput(12.585313,1.11875){\small Die totale lading is:}
\rput(12.3203125,0.69875){\small (9+5)=14 positiewe ladings}
\rput(12.430312,0.31875){\small (12+2)=14 negatiewe ladings}
\rput(1.3873438,-0.48125){\small 12 negatiewe ladings}
\rput(1.0790625,1.99875){\small NA wrywing:}
\rput(7.6434374,-1.96125){\small Dit is nou positief gelaai.}
\rput(2.0234375,-0.84125){\small Dit is nou negatief gelaai.}
\rput(12.391719,-0.28125){\small Ladings is oorgedra van die}
\rput(12.3515625,-0.64125){\small lap na die liniaal MAAR die totale}
\rput(10.865313,-0.98125){\small lading het behoue gebly!}
\end{pspicture}\end{center}
 \end{figure}       
 \par 

\begin{minipage}{.35\textwidth}

Neem kennis dat die hoeveelhede in hierdie voorbeeld maklik is om te bereken. In die regte w\^ereld sal daar net 'n klein breukdeel van die ladings oorgedra word, maar die totale lading sal nog steeds behoue bly. \par

Die proses waardeur materiale gelaai word wanneer hulle in kontak kom met ander materiale word tribo-elektriese belading. Materiale kan rangskik word in 'n tribo-elekriese reeks volgens hulle vermo\"e om meer positief of meer negatief te raak. Hierdie reeks kan ons help besluit of een materiaal belaai kan raak van 'n ander materiaal af.

Positiewe materiale is meer geneig om elektrone te verloor en negatiewe materiale is geneig om elektrone te wen. Wanner twee materiale gekies en teen mekaar gevryf word sal die meer positiewe een elektrone verloor en die negatiewe materiaal sal elektrone bykry.

Byvoorbeeld, \textbf{!!!AMBER!!! } is meer negatief as wol, so as 'n stuk wol teen amber gevryf word sal die amber meer negatief gelaai word.
\end{minipage}
\begin{minipage}{.55\textwidth}
\begin{center}
\begin{table}[H]
\centering
\begin{tabular}{|cc|}\hline
\textbf{Materiaal}&\textbf{Tribo-elektriese reeks}\\\hline
Glss& Baie positief\\\hline
Menslike hare&\\\hline
Nylon&\\\hline
Wol&\\\hline
Pels&\\\hline
Lood&\\\hline
Sy&\\\hline
Aluminium&\\\hline
Papier&\\\hline
Cotton&Neutraal\\\hline
Staal &Neutraal\\\hline
Hout&\\\hline
Amber&\\\hline
Harde rubber&\\\hline
Nikkel, Koper&\\\hline
Goud, Platinum&\\\hline
Polyester&\\\hline
Polyurethane&\\\hline
Polypropileen&\\\hline
Sikicon&\\\hline
Teflon& Baie negatief\\\hline
\end{tabular}
\caption{Tribo-elektriese reeks.}
\end{table}
\end{center}
\end{minipage}

\begin{wex}{Tribo-elekriese belading}
{
As jy katoen  en sy teen mekaar vryf, watter een word negatief belaai?
}
{
\westep{Analiseer die gegewe informasie}

Daar is twee materiale gegee en hulle word teen mekaar gevryf. Dit beteken ons is besig met die interaksie tussen die materiale. Die vraag is verwant aan die lading op die materiale en ons kan aanneem dat hulle neutraal was aan die begin. Dit beteken ons het te doen met elektrostatika en die interaksie tussen materiale wat lei tot belading is tribo-elektriese belading.


\westep{Verkry materiaal eienskappe}

Spoor die materiale op in die tribo-elektriese reeks. Die belangrike item om te weet is watter materiaal is meer positief en meer negatief in die reeks. Sy val bo katoen in ons tabel wat dit meer positief maak.

\westep{Pas die beginsels toe}
Ons weet dat wanneer twee materiale teen mekaar gevryf word, die meer negatiewe een in die reeks die elektrone gaan by kry en die meer positiewe een elektrone gaan verloor. Dit beteken dat die sy elektrone gaan verloor en die katoen elektrone gaan by kry.
\par
'n Materiaal wat meer negatief belaai raak het 'n oormaat van elektrone, dus raak die katoen, wat elektrone bykry, meer negatief belaai.
}\end{wex}

\subsection{Kragte tussen ladings}
            \nopagebreak

Die krag wat stilstaande (statiese) ladings op mekaar uitoefen word die \textbf{elektrostatiese krag} genoem. Die elektrostatiese krag tussen:

\begin{itemize}[noitemsep]
\item \textbf{soortgelyke} ladings is \textbf{afstotend}.
\item \textbf{teenoorgestelde} ladings is \textbf{aantreklik}.
\end{itemize}

In ander woorde, soortgelyke ladings stoot mekaar af en teenoorgestelde ladings trek mekaar aan.


\begin{figure}[H] % horizontal\label{m38780*id200901}
    \begin{center}
    \begin{pspicture}(0,-0.6)(10,0.8)
%\psgrid[gridcolor=gray]
\rput(-2,0){\pscircle[linewidth=1pt](0.5,0.25){0.25}
\pscircle[linewidth=1pt](2.5,0.25){0.25}
\psline[linewidth=2pt]{->}(0.85,0.25)(1.4,0.25)
\psline[linewidth=2pt]{<-}(1.6,0.25)(2.15,0.25) \rput(1,0.5){F}
\rput(2,0.5){F} \rput(0.5,0.25){-} \rput(2.5,0.25){+}
\uput[d](1.5,0){aantrekkingskrag}}

\rput(3.5,0){\pscircle[linewidth=1pt](2.5,0.25){0.25}
\pscircle[linewidth=1pt](0.5,0.25){0.25}
\psline[linewidth=2pt]{<-}(-0.4,0.25)(.15,0.25)
\psline[linewidth=2pt]{->}(2.85,0.25)(3.4,0.25) \rput(-.2,0.5){F}
\rput(3.2,0.5){F} \rput(0.5,0.25){-} \rput(2.5,0.25){-}
\uput[d](1.5,0){afstotende krag}}

\rput(9,0){\pscircle[linewidth=1pt](2.5,0.25){0.25}
\pscircle[linewidth=1pt](0.5,0.25){0.25}
\psline[linewidth=2pt]{<-}(-0.4,0.25)(.15,0.25)
\psline[linewidth=2pt]{->}(2.85,0.25)(3.4,0.25) \rput(-.2,0.5){F}
\rput(3.2,0.5){F} \rput(0.5,0.25){+} \rput(2.5,0.25){+}
\uput[d](1.5,0){afstotende krag}}
\end{pspicture}\end{center}
 \end{figure}       
      \par 
Hoe \textsl{nader} ladings aan mekaar is, hoe \textsl{sterker} word die elektrostatiese krag tussen hulle.\par
	
\begin{figure}[H] % horizontal\label{m38780*id200924}
    \begin{center}
 \begin{pspicture}(0,-2.3292189)(9.975,2.3292189)
\pscircle[linewidth=0.035277776,dimen=outer](3.45,1.9107813){0.25}
\pscircle[linewidth=0.035277776,dimen=outer](1.45,1.9107813){0.25}
\psline[linewidth=0.07055555cm,arrowsize=0.05291667cm 2.0,arrowlength=1.4,arrowinset=0.4]{<-}(.1,1.9107813)(1.1,1.9107813)
\psline[linewidth=0.07055555cm,arrowsize=0.05291667cm 2.0,arrowlength=1.4,arrowinset=0.4]{->}(3.8,1.9107813)(4.8,1.9107813)
\rput(1.9196875,2.1607811){F}
\rput(2.9196875,2.1607811){F}
\rput(1.4246875,1.9107813){+}
\rput(3.4246874,1.9107813){+}
\rput(7.5,2.0407813){sterker afstotende krag}
\rput(7.5,0.9607813){swakker afstotende krag}

\pscircle[linewidth=0.035277776,dimen=outer](4.65,0.73078126){0.25}
\pscircle[linewidth=0.035277776,dimen=outer](0.25,0.73078126){0.25}
\psline[linewidth=0.07055555cm,arrowsize=0.05291667cm 2.0,arrowlength=1.4,arrowinset=0.4]{->}(-0.1,0.69078124)(-.5,0.69078124)
\psline[linewidth=0.07055555cm,arrowsize=0.05291667cm 2.0,arrowlength=1.4,arrowinset=0.4]{->}(5.0,0.69078124)(5.4,0.69078124)
\rput(0.7396875,0.94078124){F}
\rput(4.1396875,0.94078124){F}
\rput(0.2446875,0.73078126){+}
\rput(4.6246877,0.73078126){+}
% \rput(7.445156,1.6807812){(shorter distance between charges)}
% \rput(7.405156,0.5807812){(longer distance between charges)}
\end{pspicture}   
\end{center}
 \end{figure}       
      \par 
\label{m38780*secfhsst!!!underscore!!!id162}
            

\begin{g_experiment}{Electrostatic force}
            \nopagebreak
Jy kan maklik toets dat soortgelyke ladings mekaar afstoot en teenoorgestelde ladings mekaar aantrek met hierdie eenvoudige eksperiment. \par      

Neem 'n glas staaf en vryf dit met 'n stuk sy en hang dit dan met 'n stuk tou in die middel sodat dit vrylik kan beweeg. As jy nog 'n glas staaf wat jy ook gelaai het nader bring aan die een wat hang, sal jy sien dat die eerste glas staaf \textsl{weg beweeg} van die een in jou hand, dit word \textbf{afgestoot}. Andersins, as jy 'n plastiese staaf met 'n stuk pels vryf en dit nader bring aan die glas staaf met die tou sal dit glas staaf \textsl{na die plastiese staaf toe} beweeg, dit word \textbf{aangetrek}.\par
      
	\begin{figure}[H] % horizontal\label{m38780*id200974}
    \begin{center}
  \begin{pspicture}(0,0.8)(11.2,8)
%\psgrid
%left pic
\psline[linewidth = 2pt](2, 7.5)(4, 7.5) \multiput(2.1, 7.6)(0.2,
0){10}{/} \psline(3, 7.5)(3, 4.5) \psframe(1, 3.5)(5, 4.5)
\psline{->}(0.5, 3.5)(0.5, 4.5) \psline{->}(0.5, 3.3)(0.5, 2.3)
\rput(0.75, 4.5){F} \rput(0.75, 2.5){F}
\rput{30}{%
\psline(1,0.5)(3,0.5) \psline(3,0.5)(3, 1.5) \psline(3,1.5)(1,1.5)
}
%plus signs!
\rput(1.25, 3.75){+} \rput(1.25, 4.25){+} \rput(1.5, 4){+}
\rput(1.75, 3.75){+} \rput(1.75, 4.25){+} \rput(2, 4){+}
%More plus signs!
\rput(1.25, 1.5){+} \rput(1.25, 2){+} \rput(1.5, 1.75){+}
\rput(1.5, 2.25){+} \rput(1.75, 2){+} \rput(1.75, 2.5){+}
%arrow!
\psecurve[linewidth = 2pt]{->}(2.5, 6)(2.5, 6)(3, 6.25 )(3.5,
6)(3, 5.75)(2.5, 6)
%RIGHT pic
\psline[linewidth = 2pt](8, 7.5)(10, 7.5) \multiput(8.1, 7.6)(0.2,
0){10}{/} \psline(9, 7.5)(9, 4.5) \psframe(7, 3.5)(11, 4.5)
\psline{<-}(6.5, 3.5)(6.5, 4.5) \psline{<-}(6.5, 3.3)(6.5, 2.3)
\rput(6.75, 4.5){F} \rput(6.75, 2.5){F}
\rput{30}(1,-3){%
\psline(7,0.5)(9,0.5) \psline(9,0.5)(9, 1.5) \psline(9,1.5)(7,1.5)
}
%plus signs!
%\degree[12]
%\multido{\n=0+.1}{12}{\rput{\n}{+}}
\rput(7.25, 3.75){+} \rput(7.25, 4.25){+} \rput(7.5, 4){+}
\rput(7.75, 3.75){+} \rput(7.75, 4.25){+} \rput(8, 4){+}
%Minussigns!
\rput(7.25, 1.5){-} \rput(7.25, 2){-} \rput(7.5, 1.75){-}
\rput(7.5, 2.25){-} \rput(7.75, 2){-} \rput(7.75, 2.5){-} \rput(8,
2.25){-}
%arrow!
\psecurve[linewidth = 2pt]{<-}(8.5, 6)(8.5, 6)(9, 6.25 )(9.5,
6)(9, 5.75)(9.5, 6)
\end{pspicture}
  \end{center}
 \end{figure}       
      \par 

Dit gebeur want die glas verloor klein hoeveelhede negatiewe lading wanneer jy dit vryf met sy, wat die glas meer \textbf{positief} gelaai maak. Waneer jy die plastiek met pels vryf, word klein hoeveelhede negatiewe lading oorgedra na die plastiek toe wat dit meer \textbf{negatief} gelaai maak.\par

\end{g_experiment}



\begin{wex}
{
Toepassing van elektrostatiese kragte
}
{
Twee gelaaide metaal sfere hang van toutjies en is vry om te beweeg soos in die diagram hieronder. Die sfeer aan die regterkant is positief gelaai. Die lading van die ander sfeer is nie bekend nie.
\begin{center}
\begin{pspicture}(0,-1.23)(2.74,1.25)
\pscircle[linewidth=0.04,dimen=outer](0.41,-0.82){0.41}
\pscircle[linewidth=0.04,dimen=outer](2.33,-0.82){0.41}
\psline[linewidth=0.04cm](0.4,-0.43)(0.4,1.23)
\psline[linewidth=0.04cm](2.34,-0.45)(2.34,1.21)
\rput(2.3559375,-0.815){\large +}
\rput(0.3790625,-0.835){\large ?}
\end{pspicture}
\end{center}
Die linkerkantste sfeer word nou nader gebring aan die regterkantste sfeer.      
\begin{enumerate}[noitemsep, label=\textbf{\arabic*}. ] 
\item As die linkerkantste sfeer na die regterkantste sfeer to swaai, wat kan jy s\^{e} van die lading op die linkerkantste sfeer en hoekom?
\item As die linkerkantste sfeer wegswaai van die regterkantste een, wat kan jy dan s\^{e} van die lading op die linkerkantste sfeer en hoekom?
\end{enumerate}
}
{
\westep{Analiseer die probleem}  
In die eerste geval het ons 'n sfeer met 'n positiewe lading wat \textsl{aangetrek} word na die linkerkantste sfeer. Ons moet die lading van die linkerkantste sfeer vind.\par

\westep{Identifiseer die beginsels.}  

Ons het te doen met elektrostatiese kragte tussen gelaaide voorwerpe. Daarom weet ons dat \textsl{soortgelyke} ladings \textsl{afstotend} is en dat \textsl{teenoorgestelde} ladings mekaar \textsl{aantrek}.\par
      
\westep{Pas die beginsels toe.}
\begin{enumerate}[noitemsep, label=\textbf{\alph*}. ] 
    \item In die eerste geval word die positiewe sfeer aangetrek deur die linkerkantste sfeer. Sedert die elektrostatiese krag tussen twee teenoorgestelde ladings aantreklik is weet ons dat die linkerkantste sfeer \textsl{negatief} gelaai moet wees.
    \item In die tweede geval word die positiewe sfeer afgestoor deur die ander sfeer. Soortgelyke ladings stoot mekaar af. dus weet ons dat die linkerkantste sfeer ook 'n \textsl{positiewe} lading het.
\end{enumerate}
}
\end{wex}
    

\IFact{The woord 'elektron' kom van die Griekse woord vir amber. Die antieke Grieke het gesien dat wanneer jy 'n stuk amber vryf, jy stukkies strooi daarmee kan optel.}


\subsection*{Polarisasie}
            \nopagebreak
In teenstelling met geleiers, is die elektrone in isolators (nie-geleiers) verbind aan die atome van die isolator en kan nie vrylik rondbeweeg in die materiaal nie. 'n Gelaaide voorwerp kan egter nog steeds 'n krag uitoefen op 'n neutrale isolator deur 'n verskynsel wat \textbf{polarisasie} genoem word. \par

Wanneer 'n positief gelaaide staaf naby 'n neutrale isolator, soos polistireen, gebring word, kan dit die verbinde elektrone naaste aan die staaf aantrek and die positiewe atoomkerns effens afstoot. Hierdie proses word \textsl{polarisasie} genoem. Alhoewel dit 'n baie klein (mikroskopiese) effek is, kan dit genoem krag opwek as daar genoeg atome betrokke is, om 'n ligte voorwerp soos 'n polistireen bal rond te beweeg. Onthou, die polistireen bal is \textsl{net gepolariseer} en \textsl{nie gelaai nie}. Die polistireen bal is nog steeds neutraal omdat geen ladings bygevoeg of weggevat is nie. Die prent wys gepolariseerde atome (nie volgens skaal) in die polistireen bal:\par


\begin{figure}[H] % horizontal\label{m38780*id201917}
    \begin{center}
    \begin{pspicture}(0,-2.96)(6.84125,2.96)
\psarc[linewidth=0.04](1.11125,-1.89){1.05}{176.37851}{175.91438}
\psline[linewidth=0.04cm](1.80125,0.48)(4.12125,2.94)
\psline[linewidth=0.04cm](3.42125,-1.04)(6.82125,2.74)
\psbezier[linewidth=0.04](1.80125,0.48)(1.80125,-0.32)(1.94125,-1.22)(3.44125,-1.04)
\rput(2.0571876,0.275){\large +}
\rput(2.0771875,-0.145){\large +}
\rput(2.2171874,-0.445){\large +}
\rput(2.4771874,-0.725){\large +}
\rput(2.8571875,-0.925){\large +}
\rput(3.2371874,-0.885){\large +}
\rput(2.8171875,-0.645){\large +}
\rput(2.5371876,-0.445){\large +}
\rput(1.6571875,0.215){\large +}
\rput(1.7371875,-0.105){\large +}
\rput(1.8171875,-0.485){\large +}
\rput(2.0171876,-0.725){\large +}
\rput(2.1971874,-1.005){\large +}
\rput(2.4971876,-1.205){\large +}
\rput(2.8371875,-1.285){\large +}
\rput(3.1971874,-1.265){\large +}
\rput{45.571033}(-0.47389764,-1.0318439){\psellipse[linewidth=0.04,dimen=outer](0.99125,-1.08)(0.25,0.18)}
\rput{45.571033}(-0.5237394,-1.0012594){\rput(0.92367446,-1.137003){\small +}}
\rput{45.571033}(-0.39813614,-1.0650314){\rput(1.0548837,-0.98266524){\small -}}
\rput{45.571033}(-0.39447117,-1.3009257){\psellipse[linewidth=0.04,dimen=outer](1.35125,-1.12)(0.25,0.18)}
\rput{45.571033}(-0.44431296,-1.2703412){\rput(1.2836745,-1.1770029){\small +}}
\rput{45.571033}(-0.31870967,-1.3341132){\rput(1.4148837,-1.0226653){\small -}}
\rput{45.571033}(-0.46158466,-1.581156){\psellipse[linewidth=0.04,dimen=outer](1.65125,-1.34)(0.25,0.18)}
\rput{45.571033}(-0.51142645,-1.5505716){\rput(1.5836744,-1.3970029){\small +}}
\rput{45.571033}(-0.38582316,-1.6143435){\rput(1.7148837,-1.2426652){\small -}}
\rput{45.571033}(-0.60982573,-1.8282549){\psellipse[linewidth=0.04,dimen=outer](1.87125,-1.64)(0.25,0.18)}
\rput{45.571033}(-0.6596675,-1.7976704){\rput(1.8036745,-1.6970029){\small +}}
\rput{45.571033}(-0.53406423,-1.8614424){\rput(1.9348837,-1.5426652){\small -}}
\rput{45.571033}(-0.88575757,-2.0113742){\psellipse[linewidth=0.04,dimen=outer](1.95125,-2.06)(0.25,0.18)}
\rput{45.571033}(-0.9355994,-1.9807895){\rput(1.8836745,-2.117003){\small +}}
\rput{45.571033}(-0.80999607,-2.0445616){\rput(2.0148838,-1.9626652){\small -}}
\rput{45.571033}(-1.2756802,-1.9231282){\psellipse[linewidth=0.04,dimen=outer](1.65125,-2.48)(0.25,0.18)}
\rput{45.571033}(-1.325522,-1.8925437){\rput(1.5836744,-2.5370028){\small +}}
\rput{45.571033}(-1.1999187,-1.9563158){\rput(1.7148837,-2.3826652){\small -}}
\rput{45.571033}(-0.96518403,-1.7422923){\psellipse[linewidth=0.04,dimen=outer](1.59125,-2.02)(0.25,0.18)}
\rput{45.571033}(-1.0150259,-1.7117077){\rput(1.5236745,-2.077003){\small +}}
\rput{45.571033}(-0.88942254,-1.7754798){\rput(1.6548836,-1.9226652){\small -}}
\rput{45.571033}(-0.8312254,-1.501193){\psellipse[linewidth=0.04,dimen=outer](1.37125,-1.74)(0.25,0.18)}
\rput{45.571033}(-0.88106716,-1.4706085){\rput(1.3036745,-1.7970029){\small +}}
\rput{45.571033}(-0.7554639,-1.5343806){\rput(1.4348837,-1.6426653){\small -}}
\rput{45.571033}(-0.7783943,-1.2269622){\psellipse[linewidth=0.04,dimen=outer](1.07125,-1.54)(0.25,0.18)}
\rput{45.571033}(-0.82823604,-1.1963776){\rput(1.0036745,-1.5970029){\small +}}
\rput{45.571033}(-0.7026328,-1.2601497){\rput(1.1348836,-1.4426652){\small -}}
\rput{45.571033}(-0.7141464,-0.7799101){\psellipse[linewidth=0.04,dimen=outer](0.57125,-1.24)(0.25,0.18)}
\rput{45.571033}(-0.76398814,-0.7493256){\rput(0.50367445,-1.2970029){\small +}}
\rput{45.571033}(-0.6383849,-0.8130976){\rput(0.6348837,-1.1426653){\small -}}
\rput{45.571033}(-1.0957861,-0.6713823){\psellipse[linewidth=0.04,dimen=outer](0.25125,-1.64)(0.25,0.18)}
\rput{45.571033}(-1.1456279,-0.64079773){\rput(0.18367445,-1.6970029){\small +}}
\rput{45.571033}(-1.0200247,-0.7045698){\rput(0.31488368,-1.5426652){\small -}}
\rput{45.571033}(-0.8578207,-0.9578804){\psellipse[linewidth=0.04,dimen=outer](0.71125,-1.5)(0.25,0.18)}
\rput{45.571033}(-0.9076625,-0.92729586){\rput(0.64367443,-1.5570029){\small +}}
\rput{45.571033}(-0.7820592,-0.9910679){\rput(0.7748837,-1.4026653){\small -}}
\rput{45.571033}(-1.1951805,-0.91476494){\psellipse[linewidth=0.04,dimen=outer](0.49125,-1.88)(0.25,0.18)}
\rput{45.571033}(-1.2450223,-0.8841804){\rput(0.42367446,-1.9370029){\small +}}
\rput{45.571033}(-1.119419,-0.94795245){\rput(0.55488366,-1.7826653){\small -}}
\rput{45.571033}(-1.1714507,-1.2912557){\psellipse[linewidth=0.04,dimen=outer](0.95125,-2.04)(0.25,0.18)}
\rput{45.571033}(-1.2212926,-1.2606711){\rput(0.88367444,-2.097003){\small +}}
\rput{45.571033}(-1.0956893,-1.3244432){\rput(1.0148836,-1.9426652){\small -}}
\rput{45.571033}(-1.2505633,-1.5429213){\psellipse[linewidth=0.04,dimen=outer](1.21125,-2.26)(0.25,0.18)}
\rput{45.571033}(-1.300405,-1.5123367){\rput(1.1436745,-2.317003){\small +}}
\rput{45.571033}(-1.1748018,-1.5761088){\rput(1.2748836,-2.1626651){\small -}}
\rput{45.571033}(-1.5324947,-1.7117581){\psellipse[linewidth=0.04,dimen=outer](1.27125,-2.68)(0.25,0.18)}
\rput{45.571033}(-1.5823364,-1.6811736){\rput(1.2036744,-2.7370028){\small +}}
\rput{45.571033}(-1.4567332,-1.7449456){\rput(1.3348837,-2.5826652){\small -}}
\rput{45.571033}(-1.6119212,-1.4426763){\psellipse[linewidth=0.04,dimen=outer](0.91125,-2.64)(0.25,0.18)}
\rput{45.571033}(-1.661763,-1.4120917){\rput(0.8436745,-2.697003){\small +}}
\rput{45.571033}(-1.5361596,-1.4758638){\rput(0.9748837,-2.5426652){\small -}}
\rput{45.571033}(-1.5733724,-1.174445){\psellipse[linewidth=0.04,dimen=outer](0.61125,-2.46)(0.25,0.18)}
\rput{45.571033}(-1.6232142,-1.1438605){\rput(0.54367447,-2.5170028){\small +}}
\rput{45.571033}(-1.4976109,-1.2076325){\rput(0.67488366,-2.3626652){\small -}}
\rput{45.571033}(-1.5025427,-0.9430614){\psellipse[linewidth=0.04,dimen=outer](0.37125,-2.26)(0.25,0.18)}
\rput{45.571033}(-1.5523845,-0.91247684){\rput(0.30367446,-2.317003){\small +}}
\rput{45.571033}(-1.4267813,-0.9762489){\rput(0.43488368,-2.1626651){\small -}}
\psline[linewidth=0.04cm](1.38125,1.76)(2.98125,1.76)
\psline[linewidth=0.04cm](2.04125,-2.3)(3.84125,-2.3)
\rput(0.68078125,1.79){Positief}
\rput(0.845625,1.45){belaaide staaf}
\rput(5.0623436,-2.25){gepolariseerde}
\rput(5.058125,-2.59){polistireen bal}
\end{pspicture}
\end{center}
 \end{figure}       
        \par 

Sommige materiale bestaan uit molekules wat reeds gepolariseer is. Hierdie molekules het positiewe en negatiewe kante maar is steeds neutraal in die geheel. Net soos die gepolariseerde polistireen bal aangetrek kan word deur 'n gelaaide staaf, kan hierdie materiale be\"invloed word deur 'n gelaaide voorwerp. \par

Water is 'n voorbeeld van 'n stof wat van gepolariseerde molekules gemaak is. As 'n positief gelaaide staaf naby aan 'n stroom water gebring word, roteer die molekules sodat die negatiewe kante wys na die staaf. Die stroom water word dan aangetrek omdat teenoorgestelde mladings mekaar aantrek.      

            
\section{Behoud van lading}
            \nopagebreak

In al die voorbeelde waarna ons gekyk het was die lading nie geskep of vernietig nie maar van een materiaal na 'n ander oorgedra.

\Definition{Beginsel van die behoud van lading}{
Die beginsel van die behoud van lading s\^e dat die netto lading in 'n geslote stelsel konstant bly gedurende enige fisiese proses.
}

\subsection*{Geleiers en isolators}
    \nopagebreak

Sommige materiale laat elektrone toe om relatief vrylik rond te beweeg (bv. meeste metale, die menslike liggaam). Hierdie materiale word \textbf{geleiers} genoem. \par
\mindsetvid{Conductors and insulators}{VPfnt}

Ander materiale laat nie die ladingdraers, die elektrone, toe om deur hulle te beweeg nie (bv. glas, plastiek). Die elektrone is gebind aan die materiaal. Hierdie materiale word \textbf{nie-geleiers} of \textbf{isolators} genoem. \par

\Tip{Die invloed wat die vorm op die ladingsverspreiding het is die rede hoekom ons net identiese geleiers oorweeg as ons werk met die deel van lading.}

Wanneer 'n oormaat lading op 'n isolator geplaas word, sal dit bly waar dit geplaas is en daar sal gekonsentreerde lading in daardie area op die voorwerp wees. As 'n oormaat lading egter op 'n geleier geplaas word sal die soorgelyke ladings mekaar afstoot and versprei oor die oppervlak van die geleier. Wanneer twee geleiers aan mekaar raak, word die totale lading tussen hulle gedeel. As die twee geleiers identies is sal elkeen die helfte van die totale netto lading verkry. \par

\subsection*{Rangskikking van lading}

Die elektrostatiese krag beheer die rangskikking van lading op die oppervlak van 'n geleier. Dit is moontlik want ladings kan rondbeweeg in 'n geleier. Wanneer ons 'n lading op 'n sferiese geleier plaas sal die afstotende kragte tussen die individuele ladings veroorsaak dat die ladings egalig versprei oor die oppervlak van die sfeer. Geleiers met onre\"elmatige vorms sal egter gekonsentreerde lading kry naby enige punte in die geleier. In die figuur hieronder wys ons 'n ho\"er konsentrasie lading met meer $-$ of $+$ tekens terwyl eweredig verspreide lading vertoon word as meer egalige verspreding van $-$ en $+$ tekens.\par

\IFact{Hierdie versameling van lading kan toegelaat word om om van die geleier af te lek as die punt skerp genoeg is. Dit is vir hierdie rede dat geboue gereeld 'n weerligafleier op die dak het om enige opgeboude lading te verwyder. Dit verminder die kans dat die gebou deur weerlig geslaan word. Hierdie ``uitspreiding van lading'' sal nie gebeur as ons die lading op 'n isolator sit nie want die lading op 'n isolator kan nie rondbeweeg nie.}
      \label{m38781*id201196}
	\begin{figure}[H] % horizontal\label{m38781*id201199}
    \begin{center}
\begin{pspicture}(-2,-1.2)(3.3,2)
%\psgrid
\pscircle[linewidth=1pt](-1,1){0.5}
\psline[linewidth=4pt](-1,0.5)(-1,-1)
\psline[linewidth=5pt](-1.4,-1)(-0.6,-1) \degrees[1.1]
\multido{\n=0.0+.1}{11}{%
\uput{0.6}[\n](-1,1){-}}

\psellipse[fillcolor=lightgray](2,1)(0.75,0.5)
\psline[linewidth=4pt](2,0.5)(2,-1)
\psline[linewidth=5pt](2.4,-1)(1.6,-1) \rput(3.034,1.122){-}
\rput(2.987,1.239){-} \rput(2.742,1.495){-} \rput(2.272,1.676){-}
\rput(1.728,1.676){-} \rput(1.258,1.495){-} \rput(1.013,1.239){-}
\rput(0.966,1.122){-} \rput(0.950,1.000){-} \rput(0.966,0.878){-}
\rput(1.013,0.761){-} \rput(1.258,0.505){-} \rput(1.728,0.324){-}
\rput(2.272,0.324){-} \rput(2.742,0.505){-} \rput(2.987,0.761){-}
\rput(3.034,0.878){-} \rput(3.050,1.000){-}
\end{pspicture}
    \end{center}
\caption{Verspreiding van lading}
\label{Figure:chargedistributions}
 \end{figure}       
 



Wanneer twee identiese geleidende sfere op ge\"isoldeerde staanders aanmekaar raak, sal hulle die lading eweredig deel. As die aanvanklike lading op die eerste sfeer $Q_{1}$ is en die aanvanklike lading op die tweede sfeer $Q_{2}$ is, sal die finale lading op elke sfeer
    \begin{equation*}
    Q=\frac{{Q}_{1}+{Q}_{2}}{2}
      \end{equation*}
wees nadat hulle in kontak met mekaar gebring is.


\section{Kwantisering van lading}
            \nopagebreak

Die basiese eenheid van lading, genoem die element\^ere lading, \textsl{e}, is die hoeveelheid lading wat deur een elektron gedra word\par
            
\subsection{Eenheid van lading}

Die lading op 'n enkele elektron is ${q}_{e}=1,6x{10}^{-19}\phantom{\rule{2pt}{0ex}}\mathsf{C}$. Alle ander ladings in die heelal is 'n heeltal meervoud van hierdie waarde. Dit word lading kwantisering genoem.

\IFact{IN 1909 het Robert Millikan en Harvey Fletcher die lading van 'n elektron gemeet. Die eksperiment word nou geken as die Millikan oliedruppel eksperiment. Millikan en Fletcher het olie druppels in die ruimte tussen twee gelaaide plate gespuit en hulle kennis van die elektriese krag gebruik om die lading van 'n elektron te meet.}


Lading word in eenhede gemeet wat \textbf{coulomb (C)} genoem word. Een coulomb is 'n reuse hoeveelheid. In elektrostatika werk ons dikwels met ladings van mikrocoulomb ($1\phantom{\rule{2pt}{0ex}}\mu \phantom{\rule{2pt}{0ex}}\mathsf{C}=1\ensuremath{\times}{10}^{-6}\phantom{\rule{2pt}{0ex}}\mathsf{C}$) en nanocoulomb ($1\phantom{\rule{2pt}{0ex}}\phantom{\rule{2pt}{0ex}}\mathsf{nC}=1\ensuremath{\times}{10}^{-9}\phantom{\rule{2pt}{0ex}}\mathsf{C}$). \par

  
\begin{wex}{Lading kwantisering}
{
'n Voorwerp het $-1,92\ensuremath{\times}{10}^{-17}\phantom{\rule{2pt}{0ex}}\mathsf{C}$ se oortollige lading. Hoeveel oortollige elektrone is daar?
}
{

\westep{Analiseer die probleem en identifiseer die beginsels}

Ons moet die aantal elektrone vind van 'n lading uitwerk. Ons weet dat lading gekwantiseerd is en dat elektrone die basis eenheid van lading dra wat $-1,6\ensuremath{\times}{10}^{-19}\phantom{\rule{2pt}{0ex}}\mathsf{C}$ is.


\westep{Pas die beginsels toe}

Omdat elke elektron dieselfde hoeveelheidd lading dra, moet die totale lading 'n heelgetal elektrone bevat. Om die hoeveelheid elektrone te bereken, deel ons die totale lading deur die lading van 'n enkele elektron.
\begin{align*}
N &= \frac{-1,92\ensuremath{\times}{10}^{-17}}{-1,6\ensuremath{\times}{10}^{-19}}\\ 
  &= 120~\mathsf{ elektrone}
\end{align*}
}
\end{wex}

\begin{wex}{Geleidende sfere en die oordrag van lading}
{
Ek het twee identiese geleidende metaal sfere, op ge\"isoleerde staanders, wat gelaai is met verskillende ladings. Sfeer A het 'n lading van $-5~nC$ en sfeer B het 'n lading van van $-3~nC$. Ek bring die twee sfere in kontak met mekaar en daarna beweeg ek hulle weg van mekaar sodat hulle nie kontak maak nie.
\begin{enumerate}[itemsep=5pt, label=\textbf{\arabic*}.] 
    \item Wat gebeur met die ladings op die sfere?
    \item Wat is die finale lading op elke sfeer?
\end{enumerate}
}
{

\westep{Analiseer die vraag}

Ons het twee identiese metaalsfere wat negatiewe ladings het en ons bring hulle in kontak met mekaar en neem hulle dan weg van mekaar. Ons moet verduidelik wat met die ladings op elke sfeer gebeur en wat die finale lading op elke sfeer is nadat hulle van mekaar weggeneem is.
      
\westep{Identifiseer die beginsels}

Ons weet dat die lading draers in geleiers vrylik rond kan beweeg en dat lading eweredig sal versprei oor die die oppervlak van 'n geleier. \par
      
\westep{Pas die beginsels toe.}

\begin{enumerate}[noitemsep, label=\textbf{\alph*}. ] 
    \item  Wanneer die twee geleidende sfere in kontak met mekaar gebring word, is dit asof hulle een greoot geleier word en die totale lading sal dan versprei oor die hele oppervlak van die twee sfere. Wanner die sfere dan van mekaar geneem word is elkeen oor met die helfte van die oorspronklike totale lading.

    \item Voor die sfere raak is die totale lading: $-5~nC + (-3~nC) = -8~nC$. Wanneer hulle raak deel hulle die $-8~nC$ oor hul hele oppervlak. Wanneer hulle dan van mekaar geneem word is elkeen oor met die helfte van die oorspronklike lading. Dus het elkeen
    
\begin{eqnarray*}
    \frac{-8\phantom{\rule{4pt}{0ex}}\mathsf{nC}}{2}& =& -4\phantom{\rule{4pt}{0ex}}\mathsf{nC}
      \end{eqnarray*}
na die tyd.
\end{enumerate}}
\end{wex}

\begin{wex}
{Identiese sfere wat lading deel I}
{
Twee identiese sfere wat van isolator gemaak is het verskillende ladings. Sfeer 1 het 'n lading van $-96\time10^{-18}~C$. Sfeer 2 het 60 oortollige elektrone. As die twee sfere in kontak met mekaar gebring en dan verwyder word, wat se lading sal elkeen bevat? 
}
{
\westep{Analiseer die probleem}
Ons moet vasstel wat sal gebeur wanneer die sfere raak. Ons weet hulle is isolators en dus weet ons dat lading nie vrylik kan beweeg nie. Wanner hulle aan mekaar raak sal daar niks gebeur nie. 
}\end{wex}


\begin{wex}{Identiese sfere wat lading deel II}
{
Twee indentiese metaalsfere op ge\"isoleerde staanders het verskillende ladings. Sfeer 1 het 'n lading van $-9,6\ensuremath{\times}{10}^{-18}\phantom{\rule{2pt}{0ex}}\mathsf{C}$. Sfeer 2 het 60 oortollige protone. As die twee sfere in kontak met mekaar gebring word en dan van mekaar weggeneem word, hoeveel lading sal elkeen h\^e? Hoeveel elektrone of protone is dit?
}
{
Ons moet vasstel wat gebeur wanneer die sfere aan mekaar raak. Hulle is metaal so ons weet dat hulle geleiers is. Dit beteken dat die ladings vry is om rond te beweeg en dat die lading op die sfere kan verander. Ons weet dat die ladings egalig sal versprei oor die oppervlak van die twee sfere vanwe\"e die kragte tussen die ladings. Ons moet die lading op elke sfeer weet maar ons is net een sfeer se lading gegee.

Hierdie probleem is soortgelyk aan die vro\"ere voorbeeld. Hierdie keer moet ons vasstel wat die totale lading van 'n hoeveelheid protone is. Ons weet dat lading gekwantiseerd is en dat protone die basis hoeveelheid lading bevat en positief is en dus $+1,6\ensuremath{\times}{10}^{-19}\phantom{\rule{2pt}{0ex}}\mathsf{C}$. Die totale lading sal dan

\begin{align*}
{Q}_{2} &= 60\ensuremath{\times}1,6\ensuremath{\times}{10}^{\left(-19\right)}\phantom{\rule{2pt}{0ex}}\mathsf{C}\\ 
&= 9,6\ensuremath{\times}{10}^{-18}\phantom{\rule{2pt}{0ex}}\mathsf{C}
\end{align*}
wees.\par
Omdat die sfere identies in materiaal, grootte, vorm is, sal die lading eweredig versprei oor die oppervlak van die twee sfere. Elke sfeer sal die helte van die totale lading he:
\begin{align*}
Q &= \dfrac{{Q}_{1}+{Q}_{2}}{2}\\ 
  &= \dfrac{9,6\ensuremath{\times}{10}^{-18}+\left(-9,6\ensuremath{\times}{10}^{-18}\right)}{2}\\ 
  &= 0\phantom{\rule{2pt}{0ex}}\mathsf{C}
\end{align*}
Dus is elke sfeer nou neutraal.\par

Daar is nie 'n netto lading nie en dus ook geen oortollige elektrone of protone nie.
}
\end{wex}


\begin{wex}{Behoud van lading - 1}
{
Twee identiese metaalsfere het verskillende ladings. Sfeer 1 het 'n lading van $-9,6\ensuremath{\times}{10}^{-18}\phantom{\rule{2pt}{0ex}}\mathsf{C}$. Sfeer 2 het 30 oortollige elektrone. As die twee sfere in kontak gebring en dan afgesonder word, wat sal die lading op elke wees? Hoeveel elektrone stem dit ooreen mee?
}
{
\westep{Analiseer die probleem}
Ons moet vasstel wat sal gebeur wanneer die sfere raak. Hulle is van metaal gemaak dus weet ons hulle is geleiers. Dit beteken dat lading kan rondbeweeg en dus kan die lading op elke sfeer ook verander. Ons moet die lading van elke sfeer weet maar ons is net een gegee.

\westep{Identifiseer die beginsels}

Hierdie probleem is soortgelyk aan 'n vorige voorbeeld. Hierdie keer moet ons die totale lading van 'n sekere aantal elektrone vasstel. Ons weet dat lading gekwantiseerd is en dat elektrone die basis eenheid van lading het wat  $-1,6\ensuremath{\times}{10}^{-19}\phantom{\rule{2pt}{0ex}}\mathsf{C}$ is. Die totale lading is dus:
    \begin{align*}
    {Q}_{2}&=30\ensuremath{\times}-1,6\ensuremath{\times}{10}^{-19}\phantom{\rule{2pt}{0ex}}\mathsf{C}\\ 
\phantom{x}&=4,8\ensuremath{\times}{10}^{-18}\phantom{\rule{2pt}{0ex}}\mathsf{C}
      \end{align*}


\westep{Pas die beginsels toe: herverdeel die lading}

As die sfere identies in materiaal, vorm en grootte is sal die lading eweredig oor die die twee sfere verdeel word. Elke sfeer sal dan die helfte van die totale lading h\^e. 
    
\begin{align*}
Q  &=\frac{{Q}_{1}+{Q}_{2}}{2}\\ \phantom{x}&=\frac{-9.6\ensuremath{\times}{10}^{-18}+\left(-4,8\ensuremath{\times}{10}^{-18}\right)}{2}\\ 
    &=7,2\ensuremath{\times}{10}^{-18}\phantom{\rule{2pt}{0ex}}\mathsf{C}
\end{align*}
 So elke sfeer het nou
\begin{equation*}
    7,2\ensuremath{\times}{10}^{-18}\phantom{\rule{2pt}{0ex}}\mathsf{C}
\end{equation*}
     se lading. \par

\westep{Pas die beginsel toe: lading kwantisering}

Omdat elke elektron dieselfde lading het weet ons dat die totale lading van 'n seker aantal elektrone gemaak is. Om die aantal elektrone te bereken deel ons die totale lading deur die lading van 'n enkele elektron:
\begin{align*}
     N &=\frac{-7,2\ensuremath{\times}{10}^{-18}}{-1,6\ensuremath{\times}{10}^{-19}}\\ 
\phantom{x}&=45\phantom{\rule{2pt}{0ex}}\mathsf{elektrone} 
    \end{align*}
}\end{wex}




\begin{i_experiment}{Die elektroskoop}
\nopagebreak
Die elektroskoop is 'n baie sensitiewe instrument wat gebruik word om elektriese lading op te spoor. Die diagram hieronder wys die goudbladelektroskoop. Die elektroskoop bestaan uit 'n glashouer met 'n metaal staaf aan die binnekant wat vas is aan twee dun velle goudfoelie. Die ander kant van die metaalstaaf het 'n metaalplaat wat aan die buitekant van die houer is. \par

\begin{figure}[H] % horizontal\label{m38781*id200546}
    \begin{center}\begin{pspicture}(0,-3.0971875)(6.2675,3.1371875)
\definecolor{color2}{rgb}{0.4,0.4,0.4}
\definecolor{color351b}{rgb}{0.6,0.6,0.6}
\psellipse[linewidth=0.04,linecolor=color2,dimen=outer](1.62,-2.6571875)(1.52,0.44)
\psline[linewidth=0.04cm,linecolor=color2](0.1,-0.0771875)(0.12,-2.6771874)
\psline[linewidth=0.04cm,linecolor=color2](3.1,-0.0771875)(3.12,-2.6771874)
\psbezier[linewidth=0.04,linecolor=color2](0.1,-0.1171875)(0.2,-0.4171875)(3.12,-0.5171875)(3.1,-0.1171875)(3.08,0.2828125)(1.74,1.2028126)(1.86,1.2428125)(1.98,1.2828125)(1.2,1.2828125)(1.36,1.2428125)(1.52,1.2028126)(0.0,0.1828125)(0.1,-0.1171875)
\psellipse[linewidth=0.04,dimen=outer,fillstyle=solid,fillcolor=color351b](1.56,1.8328125)(0.94,0.23)
\psframe[linewidth=0.04,linecolor=color2,dimen=outer,fillstyle=solid](2.5,1.9628125)(0.62,1.8228126)
\psellipse[linewidth=0.04,dimen=outer,fillstyle=solid,fillcolor=color351b](1.56,1.9328125)(0.94,0.23)
\psline[linewidth=0.04cm](0.64,1.8228126)(0.64,1.9828125)
\psline[linewidth=0.04cm](2.48,1.7828125)(2.48,1.9428124)
\psbezier[linewidth=0.04,fillstyle=solid,fillcolor=black](1.3151261,1.2328125)(1.310084,1.1228125)(1.889916,1.1355048)(1.884874,1.2328125)(1.8798319,1.3301202)(1.9,1.3258895)(1.8697479,1.3301202)(1.8394958,1.334351)(1.3907562,1.3301202)(1.3453782,1.3301202)(1.3,1.3301202)(1.320168,1.3428125)(1.3151261,1.2328125)
\psbezier[linewidth=0.04,linecolor=color2,fillstyle=solid](1.2972177,0.9730125)(1.4038053,0.9730125)(1.700252,0.9628125)(1.870126,0.9764125)(2.04,0.9900125)(2.0233457,1.3028125)(1.8601334,1.2212125)(1.6969212,1.1396126)(1.4256628,1.1728117)(1.3072103,1.2212125)(1.1887578,1.2696133)(1.1906301,0.9730125)(1.2972177,0.9730125)
\psframe[linewidth=0.04,dimen=outer,fillstyle=solid,fillcolor=black](1.66,1.6228125)(1.52,1.3428125)
\psframe[linewidth=0.04,dimen=outer,fillstyle=solid,fillcolor=black](1.68,0.9828125)(1.52,-0.0371875)
\psframe[linewidth=0.04,dimen=outer,fillstyle=solid,fillcolor=black](1.58,-0.0371875)(1.56,-0.5771875)
\psbezier[linewidth=0.04,linecolor=color2](1.5999999,-0.5534384)(1.5644444,-0.7890773)(1.4311111,-1.0328416)(1.12,-1.1384728)
\psbezier[linewidth=0.04,linecolor=color2](1.5999999,-0.5371875)(1.5644444,-0.7728263)(1.4311111,-1.0165906)(1.12,-1.1222218)
\psbezier[linewidth=0.04,linecolor=color2](1.58,-0.57343847)(1.6155556,-0.80907726)(1.748889,-1.0528415)(2.06,-1.1584728)
\psbezier[linewidth=0.04,linecolor=color2](1.58,-0.5571875)(1.6155556,-0.79282635)(1.748889,-1.0365906)(2.06,-1.1422218)
\rput{-10.598329}(-0.3782633,0.74742395){\psframe[linewidth=0.04,linecolor=color2,dimen=outer](5.54,2.6028125)(2.14,2.2228124)}
\rput(3.0746875,2.2328124){+}
\rput(2.7546875,2.3128126){+}
\rput(2.3546875,2.3928125){+}
\rput(2.0146875,2.5528126){+}
\rput(2.0546875,2.7928126){+}
\rput(2.2346876,2.9928124){+}
\rput(2.4746876,2.9528124){+}
\rput(2.7146876,2.8928125){+}
\rput(3.0546875,2.8528125){+}
\rput(3.2946875,2.7728126){+}
\rput(0.68546873,2.1528125){-}
\rput(0.98546875,2.2328124){-}
\rput(1.2454687,2.2728126){-}
\rput(1.5854688,2.2728126){-}
\rput(1.9254688,2.2528124){-}
\rput(2.1654687,2.2128124){-}
\rput(2.4254687,2.1328125){-}
\rput(2.6054688,1.9928125){-}
\rput(0.48546875,1.9928125){-}
\rput(0.46546876,1.8128124){-}
\rput(1.8146875,-0.7471875){+}
\rput(1.9546875,-0.9471875){+}
\rput(2.1946876,-1.1471875){+}
\rput(1.8746876,-1.2071875){+}
\rput(1.6746875,-1.0071875){+}
\rput(1.4946876,-1.0471874){+}
\rput(1.2746875,-1.2271875){+}
\rput(1.0146875,-1.0671875){+}
\rput(1.2546875,-0.9271875){+}
\rput(1.3946875,-0.7471875){+}
\psline[linewidth=0.027999999cm,linecolor=color2](1.68,-0.7771875)(4.02,-0.7771875)
\psline[linewidth=0.04cm,linecolor=color2](2.22,1.9828125)(4.0,1.1628125)
\psline[linewidth=0.04cm,linecolor=color2](3.86,2.2428124)(4.12,1.7628125)
\rput(5.1584377,-0.7471875){goud foelie}
\rput(4.8932815,1.1128125){metaal plaat}
\rput(5.044375,1.6728125){gelaaide staaf}
\psline[linewidth=0.04cm,linecolor=color2](3.1,-2.1771874)(4.0,-2.1771874)
\rput(5.065781,-2.1471875){glashouer}
\end{pspicture}
    \end{center}
 \end{figure}       
        \par 
Die elektroskoop spoor lading op die volgende manier op: 'n Gelaaide voorwerp, soos die positief gelaaide staaf in die diagram, word nader gebring (maar nie in kontak met) die neutrale metaal plaat van die elektroskoop nie. Dit veroorsaak negatiewe lading in die metaalplaat, goudfoelie en metaalstaaf om aangetrek te word na die positiewe staaf. Omdat die metaal (goud is ook 'n metaal!) geleidend is, is die lading vry om rond te beweeg vanaf die goudfoelie na die metaalplaat toe. Daar is nou meer negatiewe lading op die plaat en meer positiewe lading op die goudfoelie. Dit is belangrik om te onthou dat die elektroskoop steeds neutraal is (die totale positiewe en negatiewe lading is dieselfde) en dat die lading net \textsl{ge\"induseer} is om na verskillende dele van die instrument te beweeg! Die ge\"induseerde positiewe lading op die goudfoelie dwing die twee velle van mekaar omdat soortgelyke ladings mekaar afstoot. Dus weet ons dat die staaf gelaai is. As die staaf nou weggeneem word van die elektroskoop sal die lading in die elektroskoop weer eweredig versprei en die goue velle sal los hang omdat daar nie meer 'n lading op hulle is nie. \par




\subsubsection{Begronding}
\nopagebreak

As jy 'n gelaaide staaf naby aan die ongelaaide elektroskoop bring en dan terselfdertyd met jou vinger aan die metaal plaat raak, sal lading van die grond (die aarde) deur jou liggaam na die metaalplaat toe vloei. Koppeling aan die aarde sodat lading kan vloei word \textbf{begronding} genoem. Die lading wat na die plaat stroom is teenoorgesteld van di\"e op die gelaaide staaf want dit word aangetrek deur deur die lading op die staaf. Dus, in ons diagram, sal die lading wat na die plaat toe vloei negatief wees. Omdat daar nou lading op die elektroskoop bygevoeg is, is dit nie meer neutraal nie maar het oortollige negatiewe lading. As ons nou die staaf wegvat sal die goue velle mekaar nog steeds afstoot omdat daar oortollige negatiewe lading op hulle is. As ons die elektroskoop weer begrond (hierdie keer sonder die gelaaide staaf naby) sal die oortollige lading terugvloei na die aarde en die elektroskoop sal weer neutraal wees. \par

\begin{figure}[H] % horizontal\label{m38781*id200605}
    \begin{center}
%     \scalebox{0.8} % Change this value to rescale the drawing.
% {
\begin{pspicture}(0,-2.7271874)(7.7496877,2.7671876)
\definecolor{color2}{rgb}{0.4,0.4,0.4}
\definecolor{color351b}{rgb}{0.6,0.6,0.6}
\psellipse[linewidth=0.04,linecolor=color2,dimen=outer](1.62,-2.2871876)(1.52,0.44)
\psline[linewidth=0.04cm,linecolor=color2](0.1,0.2928125)(0.12,-2.3071876)
\psline[linewidth=0.04cm,linecolor=color2](3.1,0.2928125)(3.12,-2.3071876)
\psbezier[linewidth=0.04,linecolor=color2](0.1,0.2528125)(0.2,-0.0471875)(3.12,-0.1471875)(3.1,0.2528125)(3.08,0.6528125)(1.74,1.5728126)(1.86,1.6128125)(1.98,1.6528125)(1.2,1.6528125)(1.36,1.6128125)(1.52,1.5728126)(0.0,0.5528125)(0.1,0.2528125)
\psellipse[linewidth=0.04,dimen=outer,fillstyle=solid,fillcolor=color351b](1.56,2.2028124)(0.94,0.23)
\psframe[linewidth=0.04,linecolor=color2,dimen=outer,fillstyle=solid](2.5,2.3328125)(0.62,2.1928124)
\psellipse[linewidth=0.04,dimen=outer,fillstyle=solid,fillcolor=color351b](1.56,2.3028126)(0.94,0.23)
\psline[linewidth=0.04cm](0.64,2.1928124)(0.64,2.3528125)
\psline[linewidth=0.04cm](2.48,2.1528125)(2.48,2.3128126)
\psbezier[linewidth=0.04,fillstyle=solid,fillcolor=black](1.3151261,1.6028125)(1.310084,1.4928125)(1.889916,1.5055048)(1.884874,1.6028125)(1.8798319,1.7001202)(1.9,1.6958895)(1.8697479,1.7001202)(1.8394958,1.704351)(1.3907562,1.7001202)(1.3453782,1.7001202)(1.3,1.7001202)(1.320168,1.7128125)(1.3151261,1.6028125)
\psbezier[linewidth=0.04,linecolor=color2,fillstyle=solid](1.2972177,1.3430125)(1.4038053,1.3430125)(1.700252,1.3328125)(1.870126,1.3464125)(2.04,1.3600125)(2.0233457,1.6728125)(1.8601334,1.5912125)(1.6969212,1.5096124)(1.4256628,1.5428118)(1.3072103,1.5912125)(1.1887578,1.6396133)(1.1906301,1.3430125)(1.2972177,1.3430125)
\psframe[linewidth=0.04,dimen=outer,fillstyle=solid,fillcolor=black](1.66,1.9928125)(1.52,1.7128125)
\psframe[linewidth=0.04,dimen=outer,fillstyle=solid,fillcolor=black](1.68,1.3528125)(1.52,0.3328125)
\psframe[linewidth=0.04,dimen=outer,fillstyle=solid,fillcolor=black](1.58,0.3328125)(1.56,-0.2071875)
\psbezier[linewidth=0.04,linecolor=color2](1.5999999,-0.18343845)(1.5644444,-0.41907728)(1.4311111,-0.66284156)(1.12,-0.7684728)
\psbezier[linewidth=0.04,linecolor=color2](1.5999999,-0.1671875)(1.5644444,-0.40282634)(1.4311111,-0.64659065)(1.12,-0.7522218)
\psbezier[linewidth=0.04,linecolor=color2](1.58,-0.20343846)(1.6155556,-0.4390773)(1.748889,-0.6828416)(2.06,-0.7884728)
\psbezier[linewidth=0.04,linecolor=color2](1.58,-0.1871875)(1.6155556,-0.42282632)(1.748889,-0.66659063)(2.06,-0.7722218)
\rput(0.68546873,2.5228126){-}
\rput(0.98546875,2.6028125){-}
\rput(1.4054687,0.7028125){-}
\rput(1.5854688,2.6428125){-}
\rput(1.8254688,0.8428125){-}
\rput(1.3054688,-0.7771875){-}
\rput(2.4254687,2.5028124){-}
\rput(2.6054688,2.3628125){-}
\rput(0.48546875,2.3628125){-}
\rput(1.8346875,2.6028125){+}
\rput(1.9546875,-0.5771875){+}
\rput(1.7346874,-0.3171875){+}
\rput(1.8546875,1.0228125){+}
\rput(2.2346876,2.5428126){+}
\rput(1.3146875,-0.4371875){+}
\rput(1.8346875,0.3228125){+}
\rput(0.8146875,2.5628126){+}
\rput(1.3146875,2.6228125){+}
\rput(1.4146875,0.3428125){+}
\psline[linewidth=0.027999999cm,linecolor=color2](1.68,-0.4071875)(4.02,-0.4071875)
\psline[linewidth=0.04cm,linecolor=color2](2.22,2.3528125)(4.0,1.5328125)
\rput(5.523594,-0.3771875){goue foelie velle met}
\rput(4.8932815,1.4828125){metaal plaat}
\psline[linewidth=0.04cm,linecolor=color2](3.1,-1.8071876)(4.0,-1.8071876)
\rput(5.065781,-1.7771875){glas houer}
\rput(0.46546876,2.1828125){-}
\rput(2.0254688,2.6228125){-}
\rput(1.8454688,0.5628125){-}
\rput(1.4254688,0.1428125){-}
\rput(1.3854687,-0.2571875){-}
\rput(1.8454688,-0.4771875){-}
\rput(2.1454687,-0.7371875){-}
\rput(1.7454687,-0.6971875){-}
\rput(1.6254687,-0.5171875){-}
\rput(1.4654688,-0.6371875){-}
\rput(1.2254688,-0.5771875){-}
\rput(0.96546876,-0.6771875){-}
\rput(1.7654687,0.1228125){-}
\rput(1.4254688,0.9628125){-}
\rput(1.4254688,1.1628125){-}
\rput(1.8654687,-0.7771875){-}
\rput(5.888125,-0.7371875){oortollige negatiewe lading}
\rput(5.25,-1.0571876){stoot mekaar af}
\rput(1.1854688,2.6228125){-}
\rput(2.5054688,2.0628126){-}
\end{pspicture}
%}
\end{center}
 \end{figure}       
\end{i_experiment}
            
\summary{VPfqd}

\begin{enumerate}[noitemsep, label=\textbf{\arabic*}. ] 
\item Daar is twee soorte lading.
\item Voorwerpe kan \textbf{positief} of \textbf{negatief} gelaai of \textbf{neutraal} wees.
\item Voorwerpe wat neutraal is het dieselfde aantal positiewe en negatiewe ladings.
\item Teenoorgestelde ladings trek mekaar aan en soortgelyke ladings stoot mekaar af.
\item Lading kan nie geskep of vernietig word nie. Dit kan net oorgedra word.
\item Lading word gemeet in coulomb (C).
\item Lading is gekwantiseerd in veelvoude van die lading van 'n elektron: $1.6\times10^{-19}~\mathsf{C}$.
\item Geleiers laat lading toe om deur hulle te beweeg. 
\item Isolators laat nie toe dat lading maklik deur hulle kan beweeg nie.
\item Identiese geleidende sfere wat in kontak is deel hulle lading volgens die vergelyking:
\begin{equation*}
 Q=\frac{Q_1+Q_2}{2}
\end{equation*}
\end{enumerate}
        

\begin{eocexercises}{Electrostatics}
\noindent
\nopagebreak
\begin{enumerate}[itemsep=5pt, label=\textbf{\arabic*}. ] 
\item Wat word die twee soorte ladings genoem?

\item Voorsien bewyse vir die bestaan van twee soorte lading.

\item Vul die spasies in: Die elektrostatiese krag tussen soortgelyke ladings is \uline{\hspace{10ex}} en die elektrostatiese krag tussen teenoorgestelde ladings is \uline{\hspace{10ex}}

\item Ek het twee positief gelaaide metaal balle wat 2 m van mekaar is.
\begin{enumerate}[noitemsep, label=\textbf{\alph*}. ] 
    \item Is die elektrostatiese krag tussen die balle afstotend of aantreklik?
    \item As ek nou die balle skuif sodat hulle 1 m van mekaar is, wat gebeur met die sterkte van die elektrostatiese krag tussen hulle?
\end{enumerate}
            
\item Ek het 2 gelaaide sfere wat van toutjies hang soos in die diagram hieronder.
\begin{figure}[H] % horizontal\label{m38781*id202166}
    \begin{center}
    \begin{pspicture}(0,-1.23)(2.74,1.25)
\pscircle[linewidth=0.04,dimen=outer](0.41,-0.82){0.41}
\pscircle[linewidth=0.04,dimen=outer](2.33,-0.82){0.41}
\psline[linewidth=0.04cm](0.4,-0.43)(0.4,1.23)
\psline[linewidth=0.04cm](2.34,-0.45)(2.34,1.21)
\rput(2.3559375,-0.815){\large +}
\rput(0.3790625,-0.835){\large +}
\end{pspicture}\end{center}
 \end{figure}       
Kies die korrekte antwoord van die keuses wat volg:
Die sfere sal
\begin{enumerate}[noitemsep, label=\textbf{\alph*}. ] 
    \item na mekaar swaai vanwe\"e die aantreklike elektrostatiese krag tuseen hulle
    \item weg van mekaar swaai vanwe\"e die aantreklike elektrostatiese krag tussen hulle
    \item na mekaar swaai vanwe\"e die afstotende elektrostatiese krag tussen hulle
    \item weg van mekaar swaai vanwe\"e die afstotende elektrostatiese krag tussen hulle
\end{enumerate}
            
\item Beskryf hoe voorwerpe (isolators) gelaai kan word deur kontak of wrywing.
\item Daar word 'n perspeks liniaal en 'n stuk lap aan jou gegee.
\begin{enumerate}[noitemsep, label=\textbf{\alph*}. ] 
    \item Hoe sal jy die liniaal laai?
    \item Verduidelik hoe die liniaal gelaai raak in terme van lading. 
    \item Hoe trek die liniaal klein stukkies papier aan?
\end{enumerate}
            
\item (IEB 2005/11 HG) 'n Ongelaaide hol metaalsfeer word op 'n isolasie staander geplaas. 'n Positief gelaaide staaf word in aanraking met die sfeer gebring by punt P soos in die diagram hieronder. Dit word dan verwyder.
\begin{figure}[H] % horizontal\label{m38781*id202314}
    \begin{center}
    \begin{pspicture}(-1,0)(1,3.2)
\SpecialCoor
%\psgrid[gridcolor=lightgray]
\psframe(-1,0)(1,0.2) \psframe(-0.1,0.2)(0.1,2.2)
\pscircle[fillcolor=white,fillstyle=solid](0,2.7){0.5}
\psellipse[fillcolor=white,fillstyle=solid](0.4,3)(0.1,0.2)
\psframe[fillcolor=white,fillstyle=solid,linestyle=none](0.4,2.8)(2.4,3.2)
\psline(0.4,2.8)(2.4,2.8) \psline(0.4,3.2)(2.4,3.2)
\rput(2,0){\psellipse[fillcolor=white,fillstyle=solid](0.4,3)(0.1,0.2)}
\uput[dl](0.4,3){P} \uput[r](0.4,3){+++}
\end{pspicture}\end{center}
 \end{figure}       
Waar sal die oortollige lading op die sfeer wees nadat die staaf verwyder is?
\begin{enumerate}[noitemsep, label=\textbf{\alph*}. ] 
    \item Dit is steeds by punt P waar die staaf geraak het.
    \item Dit is eweredig oor die buitenste oppervlak van die sfeer versprei.
    \item Dit is eweredig oor die binneste en buitenste oppervlaktes van die uitgeholde sfeer versprei.
    \item Geen lading bly op die sfeer oor nie.
\end{enumerate}
            
\item Wat word die proses genoem veroorsaak dat molekules in 'n ongelaaide voorwerp in 'n seker rigting oplyn vanwe\"e 'n eksterne lading?

\item Verduidelik hoe 'n ongelaaide voorwerp aangetrek kan word deur 'n gelaaide voorwerp. Gebruik sketse om jou antwoord te illustreer. 

\item Verduidelik hoe 'n stroom water deur 'n gelaaide staaf aangetrek kan word.

\item 'n Voorwerp het oortollige lading van $-8,6\ensuremath{\times}{10}^{-18}\phantom{\rule{2pt}{0ex}}\mathsf{C}$. Hoeveel oortollige elektrone het die voorwerp?

\item 'n Voorwerp het 235 oortollige elektrone. Wat is die lading op die voorwerp? 

\item 'n Voorwerp het 235 oortollige protone. Wat is die lading van die voorwerp?

\item Twee identiese metaalsfere het verskillende ladings. Sfeer 1 het 'n lading van $-4,8\ensuremath{\times}{10}^{-18}\phantom{\rule{2pt}{0ex}}\mathsf{C}$. Sfeer 2 het 60 oortollige elektrone. As die twee sfere in kontak met mekaar gebring en dan weer geskei word, wat sal die lading op elk wees? Hoeveel elektrone stem dit ooreen mee?

\item Twee identiese ge\"isoleerde sfere het verskillende ladings. Sfeer 1 het 'n lading van $-96\ensuremath{\times}{10}^{-18}\phantom{\rule{2pt}{0ex}}C$. Sfeer 2 het 60 oortollige elektrone. As die twee sfere in kontak met mekaar gebring en dan weer geskei word, wat sal die lading op elk wees?

\item Twee indentiese metaal sfere het verskillende ladings. Sfeer 1 het 'n lading van $-4,8\ensuremath{\times}{10}^{-18}\phantom{\rule{2pt}{0ex}}\mathsf{C}$. Sfeer 2 het 30 oortollige protone. As die twee sfere in kontak gebring en dan geskei word, wat sal die lading op elk wees? Met hoeveel elektrone of protone stem dit ooreen?
            \end{enumerate}
  \label{m38781**end}
  \label{464e844ca5615087ea89d9d95dd9a43a**end}
\practiceinfo
\begin{tabular}[h]{cccccc}
 (1.) lqs  &  (2.) lqo  &  (3.) lqA  &  (4.) lqG  &  (5.) lqf  &  (6.) lqw  &  (7.) lqv  &  (8.) lqd  &  (9.) lqp  &  (10.) la2  &  (11.) lqP  &  (12.) lTf  &  (13.) lTG  &  (14.) lT7  &  (15.) lTA  &  (16.) lTo  &  (17.) lTs  & \end{tabular}
\end{eocexercises}
