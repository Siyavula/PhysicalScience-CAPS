 \chapter{Elektrostatika}
\fancyfoot[LO,RE]{Fisika: Elektrisiteit and Magnetisme}
\label{464e844ca5615087ea89d9d95dd9a43a}
 

\section{Inleiding en sleutelkonsepte}
    \nopagebreak
\label{m38780} $ \hspace{-5pt}\begin{array}{cccccccccccc}   \includegraphics[width=0.75cm]{col11305.imgs/summary_fullmarks.png} &   \includegraphics[width=0.75cm]{col11305.imgs/summary_simulation.png} &   \end{array} $ \hspace{2 pt}\raisebox{-5 pt}{} {(seksie kortkode: P10071 )} \par 
    \label{m38780*cid2}
       
Elektrostatika is die studie van elektriese lading wat staties is (dit beweeg nie). In hierdie hoofstuk kyk ons na die basiese beginsels van elektrostatika sowel as die beginsel van behoud van lading.

\section{Twee soorte lading}
            \nopagebreak

Alle voorwerpe om ons (mense insluitend!) bevat groot hoeveelhede elektriese lading. Daar is twee tipes elektriese lading: \textbf{positiewe} and \textbf{negatiewe} lading. As dieselfde hoeveelhede positiewe en negatiewe ladings in 'n voorwerp bevat word, is die \textbf{netto lading nul} en die voorwerp is \textbf{neutraal}. As daar meer van een soort lading as die ander is, is die voorwerp elektries \textbf{gelaai}. Die diagram hieronder wys hoe die verspreiding van ladings vir 'n neutraal, positief en negatief gelaaide voorwerp. \par

      \label{m38780*id200640}
	\begin{figure}[H] % horizontal\label{m38780*id200643}
    \begin{center}
    \begin{pspicture}(0,-2.3017187)(13.100625,2.3017187)
\pscircle[linewidth=0.04,dimen=outer](1.6696875,-0.48828125){0.92}
\pscircle[linewidth=0.04,dimen=outer](6.2496877,-0.48828125){0.92}
\pscircle[linewidth=0.04,dimen=outer](10.789687,-0.46828124){0.92}
\rput(1.364375,0.10171875){+}
\rput(2.124375,-0.09828125){+}
\rput(2.184375,-0.87828124){+}
\rput(1.504375,-0.49828124){+}
\rput(1.364375,-1.1182812){+}
\rput(1.004375,-0.31828126){+}
\rput(1.1751562,-0.17828125){-}
\rput(1.5751562,0.04171875){-}
\rput(2.2551563,-0.19828124){-}
\rput(1.7351563,-0.45828125){-}
\rput(1.9951563,-0.97828126){-}
\rput(1.1751562,-0.91828126){-}
\rput(5.964375,0.08171875){+}
\rput(6.884375,-0.45828125){+}
\rput(5.624375,-0.35828125){+}
\rput(6.064375,-1.1782813){+}
\rput(6.404375,-0.35828125){+}
\rput(6.584375,-0.8382813){+}
\rput(5.784375,-0.8382813){+}
\rput(6.384375,0.00171875){+}
\rput(5.755156,0.00171875){-}
\rput(5.8351564,-0.41828126){-}
\rput(5.715156,-0.97828126){-}
\rput(6.215156,-1.0582813){-}
\rput(6.8951564,-0.61828125){-}
\rput(6.615156,0.04171875){-}
\rput(10.7951565,0.26171875){-}
\rput(11.084375,0.12171875){+}
\rput(11.384375,-0.43828124){+}
\rput(10.944375,-0.25828126){+}
\rput(10.784375,-1.1382812){+}
\rput(10.264375,-0.8382813){+}
\rput(10.364375,-0.21828125){+}
\rput(10.895156,-0.39828125){-}
\rput(11.395156,-0.6982812){-}
\rput(11.075156,-1.1182812){-}
\rput(10.395156,-0.9582813){-}
\rput(10.115156,-0.21828125){-}
\rput(10.595157,-0.59828126){-}
\rput(10.955156,-0.79828125){-}
\rput(10.475156,0.14171875){-}
\rput(1.019375,2.1217186){Daar is:}
\rput(1.7278125,1.4117187){\small 6 positiewe ladings en}
\rput(1.4625,1.0517187){\small 6 negatiewe ladings}
\rput(1.66125,0.63171875){\small 6 + (-6) = 0}
\rput(1.6478125,-1.6882813){\small Daar is nul netto lading:}
\rput(1.6682812,-2.0882812){\small Die voorwerp is neutraal}
\rput(6.284844,1.4117187){\small 8 positiewe ladings en}
\rput(6.0225,1.0517187){\small 6 negatiewe ladings}
\rput(6.257344,0.63171875){\small 8 + (-6) = 2}
\rput(6.2725,-1.6882813){\small Die netto lading is +2}
\rput(6.33,-2.0882812){\small Die voorwerp is positief gelaai}
\rput(10.787812,1.4117187){\small 6 positiewe ladings en}
\rput(10.523125,1.0517187){\small 9 negatiewe ladings}
\rput(10.813281,0.63171875){\small 6 + (-9) = -3}
\rput(10.745469,-1.6882813){\small Die netto lading is -3}
\rput(10.87,-2.0882812){\small Die voorwerp is negatief gelaai}
\end{pspicture}
\end{center}
 \end{figure}       
      

\par 

Positiewe lading in materie word deur protone gedra en negatiewe lading deur elektrone. Die verandering in die algehele lading van 'n voorwerp is gewoonlik vanwe\"e die verandering in die hoeveelheid elektrone in die voorwerp.

Om 'n voorwerp meer
      \begin{itemize}
        \item \textbf{Positief gelaai te maak}: elektrone moet weggeneem word, en die voorwerp word elektron \textvf{arm}
        \item \textbf{Negatief gelaai te maak}: elektrone word bygevoeg en die voorwerp word \textbf{ryk} aan elektrone.
      \end{itemize}

So in die praktyk bly die hoeveelheid positiewe ladings (protone) dieselfde en die hoeveelheid elektrone verander.

\begin{figure}[H] % horizontal\label{m38780*id200643}
    \begin{center}
    \begin{pspicture}(0,-2.3017187)(13.100625,2.3017187)
\pscircle[linewidth=0.04,dimen=outer](1.6696875,-0.48828125){0.92}
\pscircle[linewidth=0.04,dimen=outer](6.2496877,-0.48828125){0.92}
\pscircle[linewidth=0.04,dimen=outer](10.789687,-0.46828124){0.92}
\rput(1.364375,0.10171875){+}
\rput(2.124375,-0.09828125){+}
\rput(2.184375,-0.87828124){+}
\rput(1.504375,-0.49828124){+}
\rput(1.364375,-1.1182812){+}
\rput(1.004375,-0.31828126){+}
\rput(1.1751562,-0.17828125){-}
\rput(1.5751562,0.04171875){-}
\rput(2.2551563,-0.19828124){-}
\rput(1.7351563,-0.45828125){-}
\rput(1.9951563,-0.97828126){-}
\rput(1.1751562,-0.91828126){-}
\rput(5.964375,0.08171875){+}
\rput(6.884375,-0.45828125){+}
\rput(5.624375,-0.35828125){+}
\rput(6.064375,-1.1782813){+}
\rput(6.404375,-0.35828125){+}
\rput(6.584375,-0.8382813){+}
% \rput(5.784375,-0.8382813){+}
% \rput(6.384375,0.00171875){+}
% \rput(5.755156,0.00171875){-}
% \rput(5.8351564,-0.41828126){-}
\rput(5.715156,-0.97828126){-}
\rput(6.215156,-1.0582813){-}
\rput(6.8951564,-0.61828125){-}
\rput(6.615156,0.04171875){-}
\rput(10.7951565,0.26171875){-}
\rput(11.084375,0.12171875){+}
\rput(11.384375,-0.43828124){+}
\rput(10.944375,-0.25828126){+}
\rput(10.784375,-1.1382812){+}
\rput(10.264375,-0.8382813){+}
\rput(10.364375,-0.21828125){+}
\rput(10.895156,-0.39828125){-}
\rput(11.395156,-0.6982812){-}
\rput(11.075156,-1.1182812){-}
\rput(10.395156,-0.9582813){-}
\rput(10.115156,-0.21828125){-}
\rput(10.595157,-0.59828126){-}
\rput(10.955156,-0.79828125){-}
\rput(10.475156,0.14171875){-}
\rput(1.019375,2.1217186){Daar is:}
\rput(1.7278125,1.4117187){\small \textbf{6} positiewe ladings and}
\rput(1.4625,1.0517187){\small 6 negatiewe ladings}
\rput(1.66125,0.63171875){\small 6 + (-6) = 0}
\rput(1.6478125,-1.6882813){\small Die netto lading is nul:}
\rput(1.6682812,-2.0882812){\small Die voorwerp is neutraal}
\rput(6.284844,1.4117187){\small \textbf{6} positiewe ladings en}
\rput(6.0225,1.0517187){\small 4 negatiewe ladings}
\rput(6.257344,0.63171875){\small 6 + (-4) = 2}
\rput(6.2725,-1.6882813){\small Die netto lading is +2}
\rput(6.33,-2.0882812){\small Die voorwerp is positief gelaai}
\rput(10.787812,1.4117187){\small \textbf{6} positiewe ladings end}
\rput(10.523125,1.0517187){\small 9 negatiewe ladings}
\rput(10.813281,0.63171875){\small 6 + (-9) = -3}
\rput(10.745469,-1.6882813){\small Die netto lading is -3}
\rput(10.87,-2.0882812){\small Die voorwerp is negatief gelaai}
\end{pspicture}
\end{center}
 \end{figure}       



\subsection{Tribo-elektriese belading}
\nopagebreak
Voorwerpe kan op vele maniere gelaai word, insluitend deur kontak met of wrywing teen ander voorwerpe. Dit beteken dat hulle negatiewe lading kan bykry of verloor. Byvoorbeeld, belading kan gebeur as jy jou voete skuur teen die tapyt. Waneer jy dan aan metaal of 'n ander persoon raak, kan jy 'n skok voel wanneer die oortollig lading wat jy opgebou het \textbf{ontlaai}.\par

\Tip{Lading, soos energie, kan nie geskep of vernietig word nie. Ons s\^e lading word \textbf{behou}.}

Wanneer jy jou voete teen die tapyt vryf, word negatiewe lading na jou toe oorgedra vanaf die tapyt. Die tapyt word dus positief belaai met \textbf{dieselfde hoeveelheid}. \par
      

'n Ander voorbeeld is om twee neutrale voorwerpe te vat, soos 'n plastiese liniaal en 'n katoen lap (sakdoek). In die begin is altwee voorwerpe neutraal.\par
      

\begin{figure}[H] % horizontal\label{m38780*id200777}
    \begin{center}
    \begin{pspicture}(0,-1.9746875)(14.900937,1.9746875)
\definecolor{color1034}{rgb}{0.6,0.6,0.6}
\psline[linewidth=0.04cm,linecolor=color1034](0.234375,0.67875)(0.234375,0.41875)
\psline[linewidth=0.04cm,linecolor=color1034](0.314375,0.69875)(0.314375,0.53875)
\psline[linewidth=0.04cm,linecolor=color1034](0.394375,0.69875)(0.394375,0.53875)
\psline[linewidth=0.04cm,linecolor=color1034](0.454375,0.69875)(0.454375,0.53875)
\psline[linewidth=0.04cm,linecolor=color1034](0.534375,0.69875)(0.534375,0.53875)
\psline[linewidth=0.04cm,linecolor=color1034](0.614375,0.67875)(0.614375,0.41875)
\psline[linewidth=0.04cm,linecolor=color1034](0.694375,0.69875)(0.694375,0.53875)
\psline[linewidth=0.04cm,linecolor=color1034](0.774375,0.69875)(0.774375,0.53875)
\psline[linewidth=0.04cm,linecolor=color1034](0.834375,0.69875)(0.834375,0.53875)
\psline[linewidth=0.04cm,linecolor=color1034](0.914375,0.69875)(0.914375,0.53875)
\psline[linewidth=0.04cm,linecolor=color1034](0.974375,0.67875)(0.974375,0.41875)
\psline[linewidth=0.04cm,linecolor=color1034](1.054375,0.69875)(1.054375,0.53875)
\psline[linewidth=0.04cm,linecolor=color1034](1.134375,0.69875)(1.134375,0.53875)
\psline[linewidth=0.04cm,linecolor=color1034](1.194375,0.69875)(1.194375,0.53875)
\psline[linewidth=0.04cm,linecolor=color1034](1.274375,0.69875)(1.274375,0.53875)
\psline[linewidth=0.04cm,linecolor=color1034](1.354375,0.67875)(1.354375,0.41875)
\psline[linewidth=0.04cm,linecolor=color1034](1.434375,0.69875)(1.434375,0.53875)
\psline[linewidth=0.04cm,linecolor=color1034](1.514375,0.69875)(1.514375,0.53875)
\psline[linewidth=0.04cm,linecolor=color1034](1.574375,0.69875)(1.574375,0.53875)
\psline[linewidth=0.04cm,linecolor=color1034](1.654375,0.69875)(1.654375,0.53875)
\psline[linewidth=0.04cm,linecolor=color1034](1.734375,0.67875)(1.734375,0.41875)
\psline[linewidth=0.04cm,linecolor=color1034](1.814375,0.69875)(1.814375,0.53875)
\psline[linewidth=0.04cm,linecolor=color1034](1.894375,0.69875)(1.894375,0.53875)
\psline[linewidth=0.04cm,linecolor=color1034](1.954375,0.69875)(1.954375,0.53875)
\psline[linewidth=0.04cm,linecolor=color1034](2.034375,0.69875)(2.034375,0.53875)
\psline[linewidth=0.04cm,linecolor=color1034](2.114375,0.67875)(2.114375,0.41875)
\psline[linewidth=0.04cm,linecolor=color1034](2.194375,0.69875)(2.194375,0.53875)
\psline[linewidth=0.04cm,linecolor=color1034](2.274375,0.69875)(2.274375,0.53875)
\psline[linewidth=0.04cm,linecolor=color1034](2.334375,0.69875)(2.334375,0.53875)
\psline[linewidth=0.04cm,linecolor=color1034](2.414375,0.69875)(2.414375,0.53875)
\psline[linewidth=0.04cm,linecolor=color1034](2.494375,0.67875)(2.494375,0.41875)
\psline[linewidth=0.04cm,linecolor=color1034](2.574375,0.69875)(2.574375,0.53875)
\psline[linewidth=0.04cm,linecolor=color1034](2.654375,0.69875)(2.654375,0.53875)
\psline[linewidth=0.04cm,linecolor=color1034](2.714375,0.69875)(2.714375,0.53875)
\psline[linewidth=0.04cm,linecolor=color1034](2.794375,0.69875)(2.794375,0.53875)
\psline[linewidth=0.04cm,linecolor=color1034](2.874375,0.67875)(2.874375,0.41875)
\psline[linewidth=0.04cm,linecolor=color1034](2.954375,0.69875)(2.954375,0.53875)
\psline[linewidth=0.04cm,linecolor=color1034](3.034375,0.69875)(3.034375,0.53875)
\psline[linewidth=0.04cm,linecolor=color1034](3.094375,0.69875)(3.094375,0.53875)
\psline[linewidth=0.04cm,linecolor=color1034](3.174375,0.69875)(3.174375,0.53875)
\psline[linewidth=0.04cm,linecolor=color1034](3.254375,0.67875)(3.254375,0.41875)
\psline[linewidth=0.04cm,linecolor=color1034](3.334375,0.69875)(3.334375,0.53875)
\psline[linewidth=0.04cm,linecolor=color1034](3.414375,0.69875)(3.414375,0.53875)
\psline[linewidth=0.04cm,linecolor=color1034](3.474375,0.69875)(3.474375,0.53875)
\psline[linewidth=0.04cm,linecolor=color1034](3.554375,0.69875)(3.554375,0.53875)
\psline[linewidth=0.04cm,linecolor=color1034](3.634375,0.67875)(3.634375,0.41875)
\psline[linewidth=0.04cm,linecolor=color1034](3.714375,0.69875)(3.714375,0.53875)
\psline[linewidth=0.04cm,linecolor=color1034](3.794375,0.69875)(3.794375,0.53875)
\psline[linewidth=0.04cm,linecolor=color1034](3.854375,0.69875)(3.854375,0.53875)
\psline[linewidth=0.04cm,linecolor=color1034](3.934375,0.69875)(3.934375,0.53875)
\psline[linewidth=0.04cm,linecolor=color1034](4.014375,0.67875)(4.014375,0.41875)
\psline[linewidth=0.04cm,linecolor=color1034](4.094375,0.69875)(4.094375,0.53875)
\psline[linewidth=0.04cm,linecolor=color1034](4.174375,0.69875)(4.174375,0.53875)
\psline[linewidth=0.04cm,linecolor=color1034](4.234375,0.69875)(4.234375,0.53875)
\psline[linewidth=0.04cm,linecolor=color1034](4.314375,0.69875)(4.314375,0.53875)
\psline[linewidth=0.04cm,linecolor=color1034](4.394375,0.67875)(4.394375,0.41875)
\psline[linewidth=0.04cm,linecolor=color1034](4.474375,0.69875)(4.474375,0.53875)
\psline[linewidth=0.04cm,linecolor=color1034](4.554375,0.69875)(4.554375,0.53875)
\psline[linewidth=0.04cm,linecolor=color1034](4.614375,0.69875)(4.614375,0.53875)
\psline[linewidth=0.04cm,linecolor=color1034](4.694375,0.69875)(4.694375,0.53875)
\psline[linewidth=0.04cm,linecolor=color1034](4.794375,0.69875)(4.794375,0.43875)
\psframe[linewidth=0.04,dimen=outer](5.234375,0.69875)(0.054375,-0.02125)
\rput(4.9290624,0.24875){+}
\rput(4.4090624,0.24875){+}
\rput(3.8690624,0.22875){+}
\rput(3.3290625,0.22875){+}
\rput(2.7090626,0.24875){+}
\rput(2.1890626,0.24875){+}
\rput(1.7090625,0.22875){+}
\rput(1.1490625,0.22875){+}
\rput(0.5090625,0.20875){+}
\rput(0.21984375,0.24875){-}
\rput(0.81984377,0.20875){-}
\rput(1.3998437,0.24875){-}
\rput(1.9198438,0.28875){-}
\rput(2.4198437,0.28875){-}
\rput(2.9598436,0.24875){-}
\rput(3.6198437,0.24875){-}
\rput(4.119844,0.28875){-}
\rput(4.639844,0.26875){-}
\psbezier[linewidth=0.04](6.042201,1.0521983)(5.814375,1.19875)(6.6262674,-0.91064626)(6.6262674,-0.53135264)(6.6262674,-0.152059)(8.434375,-1.02125)(8.401829,-0.5882467)(8.369282,-0.15524337)(7.636984,0.64375)(7.86481,0.8406855)(8.092636,1.037621)(6.270027,0.9056466)(6.042201,1.0521983)
\rput(6.6290627,0.62875){+}
\rput(7.4490623,0.30875){+}
\rput(6.6690626,-0.09125){+}
\rput(7.6490626,-0.37125){+}
\rput(7.3090625,0.76875){+}
\rput(6.4598436,0.48875){-}
\rput(6.8798437,0.08875){-}
\rput(7.9598436,-0.37125){-}
\rput(7.5398436,0.12875){-}
\rput(7.099844,0.66875){-}
\rput(2.4446876,-0.34125){\small Die liniaal het 9 positiewe ladings en}
\rput(7.749375,-1.12125){\small die katoen lap het}
\rput(7.7117186,-1.44125){\small 5 positiewe ladings and}
\rput(7.726406,-1.76125){\small 5 negatiewe ladings}
\rput(12.6725,0.91875){\small Die totale lading is:}
\rput(12.3175,0.49875){\small (9+5)=14 positiewe ladings}
\rput(12.3375,0.11875){\small (9+5)=14 negatiewe ladings}
\rput(1.3078125,-0.68125){\small 9 negatiewe ladings}
\rput(1.2048438,1.79875){\small VOOR wrywing:}
\end{pspicture}\end{center}
 \end{figure}       
      \par 


As die katoen lap nou gebruik word om die liniaal te vryf, sal negatiewe lading van die lap na die liniaal oorgedra word. Die liniaal is nou \textsl{negatief} gelaai (dit het nou oortollige elektrone) en die lap is \textsl{positief} gelaai (dit is arm in elektrone).

As jy al die positiewe en negatiewe ladings tel voor en na die eksperiment, sal jy dieselfde hoeveelheid tel, die totale lading is \textsl{behou}!. \par     

\begin{figure}[H] % horizontal\label{m38780*id200819}
    \begin{center}
    \begin{pspicture}(0,-2.1746874)(15.098437,2.1746874)
\definecolor{color1034}{rgb}{0.6,0.6,0.6}
\psline[linewidth=0.04cm,linecolor=color1034](0.2371875,0.87875)(0.2371875,0.61875)
\psline[linewidth=0.04cm,linecolor=color1034](0.3171875,0.89875)(0.3171875,0.73875)
\psline[linewidth=0.04cm,linecolor=color1034](0.3971875,0.89875)(0.3971875,0.73875)
\psline[linewidth=0.04cm,linecolor=color1034](0.4571875,0.89875)(0.4571875,0.73875)
\psline[linewidth=0.04cm,linecolor=color1034](0.5371875,0.89875)(0.5371875,0.73875)
\psline[linewidth=0.04cm,linecolor=color1034](0.6171875,0.87875)(0.6171875,0.61875)
\psline[linewidth=0.04cm,linecolor=color1034](0.6971875,0.89875)(0.6971875,0.73875)
\psline[linewidth=0.04cm,linecolor=color1034](0.7771875,0.89875)(0.7771875,0.73875)
\psline[linewidth=0.04cm,linecolor=color1034](0.8371875,0.89875)(0.8371875,0.73875)
\psline[linewidth=0.04cm,linecolor=color1034](0.9171875,0.89875)(0.9171875,0.73875)
\psline[linewidth=0.04cm,linecolor=color1034](0.9771875,0.87875)(0.9771875,0.61875)
\psline[linewidth=0.04cm,linecolor=color1034](1.0571876,0.89875)(1.0571876,0.73875)
\psline[linewidth=0.04cm,linecolor=color1034](1.1371875,0.89875)(1.1371875,0.73875)
\psline[linewidth=0.04cm,linecolor=color1034](1.1971875,0.89875)(1.1971875,0.73875)
\psline[linewidth=0.04cm,linecolor=color1034](1.2771875,0.89875)(1.2771875,0.73875)
\psline[linewidth=0.04cm,linecolor=color1034](1.3571875,0.87875)(1.3571875,0.61875)
\psline[linewidth=0.04cm,linecolor=color1034](1.4371876,0.89875)(1.4371876,0.73875)
\psline[linewidth=0.04cm,linecolor=color1034](1.5171875,0.89875)(1.5171875,0.73875)
\psline[linewidth=0.04cm,linecolor=color1034](1.5771875,0.89875)(1.5771875,0.73875)
\psline[linewidth=0.04cm,linecolor=color1034](1.6571875,0.89875)(1.6571875,0.73875)
\psline[linewidth=0.04cm,linecolor=color1034](1.7371875,0.87875)(1.7371875,0.61875)
\psline[linewidth=0.04cm,linecolor=color1034](1.8171875,0.89875)(1.8171875,0.73875)
\psline[linewidth=0.04cm,linecolor=color1034](1.8971875,0.89875)(1.8971875,0.73875)
\psline[linewidth=0.04cm,linecolor=color1034](1.9571875,0.89875)(1.9571875,0.73875)
\psline[linewidth=0.04cm,linecolor=color1034](2.0371876,0.89875)(2.0371876,0.73875)
\psline[linewidth=0.04cm,linecolor=color1034](2.1171875,0.87875)(2.1171875,0.61875)
\psline[linewidth=0.04cm,linecolor=color1034](2.1971874,0.89875)(2.1971874,0.73875)
\psline[linewidth=0.04cm,linecolor=color1034](2.2771876,0.89875)(2.2771876,0.73875)
\psline[linewidth=0.04cm,linecolor=color1034](2.3371875,0.89875)(2.3371875,0.73875)
\psline[linewidth=0.04cm,linecolor=color1034](2.4171875,0.89875)(2.4171875,0.73875)
\psline[linewidth=0.04cm,linecolor=color1034](2.4971876,0.87875)(2.4971876,0.61875)
\psline[linewidth=0.04cm,linecolor=color1034](2.5771875,0.89875)(2.5771875,0.73875)
\psline[linewidth=0.04cm,linecolor=color1034](2.6571875,0.89875)(2.6571875,0.73875)
\psline[linewidth=0.04cm,linecolor=color1034](2.7171874,0.89875)(2.7171874,0.73875)
\psline[linewidth=0.04cm,linecolor=color1034](2.7971876,0.89875)(2.7971876,0.73875)
\psline[linewidth=0.04cm,linecolor=color1034](2.8771875,0.87875)(2.8771875,0.61875)
\psline[linewidth=0.04cm,linecolor=color1034](2.9571874,0.89875)(2.9571874,0.73875)
\psline[linewidth=0.04cm,linecolor=color1034](3.0371876,0.89875)(3.0371876,0.73875)
\psline[linewidth=0.04cm,linecolor=color1034](3.0971875,0.89875)(3.0971875,0.73875)
\psline[linewidth=0.04cm,linecolor=color1034](3.1771874,0.89875)(3.1771874,0.73875)
\psline[linewidth=0.04cm,linecolor=color1034](3.2571876,0.87875)(3.2571876,0.61875)
\psline[linewidth=0.04cm,linecolor=color1034](3.3371875,0.89875)(3.3371875,0.73875)
\psline[linewidth=0.04cm,linecolor=color1034](3.4171875,0.89875)(3.4171875,0.73875)
\psline[linewidth=0.04cm,linecolor=color1034](3.4771874,0.89875)(3.4771874,0.73875)
\psline[linewidth=0.04cm,linecolor=color1034](3.5571876,0.89875)(3.5571876,0.73875)
\psline[linewidth=0.04cm,linecolor=color1034](3.6371875,0.87875)(3.6371875,0.61875)
\psline[linewidth=0.04cm,linecolor=color1034](3.7171874,0.89875)(3.7171874,0.73875)
\psline[linewidth=0.04cm,linecolor=color1034](3.7971876,0.89875)(3.7971876,0.73875)
\psline[linewidth=0.04cm,linecolor=color1034](3.8571875,0.89875)(3.8571875,0.73875)
\psline[linewidth=0.04cm,linecolor=color1034](3.9371874,0.89875)(3.9371874,0.73875)
\psline[linewidth=0.04cm,linecolor=color1034](4.0171876,0.87875)(4.0171876,0.61875)
\psline[linewidth=0.04cm,linecolor=color1034](4.0971875,0.89875)(4.0971875,0.73875)
\psline[linewidth=0.04cm,linecolor=color1034](4.1771874,0.89875)(4.1771874,0.73875)
\psline[linewidth=0.04cm,linecolor=color1034](4.2371874,0.89875)(4.2371874,0.73875)
\psline[linewidth=0.04cm,linecolor=color1034](4.3171873,0.89875)(4.3171873,0.73875)
\psline[linewidth=0.04cm,linecolor=color1034](4.3971877,0.87875)(4.3971877,0.61875)
\psline[linewidth=0.04cm,linecolor=color1034](4.4771876,0.89875)(4.4771876,0.73875)
\psline[linewidth=0.04cm,linecolor=color1034](4.5571876,0.89875)(4.5571876,0.73875)
\psline[linewidth=0.04cm,linecolor=color1034](4.6171875,0.89875)(4.6171875,0.73875)
\psline[linewidth=0.04cm,linecolor=color1034](4.6971874,0.89875)(4.6971874,0.73875)
\psline[linewidth=0.04cm,linecolor=color1034](4.7971873,0.89875)(4.7971873,0.63875)
\psframe[linewidth=0.04,dimen=outer](5.2371874,0.89875)(0.0571875,0.17875)
\rput(4.931875,0.44875){+}
\rput(4.411875,0.44875){+}
\rput(3.871875,0.42875){+}
\rput(3.331875,0.42875){+}
\rput(2.711875,0.44875){+}
\rput(2.191875,0.44875){+}
\rput(1.711875,0.42875){+}
\rput(1.151875,0.42875){+}
\rput(0.511875,0.40875){+}
\rput(0.22265625,0.44875){-}
\rput(0.8226563,0.40875){-}
\rput(1.4026562,0.44875){-}
\rput(1.9226563,0.48875){-}
\rput(2.4226563,0.48875){-}
\rput(2.9626563,0.44875){-}
\rput(3.6226563,0.44875){-}
\rput(4.1226563,0.48875){-}
\rput(4.6426563,0.46875){-}
\psbezier[linewidth=0.04](6.0450134,1.2521983)(5.8171873,1.39875)(6.62908,-0.7106463)(6.62908,-0.33135265)(6.62908,0.047941)(8.437187,-0.82125)(8.404641,-0.3882467)(8.372094,0.04475663)(7.6397963,0.84375)(7.8676224,1.0406855)(8.0954485,1.237621)(6.2728395,1.1056466)(6.0450134,1.2521983)
\rput(6.631875,0.82875){+}
\rput(7.451875,0.50875){+}
\rput(6.671875,0.10875){+}
\rput(7.651875,-0.17125){+}
\rput(7.311875,0.96875){+}
\rput(4.9826565,0.72875){-}
\rput(5.0426564,0.34875){-}
\rput(7.962656,-0.17125){-}
\rput(7.5426564,0.32875){-}
\rput(4.6226563,0.62875){-}
\rput(2.4475,-0.14125){\small Die liniaal het 9 positiewe ladings en}
\rput(7.4821873,-0.92125){\small Die katoen lap het}
\rput(7.614531,-1.24125){\small 5 positiewe ladings en}
\rput(7.3940625,-1.58125){\small 2 negatiewe ladings.}
\rput(12.585313,1.11875){\small Die totale lading is:}
\rput(12.3203125,0.69875){\small (9+5)=14 positiewe ladings}
\rput(12.430312,0.31875){\small (12+2)=14 negatiewe ladings}
\rput(1.3873438,-0.48125){\small 12 negatiewe ladings}
\rput(1.0790625,1.99875){\small NA wrywing:}
\rput(7.6434374,-1.96125){\small Dit is nou positief gelaai.}
\rput(2.0234375,-0.84125){\small Dit is nou negatief gelaai.}
\rput(12.391719,-0.28125){\small Ladings is oorgedra van die}
\rput(12.3515625,-0.64125){\small lap na die liniaal MAAR die totale}
\rput(10.865313,-0.98125){\small lading het behoue gebly!}
\end{pspicture}\end{center}
 \end{figure}       
 \par 

\begin{minipage}{.35\textwidth}

Neem kennis dat die hoeveelhede in hierdie voorbeeld maklik is om te bereken. In die regte w\^ereld sal daar net 'n klein breukdeel van die ladings oorgedra word, maar die totale lading sal nog steeds behoue bly. \par

Die proses waardeur materiale gelaai word wanneer hulle in kontak kom met ander materiale word tribo-elektriese belading. Materiale kan rangskik word in 'n tribo-elekriese reeks volgens hulle vermo\"e om meer positief of meer negatief te raak. Hierdie reeks kan ons help besluit of een materiaal belaai kan raak van 'n ander materiaal af.

Positiewe materiale is meer geneig om elektrone te verloor en negatiewe materiale is geneig om elektrone te wen. Wanner twee materiale gekies en teen mekaar gevryf word sal die meer positiewe een elektrone verloor en die negatiewe materiaal sal elektrone bykry.

Byvoorbeeld, \textbf{!!!AMBER!!! } is meer negatief as wol, so as 'n stuk wol teen amber gevryf word sal die amber meer negatief gelaai word.
\end{minipage}
\Tip{Hierdie simulasie sal jou help verstaan wat gebeur as jy een voorwerp teen 'n ander vryf. \raisebox{-5 pt}{ \includegraphics[width=0.5cm]{col11305.imgs/summary_www.png}} {(Simulasie: lbX)}}
\begin{minipage}{.55\textwidth}
\begin{center}
\begin{table}[H]
\centering
\begin{tabular}{|cc|}\hline
\textbf{Materiaal}&\textbf{Tribo-elektriese reeks}\\\hline
Glss& Baie positief\\\hline
Menslike hare&\\\hline
Nylon&\\\hline
Wol&\\\hline
Pels&\\\hline
Lood&\\\hline
Sy&\\\hline
Aluminium&\\\hline
Papier&\\\hline
Cotton&Neutraal\\\hline
Staal &Neutraal\\\hline
Hout&\\\hline
Amber&\\\hline
Harde rubber&\\\hline
Nikkel, Koper&\\\hline
Goud, Platinum&\\\hline
Polyester&\\\hline
Polyurethane&\\\hline
Polypropileen&\\\hline
Sikicon&\\\hline
Teflon& Baie negatief\\\hline
\end{tabular}
\caption{Tribo-elektriese reeks.}
\end{table}
\end{center}
\end{minipage}

\begin{wex}{Tribo-elekriese belading}
{
As jy katoen  en sy teen mekaar vryf, watter een word negatief belaai?
}
{
\westep{Analiseer die gegewe informasie}

Daar is twee materiale gegee en hulle word teen mekaar gevryf. Dit beteken ons is besig met die interaksie tussen die materiale. Die vraag is verwant aan die lading op die materiale en ons kan aanneem dat hulle neutraal was aan die begin. Dit beteken ons het te doen met elektrostatika en die interaksie tussen materiale wat lei tot belading is tribo-elektriese belading.


\westep{Verkry materiaal eienskappe}

Spoor die materiale op in die tribo-elektriese reeks. Die belangrike item om te weet is watter materiaal is meer positief en meer negatief in die reeks. Sy val bo katoen in ons tabel wat dit meer positief maak.

\westep{Pas die beginsels toe}
Ons weet dat wanneer twee materiale teen mekaar gevryf word, die meer negatiewe een in die reeks die elektrone gaan by kry en die meer positiewe een elektrone gaan verloor. Dit beteken dat die sy elektrone gaan verloor en die katoen elektrone gaan by kry.
\par
'n Materiaal wat meer negatief belaai raak het 'n oormaat van elektrone, dus raak die katoen, wat elektrone bykry, meer negatief belaai.
}\end{wex}

\subsection{Kragte tussen ladings}
            \nopagebreak

Die krag wat stilstaande (statiese) ladings op mekaar uitoefen word die \textbf{elektrostatiese krag} genoem. Die elektrostatiese krag tussen:

\begin{itemize}[noitemsep]
\item \textbf{soortgelyke} ladings is \textbf{afstotend}.
\item \textbf{teenoorgestelde} ladings is \textbf{aantreklik}.
\end{itemize}

In ander woorde, soortgelyke ladings stoot mekaar af en teenoorgestelde ladings trek mekaar aan.


\begin{figure}[H] % horizontal\label{m38780*id200901}
    \begin{center}
    \begin{pspicture}(0,-0.6)(10,0.8)
%\psgrid[gridcolor=gray]
\rput(-2,0){\pscircle[linewidth=1pt](0.5,0.25){0.25}
\pscircle[linewidth=1pt](2.5,0.25){0.25}
\psline[linewidth=2pt]{->}(0.85,0.25)(1.4,0.25)
\psline[linewidth=2pt]{<-}(1.6,0.25)(2.15,0.25) \rput(1,0.5){F}
\rput(2,0.5){F} \rput(0.5,0.25){-} \rput(2.5,0.25){+}
\uput[d](1.5,0){aantrekkingskrag}}

\rput(3.5,0){\pscircle[linewidth=1pt](2.5,0.25){0.25}
\pscircle[linewidth=1pt](0.5,0.25){0.25}
\psline[linewidth=2pt]{<-}(-0.4,0.25)(.15,0.25)
\psline[linewidth=2pt]{->}(2.85,0.25)(3.4,0.25) \rput(-.2,0.5){F}
\rput(3.2,0.5){F} \rput(0.5,0.25){-} \rput(2.5,0.25){-}
\uput[d](1.5,0){afstotende krag}}

\rput(9,0){\pscircle[linewidth=1pt](2.5,0.25){0.25}
\pscircle[linewidth=1pt](0.5,0.25){0.25}
\psline[linewidth=2pt]{<-}(-0.4,0.25)(.15,0.25)
\psline[linewidth=2pt]{->}(2.85,0.25)(3.4,0.25) \rput(-.2,0.5){F}
\rput(3.2,0.5){F} \rput(0.5,0.25){+} \rput(2.5,0.25){+}
\uput[d](1.5,0){afstotende krag}}
\end{pspicture}\end{center}
 \end{figure}       
      \par 
Hoe \textsl{nader} ladings aan mekaar is, hoe \textsl{sterker} word die elektrostatiese krag tussen hulle.\par
	
\begin{figure}[H] % horizontal\label{m38780*id200924}
    \begin{center}
 \begin{pspicture}(0,-2.3292189)(9.975,2.3292189)
\pscircle[linewidth=0.035277776,dimen=outer](3.45,1.9107813){0.25}
\pscircle[linewidth=0.035277776,dimen=outer](1.45,1.9107813){0.25}
\psline[linewidth=0.07055555cm,arrowsize=0.05291667cm 2.0,arrowlength=1.4,arrowinset=0.4]{<-}(.1,1.9107813)(1.1,1.9107813)
\psline[linewidth=0.07055555cm,arrowsize=0.05291667cm 2.0,arrowlength=1.4,arrowinset=0.4]{->}(3.8,1.9107813)(4.8,1.9107813)
\rput(1.9196875,2.1607811){F}
\rput(2.9196875,2.1607811){F}
\rput(1.4246875,1.9107813){+}
\rput(3.4246874,1.9107813){+}
\rput(7.5,2.0407813){sterker afstotende krag}
\rput(7.5,0.9607813){swakker afstotende krag}

\pscircle[linewidth=0.035277776,dimen=outer](4.65,0.73078126){0.25}
\pscircle[linewidth=0.035277776,dimen=outer](0.25,0.73078126){0.25}
\psline[linewidth=0.07055555cm,arrowsize=0.05291667cm 2.0,arrowlength=1.4,arrowinset=0.4]{->}(-0.1,0.69078124)(-.5,0.69078124)
\psline[linewidth=0.07055555cm,arrowsize=0.05291667cm 2.0,arrowlength=1.4,arrowinset=0.4]{->}(5.0,0.69078124)(5.4,0.69078124)
\rput(0.7396875,0.94078124){F}
\rput(4.1396875,0.94078124){F}
\rput(0.2446875,0.73078126){+}
\rput(4.6246877,0.73078126){+}
% \rput(7.445156,1.6807812){(shorter distance between charges)}
% \rput(7.405156,0.5807812){(longer distance between charges)}
\end{pspicture}   
\end{center}
 \end{figure}       
      \par 
\label{m38780*secfhsst!!!underscore!!!id162}
            

\begin{g_experiment}{Electrostatic force}
            \nopagebreak
Jy kan maklik toets dat soortgelyke ladings mekaar afstoot en teenoorgestelde ladings mekaar aantrek met hierdie eenvoudige eksperiment. \par      

Neem 'n glas staaf en vryf dit met 'n stuk sy en hang dit dan met 'n stuk tou in die middel sodat dit vrylik kan beweeg. As jy nog 'n glas staaf wat jy ook gelaai het nader bring aan die een wat hang, sal jy sien dat die eerste glas staaf \textsl{weg beweeg} van die een in jou hand, dit word \textbf{afgestoot}. Andersins, as jy 'n plastiese staaf met 'n stuk pels vryf en dit nader bring aan die glas staaf met die tou sal dit glas staaf \textsl{na die plastiese staaf toe} beweeg, dit word \textbf{aangetrek}.\par
      
	\begin{figure}[H] % horizontal\label{m38780*id200974}
    \begin{center}
  \begin{pspicture}(0,0.8)(11.2,8)
%\psgrid
%left pic
\psline[linewidth = 2pt](2, 7.5)(4, 7.5) \multiput(2.1, 7.6)(0.2,
0){10}{/} \psline(3, 7.5)(3, 4.5) \psframe(1, 3.5)(5, 4.5)
\psline{->}(0.5, 3.5)(0.5, 4.5) \psline{->}(0.5, 3.3)(0.5, 2.3)
\rput(0.75, 4.5){F} \rput(0.75, 2.5){F}
\rput{30}{%
\psline(1,0.5)(3,0.5) \psline(3,0.5)(3, 1.5) \psline(3,1.5)(1,1.5)
}
%plus signs!
\rput(1.25, 3.75){+} \rput(1.25, 4.25){+} \rput(1.5, 4){+}
\rput(1.75, 3.75){+} \rput(1.75, 4.25){+} \rput(2, 4){+}
%More plus signs!
\rput(1.25, 1.5){+} \rput(1.25, 2){+} \rput(1.5, 1.75){+}
\rput(1.5, 2.25){+} \rput(1.75, 2){+} \rput(1.75, 2.5){+}
%arrow!
\psecurve[linewidth = 2pt]{->}(2.5, 6)(2.5, 6)(3, 6.25 )(3.5,
6)(3, 5.75)(2.5, 6)
%RIGHT pic
\psline[linewidth = 2pt](8, 7.5)(10, 7.5) \multiput(8.1, 7.6)(0.2,
0){10}{/} \psline(9, 7.5)(9, 4.5) \psframe(7, 3.5)(11, 4.5)
\psline{<-}(6.5, 3.5)(6.5, 4.5) \psline{<-}(6.5, 3.3)(6.5, 2.3)
\rput(6.75, 4.5){F} \rput(6.75, 2.5){F}
\rput{30}(1,-3){%
\psline(7,0.5)(9,0.5) \psline(9,0.5)(9, 1.5) \psline(9,1.5)(7,1.5)
}
%plus signs!
%\degree[12]
%\multido{\n=0+.1}{12}{\rput{\n}{+}}
\rput(7.25, 3.75){+} \rput(7.25, 4.25){+} \rput(7.5, 4){+}
\rput(7.75, 3.75){+} \rput(7.75, 4.25){+} \rput(8, 4){+}
%Minussigns!
\rput(7.25, 1.5){-} \rput(7.25, 2){-} \rput(7.5, 1.75){-}
\rput(7.5, 2.25){-} \rput(7.75, 2){-} \rput(7.75, 2.5){-} \rput(8,
2.25){-}
%arrow!
\psecurve[linewidth = 2pt]{<-}(8.5, 6)(8.5, 6)(9, 6.25 )(9.5,
6)(9, 5.75)(9.5, 6)
\end{pspicture}
  \end{center}
 \end{figure}       
      \par 

Dit gebeur want die glas verloor klein hoeveelhede negatiewe lading wanneer jy dit vryf met sy, wat die glas meer \textbf{positief} gelaai maak. Waneer jy die plastiek met pels vryf, word klein hoeveelhede negatiewe lading oorgedra na die plastiek toe wat dit meer \textbf{negatief} gelaai maak.\par

\end{g_experiment}



\begin{wex}
{
Toepassing van elektrostatiese kragte
}
{
Twee gelaaide metaal sfere hang van toutjies en is vry om te beweeg soos in die diagram hieronder. Die sfeer aan die regterkant is positief gelaai. Die lading van die ander sfeer is nie bekend nie.
\begin{center}
\begin{pspicture}(0,-1.23)(2.74,1.25)
\pscircle[linewidth=0.04,dimen=outer](0.41,-0.82){0.41}
\pscircle[linewidth=0.04,dimen=outer](2.33,-0.82){0.41}
\psline[linewidth=0.04cm](0.4,-0.43)(0.4,1.23)
\psline[linewidth=0.04cm](2.34,-0.45)(2.34,1.21)
\rput(2.3559375,-0.815){\large +}
\rput(0.3790625,-0.835){\large ?}
\end{pspicture}
\end{center}
Die linkerkantste sfeer word nou nader gebring aan die regterkantste sfeer.      
\begin{enumerate}[noitemsep, label=\textbf{\arabic*}. ] 
\item As die linkerkantste sfeer na die regterkantste sfeer to swaai, wat kan jy s\^{e} van die lading op die linkerkantste sfeer en hoekom?
\item As die linkerkantste sfeer wegswaai van die regterkantste een, wat kan jy dan s\^{e} van die lading op die linkerkantste sfeer en hoekom?
\end{enumerate}
}
{
\westep{Analiseer die probleem}  
In die eerste geval het ons 'n sfeer met 'n positiewe lading wat \textsl{aangetrek} word na die linkerkantste sfeer. Ons moet die lading van die linkerkantste sfeer vind.\par

\westep{Identifiseer die beginsels.}  

Ons het te doen met elektrostatiese kragte tussen gelaaide voorwerpe. Daarom weet ons dat \textsl{soortgeluke} ladings \textsl{afstotend} is en dat \textsl{teenoorgestelde} ladings mekaar \textsl{aantrek}.\par
      
\westep{Pas die beginsels toe.}
\begin{enumerate}[noitemsep, label=\textbf{\alph*}. ] 
    \item In die eerste geval word die positiewe sfeer aangetrek deur die linkerkantste sfeer. Sedert die elektrostatiese krag tussen twee teenoorgestelde ladings aantreklik is weet ons dat die linkerkantste sfeer \textsl{negatief} gelaai moet wees.
    \item In die tweede geval word die positiewe sfeer afgestoor deur die ander sfeer. Soortgelyke ladings stoot mekaar af. dus weet ons dat die linkerkantste sfeer ook 'n \textsl{positiewe} lading het.
\end{enumerate}
}
\end{wex}
    

\IFact{The woord 'elektron' kom van die Griekse woord vir amber. Die antieke Grieke het gesien dat wanneer jy 'n stuk amber vryf, jy stukkies strooi daarmee kan optel.}


\subsection*{Polarisation}
            \nopagebreak
            \label{m38780*id201876}Unlike conductors, the electrons in insulators (non-conductors) are bound to the atoms of the
insulator and cannot move around freely through the material. However, a charged object can still
exert a force on a neutral insulator due to a phenomenon called \textbf{polarisation}.\par 
        \label{m38780*id201887}If a positively charged rod is
brought close to a neutral insulator such as polystyrene, it can attract the bound electrons
to move round to the
side of the atoms which is closest to the rod and cause the positive nuclei to move slightly
to the opposite side of the atoms. This process is called \textsl{polarisation}. Although
it is a very small (microscopic) effect, if there are many atoms and the polarised object is
light (e.g. a small polystyrene ball), it can add up to enough force to cause the object to be attracted onto the
charged rod. Remember, that the polystyrene
is \textsl{only} polarised, \textsl{not charged.}
The polystyrene ball is still neutral since no charge was added or removed from it.
The picture shows
a not-to-scale view of the polarised atoms in the polystyrene ball:\par 
        \label{m38780*id201914}
    \setcounter{subfigure}{0}
	\begin{figure}[H] % horizontal\label{m38780*id201917}
    \begin{center}
    \begin{pspicture}(0,-2.96)(6.84125,2.96)
\psarc[linewidth=0.04](1.11125,-1.89){1.05}{176.37851}{175.91438}
\psline[linewidth=0.04cm](1.80125,0.48)(4.12125,2.94)
\psline[linewidth=0.04cm](3.42125,-1.04)(6.82125,2.74)
\psbezier[linewidth=0.04](1.80125,0.48)(1.80125,-0.32)(1.94125,-1.22)(3.44125,-1.04)
\rput(2.0571876,0.275){\large +}
\rput(2.0771875,-0.145){\large +}
\rput(2.2171874,-0.445){\large +}
\rput(2.4771874,-0.725){\large +}
\rput(2.8571875,-0.925){\large +}
\rput(3.2371874,-0.885){\large +}
\rput(2.8171875,-0.645){\large +}
\rput(2.5371876,-0.445){\large +}
\rput(1.6571875,0.215){\large +}
\rput(1.7371875,-0.105){\large +}
\rput(1.8171875,-0.485){\large +}
\rput(2.0171876,-0.725){\large +}
\rput(2.1971874,-1.005){\large +}
\rput(2.4971876,-1.205){\large +}
\rput(2.8371875,-1.285){\large +}
\rput(3.1971874,-1.265){\large +}
\rput{45.571033}(-0.47389764,-1.0318439){\psellipse[linewidth=0.04,dimen=outer](0.99125,-1.08)(0.25,0.18)}
\rput{45.571033}(-0.5237394,-1.0012594){\rput(0.92367446,-1.137003){\small +}}
\rput{45.571033}(-0.39813614,-1.0650314){\rput(1.0548837,-0.98266524){\small -}}
\rput{45.571033}(-0.39447117,-1.3009257){\psellipse[linewidth=0.04,dimen=outer](1.35125,-1.12)(0.25,0.18)}
\rput{45.571033}(-0.44431296,-1.2703412){\rput(1.2836745,-1.1770029){\small +}}
\rput{45.571033}(-0.31870967,-1.3341132){\rput(1.4148837,-1.0226653){\small -}}
\rput{45.571033}(-0.46158466,-1.581156){\psellipse[linewidth=0.04,dimen=outer](1.65125,-1.34)(0.25,0.18)}
\rput{45.571033}(-0.51142645,-1.5505716){\rput(1.5836744,-1.3970029){\small +}}
\rput{45.571033}(-0.38582316,-1.6143435){\rput(1.7148837,-1.2426652){\small -}}
\rput{45.571033}(-0.60982573,-1.8282549){\psellipse[linewidth=0.04,dimen=outer](1.87125,-1.64)(0.25,0.18)}
\rput{45.571033}(-0.6596675,-1.7976704){\rput(1.8036745,-1.6970029){\small +}}
\rput{45.571033}(-0.53406423,-1.8614424){\rput(1.9348837,-1.5426652){\small -}}
\rput{45.571033}(-0.88575757,-2.0113742){\psellipse[linewidth=0.04,dimen=outer](1.95125,-2.06)(0.25,0.18)}
\rput{45.571033}(-0.9355994,-1.9807895){\rput(1.8836745,-2.117003){\small +}}
\rput{45.571033}(-0.80999607,-2.0445616){\rput(2.0148838,-1.9626652){\small -}}
\rput{45.571033}(-1.2756802,-1.9231282){\psellipse[linewidth=0.04,dimen=outer](1.65125,-2.48)(0.25,0.18)}
\rput{45.571033}(-1.325522,-1.8925437){\rput(1.5836744,-2.5370028){\small +}}
\rput{45.571033}(-1.1999187,-1.9563158){\rput(1.7148837,-2.3826652){\small -}}
\rput{45.571033}(-0.96518403,-1.7422923){\psellipse[linewidth=0.04,dimen=outer](1.59125,-2.02)(0.25,0.18)}
\rput{45.571033}(-1.0150259,-1.7117077){\rput(1.5236745,-2.077003){\small +}}
\rput{45.571033}(-0.88942254,-1.7754798){\rput(1.6548836,-1.9226652){\small -}}
\rput{45.571033}(-0.8312254,-1.501193){\psellipse[linewidth=0.04,dimen=outer](1.37125,-1.74)(0.25,0.18)}
\rput{45.571033}(-0.88106716,-1.4706085){\rput(1.3036745,-1.7970029){\small +}}
\rput{45.571033}(-0.7554639,-1.5343806){\rput(1.4348837,-1.6426653){\small -}}
\rput{45.571033}(-0.7783943,-1.2269622){\psellipse[linewidth=0.04,dimen=outer](1.07125,-1.54)(0.25,0.18)}
\rput{45.571033}(-0.82823604,-1.1963776){\rput(1.0036745,-1.5970029){\small +}}
\rput{45.571033}(-0.7026328,-1.2601497){\rput(1.1348836,-1.4426652){\small -}}
\rput{45.571033}(-0.7141464,-0.7799101){\psellipse[linewidth=0.04,dimen=outer](0.57125,-1.24)(0.25,0.18)}
\rput{45.571033}(-0.76398814,-0.7493256){\rput(0.50367445,-1.2970029){\small +}}
\rput{45.571033}(-0.6383849,-0.8130976){\rput(0.6348837,-1.1426653){\small -}}
\rput{45.571033}(-1.0957861,-0.6713823){\psellipse[linewidth=0.04,dimen=outer](0.25125,-1.64)(0.25,0.18)}
\rput{45.571033}(-1.1456279,-0.64079773){\rput(0.18367445,-1.6970029){\small +}}
\rput{45.571033}(-1.0200247,-0.7045698){\rput(0.31488368,-1.5426652){\small -}}
\rput{45.571033}(-0.8578207,-0.9578804){\psellipse[linewidth=0.04,dimen=outer](0.71125,-1.5)(0.25,0.18)}
\rput{45.571033}(-0.9076625,-0.92729586){\rput(0.64367443,-1.5570029){\small +}}
\rput{45.571033}(-0.7820592,-0.9910679){\rput(0.7748837,-1.4026653){\small -}}
\rput{45.571033}(-1.1951805,-0.91476494){\psellipse[linewidth=0.04,dimen=outer](0.49125,-1.88)(0.25,0.18)}
\rput{45.571033}(-1.2450223,-0.8841804){\rput(0.42367446,-1.9370029){\small +}}
\rput{45.571033}(-1.119419,-0.94795245){\rput(0.55488366,-1.7826653){\small -}}
\rput{45.571033}(-1.1714507,-1.2912557){\psellipse[linewidth=0.04,dimen=outer](0.95125,-2.04)(0.25,0.18)}
\rput{45.571033}(-1.2212926,-1.2606711){\rput(0.88367444,-2.097003){\small +}}
\rput{45.571033}(-1.0956893,-1.3244432){\rput(1.0148836,-1.9426652){\small -}}
\rput{45.571033}(-1.2505633,-1.5429213){\psellipse[linewidth=0.04,dimen=outer](1.21125,-2.26)(0.25,0.18)}
\rput{45.571033}(-1.300405,-1.5123367){\rput(1.1436745,-2.317003){\small +}}
\rput{45.571033}(-1.1748018,-1.5761088){\rput(1.2748836,-2.1626651){\small -}}
\rput{45.571033}(-1.5324947,-1.7117581){\psellipse[linewidth=0.04,dimen=outer](1.27125,-2.68)(0.25,0.18)}
\rput{45.571033}(-1.5823364,-1.6811736){\rput(1.2036744,-2.7370028){\small +}}
\rput{45.571033}(-1.4567332,-1.7449456){\rput(1.3348837,-2.5826652){\small -}}
\rput{45.571033}(-1.6119212,-1.4426763){\psellipse[linewidth=0.04,dimen=outer](0.91125,-2.64)(0.25,0.18)}
\rput{45.571033}(-1.661763,-1.4120917){\rput(0.8436745,-2.697003){\small +}}
\rput{45.571033}(-1.5361596,-1.4758638){\rput(0.9748837,-2.5426652){\small -}}
\rput{45.571033}(-1.5733724,-1.174445){\psellipse[linewidth=0.04,dimen=outer](0.61125,-2.46)(0.25,0.18)}
\rput{45.571033}(-1.6232142,-1.1438605){\rput(0.54367447,-2.5170028){\small +}}
\rput{45.571033}(-1.4976109,-1.2076325){\rput(0.67488366,-2.3626652){\small -}}
\rput{45.571033}(-1.5025427,-0.9430614){\psellipse[linewidth=0.04,dimen=outer](0.37125,-2.26)(0.25,0.18)}
\rput{45.571033}(-1.5523845,-0.91247684){\rput(0.30367446,-2.317003){\small +}}
\rput{45.571033}(-1.4267813,-0.9762489){\rput(0.43488368,-2.1626651){\small -}}
\psline[linewidth=0.04cm](1.38125,1.76)(2.98125,1.76)
\psline[linewidth=0.04cm](2.04125,-2.3)(3.84125,-2.3)
\rput(0.68078125,1.79){positively}
\rput(0.845625,1.45){charged rod}
\rput(4.5623436,-2.25){polarised}
\rput(5.058125,-2.59){polystyrene ball}
\end{pspicture}
\end{center}
 \end{figure}       
        \par 
        \label{m38780*id201923}Some materials are made up of molecules which are already polarised.
These are molecules which have
a more positive and a more negative side but are still neutral overall.
Just as a polarised polystyrene ball can be attracted to a charged rod, these materials
are also affected if brought close to a charged object.\par 
        \label{m38780*id201929}Water is an example of a substance which is made of polarised molecules.
If a positively charged rod is brought close to a stream of water, the molecules can rotate
so that the negative sides all line up towards the rod.
The stream of water will then be attracted to the rod since opposite charges attract.\par 
    \label{m38780*eip-275}
            \section{Conservation of charge}
            \nopagebreak
   In all of the examples we've looked at charge was not created or destroyed but it moved from one material to another.
\Definition{Principle of conservation of charge}{
            \label{m38780*eip-506}The principle of conservation of charge states that the net charge of an isolated system remains constant during any physical process.}

%%% MH - if you introduced the quantisation of charge this would make perfect sense to anyone and everyone
      
         \subsection*{Conductors and insulators}
    \nopagebreak
            \label{m38781} $ \hspace{-5pt}\begin{array}{cccccccccccc}   \includegraphics[width=0.75cm]{col11305.imgs/summary_fullmarks.png} &   \includegraphics[width=0.75cm]{col11305.imgs/summary_video.png} &   \includegraphics[width=0.75cm]{col11305.imgs/summary_presentation.png} &   \end{array} $ \hspace{2 pt}\raisebox{-5 pt}{} {(seksie kortkode: P10072 )} \par 
            \label{m38781*id201248}

Some materials allow electrons to move relatively freely
through them (e.g. most metals, the human body).
These materials are called \textbf{conductors}.\par 
      \label{m38781*id201271}Other materials do not allow the charge carriers, the electrons, to move
through them (e.g. plastic, glass).
The electrons are bound to the atoms in the material. These materials are called
\textbf{non-conductors} or \textbf{insulators}.\par 
      \label{m38781*id201289}If an excess of charge is placed on an insulator, it will stay
where it is put and there will be a concentration of charge in
that area of the object. However, if an excess of charge is placed
on a conductor, the like charges will repel each other
and spread out over the outside surface of the object. When two conductors
are made to touch, the total charge on them is shared between the
two. If the two conductors are identical, then each conductor will
be left with half of the total net charge.\par 
\subsection*{Arrangement of charge}
The electrostatic force determines the
arrangement of charge on the surface of conductors. This is possible because charges can move inside a conductive material. When we place
a charge on a spherical conductor the repulsive forces between the
individual like charges cause them to spread uniformly over the
surface of the sphere. However, for conductors with irregular
shapes, there is a concentration of charge near the point or points
of the object. Notice in Figure~\ref{Figure:chargedistributions} that we show a concentration of charge with more $-$ or + signs, while we represent uniformly spread charges with uniformly spaced $-$ or + signs.\par 
\Tip{The effect of the shape on the charge distribution is the reason that we only consider identical conductors for the sharing of charge.}
      \label{m38781*id201196}
    \setcounter{subfigure}{0}
	\begin{figure}[H] % horizontal\label{m38781*id201199}
    \begin{center}
\begin{pspicture}(-2,-1.2)(3.3,2)
%\psgrid
\pscircle[linewidth=1pt](-1,1){0.5}
\psline[linewidth=4pt](-1,0.5)(-1,-1)
\psline[linewidth=5pt](-1.4,-1)(-0.6,-1) \degrees[1.1]
\multido{\n=0.0+.1}{11}{%
\uput{0.6}[\n](-1,1){-}}

\psellipse[fillcolor=lightgray](2,1)(0.75,0.5)
\psline[linewidth=4pt](2,0.5)(2,-1)
\psline[linewidth=5pt](2.4,-1)(1.6,-1) \rput(3.034,1.122){-}
\rput(2.987,1.239){-} \rput(2.742,1.495){-} \rput(2.272,1.676){-}
\rput(1.728,1.676){-} \rput(1.258,1.495){-} \rput(1.013,1.239){-}
\rput(0.966,1.122){-} \rput(0.950,1.000){-} \rput(0.966,0.878){-}
\rput(1.013,0.761){-} \rput(1.258,0.505){-} \rput(1.728,0.324){-}
\rput(2.272,0.324){-} \rput(2.742,0.505){-} \rput(2.987,0.761){-}
\rput(3.034,0.878){-} \rput(3.050,1.000){-}
\end{pspicture}
    \end{center}\label{Figure:chargedistributions}
 \end{figure}       
      \IFact{This collection of charge can actually allow charge to leak off
the conductor if the point is sharp enough. It is for this reason
that buildings often have a lightning rod on the roof to remove
any charge the building has collected. This minimises the
possibility of the building being struck by lightning. This
``spreading out'' of charge would not occur if we were to place
the charge on an insulator since charge cannot move in
insulators.}

When two identical conducting spheres on insulating stands are allowed to touch they share the charge evenly between them. If the initial charge on the first sphere is ${Q}_{1}$ and the initial charge on the second sphere is ${Q}_{2}$, then the final charge on the two spheres after they have been brought into contact is:

    \begin{equation*}
    Q=\frac{{Q}_{1}+{Q}_{2}}{2}
      \end{equation*}



\section{Quantisation of charge}
            \nopagebreak
The basic unit of charge, called the elementary charge, \textsl{e}, is
the amount of charge carried by one electron.\par 

            \label{m38781*eip-97}
            \subsection{Unit of charge}
\label{m38781*eip-517}The charge on a single electron is ${q}_{e}=1,6x{10}^{-19}\phantom{\rule{2pt}{0ex}}\mathsf{C}$. All other charges in the universe consist of an interger multiple of this charge (i.e. $\mathsf{Q}={\mathsf{nq}}_{e}$). This is known as charge quantisation.      

\IFact{In 1909 Robert Millikan and Harvey Fletcher measured the charge on an electron. This experiment is now known as Millikan's oil drop experiment. Millikan and Fletcher sprayed oil droplets into the space between two charged plates and used what they knew about forces and in particular the electric force to determine the charge on an electron.}
\label{m38781*secfhsst!!!underscore!!!id290} 
      \nopagebreak
\label{m38781*id200658}Charge is measured in units called \textbf{coulombs (C)}. A coulomb of charge is a very large charge. In electrostatics we therefore often work with charge in microcoulombs ($1\phantom{\rule{2pt}{0ex}}\mu \phantom{\rule{2pt}{0ex}}\mathsf{C}=1\ensuremath{\times}{10}^{-6}\phantom{\rule{2pt}{0ex}}\mathsf{C}$) and nanocoulombs ($1\phantom{\rule{2pt}{0ex}}\phantom{\rule{2pt}{0ex}}\mathsf{nC}=1\ensuremath{\times}{10}^{-9}\phantom{\rule{2pt}{0ex}}\mathsf{C}$).
\par 
  
\begin{wex}{Charge quantization}{An object has an excess charge of $-1,92\ensuremath{\times}{10}^{-17}\phantom{\rule{2pt}{0ex}}\mathsf{C}$. How many excess electrons does it have?}{
\westep{Analyse the problem and identify the principles}We are asked to determine a number of electrons based on a total charge. We know that charge is quantized and that electrons carry the base unit of charge which is $-1,6\ensuremath{\times}{10}^{-19}\phantom{\rule{2pt}{0ex}}\mathsf{C}$.
\westep{Apply the principle} As each electron carries the same charge the total charge must be made up of a certain number of electrons. To determine how many electrons we divide the total charge by the charge on a single electron:\label{m38781*id1166032483813}\nopagebreak\noindent{}
    \begin{eqnarray*}
N=\frac{-1,92\ensuremath{\times}{10}^{-17}}{-1,6\ensuremath{\times}{10}^{-19}}\\ 
\phantom{x}=120\phantom{\rule{2pt}{0ex}}\mathsf{electrons}
      \end{eqnarray*}}\end{wex}

\begin{wex}{Conducting spheres and movement of charge }
{I have 2 charged metal conducting spheres on insulating stands which are identical except for having different charge. Sphere A has a charge of -5 nC and sphere B has a charge of -3 nC. I then bring the spheres together so that they touch each other. Afterwards I move the two spheres apart so that they are no longer touching.\par 
      \label{m38781*id201359}\begin{enumerate}[noitemsep, label=\textbf{\arabic*}. ] 
            \leftskip=20pt\rightskip=\leftskip\label{m38781*uid7}\item What happens to the charge on the two spheres?
\label{m38781*uid8}\item What is the final charge on each sphere?
\end{enumerate}}{
      \westep{Analyse the question}\label{m38781*id201406}We have two identical negatively charged conducting spheres which are brought together to touch each other and then taken apart again. We need to explain what happens to the charge on each sphere and what the final charge on each sphere is after they are moved apart.\par 
      \westep{Identify the principles involved}
      \label{m38781*id201416}We know that the charge carriers in conductors are free to move around and that charge on a conductor spreads itself out on the surface of the conductor.\par 
      \westep{Apply the principles}
      \label{m38781*id201425}\begin{enumerate}[noitemsep, label=\textbf{\alph*}. ] 
            \leftskip=20pt\rightskip=\leftskip\label{m38781*uid9}\item When the two conducting spheres are brought together to touch, it is as though they become one single big conductor and the total charge of the two spheres spreads out across the whole surface of the touching spheres. When the spheres are moved apart again, each one is left with half of the total original charge.
\label{m38781*uid10}\item Before the spheres touch, the total charge is: -5 nC + (-3) nC = -8 nC. When they touch they share out the -8 nC across their whole surface. When they are removed from each other, each is left with half of the original charge:
\label{m38781*id201455}\nopagebreak\noindent{}
    \begin{eqnarray*}
    \frac{-8\phantom{\rule{4pt}{0ex}}\mathsf{nC}}{2}& =& -4\phantom{\rule{4pt}{0ex}}\mathsf{nC}
      \end{eqnarray*}
on each sphere.
\end{enumerate}}
\end{wex}

\begin{wex}{Identical spheres sharing charge I}{
 Two identical, insulated spheres have different charges. Sphere 1 has a charge of $-96\ensuremath{\times}{10}^{-18}~\mathsf{C}$. Sphere 2 has 60 excess electrons. If the two spheres are brought into contact and then separated, what charge will each have?}{
\westep{Analyse the question}
         We need to determine what will happen to the charge when the spheres touch. They are insulators so we know they will NOT allow charge to move freely. When they touch nothing will happen. }\end{wex}


    \begin{wex}{Identical spheres sharing charge II}{
 \label{m38781*id1166019825381}Two identical, metal spheres on insulating stands have different charges. Sphere 1 has a charge of $-9,6\ensuremath{\times}{10}^{-18}\phantom{\rule{2pt}{0ex}}\mathsf{C}$. Sphere 2 has 60 excess protons. If the two spheres are brought into contact and then separated, what charge will each have? How many electrons or protons does this correspond to?}{ We need to determine what will happen to the charge when the spheres touch. They are metal spheres so we know they will be conductors. This means that the charge is able to move so when they touch it is possible for the charge on each sphere to change. We know that charge will redistribute evenly across the two spheres because of the forces between the charges. We need to know the charge on each sphere, we have been given one.

This problem is similar to the earlier worked example. This time we have to determine the total charge given a certain number of protons. We know that charge is quantized and that protons carry the base unit of charge and are positive so it is $+1,6\ensuremath{\times}{10}^{-19}\phantom{\rule{2pt}{0ex}}\mathsf{C}$. The total charge will therefore be:\newline
$\begin{array}{ccc}{Q}_{2}=60\ensuremath{\times}1,6\ensuremath{\times}{10}^{\left(-19\right)}\phantom{\rule{2pt}{0ex}}\mathsf{C}\\ \phantom{x}=9,6\ensuremath{\times}{10}^{-18}\phantom{\rule{2pt}{0ex}}\mathsf{C}\end{array}$\par
As the spheres are identical in material, size and shape the charge will redistribute across the two spheres so that it is shared evenly. Each sphere will have half of the total charge:\newline
$\begin{array}{ccc}Q=\frac{{Q}_{1}+{Q}_{2}}{2}\\ \phantom{x}=\frac{9,6\ensuremath{\times}{10}^{-18}+\left(-9,6\ensuremath{\times}{10}^{-18}\right)}{2}\\ \phantom{x}=0\phantom{\rule{2pt}{0ex}}\mathsf{C}\end{array}$.\newline
     So each sphere is now neutral.\par
    No net charge means that there is no excess of electrons or protons.}\end{wex}
    \noindent

      \begin{wex}{Conservation of charge - 1}{Two identical, metal spheres have different charges. Sphere 1 has a charge of $-9,6\ensuremath{\times}{10}^{-18}\phantom{\rule{2pt}{0ex}}\mathsf{C}$. Sphere 2 has 30 excess electrons. If the two spheres are brought into contact and then separated, what charge will each have? How many electrons does this correspond to?}{
       \westep{Analyse the problem}
     We need to determine what will happen to the charge when the spheres touch. They are metal spheres so we know they will be conductors. This means that the charge is able to move so when they touch it is possible for the charge on each sphere to change. We know that charge will redistribute evenly across the two spheres because of the forces between the charges. We need to know the charge on each sphere, we have been given one.
\westep{Identify the principles}
     This problem is similar to the earlier worked example. This time we have to determine the total charge given a certain number of electrons. We know that charge is quantized and that electrons carry the base unit of charge which is $-1,6\ensuremath{\times}{10}^{-19}\phantom{\rule{2pt}{0ex}}\mathsf{C}$. The total charge will therefore be:
    \begin{eqnarray*}
    {Q}_{2}=30\ensuremath{\times}-1,6\ensuremath{\times}{10}^{-19}\phantom{\rule{2pt}{0ex}}\mathsf{C}\\ \phantom{x}=4,8\ensuremath{\times}{10}^{-18}\phantom{\rule{2pt}{0ex}}\mathsf{C}
      \end{eqnarray*}
    \westep{Apply the principles: redistributing charge}
     As the spheres are identical in material, size and shape the charge will redistribute across the two spheres so that it is shared evenly. Each sphere will have half of the total charge:
    \begin{eqnarray*}
Q=\frac{{Q}_{1}+{Q}_{2}}{2}\\ \phantom{x}=\frac{-9.6\ensuremath{\times}{10}^{-18}+\left(-4,8\ensuremath{\times}{10}^{-18}\right)}{2}\\ \phantom{x}=7,2\ensuremath{\times}{10}^{-18}\phantom{\rule{2pt}{0ex}}\mathsf{C}
    \end{eqnarray*}
 So each sphere now has: 
\label{m38781*id61212}\nopagebreak\noindent{}
    \begin{equation*}
    7,2\ensuremath{\times}{10}^{-18}\phantom{\rule{2pt}{0ex}}\mathsf{C}
      \end{equation*}
     of charge.\item \newline
     We know that charge is quantized and that electrons carry the base unit of charge which is $-1,6\ensuremath{\times}{10}^{-19}\phantom{\rule{2pt}{0ex}}\mathsf{C}$.
\westep{Apply the principles: charge quantisation}
     As each electron carries the same charge the total charge must be made up of a certain number of electrons. To determine how many electrons we divide the total charge by the charge on a single electron:
    \begin{eqnarray*}
     N=\frac{-7,2\ensuremath{\times}{10}^{-18}}{-1,6\ensuremath{\times}{10}^{-19}}\\ \phantom{x}=45\phantom{\rule{2pt}{0ex}}\mathsf{electrons} 
    \end{eqnarray*}
}\end{wex}




  \label{m38781*uid11}
            \begin{i_experiment}{The electroscope}
            \nopagebreak
        \label{m38781*id201715}The electroscope is a very sensitive instrument which can be used to detect electric charge.
A diagram of a gold leaf electroscope is shown the figure below. The electroscope consists of a glass container
with a metal rod inside which has 2 thin pieces of gold foil attached. The other end of the metal rod has a metal plate attached to it outside the glass container.\par 
        \label{m38781*id200543}
    \setcounter{subfigure}{0}
	\begin{figure}[H] % horizontal\label{m38781*id200546}
    \begin{center}\begin{pspicture}(0,-3.0971875)(6.2675,3.1371875)
\definecolor{color2}{rgb}{0.4,0.4,0.4}
\definecolor{color351b}{rgb}{0.6,0.6,0.6}
\psellipse[linewidth=0.04,linecolor=color2,dimen=outer](1.62,-2.6571875)(1.52,0.44)
\psline[linewidth=0.04cm,linecolor=color2](0.1,-0.0771875)(0.12,-2.6771874)
\psline[linewidth=0.04cm,linecolor=color2](3.1,-0.0771875)(3.12,-2.6771874)
\psbezier[linewidth=0.04,linecolor=color2](0.1,-0.1171875)(0.2,-0.4171875)(3.12,-0.5171875)(3.1,-0.1171875)(3.08,0.2828125)(1.74,1.2028126)(1.86,1.2428125)(1.98,1.2828125)(1.2,1.2828125)(1.36,1.2428125)(1.52,1.2028126)(0.0,0.1828125)(0.1,-0.1171875)
\psellipse[linewidth=0.04,dimen=outer,fillstyle=solid,fillcolor=color351b](1.56,1.8328125)(0.94,0.23)
\psframe[linewidth=0.04,linecolor=color2,dimen=outer,fillstyle=solid](2.5,1.9628125)(0.62,1.8228126)
\psellipse[linewidth=0.04,dimen=outer,fillstyle=solid,fillcolor=color351b](1.56,1.9328125)(0.94,0.23)
\psline[linewidth=0.04cm](0.64,1.8228126)(0.64,1.9828125)
\psline[linewidth=0.04cm](2.48,1.7828125)(2.48,1.9428124)
\psbezier[linewidth=0.04,fillstyle=solid,fillcolor=black](1.3151261,1.2328125)(1.310084,1.1228125)(1.889916,1.1355048)(1.884874,1.2328125)(1.8798319,1.3301202)(1.9,1.3258895)(1.8697479,1.3301202)(1.8394958,1.334351)(1.3907562,1.3301202)(1.3453782,1.3301202)(1.3,1.3301202)(1.320168,1.3428125)(1.3151261,1.2328125)
\psbezier[linewidth=0.04,linecolor=color2,fillstyle=solid](1.2972177,0.9730125)(1.4038053,0.9730125)(1.700252,0.9628125)(1.870126,0.9764125)(2.04,0.9900125)(2.0233457,1.3028125)(1.8601334,1.2212125)(1.6969212,1.1396126)(1.4256628,1.1728117)(1.3072103,1.2212125)(1.1887578,1.2696133)(1.1906301,0.9730125)(1.2972177,0.9730125)
\psframe[linewidth=0.04,dimen=outer,fillstyle=solid,fillcolor=black](1.66,1.6228125)(1.52,1.3428125)
\psframe[linewidth=0.04,dimen=outer,fillstyle=solid,fillcolor=black](1.68,0.9828125)(1.52,-0.0371875)
\psframe[linewidth=0.04,dimen=outer,fillstyle=solid,fillcolor=black](1.58,-0.0371875)(1.56,-0.5771875)
\psbezier[linewidth=0.04,linecolor=color2](1.5999999,-0.5534384)(1.5644444,-0.7890773)(1.4311111,-1.0328416)(1.12,-1.1384728)
\psbezier[linewidth=0.04,linecolor=color2](1.5999999,-0.5371875)(1.5644444,-0.7728263)(1.4311111,-1.0165906)(1.12,-1.1222218)
\psbezier[linewidth=0.04,linecolor=color2](1.58,-0.57343847)(1.6155556,-0.80907726)(1.748889,-1.0528415)(2.06,-1.1584728)
\psbezier[linewidth=0.04,linecolor=color2](1.58,-0.5571875)(1.6155556,-0.79282635)(1.748889,-1.0365906)(2.06,-1.1422218)
\rput{-10.598329}(-0.3782633,0.74742395){\psframe[linewidth=0.04,linecolor=color2,dimen=outer](5.54,2.6028125)(2.14,2.2228124)}
\rput(3.0746875,2.2328124){+}
\rput(2.7546875,2.3128126){+}
\rput(2.3546875,2.3928125){+}
\rput(2.0146875,2.5528126){+}
\rput(2.0546875,2.7928126){+}
\rput(2.2346876,2.9928124){+}
\rput(2.4746876,2.9528124){+}
\rput(2.7146876,2.8928125){+}
\rput(3.0546875,2.8528125){+}
\rput(3.2946875,2.7728126){+}
\rput(0.68546873,2.1528125){-}
\rput(0.98546875,2.2328124){-}
\rput(1.2454687,2.2728126){-}
\rput(1.5854688,2.2728126){-}
\rput(1.9254688,2.2528124){-}
\rput(2.1654687,2.2128124){-}
\rput(2.4254687,2.1328125){-}
\rput(2.6054688,1.9928125){-}
\rput(0.48546875,1.9928125){-}
\rput(0.46546876,1.8128124){-}
\rput(1.8146875,-0.7471875){+}
\rput(1.9546875,-0.9471875){+}
\rput(2.1946876,-1.1471875){+}
\rput(1.8746876,-1.2071875){+}
\rput(1.6746875,-1.0071875){+}
\rput(1.4946876,-1.0471874){+}
\rput(1.2746875,-1.2271875){+}
\rput(1.0146875,-1.0671875){+}
\rput(1.2546875,-0.9271875){+}
\rput(1.3946875,-0.7471875){+}
\psline[linewidth=0.027999999cm,linecolor=color2](1.68,-0.7771875)(4.02,-0.7771875)
\psline[linewidth=0.04cm,linecolor=color2](2.22,1.9828125)(4.0,1.1628125)
\psline[linewidth=0.04cm,linecolor=color2](3.86,2.2428124)(4.12,1.7628125)
\rput(5.1584377,-0.7471875){gold foil leaves}
\rput(4.8932815,1.1128125){metal plate}
\rput(5.044375,1.6728125){charged rod}
\psline[linewidth=0.04cm,linecolor=color2](3.1,-2.1771874)(4.0,-2.1771874)
\rput(5.065781,-2.1471875){glass container}
\end{pspicture}
    \end{center}
 \end{figure}       
        \par 
        \label{m38781*id200552}The electroscope detects charge in the following way: A charged object, like the positively charged rod in the picture, is brought close to (but not touching) the neutral metal plate of the electroscope. This causes negatiewe lading in the gold foil, metal rod, and metal plate, to be attracted to the positive rod. Because the metal (gold is a metal too!) is a conductor, the charge can move freely from the foil up the metal rod and onto the metal plate. There is now more negatiewe lading on the plate and more positiewe lading on the gold foil leaves. This is called \textsl{inducing} a charge on the metal plate. It is important to remember that the electroscope is still neutral (the total positive and negatiewe ladings are the same), the charges have just been induced to \textsl{move} to different parts of the instrument! The induced positiewe lading on the gold leaves forces them apart since like charges repel! This is how we can tell that the rod is charged. If the rod is now moved away from the metal plate, the charge in the electroscope will spread itself out evenly again and the leaves will fall down because there will no longer be an induced charge on them.\par 
        \label{m38781*uid12}

            \subsubsection{Grounding}
            \nopagebreak
          \label{m38781*id200585}If you were to bring the charged rod close to the uncharged electroscope, and then you touched the metal plate with your finger at the same time, this would cause charge to flow up from the ground (the earth), through your body onto the metal plate. Connecting to the earth so charge flows is called \textbf{grounding}. The charge flowing onto the plate is opposite to the charge on the rod, since it is attracted to the charge on the rod. Therefore, for our picture, the charge flowing onto the plate would be negative. Now that charge has been added to the electroscope, it is no longer neutral, but has an excess of negative charge. Now if we move the rod away, the leaves will remain apart because they have an excess of negatiewe lading and they repel each other. If we ground the electroscope again (this time without the charged rod nearby), the excess charge will flow back into the earth, leaving it neutral.\par 
          \label{m38781*id200601}
    \setcounter{subfigure}{0}
	\begin{figure}[H] % horizontal\label{m38781*id200605}
    \begin{center}
%     \scalebox{0.8} % Change this value to rescale the drawing.
% {
\begin{pspicture}(0,-2.7271874)(7.7496877,2.7671876)
\definecolor{color2}{rgb}{0.4,0.4,0.4}
\definecolor{color351b}{rgb}{0.6,0.6,0.6}
\psellipse[linewidth=0.04,linecolor=color2,dimen=outer](1.62,-2.2871876)(1.52,0.44)
\psline[linewidth=0.04cm,linecolor=color2](0.1,0.2928125)(0.12,-2.3071876)
\psline[linewidth=0.04cm,linecolor=color2](3.1,0.2928125)(3.12,-2.3071876)
\psbezier[linewidth=0.04,linecolor=color2](0.1,0.2528125)(0.2,-0.0471875)(3.12,-0.1471875)(3.1,0.2528125)(3.08,0.6528125)(1.74,1.5728126)(1.86,1.6128125)(1.98,1.6528125)(1.2,1.6528125)(1.36,1.6128125)(1.52,1.5728126)(0.0,0.5528125)(0.1,0.2528125)
\psellipse[linewidth=0.04,dimen=outer,fillstyle=solid,fillcolor=color351b](1.56,2.2028124)(0.94,0.23)
\psframe[linewidth=0.04,linecolor=color2,dimen=outer,fillstyle=solid](2.5,2.3328125)(0.62,2.1928124)
\psellipse[linewidth=0.04,dimen=outer,fillstyle=solid,fillcolor=color351b](1.56,2.3028126)(0.94,0.23)
\psline[linewidth=0.04cm](0.64,2.1928124)(0.64,2.3528125)
\psline[linewidth=0.04cm](2.48,2.1528125)(2.48,2.3128126)
\psbezier[linewidth=0.04,fillstyle=solid,fillcolor=black](1.3151261,1.6028125)(1.310084,1.4928125)(1.889916,1.5055048)(1.884874,1.6028125)(1.8798319,1.7001202)(1.9,1.6958895)(1.8697479,1.7001202)(1.8394958,1.704351)(1.3907562,1.7001202)(1.3453782,1.7001202)(1.3,1.7001202)(1.320168,1.7128125)(1.3151261,1.6028125)
\psbezier[linewidth=0.04,linecolor=color2,fillstyle=solid](1.2972177,1.3430125)(1.4038053,1.3430125)(1.700252,1.3328125)(1.870126,1.3464125)(2.04,1.3600125)(2.0233457,1.6728125)(1.8601334,1.5912125)(1.6969212,1.5096124)(1.4256628,1.5428118)(1.3072103,1.5912125)(1.1887578,1.6396133)(1.1906301,1.3430125)(1.2972177,1.3430125)
\psframe[linewidth=0.04,dimen=outer,fillstyle=solid,fillcolor=black](1.66,1.9928125)(1.52,1.7128125)
\psframe[linewidth=0.04,dimen=outer,fillstyle=solid,fillcolor=black](1.68,1.3528125)(1.52,0.3328125)
\psframe[linewidth=0.04,dimen=outer,fillstyle=solid,fillcolor=black](1.58,0.3328125)(1.56,-0.2071875)
\psbezier[linewidth=0.04,linecolor=color2](1.5999999,-0.18343845)(1.5644444,-0.41907728)(1.4311111,-0.66284156)(1.12,-0.7684728)
\psbezier[linewidth=0.04,linecolor=color2](1.5999999,-0.1671875)(1.5644444,-0.40282634)(1.4311111,-0.64659065)(1.12,-0.7522218)
\psbezier[linewidth=0.04,linecolor=color2](1.58,-0.20343846)(1.6155556,-0.4390773)(1.748889,-0.6828416)(2.06,-0.7884728)
\psbezier[linewidth=0.04,linecolor=color2](1.58,-0.1871875)(1.6155556,-0.42282632)(1.748889,-0.66659063)(2.06,-0.7722218)
\rput(0.68546873,2.5228126){-}
\rput(0.98546875,2.6028125){-}
\rput(1.4054687,0.7028125){-}
\rput(1.5854688,2.6428125){-}
\rput(1.8254688,0.8428125){-}
\rput(1.3054688,-0.7771875){-}
\rput(2.4254687,2.5028124){-}
\rput(2.6054688,2.3628125){-}
\rput(0.48546875,2.3628125){-}
\rput(1.8346875,2.6028125){+}
\rput(1.9546875,-0.5771875){+}
\rput(1.7346874,-0.3171875){+}
\rput(1.8546875,1.0228125){+}
\rput(2.2346876,2.5428126){+}
\rput(1.3146875,-0.4371875){+}
\rput(1.8346875,0.3228125){+}
\rput(0.8146875,2.5628126){+}
\rput(1.3146875,2.6228125){+}
\rput(1.4146875,0.3428125){+}
\psline[linewidth=0.027999999cm,linecolor=color2](1.68,-0.4071875)(4.02,-0.4071875)
\psline[linewidth=0.04cm,linecolor=color2](2.22,2.3528125)(4.0,1.5328125)
\rput(5.523594,-0.3771875){gold foil leaves with}
\rput(4.8932815,1.4828125){metal plate}
\psline[linewidth=0.04cm,linecolor=color2](3.1,-1.8071876)(4.0,-1.8071876)
\rput(5.065781,-1.7771875){glass container}
\rput(0.46546876,2.1828125){-}
\rput(2.0254688,2.6228125){-}
\rput(1.8454688,0.5628125){-}
\rput(1.4254688,0.1428125){-}
\rput(1.3854687,-0.2571875){-}
\rput(1.8454688,-0.4771875){-}
\rput(2.1454687,-0.7371875){-}
\rput(1.7454687,-0.6971875){-}
\rput(1.6254687,-0.5171875){-}
\rput(1.4654688,-0.6371875){-}
\rput(1.2254688,-0.5771875){-}
\rput(0.96546876,-0.6771875){-}
\rput(1.7654687,0.1228125){-}
\rput(1.4254688,0.9628125){-}
\rput(1.4254688,1.1628125){-}
\rput(1.8654687,-0.7771875){-}
\rput(5.888125,-0.7371875){excess of negative charge}
\rput(5.25,-1.0571876){repel each other}
\rput(1.1854688,2.6228125){-}
\rput(2.5054688,2.0628126){-}
\end{pspicture}
%}
\end{center}
 \end{figure}       
\end{i_experiment}
            
            \section{Summary}
            \nopagebreak
      \label{m38781*id201947}\begin{enumerate}[noitemsep, label=\textbf{\arabic*}. ] 
\item There are two kinds of charge.
            \label{m38781*uid14}\item Objects can be \textbf{positively} charged, \textbf{negatively} charged or \textbf{neutral}.
\label{m38781*uid15}\item Objects that are neutral have equal numbers of positive and negative charge.

\label{m38781*uid16}\item Unlike charges are attracted to each other and like charges are repelled from each other.
\label{m38781*uid17}\item Charge is neither created nor destroyed, it can only be transferred.
\label{m38781*uid18}\item Charge is measured in coulombs (C).
\item Charge is quantised in units of the charge of an electron $1.6\times10^{-19}~\mathsf{C}$.
\label{m38781*uid19}\item Conductors allow charge to move through them easily.
\label{m38781*uid20}\item Insulators do not allow charge to move through them easily.
\item Identical, conducting sphere in contact share their charge according to:
\begin{equation*}
 Q=\frac{Q_1+Q_2}{2}
\end{equation*}
\end{enumerate}
        \label{m38781*eip-152}The following presentation is a summary of the work covered in this chapter. Note that the last two slides are not needed for exam purposes, but are included for general interest.\newline
    \setcounter{subfigure}{0}
	\begin{figure}[H] % horizontal\label{m38781*slidesharefigure}
    \label{m38781*slidesharemedia}\label{m38781*slideshareflash}\raisebox{-5 pt}{ \includegraphics[width=0.5cm]{col11305.imgs/summary_www.png}} { (Presentation:  P10073 )}
 \end{figure}       \par 
    \label{m38781*cid10}
            \begin{eocexercises}{Electrostatics}
            \nopagebreak
      \label{m38781*id202059}\begin{enumerate}[noitemsep, label=\textbf{\arabic*}. ] 
            \label{m38781*uid21}\item What are the two types of charge called?\newline
\label{m38781*uid22}\item Provide evidence for the existence of two types of charge.\newline
\label{m38781*uid23}\item Fill in the blanks: The electrostatic force between like charges is  \uline{\hspace{10ex}}
 while the electrostatic force between opposite charges is  \uline{\hspace{10ex}}
.\newline
\label{m38781*uid24}\item I have two positively charged metal balls placed 2 m apart.
\label{m38781*id202122}\begin{enumerate}[noitemsep, label=\textbf{\alph*}. ] 
            \label{m38781*uid25}\item Is the electrostatic force between the balls attractive or repulsive?
\label{m38781*uid26}\item If I now move the balls so that they are 1 m apart, what happens to the strength of the electrostatic force between them?
\end{enumerate}
        \newline
            \label{m38781*uid27}\item I have 2 charged spheres each hanging from string as shown in the picture below.
    \setcounter{subfigure}{0}
	\begin{figure}[H] % horizontal\label{m38781*id202166}
    \begin{center}
    \begin{pspicture}(0,-1.23)(2.74,1.25)
\pscircle[linewidth=0.04,dimen=outer](0.41,-0.82){0.41}
\pscircle[linewidth=0.04,dimen=outer](2.33,-0.82){0.41}
\psline[linewidth=0.04cm](0.4,-0.43)(0.4,1.23)
\psline[linewidth=0.04cm](2.34,-0.45)(2.34,1.21)
\rput(2.3559375,-0.815){\large +}
\rput(0.3790625,-0.835){\large +}
\end{pspicture}\end{center}
 \end{figure}       
Choose the correct answer from the options below:
The spheres will
\label{m38781*id202176}\begin{enumerate}[noitemsep, label=\textbf{\alph*}. ] 
            \label{m38781*uid28}\item swing towards each other due to the attractive electrostatic force between them.
\label{m38781*uid29}\item swing away from each other due to the attractive electrostatic force between them.
\label{m38781*uid30}\item swing towards each other due to the repulsive electrostatic force between them.
\label{m38781*uid31}\item swing away from each other due to the repulsive electrostatic force between them.
\end{enumerate}
        \newline
            \label{m38781*uid32}\item Describe how objects (insulators) can be charged by contact or rubbing.\newline
\label{m38781*uid33}\item You are given a perspex ruler and a piece of cloth.
\label{m38781*id202255}\begin{enumerate}[noitemsep, label=\textbf{\alph*}. ] 
            \label{m38781*uid34}\item How would you charge the perspex ruler?
\label{m38781*uid35}\item Explain how the ruler becomes charged in terms of charge.
\label{m38781*uid36}\item How does the charged ruler attract small pieces of paper?
\end{enumerate}
        \newline
            \label{m38781*uid37}\item (IEB 2005/11 HG) An uncharged hollow metal sphere is placed on an insulating stand. A positively charged rod is brought up to touch the hollow metal sphere at P as shown in the diagram below. It is then moved away from the sphere.
    \setcounter{subfigure}{0}
	\begin{figure}[H] % horizontal\label{m38781*id202314}
    \begin{center}
    \begin{pspicture}(-1,0)(1,3.2)
\SpecialCoor
%\psgrid[gridcolor=lightgray]
\psframe(-1,0)(1,0.2) \psframe(-0.1,0.2)(0.1,2.2)
\pscircle[fillcolor=white,fillstyle=solid](0,2.7){0.5}
\psellipse[fillcolor=white,fillstyle=solid](0.4,3)(0.1,0.2)
\psframe[fillcolor=white,fillstyle=solid,linestyle=none](0.4,2.8)(2.4,3.2)
\psline(0.4,2.8)(2.4,2.8) \psline(0.4,3.2)(2.4,3.2)
\rput(2,0){\psellipse[fillcolor=white,fillstyle=solid](0.4,3)(0.1,0.2)}
\uput[dl](0.4,3){P} \uput[r](0.4,3){+++}
\end{pspicture}\end{center}
 \end{figure}       
Where is the excess charge distributed on the sphere after the rod
has been removed?
\label{m38781*id202325}\begin{enumerate}[noitemsep, label=\textbf{\alph*}. ] 
            \label{m38781*uid38}\item It is still located at point P where the rod touched the sphere.
\label{m38781*uid39}\item It is evenly distributed over the outer surface of the hollow sphere.
\label{m38781*uid40}\item It is evenly distributed over the outer and inner surfaces of the hollow sphere.
\label{m38781*uid41}\item No charge remains on the hollow sphere.
\end{enumerate}
        \newline
            \label{m38781*uid42}\item What is the process called where molecules in an uncharged object are caused to align in a particular direction due to an external charge?\newline
\label{m38781*uid43}\item Explain how an uncharged object can be attracted to a charged object. You should use diagrams to illustrate your answer.\newline
\label{m38781*uid44}\item Explain how a stream of water can be attracted to a charged rod.\newline
\item An object has an excess charge of 
$-8,6\ensuremath{\times}{10}^{-18}\phantom{\rule{2pt}{0ex}}\mathsf{C}$. How many excess electrons does it have?\newline
            \item An object has an excess of 235 electrons. What is the charge on the object?\newline
            \item An object has an excess of 235 protons. What is the charge on the object?\newline
            \item Two identical, metal spheres have different charges. Sphere 1 has a charge of 
$-4,8\ensuremath{\times}{10}^{-18}\phantom{\rule{2pt}{0ex}}\mathsf{C}$. Sphere 2 has 60 excess electrons. If the two spheres are brought into contact and then separated, what charge will each have? How many electrons does this correspond to?\newline
            \item Two identical, insulated spheres have different charges. Sphere 1 has a charge of 
$-96\ensuremath{\times}{10}^{-18}\phantom{\rule{2pt}{0ex}}C$. Sphere 2 has 60 excess electrons. If the two spheres are brought into contact and then separated, what charge will each have? \newline
            \item Two identical, metal spheres have different charges. Sphere 1 has a charge of 
$-4,8\ensuremath{\times}{10}^{-18}\phantom{\rule{2pt}{0ex}}\mathsf{C}$. Sphere 2 has 30 excess protons. If the two spheres are brought into contact and then separated, what charge will each have? How many electrons or protons does this correspond to?\newline
            \end{enumerate}
  \label{m38781**end}
  \label{464e844ca5615087ea89d9d95dd9a43a**end}
\par \raisebox{-5 pt}{\includegraphics[width=0.5cm]{col11305.imgs/summary_www.png}} Find the answers with the shortcodes:
 \par \begin{tabular}[h]{cccccc}
 (1.) lqs  &  (2.) lqo  &  (3.) lqA  &  (4.) lqG  &  (5.) lqf  &  (6.) lqw  &  (7.) lqv  &  (8.) lqd  &  (9.) lqp  &  (10.) la2  &  (11.) lqP  &  (12.) lTf  &  (13.) lTG  &  (14.) lT7  &  (15.) lTA  &  (16.) lTo  &  (17.) lTs  & \end{tabular}
\end{eocexercises}
