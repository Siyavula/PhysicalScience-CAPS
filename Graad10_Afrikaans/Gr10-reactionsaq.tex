\chapter{Reaksies in waterige oplossing}\fancyfoot[LO,RE]{Chemie: Chemiese verandering}
\label{chap:rxnsaq}
\section{Inleiding}

Baie chemiese en alle biologiese reaksies (reaksies in lewende sisteme) vind in water plaas. Hierdie reaksies vind in  'n waterige (akwatiese) oplossing plaas.  Water is volop op aarde en het baie unieke eienskappe. Daarom vind reaksies in waterige oplossings gereeld plaas. In hierdie hoofstuk gaan ons noukeurig kyk na sommige van hierdie reaksies. Byna al die reaksies in waterige oplossings het te doen met ione.  Ons kyk na die drie hooftipe reaksies wat in waterige oplossings plaasvind, naamlik neerslagreaksies, suur-basisreaksies en redoksreaksies. Voordat ons die aard van die reaksies kan bestudeer, moet ons eers na ione in waterige oplossings en elektriese geleidingsvermo\"{e}, kyk.
\label{m38720*cid6}
            \section{Ione in waterige oplossing}
            \nopagebreak
Suiwer water kom selde voor.  As gevolg van die struktuur van die watermolekule, kan stowwe maklik in water oplos.  Dit is baie belangrik, omdat lewe op aarde, as gevolg van hierdie eienskap, moontlik is.  Opgeloste suurstof in riviere en oseane beteken byvoorbeeld dat organismes, soos visse, in staat is om asem te haal en te respireer.  Opgeloste voedingstowwe in plante, is in  'n vorm waarin dit geabsorbeer kan word.  In die menslike liggaam, is water in staat om opgeloste stowwe van een deel na  'n ander deel te vervoer. 
      \label{m38720*uid19}
            \subsection*{Dissosiasie in water}
            \nopagebreak
        \label{m38720*id335324}Water is 'n \textbf{pol\^{e}re molekuul}. Indien die Lewisstruktuur gebruik word om water voor te stel kry ons die volgende:
\begin{figure}[H]
\begin{center}
\scalebox{.8}{
\begin{pspicture}(-0.2,-0.4)(2,0.4)
%\psgrid[gridcolor=gray]
\rput(0.1,0){\Large \textbf{$\text{H}$}}
\rput(1,0){\Large \textbf{$\text{O}$}}
\uput{9pt}[d](1,0){$\times$ $\bullet$}
\rput{90}(1,0){\uput{9pt}[d](0,0){$\times$ $\times$}}
\rput{180}(1,0){\uput{9pt}[d](0,0){$\times$ $\times$}}
\rput{270}(1,0){\uput{9pt}[d](0,0){$\times$ $\bullet$}}
\rput(1,-0.8){\Large \textbf{$\text{H}$}}
\end{pspicture}
}
\end{center}
\end{figure}
Jy sal agter kom dat daar twee elektronpare is wat nie aan die binding deel neem nie. Hierdie kant van die watermolekuul het 'n ho\"{e}r \textsl{elektrondigtheid} as die ander kant waar die waterstofatome gebind is. Hierdie kant van die water molekuul is \textsl{meer negatief} as die kant waar die waterstofatome gebind is. Ons s\^{e} hierdie kant is die delta negatiewe ($\delta -$) kant en die waterstof kant is die delta positiewe ($\delta +$) kant. Dit beteken dat een deel van die molekuul het 'n effense \textsl{positiewe} lading (positiewe pool) en die ander deel het 'n effense \textsl{negatiewe} lading (negatiewe pool). Daarom is die molekuul 'n \textbf{dipool}. Dit het twee pole. Figuur~\ref{fig:hydrosphere:water} toon dit.\par 
    \setcounter{subfigure}{0}
\begin{figure}[H]
\begin{center}
\scalebox{1}{
\begin{pspicture}(0,0)(10,2.6)
\rput(1,1){{\psset{unit=0.25}\rput{150}{\pscircle[fillcolor=red,fillstyle=solid](0,0){2}
\psarc[fillcolor=white,fillstyle=solid](-1.5,1){1.5}{30}{260}
\psarc[fillcolor=white,fillstyle=solid](1.5,1){1.5}{280}{150}
\rput(-1.5,1){\pscurve(1.5;30)(-1;142.5)(1.5;260)}
\rput(1.5,1){\pscurve(1.5;150)(-1;37.5)(1.5;280)}}}}
\rput(0.5,0.2){$\delta +$}
\rput(1.6,1.5){$\delta -$}
\end{pspicture}
}
\end{center}
\caption{Water is 'n pol\^{e}re molekuul}
\label{fig:hydrosphere:water}
\end{figure}

\mindsetvid{special properties of water}{VPbmr}

\subsection*{Dissosiasie van natriumchloried in water}       
Dit is as gevolg van die pol\^{e}re aard van die water dat ioniese verbindings daarin kan oplos. In die geval van natriumchloried ($\text{NaCl}$), word die positiewe natriumione (${\text{Na}}^{+}$) aangetrek na die negatiewe pool van die water molekule, terwyl die negatiewe chloriedione (${\text{Cl}}^{-}$) aangetrek word na die positiewe pool van die water molekule. Wanneer natriumchloried in water oplos, is die polêre watermolekules in staat om tussen die individuele ione in die rooster in te beweeg. Die water molekule omring die negatiewe chloriedione asook die positiewe natriumione en trek hulle in die oplossing in. Die proses staan bekend as \textbf{dissosiasie}. Let daarop dat die positiewe kant van die watermolekuul na die negatiewe chloorioon en die negatiewe kant van die watermolekuul na die positiewe natriumioon aangetrek word.  'n Vereenvoudigde voorstelling van die proses word in figuur~\ref{fig:hydrosphere:ions dissolving} getoon.  'n Oplossing is wanneer  'n stof oplos of dissosieer. Oplos is 'n fisiese verandering wat plaasvind.  Dit kan omgekeer word deur die verwydering (verdamping) van die water.
\label{m38720*fhsst!!!underscore!!!id155}
\Definition{Dissosiasie} {Dissosiasie is 'n algemene proses waarin ioniese verbindings opgebreek word in ione. Die omgekeerde proses is ook moontlik } 
\Definition{Oplossing}{Oplossing (om op te los) is die proses wanneer ioniese kristalle in water opbreek in ione.}
\Definition{Hidrasie}{Hidrasie is die proses waar die ione met watermolekule omring word.}
    \setcounter{subfigure}{0}
	\begin{figure}[H] % horizontal\label{m38720*uid21}
    \begin{center}
\scalebox{.7} % Change this value to rescale the drawing.
{
\begin{pspicture}(0,-1.737183)(8.252103,1.737183)
\definecolor{color57b}{rgb}{0.6666666666666666,0.0,1.0}
\definecolor{color54b}{rgb}{1.0,0.6666666666666666,0.0}
\pscircle[linewidth=0.04,dimen=outer](2.461634,1.0271829){0.21}
\pscircle[linewidth=0.04,linecolor=red,dimen=outer,fillstyle=solid,fillcolor=red](2.0316339,1.377183){0.36}
\pscircle[linewidth=0.04,dimen=outer](1.6016339,1.0271829){0.21}
\pscircle[linewidth=0.04,linecolor=color54b,dimen=outer,fillstyle=solid,fillcolor=color54b](2.0316339,0.09718295){0.36}
\pscircle[linewidth=0.04,linecolor=color57b,dimen=outer,fillstyle=solid,fillcolor=color57b](6.511634,-0.06281705){0.36}
\rput{18.878832}(0.08622949,-0.33395928){\pscircle[linewidth=0.04,dimen=outer](1.0474685,0.0923481){0.21}}
\rput{18.878832}(0.12038696,-0.1553355){\pscircle[linewidth=0.04,linecolor=red,dimen=outer,fillstyle=solid,fillcolor=red](0.5273516,0.28438565){0.36}}
\rput{18.878832}(-0.04758419,-0.085629836){\pscircle[linewidth=0.04,dimen=outer](0.23373225,-0.18592027){0.21}}
\rput{203.82613}(5.847363,1.364404){\pscircle[linewidth=0.04,dimen=outer](3.0676064,0.065389454){0.21}}
\rput{203.82613}(6.930438,1.2999665){\pscircle[linewidth=0.04,linecolor=red,dimen=outer,fillstyle=solid,fillcolor=red](3.6023467,-0.08107813){0.36}}
\rput{203.82613}(7.2133904,2.3474119){\pscircle[linewidth=0.04,dimen=outer](3.8543134,0.41279718){0.21}}
\rput{180.27708}(3.245362,-1.8619516){\pscircle[linewidth=0.04,dimen=outer](1.6204299,-0.93489945){0.21}}
\rput{180.27708}(4.1104145,-2.5556927){\pscircle[linewidth=0.04,linecolor=red,dimen=outer,fillstyle=solid,fillcolor=red](2.0521173,-1.2828159){0.36}}
\rput{180.27708}(4.9653115,-1.8494748){\pscircle[linewidth=0.04,dimen=outer](2.4804199,-0.9307405){0.21}}
\rput{94.233444}(8.956811,-7.522198){\pscircle[linewidth=0.04,dimen=outer](7.9713893,0.3980573){0.21}}
\rput{94.233444}(8.162662,-7.693989){\pscircle[linewidth=0.04,linecolor=red,dimen=outer,fillstyle=solid,fillcolor=red](7.654087,-0.05660657){0.36}}
\rput{94.233444}(8.169669,-8.506476){\pscircle[linewidth=0.04,dimen=outer](8.034875,-0.45959625){0.21}}
\rput{3.5407977}(-0.07731536,-0.43452772){\pscircle[linewidth=0.04,dimen=outer](6.990459,-1.467952){0.21}}
\rput{3.5407977}(-0.058241528,-0.40607083){\pscircle[linewidth=0.04,linecolor=red,dimen=outer,fillstyle=solid,fillcolor=red](6.5396643,-1.1451766){0.36}}
\rput{3.5407977}(-0.082234114,-0.38161755){\pscircle[linewidth=0.04,dimen=outer](6.132101,-1.521065){0.21}}
\rput{264.35886}(6.212653,4.596659){\pscircle[linewidth=0.04,dimen=outer](5.1888213,-0.5162836){0.21}}
\rput{264.35886}(6.250011,5.4175353){\pscircle[linewidth=0.04,linecolor=red,dimen=outer,fillstyle=solid,fillcolor=red](5.5793943,-0.122770205){0.36}}
\rput{264.35886}(5.4538083,5.620747){\pscircle[linewidth=0.04,dimen=outer](5.273357,0.33955148){0.21}}
\rput{179.04214}(12.34774,2.5576375){\pscircle[linewidth=0.04,dimen=outer](6.1631804,1.3304267){0.21}}
\rput{179.04214}(13.189889,1.8363191){\pscircle[linewidth=0.04,linecolor=red,dimen=outer,fillstyle=solid,fillcolor=red](6.5872693,0.9732873){0.36}}
\rput{179.04214}(14.06714,2.5145116){\pscircle[linewidth=0.04,dimen=outer](7.0230603,1.31605){0.21}}
\usefont{T1}{ptm}{m}{n}
\rput(6.503509,-0.05281705){Na}
\usefont{T1}{ptm}{m}{n}
\rput(2.0113213,0.08718295){Cl}
\end{pspicture} 
}
\caption{Natriumchloried in water oplos}
\label{fig:hydrosphere:ions dissolving}
\end{center}
\end{figure}     
        \label{m38720*id335421}Die oplos van natriumchloried kan deur die volgende vergelyking voorgestel word:\par 
        \label{m38720*uid3241}$\text{NaCl (s)} \to {\text{Na}}^{+}\text{(aq)} + {\text{Cl}}^{-}\text{(aq)}$
        \par 
        \label{m38720*id333999}Die oplos van die kaliumsulfaat in kalium- en sulfaatione word hieronder getoon, as nog 'n voorbeeld:\par 
        \label{m38720*uid971321}${\text{K}}_{2}{\text{SO}}_{4}\text{(s)}\to 2{\text{K}}^{+}\text{(aq)}+\text{SO}_{4}^{2-}\text{(aq)}$
        \par 
        \label{m38720*id335781}Onthou dat \textbf{molekul\^{e}re} stowwe (bv. kovalente verbindings)  ook kan oplos, maar die meeste
sal nie ione vorm nie. Een voorbeeld is glukose.\par 
        \label{m38720*uid922381}${\text{C}}_{6}{\text{H}}_{12}{\text{O}}_{6}\text{(s)}\rightarrow{\text{C}}_{6}{\text{H}}_{12}{\text{O}}_{6}\text{(aq)}$
        \par 
        \label{m38720*id335863}Daar is 'n uitsondering, sommige molekul\^{e}re stowwe \textsl{sal} ione vorm as hulle oplos. Waterstofchloried byvoorbeeld sal ioniseer om waterstof- en chloriedione te vorm.\par 
        \label{m38720*uid98732}$\text{HCl (g)} + \text{H}_{2}\text{O} (\ell) \to \text{H}_{3}\text{O}^{+} \text{(aq)} + {\text{Cl}}^{-}\text{(aq)}$
        \par 
    \noindent
Probeer om ioniese verbindings soos kaliumpermanganaat, natriumhidroksied en kaliumnitraat in water op te los en neem waar wat gebeur. 
  \label{m38720*secfhsst!!!underscore!!!id338}

\begin{exercises}{Ione in oplossings}
{
\begin{enumerate}[noitemsep, label=\textbf{\arabic*}. ] 
\item Vermeld of elk van die volgende stowwe, ionies of molekul\^{e}r is
    \begin{enumerate}[noitemsep, label=\textbf{\alph*}. ] 
    \item kaliumnitraat ($\text{KNO}_{3}$)
    \item etanol ($\text{C}_{2}\text{H}_{5}\text{OH}$)
    \item sukrose ( 'n soort suiker) ($\text{C}_{12}\text{H}_{22}\text{O}_{11}$)
    \item natriumbromied ($\text{NaBr}$)
    \end{enumerate}
\item Skryf 'n gebalanseerde vergelyking om aan te toon hoe elk van die volgende ioniese verbindings dissosieer in water.
    \begin{enumerate}[noitemsep, label=\textbf{\alph*}. ] 
    \item natriumsulfaat ($\text{Na}_{2}\text{SO}_{4}$)
    \item kaliumbromied ($\text{KBr}$)
    \item kaliumperoefeningmanganaat ($\text{KMnO}_{4}$)
    \item natrium fosfaat ($\text{Na}_{3}\text{PO}_{4}$)
    \end{enumerate}
\item Teken 'n diagram om te wys hoe $\text{KCl}$ in water oplos.
\end{enumerate}

% Automatically inserted shortcodes - number to insert 3
\par \practiceinfo
\par \begin{tabular}[h]{cccccc}
% Question 1
(1.)	02bu	&
% Question 2
(2.)	02bv	&
% Question 3
(3.)	02bw	&
\end{tabular}
% Automatically inserted shortcodes - number inserted 3
}
\end{exercises}
%             \subsection*{Applications}
%             \nopagebreak
% \begin{activity}{Acid rain and water hardness}
% The following two topics are given: acid rain or water hardness. Choose one of the two topics and prepare a poster or class presentation on it. Some information is given below about the topics. You will have to read further on the topic as well. Some guideline questions to answer are: 
% \begin{enumerate}[noitemsep,label=\textbf{\arabic*}.]
%  \item What is hard water/acid rain?
% \item Why is it a problem?
% \item Where in South Africa is this a problem?
% \item What is the chemistry involved?
% \item What is the impact on our lives and the environment?
% \item What can be done to to improve the situation?
% \end{enumerate}
% \begin{minipage}{.5\textwidth}
% \textbf{Hard water} is water that has a high mineral content. Water that has a low mineral content is known as \textbf{soft water}. If water has a high mineral content, it usually contains high levels of metal ions, mainly calcium ($\text{Ca}^{2+}$) and magnesium ($\text{Mg}^{2+}$). The calcium ions enter the water from either ${\text{CaCO}}_{3}$ (limestone or chalk) or from mineral deposits of ${\text{CaSO}}_{4}$. The main source of magnesium is a sedimentary rock called dolomite, ${\text{CaMg(CO}}_{3}\text{)}_{2}$. Hard water may also contain other metals as well as bicarbonates and sulphates.\par 
% \label{m38720*notfhsst!!!underscore!!!id362}
% \Tip{The simplest way to check whether water is hard or soft is to use the lather/froth test. If the water is very soft, soap will make bubbles more easily when it is rubbed against the skin. With hard water this won't happen. Toothpaste will also not froth well in hard water.}
% Hard water causes scaling on kettles (scaling is a build up of mainly calcium ions). Hard water also leads to blocked water pipes. 
% \end{minipage}
% \begin{minipage}{.5\textwidth}
% The acidity of rainwater comes from the natural presence of three substances ($\text{CO}_{2}$, $\text{NO}$, and $\text{SO}_{2}$) in the lowest layer of the atmosphere. These gases are able to dissolve in water and therefore make rain more acidic than it would otherwise be. Of these gases, carbon dioxide ($\text{CO}_{2}$) has the highest concentration and therefore contributes the most to the natural acidity of rainwater. \par 
% Acid rain refers to the deposition of acidic components in rain, snow and dew. Acid rain occurs when sulphur dioxide and nitrogen oxides are emitted into the atmosphere, undergo chemical transformations and are absorbed by water droplets in clouds. The droplets then fall to earth as rain, snow, mist, dry dust, hail, or sleet. This increases the acidity of the soil and affects the chemical balance of lakes and streams. 
% \label{m38720*id338300}Although these reactions do take place naturally, human activities can greatly increase the concentration of these gases in the atmosphere, so that rain becomes far more acidic than it would otherwise be. The burning of fossil fuels in industries, vehicles etc is one of the biggest culprits. If the acidity of the rain drops to below 5, it is referred to as \textbf{acid rain}.\par 
% \label{m38720*id338311}If the water in rivers, dams and lakes becomes too acidic some plants and animals may not be able to survive. Acid rain can also dissolve some minerals from the soil and these ions can get washed into the rivers and lakes. The soil can also become to acidic which influences the ability of the soil to produce crops.\par 
% \label{m38720*id338337}Acid rain can also affect buildings and monuments, many of which are made from marble and limestone. A chemical reaction takes place between ${\text{CaCO}}_{3}$ (limestone) and sulphuric acid to produce aqueous ions which can be easily washed away. The same reaction can occur in the lithosphere where limestone rocks are present e.g. limestone caves can be eroded by acidic rainwater.
%         \label{m38720*id7435}\nopagebreak\noindent{}       
%     \begin{equation}
%     {\text{H}}_{2}{\text{SO}}_{4}+{\text{CaCO}}_{3}\to {\text{CaSO}}_{4}\ensuremath{\cdot}\text{H}{}_{2}\text{O}+{\text{CO}}_{2}\tag{17.2}
%       \end{equation}
% \end{minipage}
% \end{activity}
    \label{m38720*cid7}
            \section{Elektroliete, ionisasie en geleiding}
            \nopagebreak
      \label{m38720*id338608}Jy het geleer dat die water 'n polêre molekuul is en dat ioniese stowwe kan oplos in water.
Wanneer ione in die water teenwoordig is, is die water in staat om elektrisiteit te gelei. So  'n oplossing is bekend as 'n elektroliet.\par 
\label{m38720*fhsst!!!underscore!!!id635}
\Definition{Elektroliet} { 'n Elektroliet is 'n stof wat vrye ione bevat en optree as 'n elektriese geleidende medium. Omdat hulle oor die algemeen uit ione in oplossing bestaan is elektroliete ook bekend as ioniese oplossings.  } 
 'n Sterk elektroliet is  'n oplossing waarin baie ione teenwoordig is, en 'n swak elektroliet is  'n oplossing waarin min ione teenwoordig is. Sterk elektroliete is goeie geleiers van elektrisiteit en swak elektroliete is swak geleiers van elektrisiteit. Nie-elektroliete gelei nie elektrisiteit nie.
\textbf{Geleiding} in waterige oplossings, is 'n maatstaf van die vermoë van die water om  'n elektriese stroom te gelei. Hoe meer \textbf{ione} daar in die oplossing is, hoe hoër is die geleidingsvermoë.  Verder, hoe meer ione daar in 'n oplossing is, hoe sterker is die elektroliet. 
      \label{m38720*uid56}
            \subsection*{Faktore wat die geleidingvermoë van elektroliete be\"invloed}
            \nopagebreak
Die geleidingsvermoë van 'n elektroliet word be\"invloed deur die volgende faktore:
        \label{m38720*id339291}\begin{itemize}[noitemsep]
\label{m38720*uid58}\item Die \textbf{konsentrasie van ione} in oplossing. Hoe ho\"{e}r die konsentrasie ione in oplossing, hoe beter sal sy geleidingvermo\"{e} wees.
\label{m38720*uid57}\item Die \textbf{soort stof} wat in water oplos. Sterk elektroliete soos kaliumnitraat (${\text{KNO}}_{3}$), swak elektroliete soos asynsuur (${\text{CH}}_{3}\text{COOH}$) of 'n nie-elektroliet soos suiker, alkohol en oile sal die geleidingsvermoë van die water beïnvloed omdat die konsentrasie van ione in oplossing in elke geval verskillend sal wees. Sterk elektroliete vorm maklik ione, swak elektroliete vorm moeilik ione en nie-elektroliete vorm glad nie ione in 'n oplossing nie.

\mindsetvid{investigating solutions}{VPbon}

\label{m38720*uid59}\item Die \textbf{temperatuur}. In 'n warm oplossing is die oplosbaarheid van die stof beter.  Dus hoe ho\"{e}r die oplosbaarheid hoe beter is die geleidingsvermo\"{e}.
\end{itemize} \nopagebreak
\label{m38720*secfhsst!!!underscore!!!id739}
            \begin{g_experiment}{Elektriese geleidingsvermo\"{e}}
            \nopagebreak
            \label{m38720*id339425}\noindent{}\textbf{Doel:}
          \newline
Om die elektriese geleidingvermo\"{e} van verskillende stowwe en oplossings te ondersoek.\par 
        \label{m38720*id339438}\noindent{}\textbf{Apparaat:}
\begin{itemize}[noitemsep]
\item Tafelsout ($\text{NaCl}$) kristalle
\item verskillende vloeistowwe soos gedistilleerde water, kraanwater, seewater, suiker, olie en alkohol
\item oplossings van soute bv. $\text{NaCl}$, $\text{KBr}$, $\text{CaCl}_{2}$, $\text{NH}_{4}\text{Cl}$
\item 'n suur oplossing (bv. $\text{HCl}$) en 'n basisoplossing (bv. $\text{NaOH}$)
\item flitsselle
\item ammeter
\item geleidingsdraad, krokodilknypers en twee koolstofstafies.
\end{itemize}
        \label{m38720*eip-456}
	\par
      \label{m38720*id334346}\noindent{}\textbf{Metode:}
\begin{enumerate}[noitemsep, label=\textbf{\arabic*}.]
\item Stel die eksperiment op, koppel die stroombaan soos getoon in die diagram hieronder. In die diagram, verteenwoordig "X" die stof of oplossing wat jy wil toets.
\item Wanneer jy die vaste stof kristalle gebruik, word die krokodiknypers direk aan elke end van die kristal geheg.  Wanneer die oplossings gebruik word, word die krokodilknypers aan die twee koolstofstafies verbind en in die vloeistof geplaas. 
\item Voltooi die stroombaan en laat die stroom vir ongeveer 30 sekondes vloei. 
\item Neem die ammeterlesing.
\end{enumerate}
        \label{m38720*id334362}
\begin{minipage}{.5\textwidth}
    \setcounter{subfigure}{0}
	\begin{figure}[H] % horizontal\label{m38720*id334366}
\begin{center}
\scalebox{0.7}{
\begin{pspicture}(0,-0.6)(5,6.2)
\SpecialCoor
%\psgrid[gridcolor=lightgray]
\pnode(0,0){A}
\pnode(0,5){B}
\pnode(5,5){C}
\pnode(5,0){D}
\pnode(3.5,0){E}
\pnode(1.5,0){F}
\battery(B)(C){battery}
\psline(C)(D)
\psline[arrowsize=10pt,arrowinset=0,arrowlength=2.5]{->}(D)(E)
\psframe(1.5,-0.5)(3.5,0.5)
\uput[u](2.5,0.5){toets stof}
\rput(2.5,0){X}
\psline(4,0)(4,-0.4)(4.6,-0.4)
\uput[r](4.6,-0.4){crocodile clip}
\psline[arrowsize=10pt,arrowinset=0,arrowlength=2.5]{<-}(F)(A)
\psellipse(0,2.5)(0.5,0.5)
\rput(0,2.5){\textbf{A}}
\psline(0,5)(0,3)
\psline(0,2)(0,0)
\rput(-1.5,2.5){Ammeter}
\end{pspicture}
}
\end{center}
 \end{figure}
\end{minipage}
\begin{minipage}{.5\textwidth}   
	\begin{figure}[H] % horizontal\label{m38720*id334366}
\begin{center}
\scalebox{0.7}{
\begin{pspicture}(0,-0.6)(5,6.2)
\SpecialCoor
%\psgrid[gridcolor=lightgray]
\pnode(0,0){A}
\pnode(0,5){B}
\pnode(5,5){C}
\pnode(5,0){D}
\pnode(3.5,0){E}
\pnode(1.5,0){F}
\battery(B)(C){battery}
\psline(C)(D)
\psline[arrowsize=10pt,arrowinset=0,arrowlength=2.5]{->}(D)(E)
\psline[linewidth=.1](1.5,0)(1.5,-1.2)
\psline[linewidth=.1](3.5,0)(3.5,-1.2)
\psline(1.2,-0.5)(1.2,-1.5)(3.7,-1.5)(3.7,-0.5)
\psline(1.2,-0.7)(3.7,-0.7)
\uput[u](2.5,-2){toets stof}
\rput(2.5,-1.2){X}
\psline(4,0)(4,-0.4)(4.6,-0.4)
\uput[r](4.6,-0.4){krokodilknypers}
\psline[arrowsize=10pt,arrowinset=0,arrowlength=2.5]{<-}(F)(A)
\psellipse(0,2.5)(0.5,0.5)
\rput(0,2.5){\textbf{A}}
\psline(0,5)(0,3)
\psline(0,2)(0,0)
\rput(-1.5,2.5){Ammeter}
\end{pspicture}
}
\end{center}
 \end{figure}
\end{minipage}    
        \par 
        \label{m38720*id334372}\noindent{}\textbf{Resultate:}
          \newline
        Teken jou waarnemings aan in 'n tabel, soortgelyk aan die een hieronder:
    % \textbf{m38720*id334385}\par
          \begin{table}[H]
    % \begin{table}[H]
    % \\ '' '0'
        \begin{center}
      \label{m38720*id334385}
    \noindent
      \begin{tabular}{|l|l|}\hline
        Stof wat getoets word &
        Ammeterlesing \\ \hline
         &
       \\ \hline
         &
       \\ \hline
         &
       \\ \hline
         &
    \\ \hline
    \end{tabular}
      \end{center}
\end{table}
    \par
  \par 
        \label{m38720*id339669}Wat word waargeneem?  Kan jy die waarnemings verduidelik?\\ \pagebreak
        \label{m38720*id339864}\noindent{}\textbf{Gevolgtrekkings:}
          \newline
Oplossings wat vry bewegende ione bevat gelei elektrisiteit as gevolg van die beweging van die gelaaide deeltjies. Oplossings wat nie vry bewegende ione bevat nie, gelei nie elektrisiteit nie.
\end{g_experiment}
Onthou vir die vloei van elektrisiteit moet daar beweging van gelaaide deeltjies, soos ione, wees. Met die vaste stof NaCl kristalle, was daar geen ammeterlesing nie, dus geen elektrisiteit vloei nie. Hoewel die vaste stof uit ione bestaan,word hulle baie styf bymekaar gehou binne die kristalrooster en daarom sal daar geen stroom vloei. Gedistilleerde water, olie en alkohol gelei ook nie stroom nie aangesien hulle \textbf{kovalente verbindings} is en dus nie ione bevat nie.\par 
Die ammeter moet  'n lesing toon wanneer die sout-, suur- en basisoplossings in die stroombaan verbind word.  In soutoplossing \textbf{dissosieer} hul in ione, wat vry beweeg in die oplossing. Kyk na die volgende voorbeelde:\\
Dissosiasie van kaliumbromied:\\
        \label{m38720*id339701}\nopagebreak\noindent        
    \begin{equation*}
    \text{KBr (s)} \to {\text{K}}^{+} \text{(aq)} + {\text{Br}}^{-} \text{(aq)}
      \end{equation*} \\
Dissosiasie van tafelsout:\\
        \label{m38720*id339737}\nopagebreak\noindent          
    \begin{equation*}
    \text{NaCl (s)}\to {\text{Na}}^{+} \text{(aq)} + {\text{Cl}}^{-} \text{(aq)}
      \end{equation*}\\
Ionisasie van soutsuur:\\
        \label{m38720*id339770}\nopagebreak\noindent          
    \begin{equation*}
    \text{HCl} (\ell)  +{\text{H}}_{2}\text{O} (\ell) \to {\text{H}}_{3}{\text{O}}^{+} \text{(aq)} +{\text{Cl}}^{-} \text{(aq)}
      \end{equation*}\\
        \label{m38720*id339831}\nopagebreak\noindent
Dissosiasie van natriumhidroksied:\\          
    \begin{equation*}
    \text{NaOH (s)} \to {\text{Na}}^{+} \text{(aq)} + {\text{OH}}^{-} \text{(aq)}
      \end{equation*}
 \par 
\label{m38720**end}
            \section{Neerslagreaksies}
            \nopagebreak
Soms reageer ione in oplossing met mekaar om 'n nuwe stof, wat \textsl{onoplosbaar} is, te vorm. Dit word 'n \textbf{neerslag} genoem. Die reaksie word 'n neerslagreaksie genoem.\par 

\mindsetvid{preparing precipitates}{VPbpr}

\label{m38719*fhsst!!!underscore!!!id887}
\Definition{Neerslag} { 'n Neerslag is die onoplosbare vaste stowwe wat  in 'n oplossing vorm tydens 'n chemiese reaksie.} 
\label{m38719*secfhsst!!!underscore!!!id890}
            \begin{g_experiment}{Die reaksie van ione in oplossing}
            \nopagebreak
            \label{m38719*id339954}\noindent{}\textbf{Apparaat en materiaal:}
        \newline
4 proefbuise; koper(II) chloried oplossing; natriumkarbonaat oplossing; natriumsulfaat oplossing\par 
      \label{m38719*id339975}
    \setcounter{subfigure}{0}
\begin{figure}[H]

\begin{center}
\scalebox{0.8} % Change this value to rescale the drawing.
{
\begin{pspicture}(-5,-5)(5,5)
\psset{unit=1cm}
\newpsstyle{white} {linestyle=solid,linewidth=.1,fillstyle=solid,fillcolor=white}
\rput(-4,0){\pstTubeEssais[niveauLiquide1=40]}
\psline[linewidth=0.04]{->}(-3.8,-1)(-3,-1)
\uput[r](-3,-1){\large{$\text{CuCl}_2$}}
\rput(0,0){\pstTubeEssais[niveauLiquide1=40,aspectLiquide1=white]}
\psline[linewidth=0.04]{->}(0.2,-1)(1,-1)
\uput[r](1,-1){\large{$\text{Na}_{2}\text{CO}_3$}}
\rput(4,0){\pstTubeEssais[niveauLiquide1=40]}
\psline[linewidth=0.04]{->}(4.2,-1)(5,-1)
\uput[r](5,-1){\large{$\text{CuCl}_2$}}
\rput(8,0){\pstTubeEssais[niveauLiquide1=40,aspectLiquide1=white]}
\psline[linewidth=0.04]{->}(8.2,-1)(9,-1)
\uput[r](9,-1){\large{$\text{Na}_{2}\text{SO}_4$}}
\end{pspicture}
}
\end{center}
\end{figure}       
      \par 
      \label{m38719*id339985}\noindent{}\textbf{Metode:}
        \newline
      \label{m38719*id339992}\begin{enumerate}[noitemsep, label=\textbf{\arabic*}. ] 
            \label{m38719*uid60}\item Vul twee proefbuise met ongeveer $5~\text{ml}$ verdunde koper(II) chloried oplossing.
\label{m38719*uid61}\item Vul een proefbuis met $5~\text{ml}$ natriumkarbonaat oplossing.
\label{m38719*uid62}\item Vul een proefbuis met $5~\text{ml}$ natriumsulfaat oplossing.
\label{m38719*uid63}\item Gooi die natriumkarbonaatoplossing versigtig in een van die proefbuise wat koper(II)chloried bevat. Wat word waargeneem?
\label{m38719*uid64}\item Gooi die natriumsulfaatoplossing versigtig in die tweede proefbuis met koper(II)chloried. Wat word waargeneem?
\end{enumerate}
        \par 
      \label{m38719*id340060}\noindent{}\textbf{Resultate:}
        \newline
      \label{m38719*id340067}\begin{enumerate}[noitemsep, label=\textbf{\arabic*}. ] 
            \label{m38719*uid65}\item 'n Ligte blou neerslag vorm wanneer natriumkarbonaat met koper(II)chloried reageer.
\label{m38719*uid66}\item Geen neerslag vorm wanneer natriumsulfaat met koper(II)chloried reageer nie. Die oplossing is ligblou.
\end{enumerate}
        \par 
\end{g_experiment}
      \label{m38719*id340106}Dit is belangrik om te verstaan wat in die vorige demonstrasie gebeur het.  Ons kyk stap vir stap in elke reaksie na wat gebeur het.\par 
Vir \textbf{reaksie 1} is daar die volgende ione in die oplossing: ${\text{Cu}}^{2+}$, ${\text{Cl}}^{-}$, ${\text{Na}}^{+}$ en $\text{CO}_{3}^{2-}$.  'n Neerslag sal vorm indien enige kombinasie van katione en anione 'n vaste stof vorm. Die volgende tabel gee 'n opsomming van watter kombinasies vaste stowwe (neerslae) sal vorm in 'n oplossing. 
          \begin{table}[H]
    % \begin{table}[H]
    % \\ '' '0'
        \begin{center}
      \label{m38719*uid69}
    \noindent
      \begin{tabular}{|l|p{8cm}|}\hline
                \textbf{Sout}
               &
                \textbf{Oplosbaarheid} \\ \hline
        Nitrate &
        Almal \textbf{oplosbaar} \\ \hline
        Kalium, natrium-en ammoniumsoute. &
        Almal \textbf{oplosbaar} \\ \hline
        Chloriede, bromiede en jodiede. &
        Almal \textbf{oplosbaar} behalwe silwer, lood(II) en kwik(II) soute (bv. silwer chloried)  \\ \hline
        Sulfaate &
        Almal \textbf{oplosbaar} behalwe lood(II)sulfaat, bariumsulfaat en kalsiumsulfaat \\ \hline
        Karbonate &
        \textbf{Onoplosbaar} behalwe dié van kalium, natrium en ammonium \\ \hline
        Verbindings met fluoor &
        Byna almal \textbf{oplosbaar} behalwe dié van magnesium, kalsium, strontium(II), barium(II) en lood (II) \\ \hline
        Perchlorate en Asetate &
        Almal \textbf{oplosbaar} \\ \hline
        Chlorate &
        Almal \textbf{oplosbaar} behalwe kaliumchloraat \\ \hline
        Metaalhidroksiede en oksiede &
        Meeste \textbf{onoplosbaar} \\ \hline
    \end{tabular}
      \end{center}
    \caption{Algemene re\"{e}ls vir die oplosbaarheid van soute}
\label{tab:solubility}
\end{table}
\Tip{Soute van karbonate, fosfate, oksalate, chromate en sulfiede is oor die algemeen nie oplosbaar nie.}
Indien jy na die karbonate in die tabel kyk, is alle karbonate \textbf{onoplosbaar} behalwe kalium, natrium en ammonium. Dit beteken dat $\text{Na}_{2}\text{CO}_3$ in water oplos en in oplossings bly, terwyl $\text{CuCO}_3$  'n neerslag vorm. Die neerslag wat in die reaksie waargeneem is, moet dus $\text{CuCO}_3$. Die gebalanseerde chemiese vergelyking is:\\
$2\text{Na}^{+} \text{(aq)} + \text{CO}_{3}^{2-} \text{(aq)} + \text{Cu}^{2+} \text{(aq)} + 2\text{Cl}^{-} \text{(aq)} \to \text{CuCO}_{3} \text{(s)} +  2\text{Na}^{+} \text{(aq)} + 2\text{Cl}^{-} \text{(aq)}$ \\
Let daarop dat die natriumchloried nie  'n neerslag vorm nie, en ons skryf dit as ione in die vergelyking. \\
Vir \textbf{reaksie 2} het ons ${\text{Cu}}^{2+}$, ${\text{Cl}}^{-}$, ${\text{Na}}^{+}$ en $\text{SO}_{4}^{2-}$ in 'n oplossing. Volgens die tabel is die meeste chloriede en sulfate is oplosbaar. Die gebalanseerde chemiese vergelyking is: \\
$2{\text{Na}}^{+} \text{(aq)} + \text{SO}_{4}^{2-} \text{(aq)} + {\text{Cu}}^{2+} \text{(aq)} + 2{\text{Cl}}^{-} \text{(aq)} \to 2{\text{Na}}^{+} \text{(aq)} + \text{SO}_{4}^{2-} \text{(aq)} + {\text{Cu}}^{2+} \text{(aq)} + 2{\text{Cl}}^{-} \text{(aq)} $
Beide van hierdie reaksies is ioonuitruilreaksies.
	\par
\subsection*{Toetse vir anione}
Dit is dikwels nodig om te weet watter ione in oplossing teenwoordig is. Indien ons weet watter soute neerslaan kan toetse gedoen word om die ione in die oplossing te identifiseer. Hier is 'n paar sulke toetse.
% \Warning{As always when working with chemicals, you must work carefully as you can easily get bad chemical burns if you spill the chemicals on yourself.}

\mindsetvid{test for halides}{VPbqd}

      \label{m38719*uid70}
            \subsubsection*{Toets vir 'n chloried}
            \nopagebreak
Berei 'n oplossing van  'n onbekende sout in gedistilleerde water en voeg 'n klein hoeveelheid \textbf{silwernitraatoplossing} by. Indien 'n wit neerslag (presipitaat) vorm, is die sout is óf 'n chloried of 'n karbonaat.
        \label{m38719*id341148}\nopagebreak\noindent{}
    \begin{equation*}
    {\text{Cl}}^{-} \text{(aq)} +{\text{Ag}}^{+} \text{(aq)} + \text{NO}_{3}^{-} \text{(aq)} \to \text{AgCl} \text{(s)} +\text{NO}_{3}^{-} \text{(aq)}
      \end{equation*}
     ($\text{AgCl}$ vorm 'n wit neerslag of presipitaat)
        \label{m38719*id341211}\nopagebreak\noindent{}
    \begin{equation*}
    \text{CO}_{3}^{2-} \text{(aq)} + 2{\text{Ag}}^{+} \text{(aq)} + 2\text{NO}_{3}^{-} \text{(aq)} \to {\text{Ag}}_{2}{\text{CO}}_{3} \text{(s)} + 2\text{NO}_{3}^{-} \text{(aq)}
      \end{equation*}
    (${\text{Ag}}_{2}{\text{CO}}_{3}$ vorm 'n wit neerslag of presipitaat)\par 
        \label{m38719*id341323}Die volgende stap is om die neerslag te behandel met 'n klein hoeveelheid \textbf{gekonsentreerde saltpetersuur}. Indien die neerslag onveranderd bly, dan is die sout 'n chloried.  Indien koolstofdioksied gevorm word, en die neerslag verdwyn, is die sout is 'n karbonaat.\par 
        $\text{AgCl} \text{(s)} + {\text{HNO}}_{3} (\ell) \to $ (geen reaksie; neerslag is onveranderd)\par 
        ${\text{Ag}}_{2}{\text{CO}}_{3} \text{(s)} + 2{\text{HNO}}_{3} (\ell) \to 2{\text{Ag}^{+}} \text{(aq)} + 2\text{NO}_{3}^{-} \text{(aq)} + {\text{H}}_{2}\text{O} (\ell) + {\text{CO}}_{2} \text{(g)} $ (neerslag verdwyn)\par 

\subsubsection*{Toets vir bromiedes en jodiede}
\nopagebreak
Soos in die geval van chloor, vorm bromiede en jodiede ook neerslae wanneer hulle met silwernitraat reageer. Silwerchloried is 'n wit neerslag, maar beide silwerbromied en silwerjodied se neerslae is liggeel. Om te bepaal of die neerslag 'n bromied of 'n jodied is, word gebruik gemaak van chloorwater en koolstoftetrachloried (${\text{CCl}}_{4}$).\par 
Chloorwater stel broomgas vry uit die bromied en kleur die koolstoftetrachloried rooibruin.\\
$\text{Br}^{-} \text{(aq)} + \text{Cl}_{2} \text{(aq)} \to 2 \text{Cl}^{-} \text{(aq)} + \text{Br}_{2} \text{(g)}$
\par 
Chloorwater stel jodiumgas vry uit die jodied en kleur die koolstoftetrachloried pers.\\
$\text{I}^{-} \text{(aq)} + \text{Cl}_{2} \text{(aq)} \to 2 \text{Cl}^{-} \text{(aq)} + \text{I}_{2} \text{(g)}$
\par 

\subsubsection*{Toets vir 'n sulfaat}
\nopagebreak
Voeg 'n klein hoeveelheid bariumchloried oplossing in die die toets soutoplossing.  Indien 'n wit neerslag vorm, is die sout óf 'n sulfaat of 'n karbonaat.\par 
$\text{SO}_{4}^{2-} \text{(aq)} + {\text{Ba}}^{2+} \text{(aq)} + {\text{Cl}}^{-} \text{(aq)} \to {\text{BaSO}}_{4} \text{(s)} + {\text{Cl}}^{-} \text{(aq)} $ (${\text{BaSO}}_{4}$ is 'n wit neerslag of presipitaat)\par 
$\text{CO}_{3}^{2-} \text{(aq)} + {\text{Ba}}^{2+} \text{(aq)} + {\text{Cl}}^{-} \text{(aq)} \to {\text{BaCO}}_{3} \text{(s)} + {\text{Cl}}^{-} \text{(aq)} $ (${\text{BaCO}}_{3}$ is 'n wit neerslag of presipitaat)\par 
Indien die neerslag met salpetersuur behandel word, kan daar vasgestel word of die sout 'n sulfaat of 'n karbonaat is (soos in die toets vir 'n chloried).\par 
${\text{BaSO}}_{4} \text{(s)} + {\text{HNO}}_{3} (\ell) \to $ (geen reaksie; neerslag is onveranderd)\par 
${\text{BaCO}}_{3} \text{(s)} + 2{\text{HNO}}_{3} (\ell) \to \text{Ba}^{2+} \text{(aq)} + 2\text{NO}_{3}^{-} \text{(aq)} + {\text{H}}_{2}\text{O} (\ell) + {\text{CO}}_{2} \text{(g)} $ (neerslag verdwyn)\par 

\subsubsection*{Toets vir 'n karbonaat}
\nopagebreak
 'n Positiewe toets vir 'n karbonaat is om 'n monster van die droë sout met 'n klein hoeveelheid suur te behandel. Indien koolstofdioksied gevorm word, is die sout  'n karbonaat.\par 
$\text{Acid} + \text{CO}_{3}^{2-} \text{(aq)} \to {\text{CO}}_{2} \text{(g)} $\par 
Indien die gas deur helder kalkwater geborrel word en die oplossing word melkerig is die gas koolstofdioksied.\par 
${\text{Ca}^{2+}} \text{(aq)} + 2\text{OH}^{-} \text{(aq)} + {\text{CO}}_{2} \text{(g)} \to {\text{CaCO}}_{3} \text{(s)} + \text{H}_{2}\text{O} (\ell) $ (Dit is die onoplosbare ${\text{CaCO}}_{3}$ neerslag wat die kalkwater melkerig maak.)\par 


\begin{exercises}{Neerslagreaksies en ione in oplossing}
{
\nopagebreak
\begin{enumerate}[noitemsep, label=\textbf{\arabic*}. ]
\item Silwernitraat (${\text{AgNO}}_{3}$) reageer met kaliumchloried ($\text{KCl}$) om 'n wit neerslag te vorm.
    \begin{enumerate}[noitemsep, label=\textbf{\alph*}. ] 
    \item Skryf 'n gebalanseerde vergelyking vir die reaksie wat plaasvind. Sluit die fase simbole in. 
    \item Benoem die onoplosbare sout wat vorm.
    \item Watter sout in hierdie reaksie is oplosbaar?
    \end{enumerate}
\item Bariumchloried reageer met swaelsuur om bariumsulfaat en soutsuur te vorm. 
    \begin{enumerate}[noitemsep, label=\textbf{\alph*}. ] 
    \item Skryf 'n gebalanseerde vergelyking vir die reaksie wat plaasvind. Sluit die fase simbole in.
    \item Word 'n neerslag tydens die reaksie gevorm?
    \item Beskryf hoe daar vir die teenwoordigheid van bariumsulfaat, in die produkte, getoets kan word. 
    \end{enumerate}
\item n Proefbuis bevat  'n helder, kleurlose soutoplossing.  'n Paar druppels silwernitraatoplossing word in die soutoplossing gevoeg en 'n ligte geel neerslag vorm. Watter een van die volgende soute was opgelos in die oorspronklike oplossing? Skryf die gebalanseerde 	vergelyking vir die reaksie wat plaasvind.
    \begin{enumerate}[noitemsep, label=\textbf{\alph*}. ] 
    \item $\text{NaI}$
    \item $\text{KCl}$
    \item ${\text{K}}_{2}{\text{CO}}_{3}$\label{m38719*uid86}\item ${\text{Na}}_{2}{\text{SO}}_{4}$\end{enumerate}
(IEB Paper 2, 2005)\newline
    \end{enumerate}

\practiceinfo
 \par \begin{tabular}[h]{cccccc}
 (1.) 02bx  &  (2.) 02by  &  (3.) 02bz  & \end{tabular}
}
\end{exercises}

\section{Ander reaksiestipes}
    \nopagebreak

Ons kyk na twee tipes reaksies wat in waterige oplossings plaasvind. Dit is ioonuitruilreaksies en redoksreaksies. Ioonuitruilreaksies sluit in neerslagreaksies, gasvormende reaksies en suur-basisreaksies. Redoksreaksies is elektronoordrag reaksies. Dit is belangrik om die verskil tussen hierdie twee tipes reaksies te onthou. In ioonuitruilreaksies word ione uitgeruil, maar in redoksreaksies (elektronoordragreaksies) word elektrone oorgedra. Die terme sal verder in die volgende afdelings verduidelik word. 
      \par 
\subsection*{Ioonuitruilreaksies}
      \label{m38719*uid78332}
Ioonuitruilreaksies word voorgestel deur:
	  \label{m38719*eid071534}\nopagebreak\noindent{}
	    
    \begin{equation*}
    \text{AB}\left(\text{aq}\right)+\text{CD}\left(\text{aq}\right)\to \text{AD}+\text{CB}
      \end{equation*}
	  Enige een $\text{AD}$ of $\text{CB}$ kan 'n vaste stof of 'n gas wees. Wanneer 'n vaste stof vorm staan dit bekend as 'n neerslagreaksie.  Indien 'n gas gevorm word, staan dit bekend as  'n gasvormende reaksie.  'n Suur-basisreaksie is 'n spesiale soort ioonuitruilreaksie en sal afsonderlik behandel word.
      \par 
      \label{m38719*eip-179}Die vorming van 'n neerslag of  'n gas help om die reaksie te laat plaasvind.  Die reaksie word gedryf deur die vorming van 'n neerslag of 'n gas.  Alle chemiese reaksies sal slegs plaasvind indien daar iets is wat dit laat gebeur. Vir sommige reaksies gebeur dit maklik, maar vir ander vind die reaksies moeiliker plaas.  \par 
\label{m38719*id7583}
 \Definition{Ioonuitruilreaksie} { 'n Reaksietipe waar die positiewe ione hulle onderskeie negatiewe ione uitruil as gevolg van 'n dryfkrag.} 
\label{m38719*uid10825}
\IFact{I\-oon\-uit\-ruil\-re\-ak\-sies word gebruik in i\-oon\-uit\-ruil\-chr\-oma\-to\-gra\-fie. I\-oon\-uit\-ruil\-chro\-ma\-to\-gra\-fie word gebruik om water te suiwer en is  'n metode om water te versag.  Dikwels wanneer chemici praat oor i\-oon\-uit\-rui\-ling, beteken dit i\-oon\-uit\-ruil\-chro\-ma\-to\-gra\-fie.}
	\par
Ons het reeds gekyk na neerslagreaksies.
\subsubsection*{Gasvormende reaksies}
Hierdie reaksies is soortgelyk aan neerslagreaksies met dié uitsondering dat in plaas daarvan dat 'n presipitaat vorm, vorm daar 'n gas. 'n Voorbeeld van 'n gasvormende reaksie is die reaksie tussen natriumkarbonaat en soutsuur. Die gebalanseerde vergelyking vir hierdie reaksie is: \\
$\text{Na}_{2}\text{CO}_{3} \text{(s)} + 2\text{HCl} \text{(aq)} \to \text{CO}_{2} \text{(g)} + 2\text{NaCl} \text{(aq)} + \text{H}_{2}\text{O} (\ell)$ 
            \subsubsection*{Suur-basisreaksies}
            \nopagebreak
Suur-basisreaksies vind tussen sure en basisse plaas. In die algemeen is die gevormde produkte water en 'n sout (d.w.s. 'n ioniese verbinding). 'n Voorbeeld van hierdie reaksietipe is:\label{m38719*eid1534}\nopagebreak\noindent{}
	    
    \begin{equation*}
    \text{NaOH (aq)}+\text{HCl (aq)}\to \text{NaCl (aq)}+{\text{H}}_{2}\text{O} (\ell)
      \end{equation*}
Dit is 'n spesiale geval van 'n ioonuitruilingsreaksie aangesien natrium in natriumhidroksied plekke met waterstof in waterstofchloried uitruil om natriumchloried te vorm. Ter selfde tyd verbind die hidroksied en die waterstof om water te vorm. \par 
\label{m38719*eip-454}
            \subsection*{Redoksreaksies }
            \nopagebreak
Redoksreaksies behels die uitruil van elektrone. Een atoom verloor elektrone en word meer positief, terwyl die ander atoom elektrone bykry en meer negatief raak. Om te besluit of 'n redoks reaksie plaasgevind het, kyk ons na die lading van die atome, ione of molekule betrokke. As een van die  deeltjies meer positief geword het en die ander een meer negatief het 'n redoksreaksie plaasgevind.

 So kan natrium metaal byvoorbeeld geoksideer word om natriumoksied te vorm (of somtyds ook na natriumperoksied). Die gebalanseerde vergelyking vir hierdie reaksie, is:
\label{m38719*id624}\nopagebreak\noindent{}
    \begin{equation*}
    4\text{Na}+{\text{O}}_{2}\to 2{\text{Na}}_{2}{\text{O}}
      \end{equation*}
\par \label{m38719*eip-815}
In die bogenoemde reaksie is beide natrium en suurstof neutraal en het geen lading nie. In die produk het die natriumatoom 'n lading van $+1$ en die suurstofatoom 'n lading van $-2$. Dit beteken dat natrium elektrone verloor het en suurstof elektrone bygekry het. Aangesien een soort stof meer positief geword het en  'n andereen meer negatief, kan ons aflei dat 'n redoksreaksie plaasgevind het. Ons kan ook s\^{e} dat elektrone oorgedra is van een soort stof na  'n ander. (In hierdie geval is elektrone van natrium na suurstof oorgedra).
\par \label{m38719*eip-878}
\vspace{-1cm}
            \begin{g_experiment}{Demonstrasie: Oksidasie van natriummetaal}
            \nopagebreak
            \label{m38719*eip-355}
\begin{minipage}{.6\textwidth}
Jy benodig 'n Bunsenbrander, 'n klein stukkie natriummetaal en 'n metaalspatel. Steek die Bunsenbrander aan. Plaas die natriummetaal op die spatel. Plaas die natrium in die vlam. Wanneer die reaksie voltooi is, behoort jy 'n wit poeier op die spatel waar te neem. Dit is 'n mengsel van natriumoksied (${\text{Na}}_{2}\text{O}$) en natriumperoksied (${\text{Na}}_{2}{\text{O}}_{2}$).
\par \label{m38719*eip-980}
\end{minipage}
\begin{minipage}{.35\textwidth}
 \begin{center}
  \includegraphics[width=.5\textwidth]{photos/sodium_flame_soren_wedel_nielsen_wikipedia.jpg}
 \end{center}

\end{minipage}

\Warning{Natriummetaal is baie reaktief. Natriummetaal reageer baie heftig met water en moet nooit in water geplaas word ie. Wees baie versigtig met die hantering van natriummetaal.}

\end{g_experiment}

\begin{g_experiment}{Reaksietipes}
\nopagebreak
\label{m38719*eip-190}\noindent{}\textbf{Doel: }\newline
Om deur die uitvoering van eksperimente reaksietipes te bepaal.
\par 
\label{m38719*eip-1901}\noindent{}\textbf{Apparaat: }\newline
Oplosbare soute (bv. kaliumnitraat, ammoniumchloried, natriumkarbonaat, silwernitraat, natriumbromied), soutsuur ($\text{HCl}$), natriumhidroksied ($\text{NaOH}$), broomtimolblou, sinkmetaal, koper(II)sulfaat, bekers, proefbuise
\par 
\label{m38719*eip-1902}\noindent{}\textbf{Metode: }\label{m38719*id6231}\\
\begin{minipage}{0.9\textwidth}\vspace*{1em}
\begin{itemize}[noitemsep]
\item Los elk van die soute op in 'n klein hoeveelheid water en kyk wat gebeur.
\item Probeer om pare soute (bv. kaliumnitraat en natriumkarbonaat) in water op te los en neem waar wat gebeur.
\item Los  'n bietjie natriumkarbonaat op in soutsuur en neem waar wat gebeur.
\item Meet versigtig $20{\text{cm}}^{3}$ natriumhidroksied af in 'n beker. 
\item Voeg 'n paar druppels broomtimolblou by die natriumhidroksied.
\item Voeg versigtig 'n paar druppels soutsuur by die natriumhidroksied en skud met  'n draaibeweging. Herhaal totdat jy agterkom die kleur verander.
\item Plaas sinkmetaal in die kopersulfaatoplossing en neem waar wat gebeur.
\end{itemize}\vspace*{1em}

\end{minipage}
\par 
\label{m38719*eip-1903}\noindent{}\textbf{Resultate: }\newline
Beantwoord die volgende vrae:\\
\label{m38719*id6144}
\begin{minipage}{0.9\textwidth}
\begin{itemize}[noitemsep]
\item Wat het jy waargeneem toe elk van die soute in water oplos?
\item Wat het jy waargeneem toe pare soute in water oplos?
\item Wat het jy waarneem toe natriumkarbonaat in soutsuur oplos?
\item Hoekom dink jy het ons broomtimolblou gebruik toe soutsuur en natriumhidroksied gemeng is? Dink na oor die soort reaksie wat plaasgevind het.
\item Wat neem jy waar as sinkmetaal in kopersulfaat-geplaas word?
\item Klassifiseer elke reaksie as: neerslag-, gasvormende-, suur-basis- of redoksreaksie.
\item Wat veroorsaak dat elke reaksie plaasvind (m.a.w. wat is die dryfveer)? Is dit die vorming van 'n neerslag of iets anders?
\item Watter kriteria sal jy gebruik om te bepaal watter soort reaksie plaasgevind het?
\item Probeer om gebalanseerde chemiese vergelykings vir elke reaksie te skryf.
\end{itemize}
\end{minipage}
\par 
\label{m38719*eip-1904}\noindent{}\textbf{Gevolgtrekking: }\newline
Ons kan reaksietipes klassifiseer deur eksperimente uit te voer.
\end{g_experiment}
In die eksperiment hierbo, het jy gesien hoe reaksietipes van mekaar verskil. Byvoorbeeld, 'n gasvormende reaksie lei tot borrels in die oplossing, 'n neerslagreaksie lei tot  neerslagvorming, 'n suur-basisreaksie kan geїdentifiseer word deur die byvoeging van 'n  geskikte indikator en in 'n redoksreaksie word waargeneem dat een metaal verdwyn en 'n neerslag vorm in die oplossing.\par  
\label{m38719*eip-796}
\summary{VPenn}
            \nopagebreak
            \label{m38719*eip-903}\begin{itemize}[noitemsep]
            \label{m38719*uid95}\item Die \textbf{pol\^{e}re} natuur van water beteken dat \textbf{ioniese verbindings} maklik dissosieer in waterige oplossing in die ione waaruit dit saamgestel is. 
\label{m38719*uid96}\item \textbf{Ione} in oplossing speel 'n aantal rolle. In die menslike liggaam byvoorbeeld, help ione om die interne omgewing (bv. die beheer van spierfunksie, die regulering van pH van die bloed) te reguleer. Ione in oplossing bepaal ook die hardheid van water en die pH. 
\label{m38719*uid100}\item \textbf{Geleidingsvermo\"{e}} is 'n maatstaf van 'n oplossing se vermoë om 'n elektriese stroom te gelei.
\label{m38719*uid101}\item 'n \textbf{Elektroliet} is 'n stof wat vrye ione bevat en is dus in staat om  'n elektriese stroom te gelei. Elektroliete kan verdeel word in \textbf{sterk} en \textbf{swak} elektroliete, afhangende van die mate waarin die stof in oplossing ioniseer.
\label{m38719*uid102}\item 'n \textbf{Nie-elektroliet} kan nie  'n elektriese stroom gelei nie, omdat dit nie vry ione bevat nie.
\label{m38719*uid103}\item Die \textbf{aard van die stof}, die \textbf{konsentrasie van ione} en die \textbf{temperatuur} van die oplossing beїnvloed die geleidingsvermo\"{e}.
\label{m38719*uid0253}\item Drie hooftipes reaksies plaasvind in waterige oplossings. Dit is: presipitasie-reaksies, suur-basisreaksies en redoksreaksies.
\label{m38719*uid8923}\item Neerslag- of presipitasie-reaksies en suur-basis-reaksies is soms bekend as ioonuitruilingsreaksies. Ioonuitruilingsreaksies sluit ook gasvormende reaksies in.
\label{m38719*uid104}\item 'n \textbf{Neerslag} word gevorm wanneer ione in oplossing met mekaar reageer om 'n onoplosbare produk te vorm. Oplosbaarheids"re\"{e}ls" help om die neerslag wat gevorm het, te identifiseer.
\label{m38719*uid105}\item 'n Aantal toetse kan gebruik word om te bepaal of sekere \textbf{anione} in 'n oplossing teenwoordig is.
\label{m38719*id813}\item In 'n suur-basisreaksie reageer 'n suur met 'n basis om 'n sout en water te vorm.
\label{m38719*uid823}\item Tydens 'n redoksreaksie word elektrone van een stof na 'n ander oorgedra. 
\end{itemize}
\label{m38719*eip-896} \pagebreak
            \begin{eocexercises}{Reaksies in waterige oplossings}
            \nopagebreak
            \label{m38719*id342869}\begin{enumerate}[noitemsep, label=\textbf{\arabic*}. ] 
            \label{m38719*uid107}\item Gee een woord vir elk van die volgende beskrywings:
\label{m38719*id342885}\begin{enumerate}[noitemsep, label=\textbf{\alph*}. ] 
            \label{m38719*uid108}\item die fase verandering van water van 'n gas na 'n vloeistof
\label{m38719*uid109}\item 'n gelaaide atoom
\label{m38719*uid110}\item 'n term wat gebruik word om die mineraal-inhoud van water te beskryf
\label{m38719*uid111}\item 'n gas wat swaelsuur vorm wanneer dit met water reageer
\end{enumerate}
\label{m38719*uid112}\item Pas die inligting in kolom A by die inligting in kolom B deur slegs die letter (A tot I) langs die vraagnommer (1-7) te skryf.
    % \textbf{m38719*id342952}\par
          \begin{table}[H]
    % \begin{table}[H]
    % \\ 'id2965514' '1'
        \begin{center}
      \label{m38719*id342952}
    \noindent
      \begin{tabular}{|l|l|}\hline
        \textbf{Kolom A} &
        \textbf{Kolom B} \\ \hline
        1. 'n pol\^{e}re molekuul &
        A. ${\text{H}}_{2}{\text{SO}}_{4}$ \\ \hline
        2. molekul\^{e}re oplossing &
        B. ${\text{CaCO}}_{3}$ \\ \hline
        3.  'n Mineraal wat die hardheid van water verhoog &
        C. $\text{NaOH}$ \\ \hline
        4.   'n Stof wat die waterstof-ioon konsentrasie verhoog &
        D. sout water \\ \hline
        5.  'n Sterk elektroliet &
        E. kalsium \\ \hline
        6.  'n Wit presipitaat &
        F. koolstofdioksied \\ \hline
        7. 'n Nie-geleier van elektrisiteit &
        G. kaliumnitraat \\ \hline
         &
        H. suiker water \\ \hline
         &
        I. ${\text{O}}_{2}$ \\ \hline
    \end{tabular}
      \end{center}
\end{table}
    \par

        \item Verduidelik die verskil tussen 'n swak elektroliet en 'n sterk elektroliet. Gee 'n algemene vergelyking vir elk.
           \item  Watter faktore het 'n invloed op die geleidingsvermo\"{e} van water? Hoe be\"invloed elke faktor die geleidingsvermo\"{e}?
            \item Vermeld of elk van die volgende stowwe molekulêr of ionies is. Indien ionies, gee 'n gebalanseerde reaksie vir die dissosiasie in water.
\label{m38719*id7342}\begin{enumerate}[noitemsep, label=\textbf{\alph*}. ] 
            \item metaan (${\text{CH}}_{4}$)
\item kaliumbromied
\item koolstofdioksied
\item heksaan (${\text{C}}_{6}{\text{H}}_{14}$)
\item litiumfluoried ($\text{LiF}$)
\item magnesiumchloried
\end{enumerate}
\label{m38719*uid127}\item  Drie proefbuise (X, Y en Z) bevat elk 'n oplossing van 'n onbekende kaliumsout. Die volgende waarnemings is gemaak tydens 'n praktiese ondersoek om die oplossings in die proefbuise te identifiseer. \\
A: 'n Wit presipitaat het gevorm toe silwernitraat (${\text{AgNO}}_{3}$) by proefbuis Z gevoeg is.\\
B: 'n Wit presipitaat het in proefbuise X en Y gevorm toe bariumchloried (${\text{BaCl}}_{2}$) bygevoeg is.\\
C: Die neerslag in proefbuis X het in soutsuur ($\text{HCl}$) opgelos en 'n gas is vrygestel.\\
D: Die neerslag in proefbuis Y is onoplosbaar in soutsuur.
\label{m38719*id343466}\begin{enumerate}[noitemsep, label=\textbf{\alph*}. ] 
            \label{m38719*uid128}\item Gebruik die bostaande inligting om die oplossings in elk van die proefbuise X, Y en Z te identifiseer.
\label{m38719*uid129}\item Skryf 'n chemiese vergelyking vir die reaksie wat in proefbuis X plaasgevind het voordat soutsuur bygevoeg is. 
\end{enumerate}
(DoE Exemplar Paper 2 2007)
\end{enumerate}

% Automatically inserted shortcodes - number to insert 6
\par \practiceinfo
\par \begin{tabular}[h]{cccccc}
% Question 1
(1.)	02c0	&
% Question 2
(2.)	02c1	&
% Question 3
(3.)	02c2	&
% Question 4
(4.)	02c3	&
% Question 5
(5.)	02c4	&
% Question 6
(6.)	02c5	\\ % End row of shortcodes
\end{tabular}
% Automatically inserted shortcodes - number inserted 6
\end{eocexercises}
