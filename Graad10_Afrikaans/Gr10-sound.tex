         \chapter{Klank}\fancyfoot[LO,RE]{Physics: Waves, Klank and Light}

    \setcounter{figure}{1}
    \setcounter{subfigure}{1}
    \label{9b5d72dd5f0585e544578ab90a9956a8}
         \section{Introduction}
    \nopagebreak
            \label{m38799} $ \hspace{-5pt}\begin{array}{cccccccccccc}   \end{array} $ \hspace{2 pt}\raisebox{-5 pt}{\includegraphics[width=0.5cm]{col11305.imgs/summary_www.png}} {(section shortcode: P10048 )} \par 
    \label{m38799*cid2}
\label{m38799*id183123}Het jy al gestop en dink hoe wonderlik die sintuig van gehoor is? Dis merkwaardig dat ons so 'n wye spektrum van klank kan hoor en uiwerk uit watse rigting die klank vandaan kom. Hoe word klanke gevorm wat ons kan hoor? Enige iets wat 'n versteuring in die lug veroorsaak, skep 'n puls wat weg beweeg van die punt van oorsprong. As die puls 'n mens se oor binnegaan veroorsaak dit 'n vibrasie in die oordrom, en dit is hoe 'n mens hoor. As die oorsrpong van die puls 'n reeks van pulse is dan word die versteuring 'n golf genoem. 

In die algemeen word aanvaar dat klank 'n golf is. Klank golwe is longitudinale drukgolwe wat beteken die golf bestaan uit kompressie en verdunnings van lugdruk.

\subsection*{Klank Waves}
            \nopagebreak
'n Stemvurk is 'n instrument wat deur musikante gebruik work om 'n spesifieke frekwensie klankgolf te genereer. Dit word dikwels gebruik om musikale instrumente te stem.
\begin{minipage}{.5\textwidth}
      \label{m38783*id293458}Klankgolwe wat van die stemvurk afkomstig is word veroorsaak deur vibrasies van die stemvurk wat met omliggende lugmolekules wisselwerk. Soos die lugmolekules saamgepers word, vorm dit 'n verhooging van lugdruk. Die lugmolekules na die verhooging beweeg, beweeg verder uitmekaar en vorm 'n verlaging in  lugfruk. Di\'{e} beweging van lugmolekules veroorsaak 'n konstante verandering in lugdruk. Daarvolgens is klankgolwe, lugdrukgolwe. Dit beteken dat klankgolwe vinniger sal beweeg in 'n medium waar die molekules nader aanmekaar is.\par 


      \label{m38783*id293466}Klankgolwe beweeg vinniger in vloeistowwe soos water as deur lug, want water is digter as lug (die molekules is nader aanmekaar). Klankgolwe beweeg nog vinniger in vastestowwe as in vloeistowwe.\par
      

\end{minipage}
\begin{minipage}{.5\textwidth}\begin{center}
\textbf{Stemvurk}\par
    \includegraphics[width=0.8\textwidth]{photos/TuningFork2_Flickr_amonya.jpg}\par
\textit{Foto deur amonya op Flickr}
\end{center}
\end{minipage}

    \setcounter{subfigure}{0}
	\begin{figure}[H] % horizontal\label{m38783*uid18}
    \begin{center}
\scalebox{1} % Change this value to rescale the drawing.
{
\begin{pspicture}(0,-1.3159375)(10.862187,1.3159375)
\psline[linewidth=0.04cm](1.5942917,-0.29788056)(1.8169739,-0.30129474)
\psline[linewidth=0.04cm](1.8169739,-0.30129474)(1.8345535,0.44497925)
\psline[linewidth=0.04cm](1.9030713,0.44392854)(1.8835384,-0.3852649)
\psline[linewidth=0.04cm](1.8835384,-0.3852649)(1.7636325,-0.38342634)
\psline[linewidth=0.04cm](1.5771624,-0.29761782)(1.5947421,0.44865623)
\psline[linewidth=0.04cm](1.5262243,0.4497069)(1.5066916,-0.3794867)
\psline[linewidth=0.04cm](1.6437267,-0.38158786)(1.5238208,-0.37974924)
\psline[linewidth=0.04cm](1.6269228,-0.36750528)(1.6171566,-0.7821023)
\psline[linewidth=0.04cm](1.6171566,-0.7821023)(1.7541916,-0.7842031)
\psline[linewidth=0.04cm](1.7541916,-0.7842031)(1.7636325,-0.38342634)
\psline[linewidth=0.04cm](1.8345535,0.44497925)(1.9030713,0.44392854)
\psline[linewidth=0.04cm](1.5947421,0.44865623)(1.5262243,0.4497069)
\psdots[dotsize=0.04](2.0484376,0.2296875)
\psdots[dotsize=0.04](2.1284375,0.2496875)
\psdots[dotsize=0.04](2.0884376,0.2096875)
\psdots[dotsize=0.04](2.0084374,0.1296875)
\psdots[dotsize=0.04](2.1084375,0.0896875)
\psdots[dotsize=0.04](2.1684375,0.1696875)
\psdots[dotsize=0.04](2.1284375,0.1696875)
\psdots[dotsize=0.04](2.0684376,0.1296875)
\psdots[dotsize=0.04](2.0684376,-0.0103125)
\psdots[dotsize=0.04](2.1484375,-0.0303125)
\psdots[dotsize=0.04](2.1684375,0.1296875)
\psdots[dotsize=0.04](2.2684374,0.1896875)
\psdots[dotsize=0.04](2.2884376,0.1096875)
\psdots[dotsize=0.04](2.2884376,0.0096875)
\psdots[dotsize=0.04](2.2484374,-0.0103125)
\psdots[dotsize=0.04](2.1884375,-0.1103125)
\psdots[dotsize=0.04](2.0684376,-0.1503125)
\psdots[dotsize=0.04](1.9684376,-0.0903125)
\psdots[dotsize=0.04](2.0084374,-0.0103125)
\psdots[dotsize=0.04](2.1684375,-0.1103125)
\psdots[dotsize=0.04](2.2284374,-0.1503125)
\psdots[dotsize=0.04](2.2284374,0.0296875)
\psdots[dotsize=0.04](2.1884375,0.0896875)
\psdots[dotsize=0.04](2.2284374,0.2296875)
\psdots[dotsize=0.04](2.3084376,0.1696875)
\psdots[dotsize=0.04](2.3084376,-0.0703125)
\psdots[dotsize=0.04](2.2684374,-0.1303125)
\psdots[dotsize=0.04](2.1884375,-0.2103125)
\psdots[dotsize=0.04](2.1084375,-0.2103125)
\psdots[dotsize=0.04](2.0484376,-0.1703125)
\psdots[dotsize=0.04](1.9884375,-0.1903125)
\psdots[dotsize=0.04](2.0084374,-0.2103125)
\psdots[dotsize=0.04](2.0884376,-0.2303125)
\psdots[dotsize=0.04](2.1884375,-0.2903125)
\psdots[dotsize=0.04](2.2684374,-0.2303125)
\psdots[dotsize=0.04](2.3084376,-0.1303125)
\psdots[dotsize=0.04](2.3084376,0.0096875)
\psdots[dotsize=0.04](2.0484376,0.0496875)
\psdots[dotsize=0.04](2.0884376,-0.0903125)
\psdots[dotsize=0.04](2.0284376,-0.0703125)
\psdots[dotsize=0.04](2.2284374,0.1296875)
\psdots[dotsize=0.04](2.0084374,-0.2703125)
\psdots[dotsize=0.04](2.2684374,0.2496875)
\psdots[dotsize=0.04](2.3484375,0.2496875)
\psdots[dotsize=0.04](2.3484375,0.1696875)
\psdots[dotsize=0.04](2.4284375,0.2296875)
\psdots[dotsize=0.04](2.4084375,0.2296875)
\psdots[dotsize=0.04](2.4484375,0.0496875)
\psdots[dotsize=0.04](2.4284375,0.1496875)
\psdots[dotsize=0.04](2.3484375,0.1096875)
\psdots[dotsize=0.04](2.3684375,0.0096875)
\psdots[dotsize=0.04](2.3684375,0.0096875)
\psdots[dotsize=0.04](2.3084376,0.0496875)
\psdots[dotsize=0.04](2.4484375,-0.0903125)
\psdots[dotsize=0.04](2.3684375,-0.1303125)
\psdots[dotsize=0.04](2.3284376,-0.1103125)
\psdots[dotsize=0.04](2.4084375,-0.0703125)
\psdots[dotsize=0.04](2.4684374,-0.0303125)
\psdots[dotsize=0.04](2.4684374,-0.1503125)
\psdots[dotsize=0.04](2.4284375,-0.1903125)
\psdots[dotsize=0.04](2.3484375,-0.2303125)
\psdots[dotsize=0.04](2.3084376,-0.2503125)
\psdots[dotsize=0.04](2.3884375,-0.2903125)
\psdots[dotsize=0.04](2.4884374,-0.2503125)
\psdots[dotsize=0.04](2.2284374,-0.0503125)
\psdots[dotsize=0.04](2.1084375,-0.2903125)
\psdots[dotsize=0.04](2.0084374,0.0696875)
\psdots[dotsize=0.04](2.0084374,0.2296875)
\psdots[dotsize=0.04](2.0484376,0.2696875)
\psdots[dotsize=0.04](2.4484375,0.1496875)
\psdots[dotsize=0.04](2.4684374,0.2296875)
\psdots[dotsize=0.04](2.5484376,0.2296875)
\psdots[dotsize=0.04](2.7284374,0.1896875)
\psdots[dotsize=0.04](2.7684374,-0.0703125)
\psdots[dotsize=0.04](2.5884376,-0.0103125)
\psdots[dotsize=0.04](2.6684375,0.1096875)
\psdots[dotsize=0.04](2.8684375,0.0296875)
\psdots[dotsize=0.04](3.3084376,-0.0903125)
\psdots[dotsize=0.04](3.1484375,0.0896875)
\psdots[dotsize=0.04](2.9884374,0.1296875)
\psdots[dotsize=0.04](3.0084374,0.1896875)
\psdots[dotsize=0.04](3.2084374,0.1696875)
\psdots[dotsize=0.04](2.8084376,0.2696875)
\psdots[dotsize=0.04](3.0684376,-0.0703125)
\psdots[dotsize=0.04](2.6284375,-0.2303125)
\psdots[dotsize=0.04](2.7884376,-0.2703125)
\psdots[dotsize=0.04](3.0284376,-0.2503125)
\psdots[dotsize=0.04](2.9084375,-0.0903125)
\psdots[dotsize=0.04](3.2284374,-0.2503125)
\psdots[dotsize=0.04](3.2684374,0.0296875)
\psdots[dotsize=0.04](3.1284375,0.2496875)
\psdots[dotsize=0.04](3.3284376,0.2496875)
\psdots[dotsize=0.04](3.4884374,0.2096875)
\psdots[dotsize=0.04](3.6884375,0.2496875)
\psdots[dotsize=0.04](3.6484375,0.0696875)
\psdots[dotsize=0.04](3.6084375,-0.1103125)
\psdots[dotsize=0.04](3.3484375,0.0496875)
\psdots[dotsize=0.04](3.4884374,0.0696875)
\psdots[dotsize=0.04](3.4484375,-0.1703125)
\psdots[dotsize=0.04](3.3284376,-0.2303125)
\psdots[dotsize=0.04](3.5484376,-0.2703125)
\psdots[dotsize=0.04](3.6684375,-0.2703125)
\psdots[dotsize=0.04](3.7884376,0.2496875)
\psdots[dotsize=0.04](3.8684375,0.2696875)
\psdots[dotsize=0.04](3.8284376,0.2296875)
\psdots[dotsize=0.04](3.7484374,0.1496875)
\psdots[dotsize=0.04](3.8484375,0.1096875)
\psdots[dotsize=0.04](3.9084375,0.1896875)
\psdots[dotsize=0.04](3.8684375,0.1896875)
\psdots[dotsize=0.04](3.8084376,0.1496875)
\psdots[dotsize=0.04](3.8084376,0.0096875)
\psdots[dotsize=0.04](3.8884375,-0.0103125)
\psdots[dotsize=0.04](3.9084375,0.1496875)
\psdots[dotsize=0.04](4.0084376,0.2096875)
\psdots[dotsize=0.04](4.0284376,0.1296875)
\psdots[dotsize=0.04](4.0284376,0.0296875)
\psdots[dotsize=0.04](3.9884374,0.0096875)
\psdots[dotsize=0.04](3.9284375,-0.0903125)
\psdots[dotsize=0.04](3.8084376,-0.1303125)
\psdots[dotsize=0.04](3.7084374,-0.0703125)
\psdots[dotsize=0.04](3.7484374,0.0096875)
\psdots[dotsize=0.04](3.9084375,-0.0903125)
\psdots[dotsize=0.04](3.9684374,-0.1303125)
\psdots[dotsize=0.04](3.9684374,0.0496875)
\psdots[dotsize=0.04](3.9284375,0.1096875)
\psdots[dotsize=0.04](3.9684374,0.2496875)
\psdots[dotsize=0.04](4.0484376,0.1896875)
\psdots[dotsize=0.04](4.0484376,-0.0503125)
\psdots[dotsize=0.04](4.0084376,-0.1103125)
\psdots[dotsize=0.04](3.9284375,-0.1903125)
\psdots[dotsize=0.04](3.8484375,-0.1903125)
\psdots[dotsize=0.04](3.7884376,-0.1503125)
\psdots[dotsize=0.04](3.7284374,-0.1703125)
\psdots[dotsize=0.04](3.7484374,-0.1903125)
\psdots[dotsize=0.04](3.8284376,-0.2103125)
\psdots[dotsize=0.04](3.9284375,-0.2703125)
\psdots[dotsize=0.04](4.0084376,-0.2103125)
\psdots[dotsize=0.04](4.0484376,-0.1103125)
\psdots[dotsize=0.04](4.0484376,0.0296875)
\psdots[dotsize=0.04](3.7884376,0.0696875)
\psdots[dotsize=0.04](3.8284376,-0.0703125)
\psdots[dotsize=0.04](3.7684374,-0.0503125)
\psdots[dotsize=0.04](3.9684374,0.1496875)
\psdots[dotsize=0.04](3.7484374,-0.2503125)
\psdots[dotsize=0.04](4.0084376,0.2696875)
\psdots[dotsize=0.04](4.0884376,0.2696875)
\psdots[dotsize=0.04](4.0884376,0.1896875)
\psdots[dotsize=0.04](4.1684375,0.2496875)
\psdots[dotsize=0.04](4.1484375,0.2496875)
\psdots[dotsize=0.04](4.1884375,0.0696875)
\psdots[dotsize=0.04](4.1684375,0.1696875)
\psdots[dotsize=0.04](4.0884376,0.1296875)
\psdots[dotsize=0.04](4.1084375,0.0296875)
\psdots[dotsize=0.04](4.1084375,0.0296875)
\psdots[dotsize=0.04](4.0484376,0.0696875)
\psdots[dotsize=0.04](4.1884375,-0.0703125)
\psdots[dotsize=0.04](4.1084375,-0.1103125)
\psdots[dotsize=0.04](4.0684376,-0.0903125)
\psdots[dotsize=0.04](4.1484375,-0.0503125)
\psdots[dotsize=0.04](4.2084374,-0.0103125)
\psdots[dotsize=0.04](4.2084374,-0.1303125)
\psdots[dotsize=0.04](4.1684375,-0.1703125)
\psdots[dotsize=0.04](4.0884376,-0.2103125)
\psdots[dotsize=0.04](4.0484376,-0.2303125)
\psdots[dotsize=0.04](4.1284375,-0.2703125)
\psdots[dotsize=0.04](4.2284374,-0.2303125)
\psdots[dotsize=0.04](3.9684374,-0.0303125)
\psdots[dotsize=0.04](3.8484375,-0.2703125)
\psdots[dotsize=0.04](3.7484374,0.0896875)
\psdots[dotsize=0.04](3.7484374,0.2496875)
\psdots[dotsize=0.04](3.7884376,0.2896875)
\psdots[dotsize=0.04](4.1884375,0.1696875)
\psdots[dotsize=0.04](4.2084374,0.2496875)
\psdots[dotsize=0.04](4.2884374,0.2496875)
\psdots[dotsize=0.04](4.4684377,0.2096875)
\psdots[dotsize=0.04](4.5084376,-0.0503125)
\psdots[dotsize=0.04](4.3284373,0.0096875)
\psdots[dotsize=0.04](4.4084377,0.1296875)
\psdots[dotsize=0.04](4.6084375,0.0496875)
\psdots[dotsize=0.04](5.0484376,-0.0703125)
\psdots[dotsize=0.04](4.8884373,0.1096875)
\psdots[dotsize=0.04](4.7284374,0.1496875)
\psdots[dotsize=0.04](4.7484374,0.2096875)
\psdots[dotsize=0.04](4.9484377,0.1896875)
\psdots[dotsize=0.04](4.5484376,0.2896875)
\psdots[dotsize=0.04](4.8084373,-0.0503125)
\psdots[dotsize=0.04](4.3684373,-0.2103125)
\psdots[dotsize=0.04](4.5284376,-0.2503125)
\psdots[dotsize=0.04](4.7684374,-0.2303125)
\psdots[dotsize=0.04](4.6484375,-0.0703125)
\psdots[dotsize=0.04](4.9684377,-0.2303125)
\psdots[dotsize=0.04](5.0084376,0.0496875)
\psdots[dotsize=0.04](4.8684373,0.2696875)
\psdots[dotsize=0.04](5.0684376,0.2696875)
\psdots[dotsize=0.04](5.2284374,0.2296875)
\psdots[dotsize=0.04](5.4284377,0.2696875)
\psdots[dotsize=0.04](5.3884373,0.0896875)
\psdots[dotsize=0.04](5.3484373,-0.0903125)
\psdots[dotsize=0.04](5.0884376,0.0696875)
\psdots[dotsize=0.04](5.2284374,0.0896875)
\psdots[dotsize=0.04](5.1884375,-0.1503125)
\psdots[dotsize=0.04](5.0684376,-0.2103125)
\psdots[dotsize=0.04](5.2884374,-0.2503125)
\psdots[dotsize=0.04](5.4084377,-0.2503125)
\psdots[dotsize=0.04](5.5084376,0.2296875)
\psdots[dotsize=0.04](5.5884376,0.2496875)
\psdots[dotsize=0.04](5.5484376,0.2096875)
\psdots[dotsize=0.04](5.4684377,0.1296875)
\psdots[dotsize=0.04](5.5684376,0.0896875)
\psdots[dotsize=0.04](5.6284375,0.1696875)
\psdots[dotsize=0.04](5.5884376,0.1696875)
\psdots[dotsize=0.04](5.5284376,0.1296875)
\psdots[dotsize=0.04](5.5284376,-0.0103125)
\psdots[dotsize=0.04](5.6084375,-0.0303125)
\psdots[dotsize=0.04](5.6284375,0.1296875)
\psdots[dotsize=0.04](5.7284374,0.1896875)
\psdots[dotsize=0.04](5.7484374,0.1096875)
\psdots[dotsize=0.04](5.7484374,0.0096875)
\psdots[dotsize=0.04](5.7084374,-0.0103125)
\psdots[dotsize=0.04](5.6484375,-0.1103125)
\psdots[dotsize=0.04](5.5284376,-0.1503125)
\psdots[dotsize=0.04](5.4284377,-0.0903125)
\psdots[dotsize=0.04](5.4684377,-0.0103125)
\psdots[dotsize=0.04](5.6284375,-0.1103125)
\psdots[dotsize=0.04](5.6884375,-0.1503125)
\psdots[dotsize=0.04](5.6884375,0.0296875)
\psdots[dotsize=0.04](5.6484375,0.0896875)
\psdots[dotsize=0.04](5.6884375,0.2296875)
\psdots[dotsize=0.04](5.7684374,0.1696875)
\psdots[dotsize=0.04](5.7684374,-0.0703125)
\psdots[dotsize=0.04](5.7284374,-0.1303125)
\psdots[dotsize=0.04](5.6484375,-0.2103125)
\psdots[dotsize=0.04](5.5684376,-0.2103125)
\psdots[dotsize=0.04](5.5084376,-0.1703125)
\psdots[dotsize=0.04](5.4484377,-0.1903125)
\psdots[dotsize=0.04](5.4684377,-0.2103125)
\psdots[dotsize=0.04](5.5484376,-0.2303125)
\psdots[dotsize=0.04](5.6484375,-0.2903125)
\psdots[dotsize=0.04](5.7284374,-0.2303125)
\psdots[dotsize=0.04](5.7684374,-0.1303125)
\psdots[dotsize=0.04](5.7684374,0.0096875)
\psdots[dotsize=0.04](5.5084376,0.0496875)
\psdots[dotsize=0.04](5.5484376,-0.0903125)
\psdots[dotsize=0.04](5.4884377,-0.0703125)
\psdots[dotsize=0.04](5.6884375,0.1296875)
\psdots[dotsize=0.04](5.4684377,-0.2703125)
\psdots[dotsize=0.04](5.7284374,0.2496875)
\psdots[dotsize=0.04](5.8084373,0.2496875)
\psdots[dotsize=0.04](5.8084373,0.1696875)
\psdots[dotsize=0.04](5.8884373,0.2296875)
\psdots[dotsize=0.04](5.8684373,0.2296875)
\psdots[dotsize=0.04](5.9084377,0.0496875)
\psdots[dotsize=0.04](5.8884373,0.1496875)
\psdots[dotsize=0.04](5.8084373,0.1096875)
\psdots[dotsize=0.04](5.8284373,0.0096875)
\psdots[dotsize=0.04](5.8284373,0.0096875)
\psdots[dotsize=0.04](5.7684374,0.0496875)
\psdots[dotsize=0.04](5.9084377,-0.0903125)
\psdots[dotsize=0.04](5.8284373,-0.1303125)
\psdots[dotsize=0.04](5.7884374,-0.1103125)
\psdots[dotsize=0.04](5.8684373,-0.0703125)
\psdots[dotsize=0.04](5.9284377,-0.0303125)
\psdots[dotsize=0.04](5.9284377,-0.1503125)
\psdots[dotsize=0.04](5.8884373,-0.1903125)
\psdots[dotsize=0.04](5.8084373,-0.2303125)
\psdots[dotsize=0.04](5.7684374,-0.2503125)
\psdots[dotsize=0.04](5.8484373,-0.2903125)
\psdots[dotsize=0.04](5.9484377,-0.2503125)
\psdots[dotsize=0.04](5.6884375,-0.0503125)
\psdots[dotsize=0.04](5.5684376,-0.2903125)
\psdots[dotsize=0.04](5.4684377,0.0696875)
\psdots[dotsize=0.04](5.4684377,0.2296875)
\psdots[dotsize=0.04](5.5084376,0.2696875)
\psdots[dotsize=0.04](5.9084377,0.1496875)
\psdots[dotsize=0.04](5.9284377,0.2296875)
\psdots[dotsize=0.04](6.0084376,0.2296875)
\psdots[dotsize=0.04](6.1884375,0.1896875)
\psdots[dotsize=0.04](6.2284374,-0.0703125)
\psdots[dotsize=0.04](6.0484376,-0.0103125)
\psdots[dotsize=0.04](6.1284375,0.1096875)
\psdots[dotsize=0.04](6.3284373,0.0296875)
\psdots[dotsize=0.04](6.7684374,-0.0903125)
\psdots[dotsize=0.04](6.6084375,0.0896875)
\psdots[dotsize=0.04](6.4484377,0.1296875)
\psdots[dotsize=0.04](6.4684377,0.1896875)
\psdots[dotsize=0.04](6.6684375,0.1696875)
\psdots[dotsize=0.04](6.2684374,0.2696875)
\psdots[dotsize=0.04](6.5284376,-0.0703125)
\psdots[dotsize=0.04](6.0884376,-0.2303125)
\psdots[dotsize=0.04](6.2484374,-0.2703125)
\psdots[dotsize=0.04](6.4884377,-0.2503125)
\psdots[dotsize=0.04](6.3684373,-0.0903125)
\psdots[dotsize=0.04](6.6884375,-0.2503125)
\psdots[dotsize=0.04](6.7284374,0.0296875)
\psdots[dotsize=0.04](6.5884376,0.2496875)
\psdots[dotsize=0.04](6.7884374,0.2496875)
\psdots[dotsize=0.04](6.9484377,0.2096875)
\psdots[dotsize=0.04](7.1484375,0.2496875)
\psdots[dotsize=0.04](7.1084375,0.0696875)
\psdots[dotsize=0.04](7.0684376,-0.1103125)
\psdots[dotsize=0.04](6.8084373,0.0496875)
\psdots[dotsize=0.04](6.9484377,0.0696875)
\psdots[dotsize=0.04](6.9084377,-0.1703125)
\psdots[dotsize=0.04](6.7884374,-0.2303125)
\psdots[dotsize=0.04](7.0084376,-0.2703125)
\psdots[dotsize=0.04](7.1284375,-0.2703125)
\rput(4.0528126,-1.1053125){\small kompressie}
\psline[linewidth=0.04cm](3.9484375,-0.9903125)(3.9484375,-0.6303125)
\psline[linewidth=0.04cm](2.2484374,-0.6303125)(5.7484374,-0.6303125)
\psline[linewidth=0.04cm,arrowsize=0.05291667cm 2.0,arrowlength=1.4,arrowinset=0.4]{->}(2.2684374,-0.6303125)(2.2884376,-0.3303125)
\psline[linewidth=0.04cm,arrowsize=0.05291667cm 2.0,arrowlength=1.4,arrowinset=0.4]{->}(3.9484375,-0.6103125)(3.9684374,-0.3103125)
\psline[linewidth=0.04cm,arrowsize=0.05291667cm 2.0,arrowlength=1.4,arrowinset=0.4]{->}(5.7284374,-0.6303125)(5.7484374,-0.3303125)
\psline[linewidth=0.04cm](4.7284374,0.9896875)(4.7284374,0.6296875)
\psline[linewidth=0.04cm](3.0284376,0.6296875)(6.5284376,0.6296875)
\psline[linewidth=0.04cm,arrowsize=0.05291667cm 2.0,arrowlength=1.4,arrowinset=0.4]{->}(3.0484376,0.6296875)(3.0684376,0.3296875)
\psline[linewidth=0.04cm,arrowsize=0.05291667cm 2.0,arrowlength=1.4,arrowinset=0.4]{->}(4.7284374,0.6096875)(4.7484374,0.3096875)
\psline[linewidth=0.04cm,arrowsize=0.05291667cm 2.0,arrowlength=1.4,arrowinset=0.4]{->}(6.5084376,0.6296875)(6.5284376,0.3296875)
\rput(4.685625,1.1146874){\small verdunning}
\rput(0.5132812,0.2346875){\small stem-}
\rput(0.4678125,-0.1053125){\small vurk}
\psline[linewidth=0.04cm](0.9284375,0.0096875)(1.4084375,0.0096875)
\rput(9.066093,0.2146875){\small kolom van lug}
\rput(9.035313,-0.1053125){\small voor die stemvurk}
\end{pspicture}
}
\end{center}
\caption{Klankgolwe is drukgolwe wat 'n medium benodig om deur te beweeg.}
 \end{figure}       

\Tip{Klankgolwe is drukgolwe, wat beteken dat hoogdruk (kompressie) en laagdruk (verdunning) areas gevorm word deur die klankbron wat vibreer. Die kompressies en verdunnings onstaan as gevolg van longitudinale vibrasies in die bron wat dan longitudinale klankgolwe veroorsaak.
}
	\par


\begin{activity}{Bou jou eie telefoon} 
Het jy al oot gewonder hoe jy blikkies kan gebruik om 'n telefoon te maak. Probeer die volgende!
Wat jy benodig:
\begin{itemize}
 \item Twee blikkies of papier koppies
  \item Tou
  \item Tandestokkies of klein stokkies
\end{itemize}

Probeer die volgende:
\begin{enumerate}[noitemsep, label=\textbf{\arabic*}. ] 
\item Bind 'n tandestokkie aan die ente van die tou. 
\item Druk die tandestokkies deur die onderkante van die blikkies. Trek die tou styf sodat die tandestokkies aan die onderkant van die blikkies le. (Jy sal dalk verskillende blikkies of papier koppies met ander draad of tou wil probeer om te sien watse kombinasie werk die beste.)
\item Hou die tou styf en praat in een van die blikkies. Die persoon aan die anderkant behoord jou te kan hoor. Hoekom moet die tou styf wees?
\item Probeer om meer as twee mense in die geselskap te he deur 'n tou in die middel van eerste tou te bind. Kan almal mekaar hoor?
\end{enumerate}
Klank benodig 'n medium om deur te beweeg. Gewoontlik beweeg dit deur lug, maar dit kan baie vinnger en verder deur die tou beweeg. Die tou moet styf wees anders kan die klankgolf nie daardeur beweeg nie. Die blikkie help om die klank aan die ontvanger se kant te versterk.	
\end{activity}

    \label{m38799*cid3}
            \section{Spoed van klank}
            \nopagebreak
      \label{m38799*id183885}Die spoed van klank is afhanklik van die medium waardeur dit beweeg. Klank beweeg vinniger in vastestowwe as in vloeistowwe en vinniger in vloeistowwe as in 'n gas. Dit is omdat die digtheid van vastestowwe ho\"{e}r is as die van vloeistowwe, wat beteken die molekules is nader aanmekaar. Dus, kan die klank makliker oorgedra word.\par 
      \label{m38799*id183891}Die spoed van klank hang ook af van die temperatuur van die medium. Hoe warmer die medium hoe vinniger beweeg die deeltjies, hoe vinnger die deeltjies beweeg hoe viniger sal klank deur die medium voortgeplant word. Wanneeer ons 'n voorwerp warm maak, verkry die deeltjies meer kinetiese energie en sal vinniger vibreer en vinniger beweeg. Klank kan dus makliker in warmer mediums voortgeplant word.\par  

 
      \label{m38799*id183897} Aangesien klankgolwe drukgolwe is, sal die spoed van klank in 'n medium afhanklik van die druk van die medium. Op seevlak is die lugdruk ho\"{e}r as op 'n berg. Klank sal dus vinniger beweeg op seevlak, waar die druk ho\"{e}r is, as op plekker wat ho\"{e}r as seevlak is.  

\Definition{   \label{id2448438}\textbf{ Spoed van klank }} { \label{m38799*meaningfhsst!!!underscore!!!id164}
     Die spoed van klank in lug op seevlak by, 'n temperatuur van $21{}^{\circ }\text{C}$, onder normale atmosferiese toestande, is $344\phantom{\rule{2pt}{0ex}}\text{m}\ensuremath{\cdot}\text{s}{}^{-1}$. 
       } 

\begin{i_experiment}{Bepaal die spoed van klank in lug}
Die spoed van klank kan gemeet word deur kennis te dra van die feit dat die spoed van lig vinnger as die van klank. Lig beweeg teen ongeveer 300 000 $\text{m}\cdot\text{s}^{-1}$ (jy sal meer hieroor leer in 'n volgende hoofstuk) terwyl klank teen ongeveer 300 $\text{m}\cdot\text{s}^{-1}$ beweeg. Die groot verskil in spoed beteken dat as lig en klank van 'n bron 300 $\text{m}$ ver afkomstig is, die lig byna onmiddellik waargeneem sal word, maar die klank sal eers 'n sekonde later gehoor word. As 'n afsitter se pistool gevuur word, sal die rook gesien word voor die klank 'n tydjie later gehoor word. As jy weet wat die afstand is tot by die afsitter en jy weet hoe lank die klank vat om jou te bereik,  kan jy die spoed van die klank bereken (afstand gedeel deur tyd). Jy het nie 'n geweer nodig nie, maar enige iets wat jy kan sien 'n harde geluid maak.    


Wat jy benodig:
\begin{itemize}
 \item Afsittersgeveer of enige iets wat 'n harde geluid sal maak en sigbaar is. 
  \item Stophorlosie
  \end{itemize}

Probeer die volgende:
\begin{enumerate}[noitemsep, label=\textbf{\arabic*}. ] 
\item Meet die presiese afstand tussen twee punte in 'n reguitlyn.
\item Iemand by die punt van oosprong van die klank.
\item Iemand by die ander punt met 'n stophorlosie.
\item Die persoon met die stophorlosie moet die horlosie begin as hy die ander persoon klank sien maak en die horlosie stop as hy die klank hoor. Doen dit 'n paar keer en skryf die tye neer. 

\end{enumerate}
Jy kan nou die spoed van klank bereken deur die afstand deur die tyd te deel. Onthou om in S.I. eenhede te werk, afstand in meter en tyd in secondes. As jy meer as een lesing geneem het, neem die gemiddeld van al die lesings. Gebruik die gemiddelde tye om die spoed te bereken: 

\begin{equation*}
 v = \frac{D}{t}
\end{equation*}
Bespreek wat die berekende spoed mag be\"{i}nvloed.
\end{i_experiment}

\begin{table}[H]
 \begin{tabular}{|c|c|}\hline
Materiaal	& $v$ ($\text{m}\cdot \text{s}^{-1}$)\\ \hline \hline
aluminium	&6420 \\ \hline
baksteen	&3650 \\ \hline
koper	&4760	 	 \\ \hline
glas &5100	 \\ \hline 	 	 
goud	&3240	 \\ \hline 	
lood	&2160	 \\ \hline 
water, see	&1531 \\ \hline
lug, 0 ℃	&331 \\ \hline
lug, 20 ℃	&343 \\ \hline
\end{tabular}
\caption{Die spoed van klank in verskillde materiale.}
\end{table}

         \subsection*{SONAR}
            \nopagebreak
      \label{m38800*id185202}
\begin{minipage}{.5\textwidth}
      	\begin{figure}[H] % horizontal\label{m38800*id185205}
    \begin{center}

\includegraphics[width=.8\textwidth]{photos/sonar_NOAANationalOceanService_flickr.jpg}
% 
% \scalebox{0.8} % Change this value to rescale the drawing.
% {
% \begin{pspicture}(0,-3.93)(11.42,3.91)
% \definecolor{color78b}{rgb}{0.6,0.6,0.6}
% \psbezier[linewidth=0.04](0.52,-1.6283582)(3.3282945,-0.05)(4.980751,-1.0964925)(5.8,-2.11)(6.619249,-3.1235075)(8.99,-3.91)(10.64,-3.71)
% \psbezier[linewidth=0.04,fillstyle=solid,fillcolor=color78b](1.94,2.61)(1.9801459,1.77)(1.84,1.17)(2.44,1.15)(3.04,1.13)(6.68,1.01)(7.358321,1.29)(8.036642,1.57)(9.52,2.81)(8.58,2.59)(7.64,2.37)(1.96,2.43)(1.92,2.59)
% \psline[linewidth=0.04cm](2.52,2.49)(2.52,3.27)
% \psline[linewidth=0.04cm](2.52,3.27)(4.44,3.27)
% \psline[linewidth=0.04cm](4.44,3.27)(4.44,2.45)
% \psline[linewidth=0.04cm](3.14,3.81)(3.14,3.27)
% \psline[linewidth=0.04cm](3.76,3.81)(3.76,3.27)
% \psellipse[linewidth=0.04,dimen=outer](3.45,3.82)(0.33,0.09)
% \pscircle[linewidth=0.04,dimen=outer,fillstyle=solid,fillcolor=color78b](2.74,3.05){0.14}
% \pscircle[linewidth=0.04,dimen=outer,fillstyle=solid,fillcolor=color78b](3.18,3.05){0.14}
% \pscircle[linewidth=0.04,dimen=outer,fillstyle=solid,fillcolor=color78b](3.7,3.05){0.14}
% \pscircle[linewidth=0.04,dimen=outer,fillstyle=solid,fillcolor=color78b](4.12,3.05){0.14}
% \pscircle[linewidth=0.04,dimen=outer,fillstyle=solid,fillcolor=color78b](2.96,2.77){0.14}
% \pscircle[linewidth=0.04,dimen=outer,fillstyle=solid,fillcolor=color78b](3.44,2.77){0.14}
% \pscircle[linewidth=0.04,dimen=outer,fillstyle=solid,fillcolor=color78b](3.9,2.75){0.14}
% \rput(7.5798435,2.24){SAS Sonar}
% \psframe[linewidth=0.04,dimen=outer,fillstyle=solid](3.24,1.15)(3.02,1.05)
% \psframe[linewidth=0.04,dimen=outer,fillstyle=solid](4.58,1.13)(4.36,1.03)
% \pscustom[linewidth=0.04]
% {
% \newpath
% \moveto(9.94,1.31)
% }
% \psline[linewidth=0.04cm](0.0,1.61)(11.4,1.61)
% \psline[linewidth=0.04cm](3.12,1.15)(3.76,-0.85)
% \psline[linewidth=0.04cm](3.76,-0.85)(4.48,1.13)
% \psline[linewidth=0.04cm](3.2972062,0.25496995)(3.4476187,0.11434252)
% \psline[linewidth=0.04cm](3.4476187,0.11434252)(3.4875562,0.316345)
% \psline[linewidth=0.04cm](4.229458,0.10226333)(4.187863,0.30393106)
% \psline[linewidth=0.04cm](4.187863,0.30393106)(4.0386105,0.1620733)
% \rput(3.2217188,-1.2){seabed}
% \rput(1.949375,0.62){transmitter}
% \rput(5.47,0.62){receiver}
% \psline[linewidth=0.04cm,arrowsize=0.05291667cm 2.0,arrowlength=1.4,arrowinset=0.4]{->}(2.56,0.81)(3.0,1.03)
% \psline[linewidth=0.04cm,arrowsize=0.05291667cm 2.0,arrowlength=1.4,arrowinset=0.4]{->}(4.92,0.79)(4.64,1.03)
% %\psline[linewidth=0.04cm,arrowsize=0.05291667cm %2.0,arrowlength=1.4,arrowinset=0.4]{->}(6.36,-0.09)(6.36,-0.09)
% \rput(7.020781,-0.46){sea}
% \end{pspicture}
% }
\end{center}

 \end{figure}       
\end{minipage}
\begin{minipage}{.5\textwidth}
      \label{m38800*id185212}Skepe maak gebruik van die weerkaatsingseienskappe van klankgolwe om die diepte van die oseaan te bereken. 'n Klankgolf gestuur en weerkaats van die seebodem af. Die spoed van klank in water is bekend en die tyd wat dit neem vir 'n klankgolf om terug te keer kan gemeet word. So kan die afstand na die bodem van die oseaan bereken word. Dit word sonar genoem, afkomstig van die engels vir \textbf{So}und \textbf{N}avigation \textbf{A}nd \textbf{R}anging.\par    
      \label{m38800*uid13}
     \end{minipage}
            

\begin{wex}{SONAR}
{'n Skip stuur 'n sein na die bodem van die oseaan om die diepte te bereken. Die spoed van klank in water is is 1450 $\text{m.s^{-1}}$ As die weerkaatse sein 1.5 sekondes later ontvang word, hoe diep is die oseaan by daardie punt?}
{
\westep{Identifiseer wat gegee word en wat gevra word:}
\begin{eqnarray*}
s &=& 1450 \ \text{m.s^{-1}}\\
t &=& 1,5 \ \text{s} \ \text{soontoe \ en \ terug}\\
\therefore t &= & 0,75 \ \text{s} \ \text{een \ rigting}\\
d &=& ?
\end{eqnarray*}

\westep{Bereken die afstand:}
\begin{eqnarray*}
\text{Afstand} &=& \text{spoed} \times \text{tyd} \\
d &=& s \times t \\
&=& 1450 m.s^{-1} \times 0,75 s \\
&=& 1087,5 \ \text{m}
\end{eqnarray*}

}
\end{wex}
    \noindent
    \label{m38800*eip-656}
          
\section*{Eggoplasing}
            \nopagebreak
        \label{m38800*id185251}Diere soos dolfyne en vlermuise maak gebuik van klankgolwe om hulle pad te vind. Net soos skepe, gebruik vlermuise ook sonar om afstande na voorwerpe te bereken en sodoende hindernisse te vermy. Klankgolwe word uitgestuur en weerkaats van voorwerpe af rondom die vlermuis. Vlermuise en dolfyne maak gebruik van die weerkaatse klanke om 'n prentjie te vorm van wat om hulle is. Dit word eggoplasing genoem.\par



\section{Eienskappe van 'n Klankgolf}
            \nopagebreak
      \label{m38799*id183478}Aangesien klank 'n golf is, kan ons die eienskappe van klank in verband bring met di\'{e} golwe. Die basiese eienskappe van klank is toonhoogte en hardheid.\par 

    \begin{figure}[h!tbp]
\begin{center}
%\scalebox{0.8}
%{
\begin{pspicture}(-5,-1)(5,1)%\psgrid
\psplot[xunit=0.0055,]{-360}{360}{x sin }
\psline[linestyle=dashed](-2,0)(2.5,0)
\psline[linestyle=dashed](-2,1)(2.5,1)
\psline[linestyle=dashed](-2,-1)(2.5,-1)
\psline{<->}(-2,-1.2)(-2,1.2)
\rput(-3.5,0){Klank A}
\end{pspicture}
%}
\end{center}

\begin{center}
%\scalebox{0.8}
%{
\begin{pspicture}(-5,-1)(5,2)%\psgrid
\psplot[xunit=0.0055,]{-360}{360}{-1 0.5 x mul sin mul}
\psline[linestyle=dashed](-2,0)(2.5,0)
\psline[linestyle=dashed](-2,1)(2.5,1)
\psline[linestyle=dashed](-2,-1)(2.5,-1)
\psline{<->}(-2,-1.2)(-2,1.2)
\rput(-3.5,0){Klank B}
\end{pspicture}
%}
\end{center}

\begin{center}
%\begin{minipage}{0.3\textwidth}
%\scalebox{0.8}
%{
\begin{pspicture}(-5,-1)(5,2)%\psgrid
\psplot[xunit=0.0055,]{-360}{360}{-1.5 0.5 x mul sin mul}
\psline[linestyle=dashed](-2,0)(2.5,0)
\psline[linestyle=dashed](-2,1)(2.5,1)
\psline[linestyle=dashed](-2,-1)(2.5,-1)
\psline{<->}(-2,-1.2)(-2,1.2)
\rput(-3.5,0){Klank C}
\end{pspicture}
%}
%\end{minipage}
\end{center}
\caption{Toonhoogte en hardheid van klank. Klank B het 'n \emph{laer} toonhoogte (laer frekwensie) as Klank A en is \emph{sagter} (kleiner amplitude) as Klank C.}\label{fig:pitchetc}
\end{figure}
      \label{m38799*uid2}
            \subsubsection{Toonhoogte}
            \nopagebreak
            
Die frekwensie van 'n klankgold is wat jy hoor as toonhoogte. 'n Ho\"er frekwensie klank het 'n ho\"er toonhoogte en 'n laer frekwensie klank het 'n laer toonhoogte. In Figuur~\ref{fig:pitchetc} het Klank A 'n ho\"er toonhoogte as Klank B. 'n Vo\"eltjie se sang sal byvoorbeeld 'n laer toonhoogte h\^e as die brul van 'n leeu.\par

Die menslike oor kan 'n wye verskeidenheid frekwensies hoor. Frekwensies van 20 tot 20 000 Hz is hoorbaar vir mense. Enige klank met 'n frekwensie onder 20 Hz word genoem \textbf{infraklank} en 'n klank met 'n frekwensie bo 20 000 Hz word genoem \textvf{ultraklank}.\par


Tabel~\ref{p:wsl:s11:rangeoff} lys die frekwensiebereik van sommige algemene diere rin vergeleke met mense.

\begin{table}[htbp]
\begin{center}
\caption{Frekwensiebereik}
\label{p:wsl:s11:rangeoff}
\begin{tabular}{|l|c|c|}\hline
&onderste frekwensie (Hz) & boonste frekwensie (Hz)\\\hline\hline
Mense & 20 & 20 000\\\hline
Honde & 50 & 45 000\\\hline
Katte & 45 & 85 000\\\hline
Vlermuise & 20 & 120 000\\\hline
Dolfyne & 0,25 & 200 000\\\hline
Olifante & 5 & 10 000\\\hline
\hline
\end{tabular}
\end{center}
\end{table}
    \par
\begin{activity}{Frekwensiebereik}
\nopagebreak
Gebruik die informasie in Tabel \ref{p:wsl:s11:rangeoff} om die laagste en hoogste golflengtes wat elke spesie kan hoor te bereken. Aanvaar dat die spoed van klank in lug 344 $m \cdot s^{-2}$ is.
\end{activity}
 
\subsubsection{Hardheid}
\nopagebreak

Die amplitude van 'n klankgolf beheer sy hardheid of volume. 'n Groter amplitude  beteken 'n harder klank en 'n kleiner ampitude beteken 'n sagter klank. In Figuur~\ref{fig:pitchetc} is Klank C harder as Klank B. Die vibrasie van 'n bron bepaal die amplitude van die golf wat geskep word. Energie word in die medium in gestraal deur die vibrasie. Meer energie stem ooreen met 'n groter amplitude want die molekules beweeg verder heen en weer. \par

Die hardheid van 'n klank word ook bepaal deur die sensitiwiteit van die oor. Die menslike oor is meer sensitief vir sekere frekwensies as ander. Die volume wat ons hoor is dus afhanklik van beide die amplitude van 'n klank asook die frekwensie van die klank.\par


\begin{exercises}{Klank, frekwensie en amplitude}
\nopagebreak
Bestudeer die volgende diagram wat 'n musieknoot voorstel
Skets die diagram vir 'n noot wat
\begin{enumerate}[noitemsep, label=\textbf{\arabic*}. ] 
\item 'n ho\"er toonhoogte het
\item harder is
\item sagter is
\end{enumerate}



\begin{figure}[H] % horizontal\label{m38799*id184017}
    \begin{center}
    \begin{pspicture}(-3,-1)(5,1)%\psgrid
\psplot[xunit=0.0055,]{-360}{360}{x sin }
\psline[linestyle=dashed](-2,0)(2.5,0)
\psline[linestyle=dashed](-2,1)(2.5,1)
\psline[linestyle=dashed](-2,-1)(2.5,-1)
\psline{<->}(-2,-1.2)(-2,1.2)
\end{pspicture}
    \end{center}
 \end{figure}               
 \par 
  \label{m38799**end}
\par \raisebox{-5 pt}{\includegraphics[width=0.5cm]{col11305.imgs/summary_www.png}} Vind die antwoorde met die kortkodes:
 \par \begin{tabular}[h]{cccccc}
 (1.) l2o  & \end{tabular}

\end{exercises}

\begin{activity}{Vergelyk instrumente wat klank genereer}
Die grootte en vorm van instrument be\"invloed die klanke wat hulle produseer. Kry 'n paar instrumente wat verskillende fisiese eienskappe het en vergelyk hulle klanke. Jy kan:\par\hspace{1cm}
\begin{minipage}{.28\columnwidth}
Die klank van vuvuzelas met verskillende groottes te vergelyk deur deur hulle te blaas.
\par
Jy sal 'n paar verskillende vuvuzelas moet opspoor. Maak beurte om deur hulle te blaas, een op 'n slag en maak 'n nota van watter jy dink is harder (amplitude)en watter het ho\"er toonhoogte (frekwensie).
\end{minipage}\hspace{.05\columnwidth}
\begin{minipage}{.28\columnwidth}
Vergelyk die klanke wat jy hoor as jy verskillende stemvurke tik.
\par
Jy sal 'n paar verskillende stemvurke benodig. Maak beurte om hulle te tik, een op 'n slag en maak 'n nota van watter jy dink is harder (amplitude) en watter het ho\"er toonhoogte (frekwensie).
\end{minipage}\hspace{.05\columnwidth}
\begin{minipage}{.28\columnwidth}

Gebruik 'n sein generator om klanke met verskillende frekwensies en amplitudes te genereer en gebruik 'n ossilloskoop om die verskillende seine te bestudeer.

\end{minipage}

\end{activity}

\Note{The display of the oscilloscope will show you a transverse wave pattern. This does not mean that klank waves are transverse waves but just shows that the pressure being measured is fluctuating because of a pressure wave.}


\section*{Intensiteit van klank}
\nopagebreak

\par
Intensiteit is een indikator van amplitude. Intensiteit is die energie wat deur 'n eenheid oppervlakte elke sekonde gestraal word.\par

Die eenheid van intensiteit is die \textbf{desibel} (simbool: dB).

\begin{table}[H]
\begin{center}
\begin{tabular}{|l|c|c|}\hline
\textbf{Bron}&\textbf{Intensiteit} (dB) & \textbf{Keer groter as gehoordrumpel}\\\hline
Vuurpyl lansering &180 & $10^{18}$\\
Straal vliegtuig & 140 & $10^{14}$ \\
Peindrumpel & 120 & $10^{12}$\\
Rock orkes & 110 & $10^{11}$\\
%Subway Train & 90 & $10^{9}$\\
Fabriek & 80 & $10^{8}$\\
Stadsverkeer & 70 & $10^{7}$\\
Normale gesprek & 60 & $10^{6}$\\
Biblioteek & 40 & $10^{4}$\\
Fluister & 20 & $10^{2}$\\
Gehoordrumpel & 0 & 0\\
\hline
\end{tabular}
\end{center}
\caption{Voorbeelde van klank intensiteite.}
\label{p:wsl:s11:intensity}
\end{table}

Vuvuzelas is prominent by sokker wedstryde in Suid-Afrika. Die intensiteit van die klank van 'n vuvuzela is afhanklik van die afstand na die vuvuzela toe. In Tabel~\ref{table:vuvuzelas} kan jy sien hoe die intensiteit verskil

\begin{minipage}{.5\textwidth}
\begin{table}[H]
\begin{tabular}{ccccc}\hline
Frekwensie (Hz) & In oor & Oorskulp & 1 m & 2 m \\ \hline
125&   36  	&62	 &38	& 35 \\ \hline
250&	92 &	106	& 82	&	 85 \\ \hline
500&	103 & 121&	 102&	 101 		\\ \hline
1 000&	106 & 122&	 108&	 100 	\\  \hline
2 000&	101 & 122&	 110&	 101 	\\ \hline
4 000&	97 & 109&	 110&	 102 	\\ \hline
5 000&	93 & 111&	 109&	 100 		\\ \hline
8 000&  87 & 110&	 107&		 98 	\\ \hline
\end{tabular}
\label{table:vuvuzelas}
\caption{Gemiddelde vuvuzela intensiteit metings by verskillende frekwensies by 4 afsonderlike afstande van die opening van die vuvuzela af (dBA). Tabel geneem uit South African Medical Journal (Cape Town, South Africa) \textbf{100} (4): 192}
\end{table}
\end{minipage}
\begin{minipage}{.5\textwidth}
\begin{center}
\textbf{Vuvuzelas in aksie}\par
\includegraphics[width=.8\columnwidth]{photos/vuvuzelas_dundas_football_club.jpg}\par
Foto deur Dundas Football Club op Flickr.	
\end{center}
\end{minipage}

            
\section{Ultraklank}
            \nopagebreak
Ultraklank is klank met 'n frekwensie meer as 20kHz. Sommige diere, soos honde, dolfyne en vlermuise, se gehoorbereik is ho\"er as di\'e van mense en kan ultraklank hoor.
    % \textbf{m38800*eip-558}\par
          \begin{table}[H]
    % \begin{table}[H]
    % \\ '' '0'
        \begin{center}
      \label{m38800*eip-558}
    \noindent
    \tabletail{%
        \hline
        \multicolumn{3}{|p{\mytableboxwidth}|}{\raggedleft \small \sl continued on next page}\\
        \hline
      }
      \tablelasttail{}
      \begin{xtabular}[t]{|l|c|c|}\hline
        Toepassing &
        Laagste frekwensie (kHz) &
        Hoogste frekwensie (kHz)% make-rowspan-placeholders
     \tabularnewline\cline{1-1}\cline{2-2}\cline{3-3}
      %--------------------------------------------------------------------
        Skoonmaak (bv. Juweliersware) &
        20 &
        40% make-rowspan-placeholders
     \tabularnewline\cline{1-1}\cline{2-2}\cline{3-3}
      %--------------------------------------------------------------------
        Toets vir foute in materiale &
        50 &
        500% make-rowspan-placeholders
     \tabularnewline\cline{1-1}\cline{2-2}\cline{3-3}
      %--------------------------------------------------------------------
        Sweis van plastiek &
        15 &
        40% make-rowspan-placeholders
     \tabularnewline\cline{1-1}\cline{2-2}\cline{3-3}
      %--------------------------------------------------------------------
     Gewas wegsnyding &
        250 &
        2000% make-rowspan-placeholders
     \tabularnewline\cline{1-1}\cline{2-2}\cline{3-3}
      %--------------------------------------------------------------------
    \end{xtabular}
      \end{center}
    \caption{Verskillende gebruike van ultraklank en die toepaslike frekwensies.}
\end{table}
    \par
  \par 

\begin{minipage}{.5\textwidth}
\begin{center}
\textbf{Ultraklank beeld van 'n ongebore baba}\par
\includegraphics[width=.8\columnwidth]{photos/ultrasound_mbaylor_flickr.jpg}
\par\textit{Beeld deur mbaylor op Flickr.}
\end{center}
\end{minipage}
\begin{minipage}{.5\textwidth}

Die mees algemene gebruik van ultraklank is beeldskepping in industri\"ele en mediese toepassings. The gebruik van ultraklank om beelde te skep is basseer op die weerkaatsing en oordrag van 'n gold by 'n grens (wanneer die golf van een materiaal na 'n ander beweeg). Wanneer 'n ultraklank golf binne-in 'n voorwerp wat van verskillende materiale gemaak is beweeg, word die goolf gedeeltelik weerkaats en oorgedra as dit by 'n grens kom, bv. tussen been en spier of spier en vet. Die deel van die golf wat weerkaats is kan dan opgeneem word en gebruik word om 'n beeld te vorm van die voorwerp. \par

Ultraklank word in mediese sirkels gebruik om spier en sagte weefsel te visualiseer, wat dit handig maak om na organe te kyk. Dit word ook gereeld gebruik in verwagtende moeders. Ultraklank is 'n veilige, nie-indringende metode om binne die menslike liggaam te kyk. \par
      
\end{minipage}

Bronne van ultraklank kan gebruik word om gelokaliseerde areas in biologiese weefsel te verhit, met toepassing in fisioterapie en kanker behandeling. Gefokusde ultraklank bronne word ook gebruik om nierstene op te breek.\par

Ultrasoniese skoonmakers, soms genoem supersoniese skoonmakers, word by frekwensies van 20-40kHz gebruik om juweliersware, lense en ander optiese dele, horlosies, tandheelkundige instrumente, chirurgiese instrumente en ander industri\"ele dele skoon te maak.

Hierdie skoonmakers bestaan uit 'n houer met 'n vloeistof in met die voorwerp wat skoongemaak word binne-in geplaas. Ultrasoniese golwe word dan in die vloeistof oordgedra. Die meganisme wat die skoonmaker maak werk is die energie wat vrygelaat word as daar miljoene mikroskopiese borrels bars in die vloeistof.\par
\label{m38800*notfhsst!!!underscore!!!id482}


\IFact{Ultraklank gen\-e\-ra\-tor\-/\-luid\-spre\-ker stelses word verkoop wat beweer hulle kan knaagdiere en insekte wegjaag, maar daar is geen wetenskaplike bewyse dait hierdie toestelle werk nie; beheerde toetse wys dat knaagdiere vinnig leer dat die stelsels skadeloos is.}

\section*{Fisika van die oor en die gehoor [Vir Kennis]}
            \nopagebreak
\begin{figure}[H]
\begin{center}
\includegraphics[width=0.65\textwidth]{HumanEar-GrayScale.pdf}
\end{center}
\caption{Diagram van die menslike oor }
\label{Human Ear}
\end{figure}


Die menslike oor is opgedeel in drie hoof dele: die buite-, middel- en binne-oor. Laat ons 'n klankgolf volg soos dit beweeg van die pinna (buitenste deel van oor) na die gehoorsenuwee, wat 'n sein na die brein stuur. Die pinna is die deel van die oor waaraan ons tipies dink as ons van ore praat. Sy hoofdoel is om klankgolwe te vang en te fokus. Die golf beweeg dan deur die oorkanaal waarna dit die oordrom ontmoet. Die lugdrukvariasies van die klankgold maak dat die oordrom vibreer. Die 3 klein beentjies van die middel-oor, die malleus (hamer), incus (aanbeeldbeentjie) en stiebeuels dra dan die sein na die elliptiese venster toe. Dit is die begin van die binne-oor. Vandaar word die klank deur die vloeistof in die binne-oor versend en word as klank interpreteer deur die brein. Die binne-oor, wat van halfronde kanale gemaak is, die cochlea en die gehoorsenuwee is gevul met vloeistof. Die vloeistof laat die liggaam toe om vinnige bewegings te voel om om balans te hou.
\par

Daar is klanke wat die peindrumpel kan oorskry. Blootstelling aan hierdie klanke kan onmiddelike gehoorskade veroorsaak. In werklikheid kan blootstelling aan klanke oor 80 dB jou gehoor beskadig oor tyd. Stappe kan geneem word om skade te vermy, bv. deur oorpluisies of oormowwe aan te sit. Die beperking van durasie van blootstelling asook die afstand tussen jou en die bron van die klank is ook belangrike stappe om te volg om jou gehoor te beskerm.\par

\pagebreak
\begin{groupdiscussion}{Importance of Safety Equipment }
\label{m38800*id185111}Working in groups of 5, discuss the importance of safety equipment such as ear protectors for workers in loud environments, e.g. those who use jack hammers or direct aeroplanes to their parking bays. Write up your conclusions in a one page report. Some prior research into the importance of safety equipment might be necessary to complete this group discussion. 
\end{groupdiscussion} 
            \section{Summary}
            \nopagebreak
      \label{m38800*id185628}\begin{enumerate}[noitemsep, label=\textbf{\arabic*}. ] 
            \label{m38800*uid14}\item Klank waves are longitudinal waves
\label{m38800*uid15}\item The \textbf{frequency} of a klank is an indication of how high or low the \textsl{pitch} of the klank is.
\label{m38800*uid16}\item The human ear can hear frequencies from 20 to 20~000 Hz.
\textbf{Infrasound} waves have frequencies lower than 20 Hz.
\textbf{Ultrasound} waves have frequencies higher than 20~000 Hz.
\label{m38800*uid17}\item The \textbf{amplitude} of a klank determines its \textsl{loudness} or volume.
\label{m38800*uid18}\item The \textbf{tone} is a measure of the \textsl{quality} of a klank wave.
\label{m38800*uid19}\item The speed of klank in air is around $340\phantom{\rule{2pt}{0ex}}\text{m}\ensuremath{\cdot}\text{s}{}^{-1}$. It is dependent on the temperature, height above sea level and the phase of the medium through which it is travelling.
\label{m38800*uid20}\item Klank travels faster when the medium is hot.
\label{m38800*uid21}\item Klank travels faster in a solid than a liquid and faster in a liquid than in a gas.
\label{m38800*uid22}\item Klank travels faster at sea level where the air pressure is higher.
\label{m38800*uid23}\item The intensity of a klank is the energy transmitted over a certain area. Intensity is a measure of frequency.
\label{m38800*uid24}\item Ultrasound can be used to form pictures of things we cannot see, like unborn babies or tumors.
\label{m38800*uid25}\item Echolocation is used by animals such as dolphins and bats to ``see'' their surroundings by using ultrasound.
\label{m38800*uid26}\item Ships use sonar to determine how deep the ocean is or to locate shoals of fish.
\end{enumerate}
    \label{m38800*cid9}
            


\begin{eocexercises}{Klank}
            \nopagebreak
      \label{m38800*id185882}\begin{enumerate}[noitemsep, label=\textbf{\arabic*}. ] 
            \label{m38800*uid27}\item Choose a word from column B that best describes the concept in column A.
    % \textbf{m38800*id185898}\par
          \begin{center}
\begin{tabular}{ll}
\textbf{Column A} & \textbf{Column B} \\ \hline
pitch of klank \ \ \ & amplitude \\
loudness of klank \ \ \ \ \ \ \ \ \ & frequency \\
quality of klank \ \ \ & speed \\
& waveform \\
\end{tabular}
\end{center}
    \par
          \label{m38800*uid28}\item A tuning fork, a violin string and a loudspeaker are producing sounds. This is because they are all in a state of:
\label{m38800*id185988}\begin{enumerate}[noitemsep, label=\textbf{\alph*}. ] 
            \label{m38800*uid29}\item compression
\label{m38800*uid30}\item rarefaction
\label{m38800*uid31}\item rotation
\label{m38800*uid32}\item tension
\label{m38800*uid33}\item vibration
\end{enumerate}
                \label{m38800*uid34}\item What would a drummer do to make the klank of a drum give a note of lower pitch?
\label{m38800*id186066}\begin{enumerate}[noitemsep, label=\textbf{\alph*}. ] 
            \label{m38800*uid35}\item hit the drum harder
\label{m38800*uid36}\item hit the drum less hard
\label{m38800*uid37}\item hit the drum near the edge
\label{m38800*uid38}\item loosen the drum skin
\label{m38800*uid39}\item tighten the drum skin
\end{enumerate}
                \label{m38800*uid40}\item What is the approximate range of audible frequencies for a healthy human?
\label{m38800*id186144}\begin{enumerate}[noitemsep, label=\textbf{\alph*}. ] 
            \label{m38800*uid41}\item 0.2 Hz $\to $ 200 Hz
\label{m38800*uid42}\item 2 Hz $\to $ 2 000 Hz
\label{m38800*uid43}\item 20 Hz $\to $ 20 000 Hz
\label{m38800*uid44}\item 200 Hz $\to $ 200 000 Hz
\label{m38800*uid45}\item 2 000 Hz $\to $ 2 000 000 Hz
\end{enumerate}
                \label{m38800*uid46}\item X and Y are different wave motions. In air, X travels much faster than Y but has a much shorter wavelength. Which types of wave motion could X and Y be?
    % \textbf{m38800*id186268}\par
          \begin{table}[H]
    % \begin{table}[H]
    % \\ 'id2920184' '1'
        \begin{center}
      \label{m38800*id186268}
    \noindent
    \tabletail{%
        \hline
        \multicolumn{3}{|p{\mytableboxwidth}|}{\raggedleft \small \sl continued on next page}\\
        \hline
      }
      \tablelasttail{}
      \begin{xtabular}[t]{|l|l|l|}\hline
         &
        \uline{X} &
        \uline{Y}% make-rowspan-placeholders
     \tabularnewline\cline{1-1}\cline{2-2}\cline{3-3}
      %--------------------------------------------------------------------
        A &
        microwaves &
        red light% make-rowspan-placeholders
     \tabularnewline\cline{1-1}\cline{2-2}\cline{3-3}
      %--------------------------------------------------------------------
        B &
        radio &
        infra red% make-rowspan-placeholders
     \tabularnewline\cline{1-1}\cline{2-2}\cline{3-3}
      %--------------------------------------------------------------------
        C &
        red light &
        sound% make-rowspan-placeholders
     \tabularnewline\cline{1-1}\cline{2-2}\cline{3-3}
      %--------------------------------------------------------------------
        D &
        klank &
        ultraviolet% make-rowspan-placeholders
     \tabularnewline\cline{1-1}\cline{2-2}\cline{3-3}
      %--------------------------------------------------------------------
        E &
        ultraviolet &
        radio% make-rowspan-placeholders
     \tabularnewline\cline{1-1}\cline{2-2}\cline{3-3}
      %--------------------------------------------------------------------
    \end{xtabular}
      \end{center}
\end{table}
    \par
          \label{m38800*uid47}\item Astronauts are in a spaceship orbiting the moon. They see an explosion on the surface of the moon. Why can they not hear the explosion?
\label{m38800*id186399}\begin{enumerate}[noitemsep, label=\textbf{\alph*}. ] 
            \label{m38800*uid48}\item explosions do not occur in space
\label{m38800*uid49}\item klank cannot travel through a vacuum
\label{m38800*uid50}\item klank is reflected away from the spaceship
\label{m38800*uid51}\item klank travels too quickly in space to affect the ear drum
\label{m38800*uid52}\item the spaceship would be moving at a supersonic speed
\end{enumerate}
                \label{m38800*uid53}\item A man stands between two cliffs as shown in the diagram and claps his hands once.
    \setcounter{subfigure}{0}
	\begin{figure}[H] % horizontal\label{m38800*id186481}
   \begin{center}
{
\begin{pspicture}(0,-1.0985937)(7.6665626,1.1185937)
\psline[linewidth=0.04cm](1.260625,0.92140627)(1.260625,-1.0785937)
\psline[linewidth=0.04cm](6.260625,0.94140625)(6.260625,-1.0585938)
%\usefont{T1}{ptm}{m}{n}
\rput(0.4575,0.47140625){cliff 1}
%\usefont{T1}{ptm}{m}{n}
\rput(7.1339064,0.49140626){cliff 2}
\psline[linewidth=0.04cm](1.260625,-1.0785937)(6.240625,-1.0585938)
\psline[linewidth=0.04cm](4.260625,0.88140625)(4.240625,0.52140623)
\psline[linewidth=0.04cm,arrowsize=0.1029cm 2.04,arrowlength=1.44,arrowinset=0.4]{<->}(1.240625,0.70140624)(4.240625,0.70140624)
\psline[linewidth=0.04cm,arrowsize=0.0929cm 2.05,arrowlength=1.45,arrowinset=0.4]{<->}(4.300625,0.70140624)(6.260625,0.70140624)
\pscircle[linewidth=0.04,dimen=outer](4.250625,0.21140625){0.25}
\psline[linewidth=0.04cm](4.240625,-0.05859375)(4.240625,-0.6585938)
\psline[linewidth=0.04cm](4.240625,-0.6585938)(3.940625,-1.0585938)
\psline[linewidth=0.04cm](4.240625,-0.6585938)(4.540625,-1.0585938)
\psline[linewidth=0.04cm](4.040625,-0.25859374)(4.440625,-0.25859374)
%\usefont{T1}{ptm}{m}{n}
\rput(2.6015625,0.94140625){\footnotesize 165 m}
%\usefont{T1}{ptm}{m}{n}
\rput(5.2015624,0.94140625){\footnotesize 110 m}
\end{pspicture}
}
\end{center}
 \end{figure}       
Assuming that the velocity of klank is $330\phantom{\rule{2pt}{0ex}}\text{m}\ensuremath{\cdot}\text{s}{}^{-1}$, what will be the time interval between the two loudest echoes?
\label{m38800*id186509}\begin{enumerate}[noitemsep, label=\textbf{\alph*}. ] 
            \label{m38800*uid54}\item $\frac{2}{3}\phantom{\rule{2pt}{0ex}}\text{s}$
\label{m38800*uid55}\item $\frac{1}{6}\phantom{\rule{2pt}{0ex}}\text{s}$
\label{m38800*uid56}\item $\frac{5}{6}\phantom{\rule{2pt}{0ex}}\text{s}$
\label{m38800*uid57}\item 1 s
\label{m38800*uid58}\item $\frac{1}{3}\phantom{\rule{2pt}{0ex}}\text{s}$
\end{enumerate}
                \label{m38800*uid59}\item A dolphin emits an ultrasonic wave with frequency of 0,15 MHz. The speed of the ultrasonic wave in water is $1 500\phantom{\rule{2pt}{0ex}}\text{m}\ensuremath{\cdot}\text{s}{}^{-1}$. What is the wavelength of this wave in water?
\label{m38800*id186650}\begin{enumerate}[noitemsep, label=\textbf{\alph*}. ] 
            \label{m38800*uid60}\item 0,1 mm
\label{m38800*uid61}\item 1 cm
\label{m38800*uid62}\item 10 cm
\label{m38800*uid63}\item 10 m
\label{m38800*uid64}\item 100 m
\end{enumerate}
                \label{m38800*uid65}\item The amplitude and frequency of a klank wave are both increased. How are the loudness and pitch of the klank affected?
    % \textbf{m38800*id186726}\par
          \begin{table}[H]
    % \begin{table}[H]
    % \\ 'id2920575' '1'
        \begin{center}
      \label{m38800*id186726}
    \noindent
    \tabletail{%
        \hline
        \multicolumn{3}{|p{\mytableboxwidth}|}{\raggedleft \small \sl continued on next page}\\
        \hline
      }
      \tablelasttail{}
      \begin{xtabular}[t]{|l|l|l|}\hline
         &
        \uline{loudness} &
        \uline{pitch}% make-rowspan-placeholders
     \tabularnewline\cline{1-1}\cline{2-2}\cline{3-3}
      %--------------------------------------------------------------------
        A &
        increased &
        raised% make-rowspan-placeholders
     \tabularnewline\cline{1-1}\cline{2-2}\cline{3-3}
      %--------------------------------------------------------------------
        B &
        increased &
        unchanged% make-rowspan-placeholders
     \tabularnewline\cline{1-1}\cline{2-2}\cline{3-3}
      %--------------------------------------------------------------------
        C &
        increased &
        lowered% make-rowspan-placeholders
     \tabularnewline\cline{1-1}\cline{2-2}\cline{3-3}
      %--------------------------------------------------------------------
        D &
        decreased &
        raised% make-rowspan-placeholders
     \tabularnewline\cline{1-1}\cline{2-2}\cline{3-3}
      %--------------------------------------------------------------------
        E &
        decreased &
        lowered% make-rowspan-placeholders
     \tabularnewline\cline{1-1}\cline{2-2}\cline{3-3}
      %--------------------------------------------------------------------
    \end{xtabular}
      \end{center}
\end{table}
    \par
          \label{m38800*uid66}\item A jet fighter travels slower than the speed of sound. Its speed is said to be:
\label{m38800*id186857}\begin{enumerate}[noitemsep, label=\textbf{\alph*}. ] 
            \label{m38800*uid67}\item Mach 1
\label{m38800*uid68}\item supersonic
\label{m38800*uid69}\item subsonic
\label{m38800*uid70}\item hypersonic
\label{m38800*uid71}\item infrasonic
\end{enumerate}
                \label{m38800*uid72}\item A klank wave is different from a light wave in that a klank wave is:
\label{m38800*id186936}\begin{enumerate}[noitemsep, label=\textbf{\alph*}. ] 
            \label{m38800*uid73}\item produced by a vibrating object and a light wave is not.
\label{m38800*uid74}\item not capable of traveling through a vacuum.
\label{m38800*uid75}\item not capable of diffracting and a light wave is.
\label{m38800*uid76}\item capable of existing with a variety of frequencies and a light wave has a single frequency.
\end{enumerate}
                \label{m38800*uid77}\item At the same temperature, klank waves have the fastest speed in:
\label{m38800*id187004}\begin{enumerate}[noitemsep, label=\textbf{\alph*}. ] 
            \label{m38800*uid78}\item rock
\label{m38800*uid79}\item milk
\label{m38800*uid80}\item oxygen
\label{m38800*uid81}\item sand
\end{enumerate}
                \label{m38800*uid82}\item Two klank waves are traveling through a container of nitrogen gas. The first wave has a wavelength of 1,5~m, while the second wave has a wavelength of 4,5~m. The velocity of the second wave must be:
\label{m38800*id187073}\begin{enumerate}[noitemsep, label=\textbf{\alph*}. ] 
            \label{m38800*uid83}\item $\frac{1}{9}$ the velocity of the first wave.
\label{m38800*uid84}\item $\frac{1}{3}$ the velocity of the first wave.
\label{m38800*uid85}\item the same as the velocity of the first wave.
\label{m38800*uid86}\item three times larger than the velocity of the first wave.
\label{m38800*uid87}\item nine times larger than the velocity of the first wave.
\end{enumerate}
                \label{m38800*uid88}\item Klank travels at a speed of 340~m$\ensuremath{\cdot}$s${}^{-1}$. A straw is 0,25 m long. The standing wave set up in such a straw with one end closed has a wavelength of 1,0~m. The standing wave set up in such a straw with both ends open has a wavelength of 0,50 m.
\label{m38800*id187205}\begin{enumerate}[noitemsep, label=\textbf{\alph*}. ] 
            \label{m38800*uid89}\item calculate the frequency of the klank created when you blow across the straw with the bottom end closed.
\label{m38800*uid90}\item calculate the frequency of the klank created when you blow across the straw with the bottom end open.
\end{enumerate}
                \label{m38800*uid91}\item A lightning storm creates both lightning and thunder. You see the lightning almost immediately since light travels at $3\ensuremath{\times}{10}^{8}\phantom{\rule{0.166667em}{0ex}}\text{m}\ensuremath{\cdot}{\text{s}}^{-1}$. After seeing the lightning, you count 5~s and then you hear the thunder. Calculate the distance to the location of the storm.\newline
\label{m38800*uid92}\item A person is yelling from a second story window to another person standing at the garden gate, 50~m away. If the speed of klank is 344~m$\ensuremath{\cdot}$s${}^{-1}$, how long does it take the klank to reach the person standing at the gate?\newline
\label{m38800*uid93}\item A piece of equipment has a warning label on it that says, "Caution! This instrument produces 140 decibels." What safety precaution should you take before you turn on the instrument?\newline
\label{m38800*uid94}\item What property of klank is a measure of the amount of energy carried by a klank wave?\newline
\label{m38800*uid96}\item Person 1 speaks to person 2. Explain how the klank is created by person 1 and how it is possible for person 2 to hear the conversation.\newline
\label{m38800*uid97}\item Klank cannot travel in space. Discuss what other modes of communication astronauts can use when they are outside the space shuttle?\newline
\label{m38800*uid98}\item An automatic focus camera uses an ultrasonic klank wave to focus on objects. The camera sends out klank waves which are reflected off distant objects and return to the camera. A sensor detects the time it takes for the waves to return and then determines the distance an object is from the camera. If a klank wave (speed = 344~m$\ensuremath{\cdot}$s${}^{-1}$) returns to the camera 0,150~s after leaving the camera, how far away is the object?\newline
\label{m38800*uid99}\item Calculate the frequency (in Hz) and wavelength of the annoying klank made by a mosquito when it beats its wings at the average rate of 600 wing beats per second. Assume the speed of the klank waves is 344~m$\ensuremath{\cdot}$s${}^{-1}$.        
\label{m38800*uid100}\item How does halving the frequency of a wave source affect the speed of the waves?\newline
\label{m38800*uid101}\item Humans can detect frequencies as high as 20~000 Hz. Assuming the speed of klank in air is 344~m$\ensuremath{\cdot}$s${}^{-1}$, calculate the wavelength of the klank corresponding to the upper range of audible hearing.\newline
\label{m38800*uid102}\item An elephant trumpets at 10~Hz. Assuming the speed of klank in air is 344~m$\ensuremath{\cdot}$s${}^{-1}$, calculate the wavelength of this infrasonic klank wave made by the elephant.\newline
\label{m38800*uid103}\item A ship sends a signal out to determine the depth of the ocean. The signal returns 2,5 seconds later. If klank travels at
1450 m.s${}^{-1}$ in sea water, how deep is the ocean at that point?\newline
\label{m38783*uid44}\item A person shouts at a cliff and hears an echo from the cliff 1~s later. If the speed of klank is $344\phantom{\rule{2pt}{0ex}}\text{m}\ensuremath{\cdot}\text{s}{}^{-1}$, how far away is the cliff?\newline
\label{m38783*uid37}\item Select a word from Column B that best fits the description in Column A:
    % \textbf{m38783*id293899}\par
          \begin{center}
\begin{tabular}{ll}
\textbf{Column A} & \textbf{Column B} \\ \hline
waves in the air caused by vibrations & longitudinal waves \\
waves that move in one direction, but medium moves in another & frequency \\
waves and medium that move in the same direction & white noise \\
the distance between consecutive points of a wave which are in phase & amplitude \\
how often a single wavelength goes by & klank waves \\
half the difference between high points and low points of waves & standing waves \\
the distance a wave covers per time interval & transverse waves \\
the time taken for one wavelength to pass a point & wavelength \\
& music \\
& sounds \\
& wave speed \\
\end{tabular}
\end{center}
\end{enumerate}
  \label{m38800**end}
  \label{9b5d72dd5f0585e544578ab90a9956a8**end}
\par \raisebox{-5 pt}{\includegraphics[width=0.5cm]{col11305.imgs/summary_www.png}} Find the answers with the shortcodes:
 \par \begin{tabular}[h]{cccccc}
 (1.) l4Y  &  (2.) l41  &  (3.) l4C  &  (4.) l4a  &  (5.) l4x  &  (6.) l4c  &  (7.) l4O  &  (8.) l43  &  (9.) l4i  &  (10.) l4l  &  (11.) l4q  &  (12.) lgh  &  (13.) lgS  &  (14.) lgJ  &  (15.) lgu  &  (16.) lgz  &  (17.) lgt  &  (18.) lge  &  (19.) lgM  &  (20.) lgL  &  (21.) lgF  &  (22.) lg6  &  (23.) lgH  &  (24.) lgs  &  (25.) lgo  &  (26.) lgA  & \end{tabular}

\end{eocexercises}
